\documentclass{article}
\usepackage[utf8]{inputenc}
\usepackage{amsmath,amssymb}
\usepackage[pdfborder={0 0 0}]{hyperref}
\usepackage{csquotes}
\newcommand\set[1]{\left\{#1\right\}}

\begin{document}
\section{Exercise 8}

\subsection{Royal Flush}
10, jack, queen, king, ace in one suit
\[ \sum_{s \in \set{\clubsuit, \diamondsuit, \diamondsuit, \spadesuit}} \frac{5}{52} \cdot \frac{4}{51} \cdot \frac{3}{50} \cdot \frac{2}{49} \cdot \frac{1}{48} = 4 \cdot \frac{120}{311875200} \approx 1.5390\cdot 10^{-6} \]
Consider one suit. It is okay to pick one of five cards. Then 4 admissible cards are left. Then 3 cards. Then 2. Then 1. For four suits.

\subsection{Straight Flush}
5 cards in same suit with consecutive rank.

%\[ \sum_{s \in \set{\clubsuit, \diamondsuit, \diamondsuit, \spadesuit}} \frac{13}{52} \cdot \frac{4}{51} \cdot \frac{3}{50} \cdot \frac{2}{49} \cdot \frac{1}{48} = 4 \cdot \frac{312}{311875200} = 4.0016\cdot 10^{-6} \]
%Card 1 must match one specific color, the second card must be one of 4 neighboring cards of same suit. This continues to 5 cards.
%
%Now we need to exclude the royal flush:
%\[ 4 \cdot \frac{312}{311875200} - 4 \cdot \frac{120}{311875200} = \frac{768}{311875200} \approx 2.4625 \cdot 10^{-6} \]
%
%This result is wrong according to Wikipedia (should be more likely).

The ranks can begin with 1, 2, 3, \dots up to 10. There are 4 suits.
\[ \frac{4 \cdot 10}{311875200} \]

\subsection{Poker / Four of a kind}
4 cards of same rank.
\[ \frac{1}{1} \cdot \frac{3}{51} \cdot \frac{2}{50} \cdot \frac{1}{49} = \frac{6}{124950} \]

\subsection{Full House}
3 cards of same rank and 2 cards of same rank.
\[ \frac11 \cdot \frac{3}{51} \cdot \frac{2}{50} \cdot \frac{48}{49} \cdot \frac{3}{48} = \frac{864}{5997600} \approx 0.0001 \]
The first choice is arbitrary, the second choice must be of same rank (3 of same suit are left, 51 cards left), as well as the third choice.
The fourth choice excludes 4 cards of same rank (the fourth card must not be picked!), hence $52 - 3 - 1 = 48$. The fifth choice must be of same rank (there are 3 suits of same rank left).

\subsection{Flush}
5 cards of same suit
\[ \frac{1}{1} \cdot \frac{12}{51} \cdot \frac{11}{50} \cdot \frac{10}{49} \cdot \frac{9}{48} = \frac{11880}{5997600} \approx 0.0020 \]

Verified with \href{https://en.wikipedia.org/wiki/Poker_probability}{Wikipedia: Poker probability}

\section{Whiteboard solution}

The solutions do not necessarily respect the previous solutions (a Royal Flush is a Flush).

\[ \mathbb P(\text{\enquote{Royal Flush}}) = \frac{{4 \choose 1} {47 \choose 1}}{{52 \choose 5}} \approx 1.539 \cdot 10^{-6} \]
There are $4$ different Royal Flushs (because of $4$ suits).

\[ \Omega = \{A \subseteq \{1, \dots, 4\} \times \{2, \dots, 14\} \,|\, |A| = 5\} \]

\[ \mathbb P(\text{\enquote{Straight Flush}}) = \frac{{4 \choose 1} {9 \choose 1}}{{52 \choose 5}} = \frac{36}{{52 \choose 5}} \approx 0.0000138 \]

\[ \mathbb P(\text{\enquote{Poker}}) = \frac{{13 \choose 1} {48 \choose 1}}{{52 \choose 5}} \approx 0.0002401 \]
\[ \mathbb P(\text{\enquote{Full House}}) = \frac{{13 \choose 1} {4 \choose 3} {12 \choose 1} {4 \choose 2}}{{52 \choose 5}} \approx 0.00144 \]
\[ \mathbb P(\text{\enquote{Flush}}) = \frac{{4 \choose 1} {13 \choose 5}}{{52 \choose 5}} \]
minus Royal Flush: $- \frac{36}{{52 \choose 5}} - \frac{4}{{52 \choose 5}} \approx 0.001965$

\end{document}