\documentclass{article}
\usepackage[utf8]{inputenc}
\usepackage{amsmath,amssymb}
\usepackage{csquotes}
\usepackage[pdfborder={0 0 0}]{hyperref}
\newcommand\set[1]{\left\{#1\right\}}
\newcommand\card[1]{\left|#1\right|}

\begin{document}
\section{Exercise 10}

\begin{verbatim}
#!/usr/bin/env python

import itertools

def experiment(n):
    distribution = [0] * (n + 1)
    for outcome in itertools.product({'L', 'R'}, repeat=2*n-1):
        bags = {'L': n, 'R': n}
        for out in outcome:
            bags[out] -= 1
            if bags[out] == 0:
                break

        print(outcome, bags)

        outcome_k = max(bags.values())
        distribution[outcome_k] += 1

    return distribution


print(experiment(5))
\end{verbatim}

For $n=5$, it gives $k=1$ in 140 cases, $k=2$ in 140 cases, $k=3$ in 120 cases, $k=4$ in 80 cases and $k=5$ in 32 cases.
See OEIS A172068 \enquote{Triangular array T(n,k) = the number of n-step one dimensional walks that return to the origin exactly k times.}.


\[ \Omega = \set{0,1}^{2n-k} \]
\[ \mathcal A = \mathcal P(\Omega) \]
\[ \mathbb P(A_k) = \frac{\card{A_k}}{\card{\Omega}} \]

$0$ represents a left drawing, $1$ represents a right drawing.

\[ \implies A_k = R_k + L_k \]

Regards $L_k$, there must be a $1$ at its tail.
Before that $(n+1)$ ones and $n-k$ zeros will be distributed on $2n - k - 1$ places.
\[ \implies {2n-k-1 \choose n-1} \]



The probability that right-sided box is chosen: ${1 \over 2}^{n-1}$
The probability that left-sided box is chosen: ${1 \over 2}^{n-k}$

In the last trial, we chose a right-sided box, so $\cdot \frac12$.
The probability that right-sided box is empty and left-sided box has
$k$ elements,
\[
  {2n - k - 1 \over n-1} \cdot {1 \over 2}^{n-1} {1 \over 2} {1 \ over 2}^{n-k} = {2n-k-1 \over n-1} \cdot {1 \over 2}^{2n-k}
\]
$L_k$ has same cardinality like $R_k$, hence times $2$.

\[
  \mathbb P(A_k) = 2 \cdot {2n-k-1 \choose n-1} {1 \over 2}^{2n-k} = {2n-k-1 \over n-1} \cdot {1 \over 2}^{2n-k-1}
\]

\end{document}