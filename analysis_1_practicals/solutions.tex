\documentclass[a4paper]{article}
\usepackage{amsmath}
\usepackage{amssymb}
\usepackage{amsthm}
\usepackage{baskervald}
\usepackage{csquotes}
\usepackage{enumerate}
\usepackage[T1]{fontenc}
\usepackage[margin=1in]{geometry}
\usepackage[utf8]{inputenc}
\usepackage{mathtools}
\usepackage{mdframed}
\usepackage{pifont}
\usepackage{stmaryrd}
\usepackage{wasysym}
\usepackage{xcolor}

\theoremstyle{definition}
\newmdtheoremenv[%
  backgroundcolor=white,
  linecolor=white!60!black,
  linewidth=3pt]{ex}{Exercise}

\parskip5pt
\parindent0pt

\newcommand\abs[1]{\left|\thinspace #1\thinspace\right|}
\newcommand\set[1]{\left\{#1\right\}}
\newcommand\setdef[2]{\left\{#1\,\middle|\,#2\right\}}
\newcommand\card[1]{\left|\,#1\,\right|}
\newcommand\yes{\checkmark}
\newcommand\no{\ding{55}}

\title{Analysis 1 -- Practicals}
\author{Lukas Prokop}
\date{October 2015 to January 2016}

\begin{document}
\maketitle
\tableofcontents

\clearpage
\section{Exercise 1}
\begin{ex}
  Let $p$, $q$ and $r$ be statements. Prove the distributive law using the truth table:
  \[ p \land (q \lor r) \Leftrightarrow (p \land q) \lor (p \land r) \]
\end{ex}

\begin{table}[!h]
  \begin{center}
    \begin{tabular}{ccc|cc|ccc}
        $p$ & $q$ & $r$ & $(q \lor r)$ & $(p \land (q \lor r))$ & $(p \land q)$ & $(p \land r)$ & $(p \land q) \lor (p \land r)$ \\
      \hline
        0 & 0 & 0       & 0 & 0                                 & 0 & 0 & 0 \\
        0 & 0 & 1       & 1 & 0                                 & 0 & 0 & 0 \\
        0 & 1 & 0       & 1 & 0                                 & 0 & 0 & 0 \\
        0 & 1 & 1       & 1 & 0                                 & 0 & 0 & 0 \\
        1 & 0 & 0       & 0 & 0                                 & 0 & 0 & 0 \\
        1 & 0 & 1       & 1 & 1                                 & 0 & 1 & 1 \\
        1 & 1 & 0       & 1 & 1                                 & 1 & 0 & 1 \\
        1 & 1 & 1       & 1 & 1                                 & 1 & 1 & 1
    \end{tabular}
  \end{center}
\end{table}

Therefore the truthtable of both statements is equivalent.
Two boolean statements are equivalent iff their truthtable is equivalent.

\section{Exercise 2}
\begin{ex}
  Formalize the following colloquial combination of statements $p$, $q$ and $r$
  in propositional calculus. Furthermore always create the negation:
  \begin{itemize}
    \item \enquote{Under the assumption, that $p$ or $q$ holds, we conclude that $r$ cannot be true.}
    \item \enquote{It's a requirement for $r$, that $p$ and $q$ hold.}
    \item \enquote{$p$ or $q$ holds, but $p$ and $q$ exclude each other}
  \end{itemize}
\end{ex}

\begin{itemize}
  \item \enquote{Under the assumption, that $p$ or $q$ holds, we conclude that $r$ cannot be true.}
    \[ (p \lor q) \rightarrow \neg r \]
    Negation: $(p \lor q) \land r$
  \item \enquote{It's a requirement for $r$, that $p$ and $q$ hold.}
    \[ r \rightarrow (p \land q) \]
    Negation: $r \land (\neg p \lor \neg q)$
  \item \enquote{$p$ or $q$ holds, but $p$ and $q$ exclude each other}
    \[ (p \lor q) \land \neg (p \land q) \]
    \[ \Leftrightarrow (p \dot\lor q) \Leftrightarrow (p \oplus q) \]
    Negation: $p \leftrightarrow q$
\end{itemize}

\section{Exercise 3}
\begin{ex}
  Mister Travelmuch bought a Eurail ticket in August 1980 and has organized a large journey.
  When moving flats, he list his photo album, he tries to remember,
  which cities of Paris, Madrid and Rome he visited.

  He remembers:
  \begin{itemize}
    \item If he was not in Madrid, then he was in Paris and Rome.
    \item If he was in Paris, he was not in Madrid and not in Rome.
    \item If he was not in Paris, he was also not in Rome.
  \end{itemize}

  Use appropriate variables for the statements and help Mister Travelmuch
  determining which cities (or city) he visited in 1980.
\end{ex}

Let $M$, $P$ and $R$ be visits to Madrid, Pairs and Rome respectively.
We formalize:

\begin{align*}
  \neg M &\implies (P \land R) \\
       P &\implies (\neg M \land \neg R) \\
  \neg P &\implies \neg R
\end{align*}

As far as all three conditions need to be satisfied, we conjoint them:
\[
    \left[\neg M \rightarrow (P \land R)\right] \land
    \left[P \rightarrow (\neg M \land \neg R)\right] \land
    \left[\neg P \rightarrow \neg R\right]
\]

We apply $(a \rightarrow b) \Leftrightarrow (\neg a \lor b)$ to all three statements:
\[
    \left[\neg(\neg M) \lor (P \land R)\right] \land
    \left[\neg P \lor (\neg M \land \neg R)\right] \land
    \left[\neg(\neg P) \lor \neg R\right]
\]
\dots and $\neg(\neg A) \Leftrightarrow A$:
\[
    \left[M \lor (P \land R)\right] \land
    \left[\neg P \lor (\neg M \land \neg R)\right] \land
    \left[P \lor \neg R\right]
\]
\dots and the distributive law holds:
\[
    \left[(M \lor P) \land (M \lor R)\right] \land
    \left[(\neg P \land \neg M) \lor (\neg P \land \neg R)\right] \land
    \left[P \lor \neg R\right]
\]
We reorder statements:
\[
    \left[(M \lor P) \land (M \lor R) \land (P \lor \neg R)\right] \land
    \left[(\neg P \land \neg M) \lor (\neg P \land \neg R)\right]
\]
\dots and again the distributive law:
\[
    \left[(M \lor P) \land (M \lor R) \land (P \lor \neg R) \land (\neg P \land \neg M)\right] \lor
    \left[(M \lor P) \land (M \lor R) \land (P \lor \neg R) \land (\neg P \land \neg R)\right]
\]
\[
    \left[(M \lor P) \land \neg P \land \neg M\right] \lor
    \left[(M \lor P) \land (M \lor R) \land (P \lor \neg R) \land (\neg P \land \neg R)\right]
\]

The left-hand side cannot be satisfied, but $M \land \neg P \land \neg R$ holds for the right side.
So,
\begin{itemize}
  \item In 1980, he was in Madrid.
  \item In 1980, he was not in Paris.
  \item In 1980, he was not in Rome.
\end{itemize}

\section{Exercise 4}
\begin{ex}
  Let $X$ be a set. Formalize the following colloquial combination of statements
  $p(x)$, $q(x)$, $r(x)$ and $s(x, y)$ with the help of quantifiers. Also create
  the negation:
  \begin{enumerate}
    \item \enquote{For all elements $x$ of the set $X$ for which $p(x)$ holds, also $q(x)$ or $r(x)$ holds.}
    \item \enquote{For all $x$ in $X$, there is one $y$ in $Y$ such that $s(x,y)$ holds.}
    \item \enquote{If $p(x)$ is not wrong for all $x$ in $X$, then $q(y)$ is true for at least one $y$ in $Y$.}
  \end{enumerate}
\end{ex}

\begin{enumerate}
  \item \enquote{For all elements $x$ of the set $X$ for which $p(x)$ holds, also $q(x)$ or $r(x)$ holds.}
    \[ \forall x \in X: p(x) \rightarrow q(x) \lor r(x) \]
    \[ \text{negation: }  \exists x \in X: p(x) \land (\neg q(x) \land \neg r(x)) \]
  \item \enquote{For all $x$ in $X$, there is one $y$ in $Y$ such that $s(x,y)$ holds.}
    \[ \forall x \in X \exists y \in Y: s(x, y) \]
    \[ \text{negation: }  \exists x \in X \forall y \in Y: \neg s(x, y) \]
  \item \enquote{If $p(x)$ is not wrong for all $x$ in $X$, then $q(y)$ is true for at least one $y$ in $Y$.}
    \[ (\exists x \in X: p(x)) \rightarrow (\exists y \in X: q(y)) \]
    \[ \text{negation: }  (\exists x \in X: p(x)) \land (\forall y \in X: \neg q(y)) \]
\end{enumerate}


\section{Exercise 5}
\begin{ex}
  Prove in three ways (direct, indirect, by contradiction):
  \[ \forall x \in \mathbb R: x^3 + 2x > 0 \Rightarrow x > 0 \]
\end{ex}

Consider $\phi$ to be given and $\varphi$ to be our conclusion.
Then the three ways of proving work as follows:
\begin{description}
  \item[Direct proof] $\phi \implies \varphi$
  \item[Indirect proof] $\neg\varphi \implies \neg\phi$ \\
    Because $\varphi \lor \neg\phi \Leftrightarrow \neg\phi \lor \varphi \Leftrightarrow \phi \rightarrow \varphi$.
  \item[Proof by contradiction] $(\neg (\phi \implies \varphi) \implies \bot) \implies (\phi \implies \varphi)$ \\
    Because $((\phi \rightarrow \varphi) \lor \bot) \rightarrow (\phi \rightarrow \varphi)
    \Leftrightarrow (\phi \rightarrow \varphi) \rightarrow (\phi \rightarrow \varphi)$.
\end{description}

\begin{description}
  \item[Direct proof]
    Assume,
    \[ x(x^2 + 2) > 0 \]
    This requires that
    \begin{itemize}
      \item both factors are non-zero
      \item and
        \begin{itemize}
          \item both factors are negative, or
          \item both factors are positive
        \end{itemize}
    \end{itemize}
    So,
    \[ (x \neq 0 \land (x^2 + 2) \neq 0) \land \left[(x < 0 \land (x^2 + 2) < 0) \lor (x > 0 \land (x^2 + 2) > 0)\right] \]
    As far as a square cannot be negative, $(x^2 + 2) < 0$ does not hold.
    \[ (x \neq 0 \land (x^2 + 2) \neq 0) \land \left[(x > 0 \land (x^2 + 2) > 0)\right] \]
    Therefore it must hold that
    \[ (x \neq 0) \land (x^2 + 2 \neq 0) \land (x > 0) \land (x^2 + 2 > 0) \]
    And so it holds that $x > 0$.
  \item[Indirect proof]
    Assume $x \leq 0$.
    Then $x \cdot x^2 \leq 0$.
    And also $x \cdot (x^2 + 2) \leq 0$.
    Which is $x^3 + 2x \leq 0$.
  \item[Proof by contradiction]
    Assume $x (x^2 + 2) > 0 \implies x \leq 0$.
    \[ \forall x \in \mathbb R: x \cdot \underbrace{(x^2 + 2)}_{\geq 2} > 0 \implies x \leq 0 \]
    \[ \forall x \in \mathbb R: \underbrace{x}_{\Rightarrow \geq 0} \cdot \underbrace{(x^2 + 2)}_{\geq 2} > 0 \implies x \leq 0 \]
    \[ \forall x \in \mathbb R: x > 0 \implies x \leq 0 \]
    \[ \text{\lightning} \]
    \[ \Rightarrow \forall x \in \mathbb R: x \cdot (x^2 + 2) > 0 \implies x > 0 \]
\end{description}

\section{Exercise 6}

\begin{ex}
  Let $p$, $q$ and $r$ be statements. Show that
  \begin{itemize}
    \item $(p \rightarrow q) \iff \neg (p \land \neg q)$  \hspace{30pt} \enquote{proof by contradiction}
    \item $\left[p \rightarrow (q \lor r)\right] \iff \left[(p \land \neg q) \rightarrow r\right]$
  \end{itemize}
\end{ex}

\subsection{Exercise 6a}

\[ (p \rightarrow q) \iff \neg (p \land \neg q) \]
\[ (\neg p \lor q) \iff (\neg p \lor q) \]

\subsection{Exercise 6b}

\begin{align*}
  \left(p \rightarrow (q \lor r)\right) &\iff \left((p \land \neg q) \rightarrow r\right) \\
  \neg p \lor (q \lor r) &\iff \neg (p \land \neg q) \lor r \\
  (\neg p \lor q) \lor r &\iff (\neg p \lor q) \lor r
\end{align*}

\section{Exercise 7}
%
\begin{ex}
  Let $A$, $B$, $C$ and $D$ be sets. Prove that
  \begin{itemize}
    \item $(A \setminus B) \cap (A \setminus C) = A \setminus (B \cup C)$
    \item $(A \setminus B) \cap (C \setminus D) = (A \setminus D) \cap (C \setminus B)$
    \item $B \subseteq A \implies B = A \setminus (A \setminus B)$
  \end{itemize}
\end{ex}

\subsection{Exercise 7a}
\[ (A \setminus B) \cap (A \setminus C) = A \setminus (B \cup C) \]
%
Let $a$ be a variable which is true if the considered element is contained in $A$.
$\neg a$ analogously means not contained. Same for $b$ and $c$. Then:
\begin{align*}
  (a \land \neg b) \land (a \land \neg c) &= a \land \neg (b \lor c) \\
  a \land \neg b \land a \land \neg c &= a \land (\neg b \land \neg c) \\
  a \land \neg b \land \neg c &= a \land \neg b \land \neg c \\
  \top &= \top
\end{align*}

\subsection{Exercise 7b}
\begin{align*}
  (A \setminus B) \cap (C \setminus D) &= (A \setminus D) \cap (C \setminus B) \\
  (a \land \neg b) \land (c \land \neg d) &= (a \land \neg d) \land (c \land \neg b) \\
  a \land \neg b \land c \land \neg d &= a \land \neg b \land c \land \neg d \\
  \top &= \top
\end{align*}

\subsection{Exercise 7c}
\[ B \subseteq A \Rightarrow B = A \setminus (A \setminus B) \]
%
\begin{align*}
  \forall x \in X: (x \in B \rightarrow x \in A) &\implies \left[
          x \in B \leftrightarrow x \in A \land \neg (x \in A \land x \not\in B)))
      \right] \\
  \forall x \in X: (x \in B \rightarrow x \in A) &\implies \left[
          x \in B \leftrightarrow x \in A \land (\underbrace{x \not\in A}_{\bot} \lor x \in B)))
      \right] \\
  \forall x \in X: (x \in B \rightarrow x \in A) &\implies \left[
          x \in B \leftrightarrow x \in A \land x \in B
      \right] \\
  \forall x \in X: (x \in B \rightarrow x \in A) &\implies \left[
          (x \in B \rightarrow x \in A \land \underbrace{x \in B}_{\top}) \land
          \underbrace{(x \in A \land x \in B \rightarrow x \in B)}_{\top}
      \right] \\
  \forall x \in X: (x \in B \rightarrow x \in A) &\implies (x \in B \rightarrow x \in A)
\end{align*}


\section{Exercise 8}

\begin{ex}
  Let $X$ be a set with $X \neq \emptyset$ and $X \neq \set{\emptyset}$. Of which of the following sets is
  (a) the set $X$, (b) the set $\set{X}$, element of subset?
\end{ex}

\begin{table}[!h]
 \begin{center}
  \begin{tabular}{rclcl}
   \hline\hline
    $\downarrow \ S \qquad \text{op} \rightarrow$
        & \multicolumn{2}{l}{$x \in S$} & \multicolumn{2}{l}{$X \subseteq S$} \\
   \hline
    $\set{\set{X}, X}$
        & \yes & 2nd argument            & \no  & impossible to build \\
    $X$
        & \no  & impossible to build     & \yes & $X = X$ \\
    $\emptyset \cap \set{X} = \emptyset$
        & \no  &                         & \no  & unless $X = \emptyset$ \\
    $\set{X} \setminus \set{\set{X}} = \set{X}$
        & \yes & 1st argument            & \no  & impossible to build \\
    $\set{X} \cup X$
        & \yes & 1st argument            & \yes & $X = X$ \\
    $\set{X} \cup \set{\emptyset}$
        & \yes & 1st argument            & \no  & impossible to build \\
   \hline
  \end{tabular}
 \end{center}
\end{table}


\section{Exercise 9}

\centerline{ $(0, \infty)$ is the set $\mathbb{R}_{>0}$ }

\subsection{Exercise 9a}

Prove in three ways the following statement:
\[
    \forall x \in (0, \infty) \forall y \in (0, \infty):
    x \neq y \implies \frac xy + \frac yx > 2
\]

\paragraph{direct proof}
\begin{align*}
  x &\neq y \\
  x - y &\neq 0 \\
  (x - y)^2 &\neq 0 \\
  (x - y)^2 &> 0 \\
  x^2 - 2xy + y^2 &> 0    & x,y \in \mathbb{R}_{>0} \Rightarrow xy > 0 \\
  \frac{x^2}{xy} - \frac{2xy}{xy} + \frac{y^2}{xy} &> 0 \\
  \frac{x}{y} - 2 + \frac yx &> 0 \\
  \frac xy + \frac yx &> 2
\end{align*}

\paragraph{indirect proof}
\begin{align*}
  \forall x \in (0, \infty) \forall y \in (0, \infty): &\frac xy + \frac yx \leq 2 \Rightarrow x = y \\
   \frac{x^2}{xy} + \frac{y^2}{xy} &\leq 2 \\
   x^2 + y^2 &\leq 2xy \\
   x^2 - 2xy + y^2 &\leq 0 \\
   (x - y)^2 &\leq 0 \\
   (x - y)^2 &= 0 \\
   x - y &= 0 \\
   x &= y
\end{align*}

\paragraph{proof by contradiction}
\[ \forall x \in (0, \infty) \forall y \in (0, \infty): x \neq y \Rightarrow \frac xy + \frac yx \leq 2 \]
\begin{align*}
  x - y &\neq 0 \\
  x^2 - 2xy + y^2 &\neq 0 \\
  x^2 - 2xy + y^2 &> 0 \\
  \frac{x^2}{xy} - 2 + \frac{y^2}{xy} &> 0 \\
  \underbrace{\frac{x}{y}}_{>0} + \underbrace{\frac{y}{x}}_{>0} &> 2 \\
    &\text{\lightning}
\end{align*}

\subsection{Exercise 9b}
%
Let $z = \frac{x}{y}$ and illustrate the inequality geometrically.

\[ \frac{x}{y} + \frac{y}{x} > 2 \Rightarrow z + z^{-1} > 2 \]

\begin{figure}[!ht]
  \begin{center}
    \includegraphics[width=200pt]{img/9b.png}
    \caption{Plot for $z + z^{-1} > 2$ (Exercise 9b)}
  \end{center}
\end{figure}

\section{Reminder}

If $n < m$, then the \emph{empty sum} $\sum_{k=m}^n a_k$ has value $0$,
and the \emph{empty product} $\prod_{k=m}^n a_k$ has value $1$.

\section{Exercise 10}
\begin{ex} \hfill{}
  \begin{itemize}
    \item Provide a concise definition of \enquote{n is an even number}
          and \enquote{n is an odd number} using the existence quantifier.
    \item Prove $\forall n \in \mathbb Z: n \text{ is even} \Leftrightarrow n^2 \text{ is even}$ \\
          \textbf{Hint:} Prove $\Leftarrow$ using an indirect proof.
  \end{itemize}
\end{ex}

\subsection{Exercise 10a}
\begin{align*}
  n \text{ is even} &\Rightarrow \exists a \in \mathbb Z: n = 2a \\
  n \text{ is odd} &\Rightarrow \exists a \in \mathbb Z: n = 2a + 1
\end{align*}

\subsection{Exercise 10b}
\[
    \forall n \in \mathbb Z \exists a_1 \in \mathbb Z \exists a_2 \in \mathbb Z:
    n = 2a_1 \Leftrightarrow n^2 = 2a_2
\]

\begin{description}
  \item[Direction $\Rightarrow$] \hfill{} \\
    \[ n = 2a_1 \Rightarrow n^2 = 4a_1^2 \]
    Let $a_1 = \sqrt{\frac{a_2}{2}}$.
    \[ n^2 = 4\left(\sqrt{\frac{a_2}{2}}\right)^2 \Rightarrow n^2 = 2a_2 \]
    Such an $a_2$ always exists. Proof finished.
  \item[Direction $\Leftarrow$] \hfill{} \\
    \[ n^2 \neq 2a_2 \Rightarrow n \neq 2a_1 \]
    Taking the square root preserves the parity of the value\footnote{
    Because an even number times an even number yields an even number.
    An odd number times an odd number yields an odd number.}.
    \[ n = \sqrt{2a_2 + 1} \]
    So $\sqrt{2a_2 + 1}$ gives an odd number.
    But this structure cannot match $2a_1$, which represents an even number.
    This shows a contradiction and $n \neq 2a_1$ holds.
\end{description}

\section{Exercise 11}
\begin{ex}
  For the following statement give
  \begin{enumerate}
    \item an indirect proof
    \item a proof using Exercise 6b
  \end{enumerate}
  \[
      \forall x \in \mathbb R \forall y \in \mathbb R:
      xy \not\in \mathbb Q \Rightarrow x \not\in \mathbb Q \lor y \not\in \mathbb Q
  \]
\end{ex}

\subsection{Exercise 11.1}
\[ \forall x \in \mathbb R \forall y \in \mathbb R: x \in \mathbb Q \land y \in \mathbb Q \Rightarrow xy \in \mathbb Q \]
\[ \forall x \in \mathbb R \forall y \in \mathbb R \exists p_0, p_1 \in \mathbb R \exists q_0, q_1 \in \mathbb R \setminus \set{0}: \]
\[
    \left(x = \frac{p_0}{q_0}\right) \land \left(y = \frac{p_1}{q_1}\right)
    \Rightarrow \left(\exists p_2 \in \mathbb R \exists q_2 \in \mathbb R \setminus \set{0}: xy = \frac{p_2}{q_2} \right)
\] \[
    xy = \frac{\overbrace{p_0 p_1}^{\in \mathbb R}}{\underbrace{q_0 q_1}_{\in \mathbb R \setminus \set{0}}}
    \Rightarrow xy = \frac{p_2}{q_2} \text{ for } p_2 = p_0 \cdot p_1 \text{ and } q_2 = q_0 \cdot q_1
\]

\subsection{Exercise 11.2}
\[ (p \Rightarrow (q \lor r)) \Leftrightarrow ((p \land \neg q) \Rightarrow r) \]
\[
    (xy \not\in \mathbb Q \Rightarrow \left(x \not\in \mathbb Q \lor y \not\in \mathbb Q\right)) \Leftrightarrow
    ((xy \not\in \mathbb Q \land x \in \mathbb Q) \Rightarrow y \not\in \mathbb Q)
\]

\[
    \forall x,y \in \mathbb R: \left(
      \substack{
      \left(\not\exists p_2 \in \mathbb R \not\exists q_2 \in \mathbb R \setminus \set{0}: xy = \frac{p_2}{q_2}\right) \land \\
      \left(\exists p_0 \in \mathbb R \exists q_0 \in \mathbb R \setminus \set{0}: x = \frac{p_0}{q_0}\right)}
    \right) \Rightarrow \left(\not\exists p_1 \in \mathbb R \not\exists q_1 \in \mathbb R \setminus \set{0}: y = \frac{p_1}{q_1}\right)
\]

Recognize $p_2 = p_0 \cdot p_1$ and $q_2 = q_0 \cdot q_1$.

Therefore the conjunction yields the $y \not\in \mathbb Q$ because $x \in \mathbb Q$.

\[
    \forall x \in \mathbb R \forall y \in \mathbb R:
    \left(\not\exists p_1 \in \mathbb R \exists q_1 \in \mathbb R \setminus \set{0}: y = \frac{p_1}{q_1}\right) \Rightarrow
    \left(\not\exists p_1 \in \mathbb R \exists q_1 \in \mathbb R \setminus \set{0}: y = \frac{p_1}{q_1}\right)
\]

This statement is true. The proof is complete.

\section{Exercise 18}
\begin{ex}
  Let $n \in \mathbb{N}_+$. Show that
  \[ \prod_{k=2}^n \left(1 - \frac1k\right) = \frac1n . \]
\end{ex}

\begin{description}
  \item[Induction base] $n = 1$
    \[ \prod_{k=2}^1 \dots = 1 = \frac11 \qquad\checkmark \]
  \item[Induction step] $n \rightarrow n + 1$
    \begin{align*}
      \prod_{k=2}^{n+1} \left(1 - \frac{1}{k}\right) &= \frac{1}{n+1} \\
      \prod_{k=2}^{n} \left(1 - \frac{1}{k}\right) \left(1 - \frac{1}{n+1}\right) &= \frac{1}{n+1} \\
      \frac 1n \left(1 - \frac{1}{n+1}\right) &= \frac{1}{n+1} \\
      \frac 1n \cdot \frac{n+1-1}{n+1} &= \frac{1}{n+1} \\
      \frac nn &= 1 \qquad\checkmark
    \end{align*}
\end{description}

Actually, can be rewritten as
\[ \prod_{k=2}^n \left(\frac{k-1}{k}\right) \]
\[ = \frac12 \cdot \frac23 \cdot \frac34 \cdot \frac45 \dots \frac{n-1}{n} \]
\[ = \frac1n \]
So this is the multiplication equivalent of telescoping sums.

\section{Exercise 19}

\begin{ex}
  $X$ and $Y$ are non-empty sets and $f: X \rightarrow Y$ is a mapping.
  Furthermore let $A \subseteq X$ and $B \subseteq Y$.
  \begin{enumerate}
    \item Prove that $A \subseteq f^{-1}(f(A))$ and $B \supseteq f(f^{-1}(B))$
    \item Show (by providing counterexamples) that in the inclusions of (1)
      no equivalence is given.
  \end{enumerate}
\end{ex}

\subsection{Exercise 19.1}

%A function is invertible iff it is injective:
%\[ \forall y \in Y \not\exists x_1, x_2 \in X: f(x_1) = f(x_2) = y \]
%\[ \forall y \in Y \forall x_1, x_2 \in X: x_1 \neq x_2 \Rightarrow f(x_1) \neq f(x_2) \]

Show that,
\[ a \in A \Rightarrow a \in f^{-1}(f(A)) \]

So we take $a$ and map it to the codomain:
\[ f(a) \in f(A) \]
We denote the result as $y$:
\[ y \coloneqq f(a) \]
Because
\[ f^{-1}(x) = \setdef{x \in A}{f(x) \in B} \]
we know that $a$ originates from:
\[ a \in f^{-1}(f(A)) \]

It is very important here to distinguish between \emph{domain}/\emph{codomain}
and \emph{function}/\emph{inverse function}. Because an inverse function implies
that the corresponding function is injective. Assuming this fact, the exercise
is immediate. But we are talking about domains and co-domains here.

As second exercise we need to show that,
\[ B \supseteq f\left(f^{-1}\left(B\right)\right) \]

We need the definition that,
\[ f^{-1}(B) = \setdef{x' \in X}{f(x') \in B} \]

\[ y' \in f(f^{-1}(B)) \]
Does $y' \in B$ hold? Yes, because \dots
\[ y' \in f\left(f^{-1}(B)\right) \Rightarrow \exists x' \in f^{-1}(B) \]
\[ \Rightarrow y' \in B \]

\subsection{Exercise 19.2}

Show that,
\[ \exists f: A \subsetneq f^{-1}(f(A)) \]

We use a surjective, but not injective function.

\[ f: \set{1,2} \rightarrow \set{a} \]
\[ 1 \mapsto a \]
\[ 2 \mapsto a \]

\begin{align*}
  A &= \set{1} \\
  f(A) &= \set{a} \\
  f^{-1}(f(A)) &= \set{1, 2}
\end{align*}

Show that,
\[ \exists f: A \subsetneq f(f^{-1}(A)) \]

We use an injective, but not surjective function.

\[ f: \set{1} \rightarrow \set{a, b} \]
\[ 1 \mapsto a \]

\begin{align*}
  B &= \set{b} \\
  f^{-1}(B) &= \emptyset \\
  f(f^{-1}(B)) &= \emptyset
\end{align*}

\section{Exercise 20}

\begin{ex}
  Prove the following variant of Bernoulli's inequality: For $x \in \mathbb{R}$
  with $0 < x < 1$ and $n \in \mathbb{N}_+$ it holds that
  \[ (1 - x)^n < \frac{1}{1 + nx}. \]
\end{ex}
\begin{align*}
  (1 + x)^n &\geq 1 + nx \\
  \frac{(1 + x)^n}{1 + nx} &\geq \frac{1 + nx}{1 + nx} \\
  \frac{(1 + x)^n}{1 + nx} &\geq 1 \\
  \frac{(1 - x)^n (1 + x)^n}{(1 - x)^n (1 + nx)} &\geq 1 \\
  \frac{(1 - x)^n (1 + x)^n}{(1 + nx)} &\geq (1 - x)^n \\
  \frac{\left((1 - x)(1 + x)\right)^n}{(1 + nx)} &\geq (1 - x)^n \\
  \frac{\left(1 - x^2\right)^n}{(1 + nx)} &\geq (1 - x)^n \\
  \frac{\overbrace{\left(1 - x^2\right)^n}^{\text{interval }(0,1)}}{(1 + nx)} &\geq (1 - x)^n \\
  \frac{1}{(1 + nx)} &> (1 - x)^n \\
\end{align*}

% (1-x)/(1+nx) < 1/(1+(n+1)x)
% (1-x)(1+(n+1)x) < 1 + nx
% 1 + nx + x - x - x^2(n+1) < 1 + nx
% -x^2 (n+1) < 0
% true

\section{Exercise 21}

\begin{ex}
  $X$ and $Y$ are nonempty sets and $f: X \rightarrow Y$ is a mapping.
  \begin{itemize}
    \item[a)]
      Show that the following holds: For all $A, B \subseteq X$
      \[ f(A \cap B) \subseteq f(A) \cap f(B). \]
    \item[b)]
      Show that the following statements are equivalent:
      \begin{enumerate}
        \item $f$ is injective.
        \item For all $A, B \subseteq X$ it holds that $f(A \cap B) \supseteq f(A) \cap f(B)$
        \item For all $A, B \subseteq X$ it holds that $f(A \cap B) = f(A) \cap f(B)$
      \end{enumerate}
  \end{itemize}
\end{ex}

\subsection{Exercise 21a}

Let $C = A \cap B$. Case distinction:
\begin{description}
  \item[$A = B = C$]
    \begin{align*}
         f(A \cap B) &= \setdef{f(x)}{x \in A} \\
      f(A) \cap f(B) &= f(A) \\
                     &= \setdef{f(x)}{x \in A}
    \end{align*}
  \item[$C = A \dot\lor C = B$]
    wlog. $C = A$.
    \begin{align*}
         f(A \cap B) &= f(A) \\
                     &= \setdef{f(x)}{x \in A} \\
      f(A) \cap f(B) &= \setdef{f(x)}{x \in A} \cap \setdef{f(x)}{x \in B} \\
                     &= \setdef{f(x)}{x \in A \land x \not\in (B \setminus A)} \\
                     &= \setdef{f(x)}{x \in A}
    \end{align*}
  \item[$C = \emptyset$]
    \begin{align*}
      f(A \cap B)    &= f(\emptyset) \\
                     &= \emptyset \\
      f(A) \cap f(B) &= \setdef{f(x)}{x \in A} \cap \setdef{f(x)}{x \in B}
    \end{align*}
\end{description}

So,
\[ C \neq \emptyset \Rightarrow f(A \cap B) = f(A) \cap f(B) \]

But if $C = 0$, we get zero values on the left-hand side and zero to $\card{A} + \card{B}$ values on the right-hand side.
So,
\[ C = \emptyset \Rightarrow f(A \cap B) \subseteq f(A) \cap f(B) \]

\subsection{Exercise 21b}

We prove $3$ with $1$:

Let $C = A \cap B$. $f$ is injective, meaning
\[ \forall x_1, x_2 \in X: x_1 \neq x_2 \Rightarrow f(x_1) \cap f(x_2) \]

Case distinction:
\begin{description}
  \item[$A = B = C$]
    \begin{align*}
         f(A \cap B) &= \setdef{f(x)}{x \in A} \\
      f(A) \cap f(B) &= f(A) \\
                     &= \setdef{f(x)}{x \in A}
    \end{align*}
  \item[$C = A \dot\lor C = B$] \hfill{} \\
    wlog. $C = A$ meaning $A \subsetneq B$
    \begin{align*}
         f(A \cap B) &= f(A) \\
                     &= \setdef{f(x)}{x \in A} \\
      f(A) \cap f(B) &= \setdef{f(x)}{x \in A} \cap \setdef{f(x)}{x \in B} \\
                     &= \setdef{f(x)}{x \in A \land x \not\in (B \setminus A)} \\
                     &= \setdef{f(x)}{x \in A}
    \end{align*}
  \item[$C = \emptyset$]
    \begin{align*}
      f(A \cap B)    &= f(\emptyset) \\
                     &= \emptyset \\
      f(A) \cap f(B) &= \emptyset
    \end{align*}
    Every element in $A$ is distinct from values in $B$.
    Therefore $\forall x_1 \in A, x_2 \in B: f(x_1) \neq f(x_2)$ because of injectivity.
    The intersection of all $f(x_i)$ is therefore empty.
\end{description}

\section{Exercise 22}

\begin{ex}
  Let $n \in \mathbb{N}$. Use the following idea to derive an equation for the sum of powers of three.
  \[ \sum_{k=1}^n \left(k^4 - (k - 1)^4\right) \]
  This sum can be written in two different ways:
  \begin{itemize}
    \item As telescoping sum (the initial and trailing value will be left)
    \item (First resolve the parentheses.)
      As combination of sums of the third, second, first and zero-th power.
      With that (and known equations for sums of smaller powers)
      we can compute $\sum_{k=1}^n k^3$.
  \end{itemize}
\end{ex}

We look at the telescoping sum:
\begin{align*}
  \sum_{k=1}^n (k^4 - (k-1)^4)
    &= \left(1^4 - (1 - 1)^4\right) + \left(2^4 - (2-1)^4\right) + \left(3^4 - (3-1)^4\right) \\
    &+ \dots + \left((n-1)^4 - ((n-1)-1)^4\right) + \left(n^4 - (n-1)^4\right) \\
    &= -0^4 + n^4 \\
    &= n^4
\end{align*}

Then we use the combination of sums of lower powers.
\begin{align*}
  \sum_{k=1}^n (k^4 - (k-1)^4)
    &= \sum_{k=1}^n (k^4 - (k^4 - 4k^3 + 6k^2 - 4k + 1)) \\
    &= \sum_{k=1}^n (k^4 - k^4 + 4k^3 - 6k^2 + 4k - 1) \\
    &= \sum_{k=1}^n (4k^3 - 6k^2 + 4k - 1) \\
    &= \sum_{k=1}^n 4k^3 - \sum_{k=1}^n 6k^2 + \sum_{k=1}^n 4k - \sum_{k=1}^n 1 \\
    &= \sum_{k=1}^n 4k^3 - 6\frac{2n^3 + 3n^2 + n}{6} + 4\frac{n(n+1)}{2} - n \\
    &= \sum_{k=1}^n 4k^3 - 2n^3 - n^2
\end{align*}

Therefore,
\[ n^4 = \sum_{k=1}^n 4k^3 - 2n^3 - n^2 \]
\[ \sum_{k=1}^n 4k^3 = n^4 + 2n^3 + n^2 \]
\[ \sum_{k=1}^n k^3 = \frac{n^4 + 2n^3 + n^2}{4} \]

\section{Exercise 23}

\begin{ex}
  Let $n \in \mathbb{N}$. Prove that
  \[ \sum_{k=0}^n \binom nk = 2^n \]
  and if $n \geq 1$,
  \[ \sum_{k=0}^n (-1)^k \binom nk = 0 \]
\end{ex}

Binomial theorem with $x=1, y=1$:

\begin{align*}
  \sum_{k=0}^n \binom nk 1^n 1^{n-k} &= (1 + 1)^n \\
  \sum_{k=0}^n \binom nk &= 2^n
\end{align*}

If $n \geq 1$,
\begin{align*}
  \sum_{k=0}^n (-1)^k \binom{n}{k}
    &= \sum_{k=0}^n (-1)^k \left(\binom{n-1}{k} + \binom{n-1}{k-1}\right) \\
    &= \sum_{k=0}^n (-1)^k \binom{n-1}{k} + \sum_{k=0}^n (-1)^k \binom{n-1}{k-1} \\
    &= (-1)^n \binom{n-1}{n} + \sum_{k=0}^{n-1} (-1)^k \binom{n-1}{k} + \sum_{k=0}^n (-1)^k \binom{n-1}{k-1} \\
    &= \underbrace{(-1)^n \binom{n-1}{n}}_0 + \sum_{k=0}^{n-1} (-1)^k \binom{n-1}{k} + \sum_{k=1}^{n} (-1)^k \binom{n-1}{k-1} + \underbrace{(-1)^0 \binom{n-1}{-1}}_0 \\
    &= \sum_{k=0}^{n-1} (-1)^k \binom{n-1}{k} + \sum_{k=0}^{n-1} (-1)^{k+1} \binom{n-1}{k} \\
    &= \sum_{k=0}^{n-1} (-1)^k \binom{n-1}{k} - (-1) \sum_{k=0}^{n-1} (-1)^{k+1} \binom{n-1}{k} \\
    &= \sum_{k=0}^{n-1} (-1)^k \binom{n-1}{k} - \sum_{k=0}^{n-1} (-1)^{k + 2} \binom{n-1}{k} \\
    &= \sum_{k=0}^{n-1} (-1)^k \binom{n-1}{k} - \sum_{k=0}^{n-1} (-1)^k \binom{n-1}{k} \\
    &= 0
\end{align*}

\clearpage
\section{Exercise 24}
\begin{ex}
  Let $k, n \in \mathbb N_+$ with $k \leq n$. Determine the number of vectors
  of length $k$ with pairwise different entries from $M_n = \set{1, 2, \ldots, n}$.
\end{ex}

This question is covered by the field of combinatorics.

\[ (a_0, a_1, a_2, \dots) \neq (a_0, a_2, a_1, \dots) \]
The order of elements is relevant. Therefore a variation, not combination, is given.
The number of combinations without repetitions would be given by the binomial
coefficient $n \choose k$ (the number of ways to choose $k$ of $n$ elements
disregarding their order). For variations the formula $n^k$ holds to select $k$
elements among $n$ arbitrarily often (hence with repetition).

We model the given situation as
\begin{itemize}
  \item \enquote{variation without repetition}
  \item i.e. \enquote{$k$-permutations of $n$}
  \item i.e. the $k$-th falling factorial power $n^{\underline{k}}$ of $n$
\end{itemize}

The formula is given by,
\[ P_k^n = \frac{n!}{(n-k)!} \]

We can estimate it in the following way:
Consider a combination without repetition represented by the formula $n \choose k$:
\[ {n \choose k} = \frac{n!}{k! (n - k)!} \]

So because we have a variation, not combination, the order of elements is relevant.
Therefore given some combination, there are $k!$ possible arrangements. Given the vector
(and also combination) $(1, 2, 3)$ there are $k!$ possible arrangements (variations)
$\set{(1, 2, 3), (1, 3, 2), (2, 1, 3), (2, 3, 1), (3, 1, 2), (3, 2, 1)}$.
Indeed it holds that
\[ \frac{3!}{(3 - 3)!} = \frac61 = 6 \]

This argument explains why $k!$ in the denominator is omitted for variations w/o repetitions.

\begin{table}[!h]
  \begin{center}
    \begin{tabular}{c|cccccc}
      combinations & \multicolumn{6}{c}{variations} \\
    \hline
      $(123)$ & $(123)$ & $(132)$ & $(213)$ & $(231)$ & $(312)$ & $(321)$ \\
      $(124)$ & $(124)$ & $(142)$ & $(214)$ & $(241)$ & $(412)$ & $(421)$ \\
      $(134)$ & $(134)$ & $(143)$ & $(314)$ & $(341)$ & $(413)$ & $(431)$ \\
      $(234)$ & $(234)$ & $(243)$ & $(324)$ & $(342)$ & $(423)$ & $(432)$
    \end{tabular}
    \caption{Combinations and variations for $n=4$ of $k=3$}
  \end{center}
\end{table}

\section{Exercise 25}
\begin{ex}
  Let $K$ be a field and $a,b,c \in K$. Show (using the field axioms):
  \begin{enumerate}[(a)]
    \item $-(-a) = a$.
    \item $(-a)(-b) = ab$.
    \item $a + b = a + c \Rightarrow b = c$.
    \item From $a \neq 0$ and $ab = ac$ follows $b = c$.
    \item Is $a \neq 0$, then there is exactly one $x \in K$ with $ax + b = c$.
  \end{enumerate}
\end{ex}

The field axioms are defined as follows:
\begin{itemize}
  \item[\textbf{A1}] $\forall a,b \in K: a + b = b + a$
  \item[\textbf{A2}] $\forall a,b,c \in K: (a + b) + c = a + (b + c)$
  \item[\textbf{A3}] $\exists 0 \in K \forall a \in K: a + 0 = a$
  \item[\textbf{A4}] $\forall a \in K \exists \tilde{a}: a + \tilde{a} = 0$
  \item[\textbf{M1}] $\forall a,b \in K: a \cdot b = b \cdot a$
  \item[\textbf{M2}] $\forall a,b,c \in K: a \cdot (b \cdot c) = (a \cdot b) \cdot c$
  \item[\textbf{M3}] $\exists 1 \in K: a \cdot 1 = a \forall a \in K$ (neutral element)
  \item[\textbf{M4}] $\forall a \in K \setminus \set{0} \exists \hat a: \hat a \cdot a = 1$
  \item[\textbf{D}] $\forall a,b,c \in K: a \cdot (b + c) = a \cdot b + a \cdot c$
\end{itemize}

\subsection{Exercise 25.a}

\begin{align*}
                  A4 &\Rightarrow \forall a \in K \exists -a: a + (-a) = 0 \\
  \text{equivalence} &\Rightarrow a + (-a) - (-a) = 0 - (-a) \\
                  A1 &\Rightarrow a + (-a) - (-a) = -(-a) + 0 \\
                  A3 &\Rightarrow a + (-a) - (-a) = -(-a) \\
                  A4 &\Rightarrow a + 0 = -(-a) \\
                  A3 &\Rightarrow a = -(-a)
\end{align*}

\subsection{Exercise 25.b}

We have proven in the lecture: \textbf{M5}: $-a = (-1) \cdot a$

% TODO: show that $-(a a) = (-1) (a a) = (-a) a$

First, we show \textbf{M7}
\begin{align*}
     &= a \cdot (-a) \\
  M5 &\Rightarrow a \cdot (-1) \cdot a \\
  M1 &\Rightarrow (-1) \cdot a \cdot a \\
     &\Rightarrow - (a \cdot a) \\
\end{align*}

Secondly, we show (actually we have already shown that in the lecture) \textbf{M6}
\begin{align*}
                        D &\Rightarrow \forall a,b,c \in K: a \cdot (b + c) = a \cdot b + a \cdot c \\
                          & [\text{we choose } \quad a \coloneqq a, \quad b \coloneqq a, \quad c \coloneqq (-a)] \\
                          &\Rightarrow a \cdot (a + (-a)) = a \cdot a + a \cdot (-a) \\
                       A3 &\Rightarrow a \cdot 0 = a \cdot a + a \cdot (-a) \\
  \text{previous theorem} &\Rightarrow a \cdot 0 = a \cdot a + (- (a \cdot a)) \\
                       A4 &\Rightarrow a \cdot 0 = 0
\end{align*}

Finally, we show
\begin{align*}
  \text{previous theorem} &\Rightarrow (-a) \cdot 0 = 0 \\
                       A4 &\Rightarrow (-a) \cdot (b + (-b)) = 0 \\
                        D &\Rightarrow (-a) \cdot b + (-a) (-b) = 0 \\
       \text{equivalence} &\Rightarrow ab + (-a) b + (-a) (-b) = ab + 0 \\
                       M1 &\Rightarrow ab + (-a) b + (-a) (-b) = 0 + ab \\
                       A3 &\Rightarrow ab + (-a) b + (-a) (-b) = ab \\
                       M6 &\Rightarrow ab - a b + (-a) (-b) = ab \\
                       A4 &\Rightarrow 0 + (-a) (-b) = ab \\
                       A3 &\Rightarrow (-a) (-b) = ab
\end{align*}

\subsection{Exercise 25.c}

\begin{align*}
                     & a + b = a + c \\
  \text{equivalence} &\Rightarrow a + b + (-a) = a + c + (-a) \\
                  A1 &\Rightarrow (a + (-a)) + b = (a + (-a)) + c \\
                  A4 &\Rightarrow 0 + b = 0 + c \\
                  A3 &\Rightarrow b = c
\end{align*}

\subsection{Exercise 25.d}

\begin{align*}
                     & a \neq 0 \land ab = ac \\
  \text{equivalence} &\Rightarrow aba^{-1} = aca^{-1} \\
                  M1 &\Rightarrow aa^{-1}b = aa^{-1}c \\
                  M4 &\Rightarrow 1b = 1c \\
                  M3 &\Rightarrow b = c
\end{align*}

\subsection{Exercise 25.e}

Proof by contradiction.
Assume $x_1, x_2 \in K$ with $x_1 \neq x_2$ then $\exists r \in K$:
\[ ax_1 = r \qquad ax_2 = r \]
\[ ax_1 = ax_2 \]

\begin{align*}
                     &\Rightarrow ax_1 = ax_2 \\
  \text{equivalence} &\Rightarrow a^{-1} ax_1 = a^{-1} ax_2 \\
                  M4 &\Rightarrow 1x_1 = 1x_2 \\
                  M3 &\Rightarrow x_1 = x_2
\end{align*}

This is a contradiction to our assumption $x_1 \neq x_2$.
Therefore $x$ is distinct.

\section{Exercise 26}
\begin{ex}
  Let $n \in \mathbb N_+$. Prove that
  \[ \sum_{k=0}^n \binom{2n}{2k} = \sum_{k=1}^n \binom{2n}{2k-1} = 2^{2n-1}. \]
\end{ex}

\subsection{Exercise 26.1: $\sum_{k=0}^n \binom{2n}{2k} = \sum_{k=1}^n \binom{2n}{2k-1}$ - approach 1}

\begin{proof}
  \begin{align*}
    \sum_{k=0}^n \binom{2n}{2k}
      &= \sum_{k=1}^{n-1} \binom{2n}{2k} + 1 + 1 \\
      &= \sum_{k=1}^{n-1} \left[\binom{2n-1}{2k} + \binom{2n-1}{2k-1}\right] + 1 + 1 \\
      &= \sum_{k=1}^{n-1} \binom{2n-1}{2k} + \sum_{k=1}^{n-1} \binom{2n-1}{2k-1} + 1 + 1 \\
      &= \sum_{k=2}^{n} \binom{2n-1}{2(k - 1)} + \sum_{k=1}^{n-1} \binom{2n-1}{2k-1} + 1 + 1 \\
      &= \sum_{k=1}^{n} \binom{2n-1}{2k-2} + \sum_{k=1}^{n-1} \binom{2n-1}{2k-1} + 1 \\
      &= \sum_{k=1}^{n-1} \binom{2n-1}{2k-2} + \sum_{k=1}^{n-1} \binom{2n-1}{2k-1} + \binom{2n-1}{2n-2} + 1 \\
      &= \sum_{k=1}^{n-1} \left[\binom{2n-1}{2k-2} + \binom{2n-1}{2k-1}\right] + \binom{2n-1}{2n-2} + 1 \\
      &= \sum_{k=1}^{n-1} \binom{2n}{2k-1} + \left[1 + \binom{2n-1}{2n-2}\right] \\
      &= \sum_{k=1}^{n-1} \binom{2n}{2k-1} + \left[\binom{2n-1}{2n-1} + \binom{2n-1}{2n-2}\right] \\
      &= \sum_{k=1}^{n-1} \binom{2n}{2k-1} + \binom{2n}{2n-1} \\
      &= \sum_{k=1}^{n} \binom{2n}{2k-1}
  \end{align*}
\end{proof}

\subsection{Exercise 26.1: $\sum_{k=0}^n \binom{2n}{2k} = \sum_{k=1}^n \binom{2n}{2k-1}$ - approach 2}

\begin{proof}
  Idea: Consider $(1-1)^{2n}$ and split even/odd $k$s.
  \begin{align*}
    (1 - 1)^{2n} &= \sum_{k=0}^{2n} \binom{2n}{k} (-1)^k 1^{2n-k} & \text{[binomial theorem]} \\
               0 &= \sum_{k=0}^n \binom{2n}{2k} (-1)^{2k} + \sum_{k=1}^n \binom{2n}{2k-1} (-1)^{2k-1} \\
                 &= \sum_{k=0}^n \binom{2n}{2k} + \sum_{k=1}^n \binom{2n}{2k-1} (-1) & \text{[$(-1)^\text{even}$ is $1$, $(-1)^\text{odd}$ is $-1$]} \\
                 &= \sum_{k=0}^n \binom{2n}{2k} - \sum_{k=1}^n \binom{2n}{2k-1} \\
    \sum_{k=1}^n \binom{2n}{2k-1} &= \sum_{k=0}^n \binom{2n}{2k} \\
  \end{align*}
\end{proof}

\subsection{Exercise 26.2: $\sum_{k=0}^n \binom{2n}{2k}$}

\begin{proof}
  Idea: Consider $(1+1)^{2n}$ and odd+even provides the factor 2 we need to divide with.
  \begin{align*}
    (1 + 1)^{2n} &= \sum_{k=0}^{2n} \binom{2n}{k} 1^k 1^{2n-k} & \text{[binomial theorem]} \\
          2^{2n} &= \sum_{k=0}^{2n} \binom{2n}{k} \\
                 &= \sum_{k=0}^{n} \binom{2n}{2k} + \sum_{k=1}^{n} \binom{2n}{2k-1} & \text{[split even and odd]} \\
                 &= \sum_{k=0}^{n} \binom{2n}{2k} + \sum_{k=0}^{n} \binom{2n}{2k} & \text{[from previous result]} \\
                 &= 2\sum_{k=0}^{n} \binom{2n}{2k} \\
    \frac{2^{2n}}{2} &= \sum_{k=0}^{n} \binom{2n}{2k} \\
        2^{2n-1} &= \sum_{k=0}^{n} \binom{2n}{2k}
  \end{align*}
\end{proof}

\section{Exercise 27}
\begin{ex}
  Let $x \in \mathbb R \setminus \set{0}$. Show:
  Let $x + \frac1x \in \mathbb{Z}$, then $x^n + \frac1{x^n} \in \mathbb{Z}$
  for all $n \in \mathbb{N}$ (Remark: Consider $(x + \frac1x)^n$.)
\end{ex}

So we need to show that,
\[
  x \in \mathbb{R} \setminus \set{0}: \forall n \in \mathbb N:
  x + \frac1x \in \mathbb Z \implies x^n + \frac1{x^n} \in \mathbb Z
\]

First we need to cover some fundamentals,
\begin{itemize}
  \item $a, b \in \mathbb Z \implies (a + b) \in \mathbb Z$
  \item $a, b \in \mathbb Z \implies (a \cdot b) \in \mathbb Z \implies \forall n \in \mathbb N: a^n \in \mathbb Z$
\end{itemize}

\begin{proof}
  \begin{description}
    \item[IB: $n = 0$]
      \begin{align*}
        \forall x \in \mathbb{R} \setminus \set{0}: x + \frac1x \in \mathbb{Z} & \\
        &\Rightarrow x = 1: 1 + \frac11 \in \mathbb Z \\
        &\Rightarrow \forall x \in \mathbb{R} \setminus \set{0}: x^0 + \frac1{x^0} \in \mathbb Z \\
        &\Rightarrow \forall x \in \mathbb{R} \setminus \set{0}: n = 0: x^n + \frac1{x^n} \in \mathbb Z \\
      \end{align*}
    \item[IB: $n = 1$]
      \begin{align*}
        \forall x \in \mathbb{R} \setminus \set{0}: x + \frac1x \in \mathbb{Z} & \\
        &\Rightarrow \forall x \in \mathbb{R} \setminus \set{0}: x^1 + \frac1{x^1} \in \mathbb Z \\
        &\Rightarrow \forall x \in \mathbb{R} \setminus \set{0}: n = 1: x^n + \frac1{x^n} \in \mathbb Z \\
      \end{align*}
    \item[IS: $n \rightarrow n + 1$]
      Okay, how does the induction step for an implication look like?
      \begin{align*}
        \left((a \rightarrow b) \rightarrow (a \rightarrow d)\right)
          &= \neg (\neg a \lor b) \lor (\neg a \lor d) \\
          &= (a \land \neg b) \lor \neg a \lor d \\
          &= ((a \lor \neg a) \land (\neg a \lor \neg b)) \lor d \\
          &= (\neg a \lor \neg b) \lor d \\
          &= (a \land b) \rightarrow d
      \end{align*}
      Therefore we can assume
      \[ \left(x + \frac1x \in \mathbb Z\right) \land \left(x^n + \frac1{x^n} \in \mathbb Z\right) \]
      and need to prove that this follows:
      \[ x^{n+1} + \frac1{x^{n+1}} \in \mathbb Z \]

      \begin{align*}
        \left(x^n + \frac1{x^n} \in \mathbb Z\right)
          &= \left(x^n + \frac1{x^n}\right)\left(x + \frac1x\right) \in \mathbb Z \\
          &= \left(x^n \cdot x + \frac1{x^n}\cdot x + x^n \cdot \frac1x + \frac1{x^n}\cdot \frac1x\right) \in \mathbb Z \\
          &= \left(x^{n+1} + x^{-n+1} + x^{n-1} + x^{-n-1}\right) \in \mathbb Z \\
          &= \left(x^{n+1} + x^{-n-1} + x^{n-1} + x^{-n+1}\right) \in \mathbb Z \\
          &= \left(x^{n+1} + \frac1{x^{n+1}}\right) + \underbrace{\left(x^{n-1} + \frac1{x^{n-1}}\right)}%
            _{\substack{\in \mathbb Z \text{ because of induction hypothesis} \\ \text{and we have a 2-step induction}}} \in \mathbb Z \\
          &= \left(x^{n+1} + \frac1{x^{n+1}}\right) \in \mathbb Z
      \end{align*}
  \end{description}
\end{proof}

%\begin{align*}
%  x + \frac1x &\in \mathbb{Z} \\
%  \left(x + \frac1x\right)^n &\in \mathbb{Z} \\
%  \sum_{k=0}^n \binom nk x^{n-k} \left(\frac1x\right)^k &\in \mathbb{Z} \\
%  \binom n0 x^n \left(\frac1x\right)^0 + \binom nn x^{n-n} \left(\frac1x\right)^n + \sum_{k=1}^{n-1} \binom nk x^{n-k} \left(\frac1x\right)^k &\in \mathbb{Z} \\
%  x^n + \frac1{x^n} + \sum_{k=1}^{n-1} \underbrace{\binom nk}_{\in \mathbb N} \underbrace{\underbrace{x^{n-k}}_{x^{n-k}} \underbrace{\left(\frac1x\right)^k}_{x^{-k}}}_{x^{n-2k}} &\in \mathbb{Z}
%\end{align*}

\section{Exercise 28}
\begin{ex}
  Let $K$ be an ordered field and $a, b \in K_+$. Show:
  \[ a < b \Rightarrow a^2 < b^2 \]
  Especially the mapping $f: K_+ \cup \set{0} \rightarrow K_+ \cup \set{0},
  a \mapsto a^2$ is injective.
\end{ex}

We already know,
\begin{description}
  \item[\textbf{U1}] $\forall a, b \in K: a < b \Leftrightarrow b > a$
  \item[\textbf{U2}] $\forall a \in K: a^2 = a\cdot a$
  \item[\textbf{U3}] $\forall c \in K_+: a > b \Rightarrow ac > bc$
  \item[\textbf{M1}] $\forall a, b \in K: a \cdot b = b \cdot a$
\end{description}

\begin{proof}
  \begin{align*}
    a < b: \hspace{10pt}
      & U1 \Rightarrow b > a \\
      & U1 \Rightarrow b\cdot a > a\cdot a & \text{[yes, $a$ originates from $K_+$]} \\
      & U2 \Rightarrow b\cdot a > a^2 \\
    b > a: \hspace{10pt}
      & U1 \Rightarrow b\cdot b > a\cdot b & \text{[yes, $b$ originates from $K_+$]} \\
      & U2 \Rightarrow b^2 > a\cdot b \\
      & M1 \Rightarrow b^2 > b\cdot a \\
    b^2 > b\cdot a \land b\cdot a > a^2: \hspace{10pt}
      & U3 \Rightarrow b^2 > a^2 \\
      & U1 \Rightarrow a^2 < b^2
  \end{align*}
  \[ \Rightarrow \forall a, b \in K_+: a < b \Rightarrow a^2 < b^2 \]
\end{proof}

Injectivity:
\[ \forall a_1, a_2 \in K_+ \cup \set{0}: a_1 \neq a_2 \Rightarrow a_1^2 \neq a_2^2 \]

\begin{proof}
  First we consider $a = 0$. In this case, $a = 0$ and $a^2 = a \cdot a = 0 \cdot 0 = 0$
  according to the axiom $0 \cdot a = 0$ we have proven in the lecture.
  So for $a = 0$, there is only one $a$ for which the square is zero, which is $0$.

  We can proceed in $K_+$. Proof by contradiction:
  \[ \exists a_1, a_2 \in K_+: a_1 \neq a_2 \Rightarrow a_1^2 = a_2^2 \]

  \[
    a_1 \neq a_2 \Leftrightarrow a_1 < a_2 \dot\lor a_1 > a_2
  \]
  because $a_1$ and $a_2$ are elements of an ordered field.

  \begin{description}
    \item[Case 1: $a_1 < a_2$]
      \[ a_1 < a_2 \Rightarrow a_1^2 < a_2^2 \]
    \item[Case 2: $a_1 > a_2$]
      \[ a_1 > a_2 \Rightarrow a_1^2 > a_2^2 \]
  \end{description}

  Therefore either $a_1^2 < a_2^2$ or $a_1^2 > a_2^2$. So
  \[ a_1^2 \neq a_2^2 \]
  This contradicts and therefore $\not\exists a_1, a_2 \in K_+: a_1 \neq a_2 \Rightarrow a_1^2 = a_2^2$
  or because we covered $a = 0$,
  \[ \not\exists a_1, a_2 \in K_+ \cup \set{0}: a_1 \neq a_2 \Rightarrow a_1^2 = a_2^2 \]
\end{proof}

\section{Exercise 29}
\begin{ex}
  Let $K$ be an ordered field and $a,b \in K$. Show:
  \[ \abs{a + b} = \abs{a} + \abs{b} \Leftrightarrow ab \geq 0 \]
\end{ex}

Triangular inequality:
\[ \forall a, b \in K: \abs{a + b} \leq \abs{a} + \abs{b} \]

Absolute values are defined with,
\[
  \abs{a} = \begin{cases}
     a & a \in K_+ \\
     0 & a = 0 \\
    -a & a \in K_-
  \end{cases}
\]

\begin{proof}
  Case distinction:
  \begin{description}
    \item[$a = 0, b = 0$]
      \begin{align*}
        \abs{a + b} &\leq \abs{a} + \abs{b} \\
        \abs{a + 0} &\leq \abs{a} + \abs{0} \\
        A3 \Rightarrow \abs{a} &\leq \abs{a} + 0 \\
        A3 \Rightarrow \abs{a} &= \abs{a} \\
      \end{align*}
    \item[$a > 0, b = 0$]
      \begin{align*}
        \abs{a + b} &\leq \abs{a} + \abs{b} \\
        \abs{a + 0} &\leq \abs{a} + \abs{0} \\
        A3 \Rightarrow \abs{a} &\leq \abs{a} + 0 \\
        A3 \Rightarrow \abs{a} &= \abs{a} \\
      \end{align*}
    \item[$a = 0, b > 0$]
      \begin{align*}
        \abs{a + b} &\leq \abs{a} + \abs{b} \\
        \abs{0 + b} &\leq \abs{0} + \abs{b} \\
        A3 \Rightarrow \abs{b} &\leq 0 + \abs{b} \\
        A3 \Rightarrow \abs{b} &= \abs{b} \\
      \end{align*}
    \item[$a > 0, b > 0$]
      \begin{align*}
        \underbrace{\abs{a + b}}_{\in K_+} &\leq \underbrace{\abs{a}}_{\in K_+} + \underbrace{\abs{b}}_{\in K_+} \\
        (a + b) &\leq (a) + (b) \\
        A2 \Rightarrow a + b &\leq a + b \\
        a + b &= a + b \\
      \end{align*}
  \end{description}
\end{proof}

\section{Exercise 33}

\[ [a_n, b_n], [c_n, d_n], a_n \leq \alpha \leq b_n, c_n \leq \gamma \leq d_n \]
\[ \forall \varepsilon > 0 \exists N(\varepsilon): \abs{a_n - b_n} < \varepsilon \forall n \geq N(\varepsilon) \]
\[
    \left[\frac{1}{b_n}, \frac{1}{a_n}\right] \rightarrow \frac{1}{b_n} \leq \frac{1}{b_{n+1}} \leq \frac{1}{\alpha} \leq \frac{1}{a_{n+1}} \leq \frac{1}{a_n}
\] \[
  \card{\frac{1}{b_n} - \frac{1}{a_n}}
  = \frac{a_n - b_n}{a_n b_n}
  = \frac{\card{a_n - b_n}}{\card{a_n}\card{b_n}} \leq \frac{\varepsilon}{\card{a_1} \alpha}
  = \varepsilon'
\]

Important: our approximation $a_n \geq a_1 > 0$ and $b_n \geq \alpha$ is independent of $n$!
\[
  \forall \varepsilon' > 0 \exists N(\varepsilon'): \card{\frac1{b_n} - \frac1{a_n}} < \varepsilon'
\] \[
  \card{a_n c_n - b_n d_n} = \card{a_n c_n - \alpha c_n + \alpha c_n - \alpha c_n + \alpha \gamma - b_n \gamma + b_n \gamma - b_n d_n}
\] \[
  \leq \underbrace{\card{a_n - \alpha}}_{<\varepsilon} \underbrace{\card{c_n}}_{\leq \gamma} + \card{\alpha} \underbrace{\card{c_n - \gamma}}_{<\varepsilon} + \underbrace{\card{\gamma}}_{1} \underbrace{\card{\alpha - b_n}}_{<\varepsilon} + \underbrace{\card{b_n}}_{b_1} \underbrace{\card{\gamma - d_n}}_{<\varepsilon} < \varepsilon\underbrace{(2\gamma + \alpha + b_1)}_{=c} = \varepsilon'
\]

\section{Exercise 34}

\begin{ex}
  Let $f: X \rightarrow Y$ be a mapping. Prove that:
  \begin{enumerate}
    \item If a mapping $g: Y \rightarrow X$ with $g \circ f = \text{id}_X$ exists, $f$ is injective.
    \item If a mapping $h: Y \rightarrow X$ with $f \circ h = \text{id}_Y$ exists, $f$ is surjective.
  \end{enumerate}
  \textbf{Remark.} $\text{id}_X$ is the identity function over the set $X$.
    The identity function is always defined as $f: X \rightarrow X$ with $x \mapsto x$.
\end{ex}

\subsection{Exercise 34.1}
%
So given that $g: Y \rightarrow X$ exists with $g \circ f = \text{id}_X$,
let $x \in X$.
\[ x \in X \Rightarrow f(x) \in Y \Rightarrow g(f(x)) = x \Leftrightarrow \text{id}_X(x) = x \]

To show injectivity, we need to show for all $x_1, x_2 \in X$:
\[ f(x_1) = f(x_2) \Rightarrow x_1 = x_2 \]

Consider two arbitrary values $x_1, x_2 \in X$.
\[ f(x_1) = f(x_2) \]
\[ \Rightarrow g(f(x_1)) = g(f(x_2)) \]
\[ \Rightarrow x_1 = x_2 \]

As far as $x_1$ and $x_2$ are two arbitrary elements of $X$, this holds for any pair of elements of $X$.
We have directly proven injectivity of $f$.

\subsection{Exercise 34.2}
%
Given that $h: Y \rightarrow X$ exists with $f \circ h = \text{id}_Y$,
let $y \in Y$.
\[ y \in Y \Rightarrow h(y) \in X \Rightarrow f(h(y)) \in Y \Leftrightarrow \text{id}_Y(y) = y \]

To show surjectivity, we need to show for all $y_1, y_2 \in Y$:
\[ \forall y \in Y \exists x \in X: f(x) = y \]

Consider an arbitrary value $y \in Y$. Because of the existence of the identity function, it holds that:
\[ f(h(y)) = y \]
We define $h(y)$ as an intermediate value with a different name:
\[ x \coloneqq h(y) \]
\[ \Rightarrow \exists x \in X: f(x) = y \]

We have show that for any arbitrary value $y \in Y$. So it holds for any value of $Y$:
\[ \Rightarrow \forall y \in Y \exists x \in X: f(x) = y \]

We have directly proven surjectivity of $f$.


\section{Exercise 35}

This exercise was delayed until 26th of November 2015 (later then the other exercises here).

\[ (a_n)_{n \in \mathbb N} \text{ is sequence with } \lim_{n \to \infty} a_n = a \in \mathbb R \]
\[ (b_n)_{n \in \mathbb N} \text{ is sequence with } \lim_{n \to \infty} b_n = b \in \mathbb R \]
Furthermore $b \neq 0$.

\subsection{Part 1}

\[ \lim_{n\to\infty} a_n = a \land \lim_{n\to\infty} b_n = b \neq 0 \]
Let $\varepsilon > 0$ be arbitrary.

Claim: $\exists k \in \mathbb N \forall n \geq k: \card{b_n} > \frac{\card{b}}{2}$.

Proof: Let $\varepsilon > 0$. Consider $\varepsilon = \frac{\card{b}}{2}$.

For $\varepsilon = \frac{\card{b}}{2} > 0$:
\[ \exists k \in \mathbb N: \forall n \geq k: \card{b_n - b} < \frac{\card{b}}{2} = \varepsilon \]
\begin{align*}
  \forall n \geq k: \card{b_n}
      &= \card{b_n - b + b} \geq \card{\card{b} - \underbrace{\card{b - b_n}}_{<\frac{\card{b}}{2}}} \\
      &> \card{b} - \card{b - b_n} \\
      &> \card{b} - \frac{\card{b}}{2} \\
      &= \frac{\card{b}}{2}
\end{align*}

Claim:
\[ \text{sequence } \left(\frac{1}{b_n}\right)_{n \in \mathbb N} \land \exists \lim\left(\frac{1}{b_n}\right) = \frac1b \]

Proof:
For $\frac{\varepsilon \card{b}^2}{2}:$
\[ \exists N \in \mathbb N: \forall n \geq N: \card{b_n - b} < \frac{\varepsilon \card{b}^2}{2} \]

It holds that $\forall n \geq N:$
\[ \card{\frac1{b_n} - \frac1b} = \card{\frac{b - b_n}{b_n \cdot b}} = \frac{\card{b - b_n}}{\card{b_n} \cdot \card{b}} < \frac{\varepsilon \cdot \frac{\card{b}^2}{2}}{\frac{\card{b}}{b} \card{b}} = \varepsilon \]
\[ \lim_{n\to\infty} \frac{a_n}{b_n} = \lim_{n\to\infty} a_n \cdot \lim_{n\to\infty} \frac{1}{b_n} = a \cdot \frac1b \cdot \frac{a}{b} \]

Or a direct proof:
\begin{align*}
  \card{\frac{a_n}{b_n} - \frac{a}{b}}
    &= \card{\frac{a_n b - ab_n}{b_n b}} \\
    &= \frac{\card{a_n b - a b + a b - a b_n}}{\card{b_n} \card{b}} \\
    &\leq \frac{\card{b} \card{a_n - a} + \card{a} \card{b_n - b}}{\frac{\card{b}}{2} \cdot \card{b}} \\
    &\leq C \cdot \varepsilon
\end{align*}

\subsection{Part 2}


\section{Exercise 36}

\[ A = \setdef{\frac{1}{2^m} + \frac{1}{n}}{m,n \in \mathbb N_+} \]

Assumption: $\min{a} = 0$

\[ 0 \not\in A \]
\[ \frac{1}{2^N} + \frac{1}{N} < 2\varepsilon \]
\[ \forall \varepsilon > 0 \exists N \in \mathbb N_+: (m \geq N \Rightarrow \card{\frac{1}{2^m} - 0} < \varepsilon) \]
\[ \frac{1}{2^N} < \varepsilon \]
\[ n \geq N \Rightarrow \card{\frac1n - 0} < \varepsilon \]

Assume $\exists s > 0$ is our lower bound.
\[ \exists m: \frac{1}{2^m} < \frac{s}{2} = \varepsilon \]
\[ \varepsilon = \frac{s}{2}: \exists N: \frac{1}{N} < \frac{S}{2} \]
\[ \rightarrow \underbrace{\frac{1}{sm} + \frac{1}{N}}_{\in A} < s \]
\[ \Rightarrow \inf{A} = 0 \]

Remark: When starting this exercise, always estimate whether a maximum/minimum exists.
If so, you can save time to prove supremum/infimum.

\[ \frac{1}{2^{m+1}} < \frac{1}{2^m} \forall m \]
\[ \frac{1}{N+1} < \frac{1}{N} \forall N \]

Therefore $\max{A}$ is when $m,n$ is as small as possible:
\[ \frac12 + \frac11 = \frac32 \]
\[ \max(A) = \frac32 = \sup(A) \]

\[ B = \setdef{\frac{x}{1+x}}{x \in \mathbb R, x \geq 0} \]

$\min(B) = 0$ because $0 \leq \frac{x}{1 + x} \forall x \geq 0 \land \left.\frac{x}{1+x} \right|_{x = 0} = 0$.

\[ \frac{x}{1+x} < 1 \Leftrightarrow x < 1 + x \Leftrightarrow 0 < 1 \forall x \geq 0 \]

Is $1$ an upper bound and $1 \not\in B$?

Assume $\exists s < 1$:
\[ \frac{x}{1 + x} \leq s \]
\[ x \leq s(1 + x) \]
\[ x(1-s) \leq 0 \]
\[ 1 - s > 0 \]
\[ \Rightarrow \sup(B) = 1 \land \not\exists \max(B) \]

\section{Exercise 37}

\[ I = [a, b) = \setdef{x \in \mathbb R}{a \leq x < b} \]
\[ a = \min{[a,b)} \Rightarrow a \text{ is } \inf([a, b)) \]
\[ a \in [a,b): \forall x \in [a,b): a \leq x \Rightarrow \min(I) = a \Rightarrow \inf(I) = a \]

$b$ is upper bound:
\[ b \not\in [a,b) \text{ by definition } \forall x \in [a,b): b > x \]

Claim: $b$ is the smallest upper bound. \\
Assume: $\exists b' < b: b'$ is upper bound.

\[ b' \in [a,b) \quad\underbrace{\lightning}_{\text{because } \mathbb R \text{ is complete}} \]

\section{Exercise 38}

\begin{ex}
  Let $A$ and $B$ two non-empty, bounded by below subseteq of $\mathbb R$. Prove that
  \[ \inf(A \cup B) = \min\set{\inf(A), \inf(B)} \]
\end{ex}

Without loss of generality, $\inf{A} \leq \inf{B}$:

Let $a \in A$ and $b \in B$ arbitrary.
This implies that $a \geq \inf{A}$ and $b \geq \inf{B} \geq \inf{A}$.
\[ \Rightarrow \forall x \in (A \cup B): x \geq \inf{A} \]

\[ \Rightarrow \inf(A) \geq \inf(A \cup B) \]

Because extending a set $A$ with additional elements,
the infimum cannot be increased, but only decreased.

\[ \Rightarrow \inf(A) \leq \inf(A \cup B) \]

\begin{align*}
  x \in A: & \inf\set{\inf{A}, \inf{B}} \leq \inf(A) \leq x \\
  x \in B: & \inf\set{\inf{A}, \inf{B}} \leq \inf(B) \leq x
\end{align*}
\[ \forall x \in A \cup B: \underbrace{\min\set{\inf{A}, \inf{B}}}_{\text{lower bound}} \leq x \]

\[ \Rightarrow \inf(A \cup B) \leq \min\set{\inf(A), \inf(B)} \]

\[ \Rightarrow \inf(A) = \inf(A \cup B) \]

\section{Exercise 39}
\subsection{Exercise 39a}
\[ \sup_{y \in Y} \inf_{x \in X} f(x, y) \leq \inf_{x \in X} \sup_{y \in Y} f(x, y) \]
\[ \underbrace{\inf_{x \in X} f(x, y) \leq f(x, y)}_{\sup_{y \in Y}} \leq \sup_{y \in Y} f(x, y) \]
\[ \sup_{y \in Y} \inf_{x \in X} f(x, y) \leq \sup_{y \in Y} f(x, y) \]
\[ \sup_{y \in Y}{\inf_{x \in X} f(x, y)} = \inf_{x \in X} \sup_{y \in Y} \inf_{x \in X} f(x, y) \leq \inf_{x \in X} \sup_{y \in Y} f(x, y) \quad \checkmark \]

\subsection{Exercise 39b}
\[ f: (x, y) \mapsto 1_{\set{ x \geq 0, y \geq 0} \cup \set{x < 0, y < 0}} \]
\[ \sup_{y \in Y} f(x, y) = 1 \forall x \]
\[ \inf_{x \in X} \sup_{y \in Y} f(x, y) = 1 \]
\[ \inf_{x \in X} f(x, y) = 0 \forall y \in [-1, 1] \]
\[ \sup_{y \in Y} \inf_{x \in X} f(x, y) = 0 < 1 \]

\section{Exercise 40}

\subsection{Exercise 40.a.1}
\[
    \frac{5+i}{2+3i} \cdot \frac{2 - 3i}{2 - 3i}
    = \frac{10 - 15i + 2i - 3i^2}{-6i + 4 + 6i - 9i^2}
    = \frac{10 - 13i + 3}{4 + 9}
    = \frac{13 - 13i}{13}
    = 1 - i
\]

\subsection{Exercise 40.a.2}
\begin{align*}
  z^2 &= \frac{1 + \sqrt{3}i}{2} \\
  z^2 &= \pm \sqrt{\frac12 + \frac{\sqrt{3}i}{2}} \\
  z^2 &= \pm \sqrt{\frac{9 + 6\sqrt{3} i - 3}{12}} \\
  z^2 &= \pm \sqrt{\frac{(3 + \sqrt{3}i)^2}{12}} \\
  z^2 &= \pm \frac{3 + \sqrt{3}i}{\sqrt{12}} \\
  z^2 &= \pm \left(\frac{\sqrt{3}}{2} + i\frac12\right)
\end{align*}

\subsection{Exercise 40.b.1}
\[ M_1 = \setdef{z \in \mathbb C \setminus \set{0}}{\card{\frac1z} < 2} \]
\[
    \abs{\frac1z}
    = \abs{\frac1{a + bi}}
    = \frac{\abs{1}}{\abs{a + bi}}
    = \frac{1}{\sqrt{a^2 + b^2}}
\] \[
    \Rightarrow \frac{1}{\sqrt{a^2 + b^2}} < 2
\] \[
    \Rightarrow \frac12 < \sqrt{a^2 + b^2}
\] \[
    \Rightarrow \frac14 < a^2 + b^2
\]

Illustrated we draw a circle originating in $(0, 0)$ with radius $\frac12$.
The solution set is the whole plane excluding everything what is part of the circle.

\subsection{Exercise 40.b.2}
\[ M_2 = \setdef{z \in \mathbb C}{\Im((1 + i)z) = 0} \]
\[ \Im(z + zi) \]

TODO


\section{Exercise 41}

\[ A_n \coloneqq (-\infty, a_n)_{n \in \mathbb N} \qquad A \coloneqq \bigcup_{n \in \mathbb N} A_n \]
\[ B_n \coloneqq (b_n, \infty)_{n \in \mathbb N} \qquad B \coloneqq \bigcup_{n \in \mathbb N} B_n \]

\[ \forall n \in \mathbb N: x \in I_n \]

Show that $x = \sup{A} = \inf{B}$.

Because $I_n$ are nested intervals it holds that
\[ a_1 \leq \dots \leq a_n \leq a_{n+1} \leq x \]

Because
\[ \forall \varepsilon > 0 \exists N: N \geq n: 0 \leq x - a_x \leq b_n - a_n \leq \varepsilon \]
it holds that
\[ x = \sup(a_n) \]

Let $y \in A$.
\[ \exists n \in \mathbb N: y \in A_n \Rightarrow y < a_n \leq x \]
\[ \Rightarrow y \in A: y < x \]
Therefore $x$ is an upper bound. Is it the only upper bound?

Assume another upper bound $x'$ exists.
\[ x' < x = \lim){n\to\infty} a_n \]
\[ \Rightarrow \exists N \in \mathbb N: x' < a_n \qquad \forall n \geq M \]
\[ \varepsilon = \frac{x - x'}{2} \]
\[ \Rightarrow \exists y \in A_{n+1} \]
\[ y > x' \]
This is a contradiction and therefore $x$ is the distinct upper bound.

The proof for the infimum works analogously.

It only remains to show that $x \not\in A$.
\[ \forall a_n \neq x \Rightarrow \exists a_{n+k}: a_n < a_{n+k} \]

\section{Exercise 42}
Give the limes for the following sequences:

\subsection{Exercise 42.a}
\[ a_n = \frac{5n+2}{3n+7} \]

\begin{align*}
  \lim_{n\to\infty} a_n &= \frac{5n+2}{3n+7} \\
      &= \frac{\lim_{n\to\infty} 5n+2}{\lim_{n\to\infty} 3n+7} \\
      &= \frac{\lim_{n\to\infty} 5n + \lim_{n\to\infty} 2}{\lim_{n\to\infty} 3n + \lim_{n\to\infty} 7} \\
      &= \frac{n(5 + \frac2n)}{n(3+\frac7n)} \\
      &= \frac{5 + \overbrace{\frac2n}^{\to0}}{3+\underbrace{\frac7n}_{\to0}} \\
      &= \frac53
\end{align*}

This works only if the denominator is non-zero. $\lim_{n\to\infty} (3 + \frac7n)$ turns out to be non-zero.

\subsection{Exercise 42.b}
\[ b_n = \frac{2n^2-4n+5}{n^3+2\sqrt{n}} \]

First, we make a remark, that $\lim_{n\to\infty} \frac1{n^2} = 0$. Why, because
\[ \lim_{n\to\infty} \frac1{n^2} = \left(\lim_{n\to\infty} \frac1n\right) \cdot \left(\lim_{n\to\infty} \frac1n\right) = 0 \]
This can be generalized for $\lim_{n\to\infty} \frac1{n^k} = 0$ with $k\in\mathbb N_+$.

\begin{align*}
  \lim_{n\to\infty} b_n &= \frac{2n^2 - 4n + 5}{n^3 + 2\sqrt{n}} \\
    &= \frac{n^3\cdot\left(\frac2n - \frac4{n^2} + \frac5{n^2}\right)}{n^3\cdot\left(1 + 2\frac{n^{0.5}}{n^3}\right)} \\
    &= \frac{\frac2n - \frac4{n^2} + \frac5{n^3}}{1 + 2\cdot \frac1{n^{2.5}}} \\
    &= \frac{\frac2n - \frac4{n^2} + \frac5{n^3}}{\frac{n^{2.5}}{n^{2.5}}} \\
    &= \frac{2n^{1.5} - 4n^{0.5} + 5n^{0.5}}{n^{2.5} + 2} \cdot \frac{\frac1{n^{2.5}}}{\frac1{n^{2.5}}} \\
    &= \frac{2n^{-1} - 4n^{-2} + 5n^{-3}}{1 + 2n^{-2.5}} \\
    &= \frac01
\end{align*}

Or generally:
\[ 2n^2 - 4n + 5 \leq 2n^2 + 4n^2 + 5n^2 \leq 11n^2 \]

\[ 0 \leq b_n \leq \frac{11n^2}{n^3} = \underbrace{\frac{11}{n}}_{\to0} \]

\subsection{Exercise 42.c}
\[ c_n = \sqrt{4n^2+2n+3} \]

\begin{align*}
  c_n &= \sqrt{4n^2+2n+3} \cdot \frac{\sqrt{4n^2 + 2n + 3}}{\sqrt{4n^2 + 2n + 3} + 2n} \\
      &= \dots \\
      &= \frac{2 + \frac3n}{\sqrt{4 + \frac2n + \frac3{n^2}} + 2} \\
      &= \frac24 \\
      &= \frac12
\end{align*}

\subsection{Exercise 42.d}
\[ d_n = \binom{n}{k} n^{-k} \text{ with } n \in \mathbb N \text{ for a fixed } k \in \mathbb N_+ \]

\begin{align*}
  d_n &= \binom{n}{k} n^{-k} \\
      &= \frac{n!}{k! (n-k)! n^k} \\
      &= \frac{n \cdot (n-1) \cdot \dots \cdot 1}{k! (n-k)! n^k} \\
      &= \frac{n \cdot (n-1) \cdot \ldots \cdot 1}{k! (n-k)! n^k} \\
      &= \frac{(1 - \frac1n)(1 - \frac2n) \cdot \ldots \cdot (1 - \frac{k-1}{n}) \cdot (n-k) \cdot \ldots \cdot 1}{k! (n-k)!} \\
      &= \frac{(n-k)!}{k! (n-k)!} \\
      &= \frac{1}{k!}
\end{align*}

Or better we write:
\begin{align*}
  \frac{n!}{k! (n-k)!}
    &= \frac{\prod_{i=0}^{n-1} (n-i)}{\prod_{j=k}^{n-1} (n-j)} \\
    &= \frac{1}{k!} \prod_{j=0}^{k-1} (n-j) n^{-k} \\
    &= \frac{1}{k!} \prod_{j=0}^{k-1} \left[(n-j) \cdot \frac1n\right] \\
    &= \frac1{k!} \prod_{j=0}^{k-1} \left(1 - \frac{j}{n}\right) \\
  \lim_{n\to\infty} \frac1{k!} \prod_{j=0}^{k-1} \left(1 - \frac{j}{n}\right)
    &= \frac1{k!} \lim_{n\to\infty} \prod_{j=0}^{k-1} \left(1 - \frac{j}n\right) & \text{ [if limes exist]} \\
    &= \frac1{k!} \prod_{j=0}^{k-1} \underbrace{\lim_{n\to\infty} \left(1 - \frac{j}{n}\right)}_{=1} \\
    &= \frac1{k!} \forall j = 0, \dots, k-1
\end{align*}

\section{Exercise 43}
\begin{ex}
  Let $(a_n)_{n \in \mathbb N}$ be a sequence in $\mathbb R_+$ with $\lim_{n\to\infty} \frac{a_{n+1}}{a_n} = q$.
  Prove that
  \[
    (a_n)_{n \in \mathbb N}
    \begin{cases}
      \text{converges} & \text{ if } q < 1 \\
      \text{diverges} & \text{ if } q > 1
    \end{cases}
  \]
  In case $q=1$ no statement about the convergence of $(a_n)_{n \in \mathbb N}$ can be made.
\end{ex}

\[ \lim_{n \to \infty} \frac{a_{n+1}}{a_n} = q \]

\subsection{Examples for $q = 1$}

\[ a_n = \frac{1}{n+1} \qquad \frac{a_{n+1}}{a_n} \frac{n+1}{n+2} \to_{n\to\infty} 1 \qquad a_m \searrow 0 \]
\[ a_n = n+1 \qquad \frac{a_{n+1}}{a_n} = \frac{n+2}{n+1} \to_{n\to\infty} 1 \qquad a_n \nearrow 0 \]

\subsection{Proof for $q < 1$}

\[ \exists \underbrace{\varepsilon}_{=\frac{q+1}{2} - a} > 0: q + \varepsilon < 1 \]

If $n$ is sufficiently large:
\[ \card{\frac{a_{n+1}}{a_n} - q} < \varepsilon \Rightarrow \frac{a_{n+1}}{a_n} \in (q - \varepsilon, q + \varepsilon) \]
\begin{align*}
  0 \leq a_{n+1} &\leq (q + \varepsilon) a_n \\
  0 \leq a_{n+2} &\leq (q + \varepsilon)^2 a_n \\
           \dots & \\
  0 \leq a_{n+k} &\leq (q + \varepsilon)^k a_n \\
\end{align*}

By induction it holds that
\[ 0 \leq a_{n+k} \leq {\underbrace{(q + \varepsilon)}_{\tilde{q} < 1}}^k a_1 \to_{k\rightarrow\infty} 0 \]

This follows from the squeeze theorem.

\[ \forall q > 1 \exists \varepsilon > 0: q - \varepsilon > 1 \]

\[ a_{n+1} > (q - \varepsilon) a_n \]
\[ a_{n+k} > {\underbrace{(q - \varepsilon)}_{\tilde q > 1}}^k a_n \]

\[ \lim_{n\to\infty} \tilde q^k = +\infty \]
\[ \tilde q > 1 \]

\section{Exercise 44}
\begin{ex}
  Let $(a_n)_{n \in \mathbb N}$ be a zero sequence in $\mathbb R$ and $(b_n)_{n \in \mathbb N}$ a bounded sequence in $\mathbb R$.
  Prove that $(a_n b_n)_{n \in \mathbb N}$ is a zero sequence.
\end{ex}

Because $(b_n)_{n \in \mathbb N}$ is bounded some $d$ exists such that
\[ \forall \varepsilon > 0: \exists N \in \mathbb N: n \geq N: \card{a_n - 0} < \varepsilon \]

Consider $\lim_{n\to\infty} (a_n \cdot b_n) = 0$.

We need to show that
\[ \forall \varepsilon > 0 \exists N \in \mathbb N: n \geq N: \card{a_n \cdot b_n - 0} < \varepsilon \cdot d \]

Where $\varepsilon \cdot d$ is epsilon multiplied with constant $d$.
This is a hand-crafted value (meaning that we selected it intentionally and will turn out to solve our problem). Now we elaborate on the relation:

\begin{align*}
  \card{a_n \cdot b_n} &< \varepsilon \cdot d \\
  \card{a_n} \cdot \card{b_n} &< \varepsilon \cdot d \\
  \card{a_n} \cdot d &< \varepsilon \cdot d \\
  \card{a_n} &< \varepsilon
\end{align*}

Because $a_n < \varepsilon$ it holds that some constant exists for a sufficiently large $N$ such that $\card{a_n \cdot b_n}$ is always smaller than some constant $\varepsilon$.

\section{Exercise 45}
\begin{ex}
  Let $a,b,c \in [0,\infty)$. Show that,
  \[ \lim_{n\to\infty} \sqrt[n]{a^n + b^n + c^n} = \max\set{a,b,c} \]
\end{ex}

Without loss of generality, let $a = \max\set{a,b,c}$. Because $a,b,c$ is non-negative,
\[ a^n \leq a^n + b^n + c^n \leq 3a^n \]
\[ \sqrt[n]{a^n} \leq \sqrt[n]{a^n + b^n + c^n} \leq \sqrt[n]{3} \cdot \sqrt[n]{a^n} \]
\[ \lim_{n\to\infty} \sqrt[n]{a^n} = a \]
\[ \lim_{n\to\infty} \sqrt[n]{3} \cdot \sqrt[n]{a^n} = a \lim_{n\to\infty} \sqrt[n]{3} = a \cdot 1 \]

Due to the squeeze theorem, it holds that $\lim_{n\to\infty} \sqrt[n]{a^n + b^n + c^n} = a = \max\set{a,b,c}$.

\section{Exercise 46}
\begin{ex}
  Let $a_0 \in (0, 1)$ and a sequence $(a_n)_{n\in\mathbb N}$ is recursively defined with
  \[ a_{n+1} = 1 - \sqrt{1 - a_n} \text{ for } n \geq 0 \]
\end{ex}

\begin{description}
  \item[Induction base]
    \[ a_1 = 1 - \sqrt{1 - a_0} \Rightarrow 0 < a_1 < 1 \qquad a_1 \in (0, 1) \]
  \item[Induction step]
    Let $a_n \in (0, 1)$.
    \[ 0 < a_n < 1 \Rightarrow -1 < -a_n < 0 \Rightarrow 0 < \sqrt{1 - a} < 1 \]
    \[ 0 < \underbrace{1 - \sqrt{1 - a_n}}_{a_{n+1}} < 1 \]
    So $a_{n+1} < a_n$.
    \[ 1 - \sqrt{1 - a_n} < a_n \Leftrightarrow (1 - a_n)^2 < 1 - a_n \]
    \[ \Rightarrow 1 - a_n < \sqrt{1 - a_n} \]
    \[ \Rightarrow x^2 < x \Leftrightarrow x \in (0, 1) \]

    \[ a_{n+1} - 1 - \sqrt{1 - a_n} \]
    \[ a = 1 - \sqrt{1 - a} \]

    Therefore only $0$ or $1$ are possible limes for this sequence.
    But monotonically decreasing implies that $0$ is the limes
    (bounded below and monotonically decreasing sequences are convergent).
\end{description}

\section{Exercise 47}
\begin{ex}
  For a sequence $\left(a_n\right)_{n\in\mathbb N}$ in $\mathbb R$ we assign the sequence $\left(s_n\right)_{n\in\mathbb N}$, where
  \[ s_n = \frac{1}{n+1} \sum_{k=0}^n a_k \text{ for } n \geq 0 \]
  is the mean value of the first $n+1$ sequence numbers.
  \begin{itemize}
    \item Show that: If $\lim_{n\to\infty} a_n = a$ with $a \in \mathbb R$, then $\lim_{n\to\infty} s_n = a$.
    \item Give an example for a divergent sequence $\left(a_n\right)_{n \in \mathbb N}$ for which the sequence of mean values converges anyways.
  \end{itemize}
\end{ex}

\subsection{Exercise 47.a}
%
We show that $\exists a \in \mathbb R: \lim_{n\to\infty} a_n = \lim_{n\to\infty} s_n = a$.
%
\[
  \lim_{n\to\infty} s_n
  = \lim_{n\to\infty} \left(\frac{1}{n+1} \sum_{k=0}^n a_k\right)
  = \lim_{n\to\infty} \left(\frac1{n+1} \sum_{k=0}^n \left(a_n - (a_n - a_k)\right)\right)
\] \[
  = \lim_{n\to\infty} \left(\frac{1}{n+1} \sum_{k=0}^n a_n\right)
  - \lim_{n\to\infty} \left(\frac{1}{n+1} \sum_{k=0}^n (a_n - a_k)\right)
\] \[
  = \lim_{n\to\infty} \frac{1}{n+1} (n+1) a_n
  - \lim_{n\to\infty} \sum_{k=0}^n \frac{a_n - a_k}{n+1}
\] \[
  \left[\forall \varepsilon > 0 \exists N: n \geq N: \abs{a_n - a} < \varepsilon\right]
\] \[
  = a - \lim_{n\to\infty} \underbrace{\sum_{k=0}^{N} \frac{a_n - a_k}{n+1}}_{(N + 1) \frac{C}{n+1}} + \sum_{N+1}^n \frac{a_n - a_k}{n+1}
\] \[
  \stackrel{?}{\leq} \underbrace{\lim_{n\to\infty} \frac{(N + 1) C}{n+1}}_{\to 0}
  + \lim_{n\to\infty} \sum_{n=N+1}^n \frac{a_n - a_b}{n+1}
\] \[
  \left[\lim_{n\to\infty} \sum_{n=N+1}^n \frac{\varepsilon}{n+1} = \frac{n - N - 1}{n+1} \varepsilon \to 1\right]
\] \[
  = \lim_{n\to\infty} \frac{(N + 1) C}{n+1} +
\]

\subsection{Exercise 47.a, radical variant}
\[
  \sum_{n=N-1}^n a-\varepsilon \leq \dots \leq \frac{\sum_{k=0}^N a_k + \sum_{k=N+1}^n (a + \varepsilon)}{n+1}
\]

\subsection{Exercise 47.b}
\[ (a_n)_{n \in \mathbb N} = (-1)^n \]
\[ \Rightarrow \lim_{n\to\infty} S_n = \lim_{n\to\infty} \frac{\sum_{k=0}^n (-1)^n}{n+1} = 0 \]

\section{Exercise 48}
\begin{ex}
  Let $M \subseteq \mathbb R$ be a bounded above set and $s \in \mathbb R$.
  Prove that:
  \[
      s = \sup(M) \Leftrightarrow
      \left\{\begin{array}{ll}
        \forall x \in M: s \geq x & \text{and} \\
        \exists (x_n)_{n \in \mathbb N}, x_n \in M: \lim_{n\to\infty} x_n = s &
      \end{array}\right.
  \]
\end{ex}
\[
  \exists (x_n)_{n \in \mathbb N}, x_n \in M: \lim_{n\to\infty} x_n = s
  \Leftrightarrow \varepsilon > 0 \exists N \in \mathbb N: \abs{x_n - s} < \varepsilon
\]
We prove the first direction $\Leftarrow$.

Let $s$ be an upper bound of $M$.
Let $(x_n)_{n \in \mathbb N}$ in $M$ with $\lim_{n\to\infty} x_n = s$.
\[ \Rightarrow \forall t < s \exists N \in \mathbb N: n \geq N(s - x_n) \leq \abs{s-x_n} < \frac{\varepsilon}{2} \]
\[ \Rightarrow t < x_n \]

We prove the second direction $\Rightarrow$.

Therefore
\[ s - \frac1n \text{ is not an upper bound of } M \]
\[ \Rightarrow \exists x_n \in M: s - \frac1n < x_n \]
\[ \exists x_n \in M: s - \frac1n < x_n \]
\[ s - \frac1n \leq x_n < s \]
\[ s \leq x_n < s \Rightarrow (x_n)_{n\in\mathbb N} \to s \]


\section{Exercise 49}
\begin{ex}
  Let $(a_n)_{n\in\mathbb N}$ be a convergent sequence of non-negative real numbers
  with $\lim_{n\to\infty} a_n = a$ and $k \in \mathbb N_+$. Show that
  \[ \lim_{n\to\infty} \sqrt[k]{a_n} = \sqrt[k]{a} \]
  Hint: $a_n - a = \sqrt[k]{a_n^k} - \sqrt[k]{a^k} = (\sqrt[k]{a_n} - \sqrt[k]{a}) (\dots)$.
\end{ex}

\[ \lim_{n\to\infty} \sqrt[k]{a_n} = \sqrt[k]{a} \]

\[
    \lim_{n\to\infty} a_n = a \qquad \text{i.e.} \quad \forall \varepsilon \exists N \in \mathbb N:
    \left(\sqrt[k]{a_n} - \sqrt[k]{a}\right) \left(\sum_{i=0}^{k-1} \sqrt[k]{a_n}^{k-1-j} \cdot \sqrt[k]{a}\right)
\]

\begin{description}
  \item[Case 1: $a > 0$]
    \[ \abs{a_n - a} < \frac{a}{2} \]
    \[ \abs{a_n} > \frac{\abs{a}}{2} \]
    $\Rightarrow$ the product is always positive:
    \[ \underbrace{\left(\sqrt[k]{a_n} - \sqrt[k]{a}\right) \left(\sum_{i=0}^{k-1} \sqrt[k]{a_n}^{k-1-j} \cdot \sqrt[k]{a}\right)}_{b_n \geq b > 0} \]
    Done.
  \item[Case 2]
    \[ \sqrt[k]{a_n} < \varepsilon \Leftrightarrow a_n < \varepsilon^k = \tilde\varepsilon \]
    \[ \forall \tilde\varepsilon > 0 \exists N \in \mathbb N: n \geq N: \abs{a_n - 0} < \tilde\varepsilon \]
    \[ \Rightarrow \sqrt[r]{a_n} \]
\end{description}

\subsection{Shorter valid solution}
\[
  b_n
  = \frac{a_n - a}{\left(\sqrt[k]{a_n} - \sqrt[k]{a}\right) \left(\sum_{i=0}^{k-1} \sqrt[k]{a_n}^{k-1-j} \cdot \sqrt[k]{a}\right)}
  = \sqrt[k]{a_n} - \sqrt[k]{a}
\]
\[
  \sqrt[k]{a_n} - \sqrt[k]{a} = \frac{a_n - a}{b_n}
\]
We already know that
\[ \lim_{n\to\infty} \frac{a_n - a}{b_n} = \frac{\lim_{n\to\infty} (a_n - a)}{\lim_{n\to\infty} b_n} = \frac0b = 0 \]

\end{document}
%%% Local Variables:
%%% mode: latex
%%% TeX-master: t
%%% End:
