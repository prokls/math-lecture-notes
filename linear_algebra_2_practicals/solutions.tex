\documentclass[a4paper]{article}
\usepackage[utf8]{inputenc}
\usepackage[LGR,T1]{fontenc}
\usepackage{amsmath}
\usepackage{amssymb}
\usepackage{amsfonts}
\usepackage{amsthm}
\usepackage{baskervald}
\usepackage{bbold}
\usepackage{csquotes}
\usepackage{enumerate}
\usepackage{faktor}
\usepackage{fancyhdr}
\usepackage[margin=1in]{geometry}
\usepackage[pdfborder={0 0 0},colorlinks=true,citecolor=red]{hyperref}
\usepackage{imakeidx} % before hyperref
\usepackage{mathalfa}
\usepackage{mathtools}
\usepackage{mdframed}
\usepackage[bigdelims,vvarbb]{newtxmath}
\usepackage{rotating}
\usepackage{stmaryrd}
\usepackage{pifont}
\usepackage{wasysym}
\usepackage{xcolor}

\renewcommand*\oldstylenums[1]{\textosf{#1}}

\theoremstyle{definition}
\newmdtheoremenv[%
  backgroundcolor=white,
  linecolor=white!60!black,
  linewidth=3pt]{ex}{Exercise}

\DeclareMathOperator\kernel{kernel}

\title{Linear Algebra 2 -- Practicals}
\author{Lukas Prokop}
\date{summer term 2016}

\newcommand\meta[3]{This #1 took place on #2 (#3).\par}
\newcommand\abs[1]{|\,#1\,|}
\newcommand\set[1]{\left\{#1\right\}}
\newcommand\setdef[2]{\left\{#1\,\middle|\,#2\right\}}
\newcommand\card[1]{\left|\,#1\,\right|}
\newcommand\divides[2]{#1\,\mid\,#2}
\newcommand\mathspace{\hspace{20pt}}
\newcommand\functional[1]{\left\langle{#1}\right\rangle}
\newcommand\Q{\mathbb{Q}}
\newcommand\nope{\lightning}
\newcommand\vecfour[4]{\begin{pmatrix} #1 \\ #2 \\ #3 \\ #4 \end{pmatrix}}
\newcommand{\textgreek}[1]{\begingroup\fontencoding{LGR}\selectfont#1\endgroup}

\parindent0pt
\parskip7pt
\setcounter{tocdepth}{1}

\begin{document}
\maketitle
\tableofcontents

\clearpage
\section{Exercise 1}
\begin{ex}
  Determine the matrix representation of the linear map
  \[ f: \mathbb R_1[x] \to \mathbb R_2[x] \]
  \[ p(x) \mapsto (x-1) \cdot p(x) \]
  in regards of bases $B = \set{1-x, 1+x} \subseteq \mathbb R_1[x]$ and $C = \set{1, 1 + x, 1 + x + x^2} \subseteq \mathbb R^2[x]$.
\end{ex}

\[
  f: \mathbb R_1[x] \to \mathbb R_2[x]
\] \[
  f: p(x) \mapsto (x-1) p(x)
\] \[
  B = \set{1-x, 1+x} \eqqcolon \set{b_1, b_2}
\] \[
  C = \set{1, 1+x, 1+x+x^2} \eqqcolon \set{c_1, c_2, c_3}
\]

Find $A \in \mathbb K^{3\times 2} \eqqcolon M_C^B(f)$.

\[ \forall v \in \mathbb R_1: f(v) = w : \Phi_C(w) = A \Phi_B(v) \]

\[ f(b_1) = (1-x)(x-1) = -x^2 + 2x - 1 \]
\[ f(b_2) = (x-1)(x+1) = x^2 - 1 \]

\[ \Phi_C(f(b_1)) \]

Coefficient comparison:
\begin{align*}
  -x^2 + 2x - 1 &= \lambda_1 \cdot 1 + \lambda_2 (1 + x) + \lambda_3 (1 + x + x^2) \\
  x^2: & \lambda_3 = -1 \\
  x^1: & 2 = \lambda_2 + \lambda_3 \Rightarrow \lambda_2 = 3 \\
  x^0: & -1 = \lambda_1 + \lambda_2 + \lambda_3 \Rightarrow \lambda_1 = -3
\end{align*}

\[ \Phi_C(f(b_1)) = \begin{pmatrix} 3 \\ 3 \\ 1 \end{pmatrix} \]
\[ \Phi_C(f(b_2)): x^2 = 1 = \lambda_1 \cdot 1 + \lambda_2 (1 + x) + \lambda_3 (1 + x + x^2) \]
\begin{align*}
  x^2: & \lambda_3 = 1 \\
  x^1: & \lambda_2 + \lambda_3 = 0 \Rightarrow \lambda_2 = -1 \\
  x^0: & -1 = \lambda_1 + \lambda_2 + \lambda_3 \\
       & -1 = \lambda_1 - 1 + 1 \\
       & -1 = \lambda_1
\end{align*}

\[ \Phi_C(f(b_2)) = \begin{pmatrix} -1 \\ -1 \\ 1 \end{pmatrix} \]
\[ A = \begin{pmatrix} -3 & -1 \\ 3 & -1 \\ 1 & 1 \end{pmatrix} \]


\section{Exercise 3}
\begin{ex}
  Let $A_1, A_2, \ldots, A_k$ be quadratic $n\times n$ matrices over the field $\mathbb K$.
  Show that the product $A_1 A_2 \ldots A_k$ is invertible if and only if all $A_i$ are invertible.
\end{ex}

All $A_i$ are invertible, then $\prod A_i$ is invertible.

$A, B$ invertible, then $AB$ is invertible and $(AB)^{-1} = B^{-1} A^{-1}$.
Generalize by induction.

If $\prod A_i$ is invertible, then all $A_i$ are invertible.

Sidenote: We know that $\operatorname{rank}(A) = n - \dim{\kernel(A)}$.

\begin{description}
  \item[$k=1$] trivial
  \item[$k=2$] $A_1 A_2$ is invertible. Let $C = (A_1 A_2)^{-1}$. Then $C A_1 A_2 = I_n$.
    Let $x \in \kernel(A_2) \Rightarrow A_2 x = 0 \Rightarrow \underbrace{C A_1}_{I_n} A_2 x = C A_1 0 = 0$.
    \[ \kernel(A_2) = 0 \Rightarrow \operatorname{rank}(A_2) = n - 0: n \Rightarrow A_2 \text{ invertible} \]
    \[ A_1 = \underbrace{A_1 A_2}_{\text{invertible}} \cdot \underbrace{A_2^{-1}}_{\text{invertible}} \]
  \item[$k \to k+1$]
    Let $A_1 \ldots A_{k+1}$ is invertible $\Rightarrow (A_1, \ldots, A_k) A_{k+1}$ is invertible $\xRightarrow{k=2} A_1, \ldots, A_k$ is invertible, $A_{k+1}$ invertible $\xRightarrow{\text{induction base}}$ $A_1, \ldots, A_k, A_{k+1}$ is invertible.
\end{description}

Remark:
$A,B \in \mathbb K^{n\times n}$. $B$ is inverse of $A$
\[ \Leftrightarrow AB = I = BA \Leftrightarrow AB = I \Leftrightarrow BA = I \]

\section{Exercise 2}
%
\begin{ex}
  Let $V$ be a vector space and $f: V \to \mathbb V$ is a nilpotent linear map,
  hence there exists some $k \in \mathbb N$ such that $f^k = 0$.
\end{ex}

\subsection{Part a}
\begin{ex}
  Show that $\text{id}_V - f$ is invertible with $(\text{id}_V - f)^{-1} = \text{id}_V + f + f^2 + \ldots + f^{k-1}$.
\end{ex}

Show that: $(\text{id}_v - f)^{-1} = \sum_{i=0}^{k-1} f^i$.
\[ (\text{id}_V - f) \circ \left(\sum_{i=0}^{k-1} f^{i}\right) = \text{id}_V \circ \sum_{i=0}^{k-1} f^i - f \circ \sum_{i=0}^{k-1} f^i - \sum_{i=0}^{k-1} f^{i+1} = f^0 + \sum_{i=1}^{k-1} f^i - \sum_{i=1}^{k-1} f^i - f^k = \text{id}_V - 0 = \text{id}_V \]
and $\left(\sum_{i=0}^{k-1} f^i\right) \circ \left(\text{id}_V - f\right)$ analogously.

\subsection{Part b}
\begin{ex}
  Use part a) to determine the inverse of the matrix
  \[
    \begin{pmatrix}
      1 & 2 & 3 & 4 \\
      0 & 1 & 2 & 3 \\
      0 & 0 & 1 & 2 \\
      0 & 0 & 0 & 1
    \end{pmatrix}
  \]
\end{ex}
\[
  \begin{pmatrix}
    1 & 2 & 3 & 4 \\
    0 & 1 & 2 & 3 \\
    0 & 0 & 1 & 2 \\
    0 & 0 & 0 & 1
  \end{pmatrix}
  \eqqcolon A
  = \begin{pmatrix}
    1 & 0 & 0 & 0 \\
    0 & 1 & 0 & 0 \\
    0 & 0 & 1 & 0 \\
    0 & 0 & 0 & 1
  \end{pmatrix}
  - f_A
\] \[
  f_A = I_n - A =
  \begin{pmatrix}
    0 & -2 & -3 & -4 \\
    0 & 0 & -2 & -3 \\
    0 & 0 & 0 & -2 \\
    0 & 0 & 0 & 0
  \end{pmatrix}
\] \[
  f^2_A = f \cdot f = \begin{pmatrix} 0 & 0 & 4 & 12 \\ 0 & 0 & 0 & 4 \\ 0 & 0 & 0 & 0 \\ 0 & 0 & 0 & 0 \end{pmatrix}
\] \[
  f^3 = f^2 \cdot f = \begin{pmatrix} 0 & 0 & 0 & -8 \\ 0 & 0 & 0 & 0 \\ 0 & 0 & 0 & 0 \\ 0 & 0 & 0 & 0 \end{pmatrix}
\] \[
  f^4 = f^3 \cdot f = \begin{pmatrix} 0 & 0 & 0 & 0 \\ 0 & 0 & 0 & 0 \\ 0 & 0 & 0 & 0 \\ 0 & 0 & 0 & 0 \end{pmatrix}
\]
$\Rightarrow$ f nilpotent.

\[ A^{-1} = (\operatorname{id}_v - f)^{-1} = \operatorname{id}_v + f + f^2 + f^3 \]
\[
  = \begin{pmatrix}
    1 & 0 & 0 & 0 \\
    0 & 1 & 0 & 0 \\
    0 & 0 & 1 & 0 \\
    0 & 0 & 0 & 1
  \end{pmatrix}
  +
  \begin{pmatrix}
    0 & -2 & -3 & -4 \\
    0 & 0 & -2 & -3 \\
    0 & 0 & 0 & -2 \\
    0 & 0 & 0 & 0
  \end{pmatrix}
  +
  \begin{pmatrix} 0 & 0 & 4 & 12 \\ 0 & 0 & 0 & 4 \\ 0 & 0 & 0 & 0 \\ 0 & 0 & 0 & 0 \end{pmatrix}
  +
  \begin{pmatrix} 0 & 0 & 0 & -8 \\ 0 & 0 & 0 & 0 \\ 0 & 0 & 0 & 0 \\ 0 & 0 & 0 & 0 \end{pmatrix}
\] \[
  = \begin{pmatrix} 1 & -2 & 1 & 0 \\ 0 & 1 & -2 & 1 \\ 0 & 0 & 1 & -2 \\ 0 & 0 & 0 & 1 \end{pmatrix}
\] \[
  A \cdot A' =
  \begin{pmatrix} 1 & 2 & 3 & 4 \\ 0 & 1 & 2 & 3 \\ 0 & 0 & 1 & 2 \\ 0 & 0 & 0 & 1 \end{pmatrix} \cdot
  \begin{pmatrix} 1 & -2 & 1 & 0 \\ 0 & 1 & -2 & 1 \\ 0 & 0 & 1 & -2 \\ 0 & 0 & 0 & 1 \end{pmatrix}
  = \begin{pmatrix} 1 & 0 & 0 & 0 \\ 0 & 1 & 0 & 0 \\ 0 & 0 & 1 & 0 \\ 0 & 0 & 0 & 1 \end{pmatrix}
\]

\section{Exercise 4}
\subsection{Part a}
\begin{ex}
  Let $A$ be an invertible $n\times n$ matrix over a field $\mathbb K$ and $u,v$ are
  column vectors (hence $n\times 1$ matrices), such that $\sigma  1 + v^t A^{-1} u \neq 0$.
  Show that $(A + uv^t)$ is invertible and that
  \[ (A + uv^t)^{-1} = A^{-1} - \frac1{\sigma} A^{-1} uv^t A^{-1} \]
\end{ex}

\subsection{Part b}
\begin{ex}
  Apply this formula to determine the inverse of the matrix
  \[
    A = \begin{pmatrix}
      5 & 3 & 0 & 1 \\
      3 & 2 & 0 & 0 \\
      0 & 0 & 2 & 3 \\
      0 & 0 & 3 & 5
    \end{pmatrix}
  \]
\end{ex}

\[
  B = A + S
\] \[
  B = \begin{pmatrix} 5 & 3 & 0 & 0 \\ 3 & 2 & 0 & 0 \\ 0 & 0 & 2 & 3 \\ 0 & 0 & 3 & 5 \end{pmatrix}
  + \begin{pmatrix} 0 & 0 & 0 & 1 \\ 0 & 0 & 0 & 0 \\ 0 & 0 & 0 & 0 \\ 0 & 0 & 0 & 0 \end{pmatrix}
\] \[
  \begin{pmatrix} 0 & 0 & 0 & 1 \\ 0 & 0 & 0 & 0 \\ 0 & 0 & 0 & 0 \\ 0 & 0 & 0 & 0 \end{pmatrix}
    = \begin{pmatrix} 1 \\ 0 \\ 0 \\ 0 \end{pmatrix} \cdot \begin{pmatrix} 0 & 0 & 0 & 1 \end{pmatrix}
\]

$A$ is invertible, because it is a block matrix\footnote{That's why chose $A$ and $S$ that way}.

\[
  A^{-1} = \begin{pmatrix}
    2 & -3 & 0 & 0 \\
    -3 & 5 & 0 & 0 \\
    0 & 0 & 5 & -3 \\
    0 & 0 & -3 & 2
  \end{pmatrix}
\] \[
  \sigma = 1 + \begin{pmatrix} 0 & 0 & 0 & 1 \end{pmatrix} A^{-1} \begin{pmatrix} 1 \\ 0 \\ 0 \\ 0 \end{pmatrix} = 1 + 0 \neq 0
\]

\[ \Rightarrow B^{-1} = A^{-1} - A^{-1} \begin{pmatrix} 1 \\ 0 \\ 0 \\ 0 \end{pmatrix} \cdot \begin{pmatrix} 0 & 0 & 0 & 1 \end{pmatrix} A^{-1}
  = \begin{pmatrix}
    2 & -3 & 6 & -4  \\
    -3 & 5 & -9 & 6 \\
    0 & 0 & 5 & -3 \\
    0 & 0 & -3 & 2
  \end{pmatrix}
\]


\end{document}
