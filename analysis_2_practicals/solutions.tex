\documentclass[a4paper]{article}
\usepackage[utf8]{inputenc}
\usepackage[LGR,T1]{fontenc}
\usepackage{amsmath}
\usepackage{amssymb}
\usepackage{amsfonts}
\usepackage{amsthm}
\usepackage{baskervald}
\usepackage{bbold}
\usepackage{csquotes}
\usepackage{enumerate}
\usepackage{faktor}
\usepackage{fancyhdr}
\usepackage[margin=1in]{geometry}
\usepackage[pdfborder={0 0 0},colorlinks=true,citecolor=red]{hyperref}
\usepackage{imakeidx} % before hyperref
\usepackage{mathalfa}
\usepackage{mathtools}
\usepackage{mdframed}
\usepackage[bigdelims,vvarbb]{newtxmath}
\usepackage{rotating}
\usepackage{stmaryrd}
\usepackage{pifont}
\usepackage{wasysym}
\usepackage{xcolor}

\renewcommand*\oldstylenums[1]{\textosf{#1}}

\theoremstyle{definition}
\newmdtheoremenv[%
  backgroundcolor=white,
  linecolor=white!60!black,
  linewidth=3pt]{ex}{Exercise}

\DeclareMathOperator\kernel{kernel}

\title{Linear Algebra 2 -- Practicals}
\author{Lukas Prokop}
\date{summer term 2016}

\newcommand\meta[3]{This #1 took place on #2 (#3).\par}
\newcommand\abs[1]{\left|\,#1\,\right|}
\newcommand\set[1]{\left\{#1\right\}}
\newcommand\setdef[2]{\left\{#1\,\middle|\,#2\right\}}
\newcommand\card[1]{\left|\,#1\,\right|}
\newcommand\divides[2]{#1\,\mid\,#2}
\newcommand\mathspace{\hspace{20pt}}
\newcommand\functional[1]{\left\langle{#1}\right\rangle}
\newcommand\Q{\mathbb{Q}}
\newcommand\nope{\lightning}
\newcommand\vecfour[4]{\begin{pmatrix} #1 \\ #2 \\ #3 \\ #4 \end{pmatrix}}
\newcommand{\textgreek}[1]{\begingroup\fontencoding{LGR}\selectfont#1\endgroup}

\parindent0pt
\parskip7pt
\setcounter{tocdepth}{1}

\begin{document}
\maketitle
\tableofcontents

\section{Solution of the last lecture exam of Analysis~1}
\subsection{Exam: Exercise 1}
\begin{ex}
  Determine the limes of
  \[ \sum_{n=2}^\infty \frac{1}{n^2 - 1} \]
\end{ex}

\[ \frac{1}{3} + \frac{1}{8} + \frac{1}{15} + \frac{1}{24} + \ldots \]
does not help us. What about this representation?
\[ \frac{1}{n^2-1} = \frac{1}{(n+1)(n-1)} = \frac{a}{n+1} + \frac{b}{n-1} = \frac{a(n-1) + b(n+1)}{(n+1)(n-1)} \]
\[ a(n-1) + b(n+1) = 1 \]
\[ (a + b)n + (b-a) = 1 \]
\[ \Rightarrow a + b = 0 \land b - a = 1 \]
\[ \Rightarrow a = -\frac12 \quad b = \frac12 \]

Followingly,
\[ \sum_{n=2}^\infty \frac{1}{n^2 - 1} = \sum_{n=2}^\infty \frac{1}{(n+1)(n-1)} = \sum_{n=2}^\infty \left(\frac{\frac12}{n-1} - \frac{\frac12}{n+1}\right) \]

Okay, how to proceed? Let's build a pre-factor:
\[ \frac12 \sum_{n=2}^\infty \left(\frac{1}{n-1} - \frac{1}{n+1}\right) \]
\[ = \left(\frac11 - \frac13\right) + \left(\frac12 - \frac14\right) + \left(\frac13 - \frac15\right) + \left(\frac14 - \frac16\right) + \ldots \]
\[ = \frac11 + \frac12 = \frac32 \]

Let's describe this process of cancelling out formally as telescoping sum:
\[
  S_m \coloneqq \frac12 \sum_{n=2}^m \left(\frac1{n-1} - \frac1{n+1}\right)
  = \frac12 \sum_{n=2}^m \frac{1}{n-1} - \frac12 \sum_{n=2}^m \frac{1}{n+1}
\]

Please be aware that we explicitly define $S_m$ because we want to work with finite sums.
Only in finite sums, we are always allowed to split up sums.

\[ = \frac12 \sum_{n=2}^m \frac{1}{n-1} - \frac12 \sum_{n=4}^{m+2} \frac{1}{n-1} \]
\[ = \frac12 \left(\frac11 + \frac12\right) - \frac12 \left(\frac1m + \frac1{m+1}\right) \]

We already know $\frac1m \xrightarrow{m\to\infty} 0$. Also $\frac1{m+1} \xrightarrow{m\to\infty} 0$.
Followingly also $\frac12 \left(\frac1m + \frac1{m+1}\right) \xrightarrow{m\to\infty} 0$.



\subsection{Exam: Exercise 2}
\begin{ex}
  A recursive definition of a sequence is given:
  \[ a_0 \in \mathbb R, a_0 > 1, (a_n)_{n\in\mathbb N} \]
  \[ a_{n+1} = \frac12 \left(a_n +  1\right) \]
\end{ex}

As an example, we look at the sequence with $a_0 = 2$:
\[ a_0 = 2 \qquad a_1 = \frac32 \qquad a_2 = \frac54 \qquad a_3 \frac98 \]
Another example is $a_0 = 7$:
\[ a_0 = 7 \qquad a_1 = 4 \qquad a_2 = \frac52 \qquad a_3 \frac74 \]

\begin{ex}
  a) Show that $1 \stackrel!< a_n \stackrel!{\leq} a_0 \quad \forall n \in \mathbb N$
\end{ex}

Our examples suggest that this claim might hold.

We use induction over $n$ to prove this statement:
\begin{description}
  \item[induction base] $1 < a_0 \leq a_0$ holds trivially.
  \item[induction step]
    We are given $1 < a_n \leq a_0$ by the induction hypothesis.

    \begin{align*}
      a_{n+1} &= \frac12 (a_n + 1) \\
              &\leq \frac12 (a_0 + a_0) &[\text{induction hypothesis and } 1 < a_0]
    \end{align*}

    \begin{align*}
      a_{n+1} &= \frac12 (a_n + 1) \\
              &> \frac12 (1 + 1) & [\text{induction hypothesis}] \\
              &= 1
    \end{align*}
\end{description}

\begin{ex}
  b) Prove that $a_{n+1} \stackrel{!}{<} a_n \quad \forall n \in \mathbb N$
\end{ex}

\begin{align*}
  a_{n+1} &= \frac12 (a_n + 1) \\
          &< \frac12 (a_n + a_n)   &[\text{we have proven: } a_n > 1]
\end{align*}

\begin{ex}
  c) Does this series converge? If so, give its limit.
\end{ex}

Yes, because it is monotonically decreasing (according to exercise b)
and bounded below (according to exercise a).

\[ b_n \coloneqq a_n - 1 \qquad \forall n \in \mathbb N \]
\[ b_0 \coloneqq a_0 - 1 \]
\[ b_{n+1} = a_{n+1} - 1 = \frac12 (a_n + 1) - 1 = \frac12 (b_n + 1 + 1) - 1 = \frac12 b_n \]

\[ b_n = \frac1{2^n} b_0 \to 0 \cdot b_0 = 0 \]
\[ \Rightarrow b_n \to 0 \]
\[ \Rightarrow a_n = b_n + 1 \to 1 \]


Does it work to just show: $1 = \frac12 (1 + 1)$?
Nope, because in points of continuity this might be true even though $1$ is not its limes.

Let $a_n \to a$ and $a_{n+1} = \frac12 (a_n + 1)$.
\[ a_{n+1} \to a \qquad \frac12 (a_n + 1) \to \frac12 (a + 1) \qquad a = \frac12 (a + 1) \]

\subsection{Exam: Exercise 3}
\begin{ex}
  $f: \mathbb R \to \mathbb R$ with $x \mapsto 2x^2 + 5x- 3$.
  Show continuity with an $\varepsilon$-$\delta$-proof.
\end{ex}

If we don't need an $\varepsilon$-$\delta$-proof, we would argue with the Algebraic Continuity Theorem:
The function $f$ is a composition of continuous functions, hence a continuous function itself.

$\varepsilon$-$\delta$-definition:
\[
  \forall x_0 \in \mathbb R \forall \varepsilon > 0 \exists \delta > 0:
  \abs{x - x_0} < \delta \Rightarrow \abs{f(x) - f(x_0)} < \varepsilon
\]

If $\abs{x - x_0} < \delta$,
\begin{align*}
  \abs{f(x) - f(x_0)} &= \abs{2x^2 + 5x - 3 - (2x_0^2 + 5x_0 - 3)} \\
    &= \abs{2x^2 + 5x - 2x_0^2 - 5x_0} \\
    &\leq 2 \abs{x^2 - x_0^2} + 5 \abs{x - x_0} \\
    &= 2 \abs{(x + x_0) (x - x_0)} + 5 \abs{x - x_0} \\
    &= 2 \abs{x + x_0} \abs{x - x_0} + 5 \abs{x - x_0} \\
    &\leq 2 (\abs{x} + \abs{x_0}) \abs{x - x_0} + 5 \abs{x - x_0} \\
    &\leq 2 \left(\abs{x_0} + \delta + \abs{x_0}\right) + 5 \delta \\
  \intertext{Our goal: we are able to claim $\stackrel!{<} \varepsilon$}
    &= 4 \abs{x_0} \delta + 2 \delta^2 + 5 \delta \\
    &= 2\delta^2 + (4 \abs{x_0} + 5) \delta
\end{align*}

In general (here it does not apply), that $x_0$ might be zero. So division is not allowed and requires case distinctions (cumbersome!).

The following steps work only because we know $\varepsilon > 0$ and $\delta > 0$:
\[ 2 \delta^2 < \frac{\varepsilon}{2} \]
\[ \delta < \frac{\sqrt{\varepsilon}}{2} \]
\[ (4 \abs{x_0} + 5) \delta < \varepsilon \]
\[ \delta < \frac{\varepsilon}{4 \abs{x_0} + 5} \]

Then we can submit those results as solution:

Let $\varepsilon > 0$ and $\delta \coloneqq \min\left(\frac{\sqrt{\varepsilon}}{5}, \frac{\varepsilon}{4 \abs{x_0} + 6}\right)$.
Then the $\varepsilon$-$\delta$ definition shows that $f$ is continuous.

\section{Exam: Exercise 4}
%
\begin{ex}
  Let $f: [0,1] \to \mathbb R$ be continuous and $f(0) = f(1)$.
  Show that $\exists \xi \in [0,\frac12]$ with $f(\xi) = f(\xi + \frac12)$.

  Hint: Consider $h: [0, \frac12] \to \mathbb R$ with $h(x) = f(x) - f(x + \frac12)$.
\end{ex}

Intuition:
Let $\xi = 0$ with $f(\xi) = 0$ and $\xi = \frac12$ with $f(\xi) = \frac1{16}$.
Then the difference $f(0) - f(\frac12)$ is negative. At the same time $f(\frac12) - f(1)$ is positive.
So at some point between $x=0$ and $x=1$ the difference must be zero.

\[ \exists \xi \in [0,\frac12]: h(\xi) = 0 \]
\begin{align*}
  h(0) &= f(0) - f\left(\frac12\right) \\
  h(1) &= f\left(\frac12\right) - f(1) = f\left(\frac12\right) - f(0) = -h(0)
\end{align*}

$f(x)$ is continuous in $[0,\frac12]$. $f(x + \frac12)$ is continuous in $[0,\frac12]$.
Therefore $h$ is continuous, because it is a composition of continuous functions.

\begin{description}
  \item[Case 1: $h(0) < 0$]
    Then $h(\frac12) > 0$ and $h(0) < 0 < h(\frac12)$.
    Due to Intermediate Value Theorem it holds that
    \[ \exists \xi \in [0, \frac12]: h(\xi) = 0 \]
    \[ \Rightarrow f(\xi) = f(\xi + \frac12) \]
  \item[Case 2: $h(0) > 0$]
    Then $h(\frac12) < 0$. Remaining part analogous.
  \item[Case 3: $h(0) = 0$]
    Then by definition $f(0) = f(\frac12)$, so choose $\xi = 0$.
\end{description}

\clearpage
\section{Exercise 1}
\begin{ex}
  Investigate the function $f: \mathbb R \to \mathbb R, x \mapsto \frac12 (x \abs{x} + x^2)$
  in terms of multiple differentiability in all points $x_0 \in \mathbb R$.
\end{ex}

\[ f'(x) = \begin{cases} 0 & x \leq 0 \\ 2x & x > 0 \end{cases} \]

So this is differentiable, but in case of $x=0$, it remains questionable.

We look at the definition of differentiability:
\[
  \lim_{x \to 0} \frac{f(x) - f(0)}{x}
  = \lim_{x\to 0} \frac{f(x)}{x}
\] \[
  f'(x) = \begin{cases}
    \lim_{x\to 0} \frac{0}{x} = 0 \\
    \lim_{x\to 0^+} \frac{x^2}{x} = \lim_{x\to 0^+} x = 0
  \end{cases}
\]

It follows that $f$ is differentiable one time.

\[
  f''(x) = \begin{cases}
    0 & x < 0 \\
    2x & x > 0
  \end{cases}
\]

What about $x=0$?

\[
  \lim_{x\to 0} \frac{f'(x) - f'(0)}{x - 0}
  \begin{cases}
    \lim_{x\to 0} \frac0x = 0 \\
    \lim_{x\to 0^+} \frac{2x}{x} = \lim_{x\to0^+} 2 = 2
  \end{cases}
\]

Left and right limes differ. So it is not differentiable.

\section{Exercise 2}
\begin{ex}
  Determine, possibly using l'H\^{o}pital's rule, the following limits:
  \begin{enumerate}
    \item $\lim_{x\to1} \frac{\ln{x}}{x - 1}$
    \item $\lim_{x\to0^+} \frac{1}{x} - \frac{1}{\sin{x}}$
    \item $\lim_{x\to\frac\pi2^-} \frac{\ln(\cos{x})}{\ln(1 - \sin{x})}$
    \item $\lim_{x\to1^-} x^\frac{1}{1-x}$
    \item $\lim_{\substack{n\to\infty \\ n \in \mathbb N}} n^{\frac{1}{\sqrt{n}}}$
    \item $\lim_{x\to\infty} \frac{e^x - e^{-x}}{e^x + e^{-x}}$
  \end{enumerate}
\end{ex}

\subsection{Exercise 2.a}
%
\[ \lim_{x\to1} \frac{\ln{x}}{x - 1} \]
The conditions to apply L'H\^opital's rule are satisfied.

\[ \Rightarrow \lim_{x\to1} \frac{\frac{1}{x}}{1} = 1 \]

\subsection{Exercise 2.b}
%
\[ \lim_{x\to0^+} \frac{1}{x} - \frac{1}{\sin{x}} = \lim_{x\to0^+} \frac{\sin{x} - x}{x \sin{x}} \]
The conditions to apply L'H\^opital's rule are satisfied.

\[ \Rightarrow \lim_{x\to0^+} \frac{\cos{x} - 1}{\sin{x} + x \cos{x}} \]
The conditions to apply L'H\^opital's rule are satisfied.

\[ \Rightarrow \lim_{x\to0^+} \frac{-\sin{x}}{\cos{x} + \cos{x} - x \sin{x}} = \lim_{x\to0^+} \frac{-\sin{x}}{2 \cos{x} - x \sin{x}} = \frac02 = 0 \]

A nice hint to find out whether this function is differentiable:
\[ \sin{x} = x - \frac{x^3}{3!} + \frac{x^5}{5!} - \ldots \]
\[ \cos{x} = 1 - \frac{x^2}{2!} + \frac{x^4}{4!} - \ldots \]

\[ \frac{\sin{x} - x}{x \sin{x}} = \frac{-\frac{x^3}{3!} + \frac{x^5}{5!} - \ldots}{x^2 - \frac{x^4}{3!} + \frac{x^6}{5!}} \approx x \to 0 \]

This exploits, that it will take one run of L'H\^opital's rule (because each expression has at least degree $2$) and its limes will be 0 (because of $x$).

\subsection{Exercise 2.c}
%
\[ \lim_{x\to\frac\pi2^-} \frac{\ln(\cos(x))}{\ln(1 - \sin(x))} \]
The conditions to apply L'H\^opital's rule are partially satisfied.
We claim that $\lim_{x\to0^+} f(x) = \lim_{x\to0^+} g(x) = \infty$ is fine.

\[
  \Rightarrow \lim_{x\to\frac\pi2^-} \frac{\frac{-\sin(x)}{\cos(x)}}{\frac{-\cos(x)}{1 - \sin(x)}}
  = \lim_{x\to\frac\pi2^-} \frac{-\sin(x) \cdot \left(1 - \sin(x)\right)}{\cos(x)(-\cos(x))}
\]
The conditions to apply L'H\^opital's rule are partially satisfied.
\[
  \lim_{x\to\frac\pi2^-} \frac{-\cos(x)(1 - \sin(x)) - \sin(x) \cdot (-\cos(x))}{-\sin(x) (-\cos(x)) + \cos(x) \cdot \sin(x)} = \frac12
\]

If we want to apply the previous estimate here, we should consider
\[ \sin(x) = \cos\left(\frac\pi2 - x\right) = \cos(y) \qquad y = \frac\pi2 - x \]
\[ \cos(x) = \sin\left(\frac\pi2 - x\right) = \sin(y) \]

This gives us a different estimate of the result:
\[
  \lim_{y\to0^+} \frac{\ln(\sin(y))}{\ln(1 - \cos(y))}
  \approx \lim_{y\to0^+} \frac{\ln(y)}{\ln\left(\frac{y^2}{2}\right)}
  = \lim_{y\to0^+} \frac{\ln(y)}{2\ln(y) - \ln(2)}
  \approx \lim_{y\to0^+} \frac{\ln(y)}{2\ln(y)}
  = \frac12
\]


We define neighborhoods:
\[ N_\delta(x_0) = \set{x: \abs{x - x_0} < \delta} \]
\[ N_R(\infty) = \set{x: x > R} \]

\subsection{Exercise 2.d}
%
\[
  \lim_{x\to1^-} x^{\frac{1}{1 - x}} = \lim_{x\to1^-} e^{\ln(x) \frac{1}{1 - x}}
  = \exp\left(\lim_{x\to1^-} \underbrace{\frac{\ln(x)}{1 - x}}_{(-1) \cdot \text{Exercise a}}\right)
  = \frac1e
\]

\subsection{Exercise 2.e}
%
\[
  \lim_{n\to\infty} n^{\frac1{\sqrt{n}}} = \lim_{n\to\infty} \left(\exp\left(\frac{\ln{n}}{\sqrt{n}}\right)\right)
    = \exp\left(\lim_{n\to\infty} \frac{\ln(n)}{\sqrt{n}}\right)
\]
The conditions to apply L'H\^opital's rule are satisfied (,,$\frac{\infty}{\infty}$'')

\[
  \exp\left(\lim_{n\to\infty} \frac{\frac{1}{n}}{\frac{1}{2\sqrt{n}}}\right)
    = \exp\left(\lim_{n\to\infty} \frac{2\sqrt{n}}{n}\right)
    = \exp(0) = 1
\]

\subsection{Exercise 2.f}
%
\[
  \lim_{x\to\infty} \frac{e^x - e^{-x}}{e^x + e^{-x}}
    = \lim_{n\to\infty} \frac{e^x \left(1 - e^{-2x}\right)}{e^x \left(1 + e^{-2x}\right)}
    = \frac{\lim_{x\to\infty} 1 - \lim_{x\to\infty} \frac{1}{e^{2x}}}{\lim_{x\to\infty} 1 + \lim_{x\to\infty} \frac{1}{e^{2x}}}
\]

Remark:
\[
  \lim_{x\to\infty} \frac{\sinh(x)}{\cosh(x)}
    \overset{\text{L'H\^opital}}= \lim_{x\to\infty} \frac{\cosh(x)}{\sinh(x)}
    \overset{\text{L'H\^opital}}= \lim_{x\to\infty} \frac{\sinh(x)}{\cosh(x)}
\] \[
  y = \lim_{x\to\infty} \frac{\sinh(x)}{\cosh(x)} = \frac{1}{\lim_{x\to\infty} \frac{\sinh(x)}{\cosh(x)}}
    = \frac1y
\]

\section{Exercise 3}
%
\begin{ex}
  Show that the function $f: \mathbb R \to \mathbb R$ with $x \mapsto x + e^x$
  is bijective. Furthermore determine $(f^{-1})'(1)$ and $\lim_{y\to\infty}(f^{-1})'(y)$.
\end{ex}

If the function is strictly monotonically increasing,
it is injective.

\[ f'(x) = 1 + e^x > 0 \qquad \forall x \in \mathbb R \]

We show that it is strictly monotonically increasing:

Let $x_1, x_2 \in \mathbb R$ with $x_1 < x_2$.
\[ \frac{f(x_2) - f(x_1)}{x_2 - x_1} = f'(\alpha) \qquad \text{with } \alpha \in (x_1, x_2) \]
\[ f(x_2) - f(x_1) = f'(\alpha)(x_2 - x_1) > 0  \]

Is $f$ surjective?

For an arbitrary $y_0 \in \mathbb R$ it holds that $\exists x_0 \in \mathbb R: f(x_0) = y_0$:
\[ \exists f(a), f(b) \in \mathbb R: f(a) \leq y_0 < f(b) \]

It holds that
\[ \lim_{x\to-\infty} x + \underbrace{e^x}_{\to 0} = -\infty \]
\[ \lim_{x\to+\infty} x + e^x = \infty \]

Formally:
\[ \forall y_0 \exists x_0: \forall x < x_0: f(x) < y_0 \]

From the Intermediate Value Theorem it follows that
\[ \Rightarrow \exists c \in [a,b): f(c) = y_0 \qquad c \eqqcolon x_0 \]
So it is surjective.

From injectivity and surjectivity it follows that it is bijective.

\subsection{Determine $(f^{-1})'(1)$}
%
\[ f(x) = x + e^x \]
\[ f'(x) = 1 + e^x \]

We apply the inverse function theorem:
\[ \left(f^{-1}\right)'(y) = \frac{1}{f'(f^{-1}(y))} \]

\[ y = 1 = f(x) \]
\[ x = f^{-1}(1) \]

An educated guess gives us that $x = 0$. In general determining $x$ is more difficult.

\[ \left(f^{-1}\right)'(1) = \frac{1}{f'(0)} = \frac{1}{1 + e^0} = \frac12 \]

\subsection{Determine $\lim_{y\to\infty} \left(f^{-1}\right)'(y)$}
%
\[
  \lim_{y\to\infty} \left(f^{-1}\right)'(y)
  = \lim_{y\to\infty} \frac{1}{1 + e^x}
\]

As $x$ grows to infinity, also $y$ grows to infinity.
From bijectivity it follows that any value can be reached with $x$ as well as $f(x)$.

\[ \underbrace{f'(\underbrace{f^{-1}(\underbrace{y}_{\to\infty})}_{\to\infty})}_{\to\infty} \]

\section{Exercise 4}
\begin{ex}
  Let $D \subseteq \mathbb R$ be an open interval and $f: D \to \mathbb R$ be differentiable
  in $x_0 \in D$.
  Show
  \[ \lim_{h\to0} \frac{f(x_0 + h) - f(x_0 - h)}{2} = f'(x_0) \]
\end{ex}

\[ = \lim_{h\to0} \frac{f(x_0 + h) - f(x_0) + f(x_0) - f(x_0 - h)}{2h} \]
\[ = \lim_{h'\to 0} \frac12 \cdot \left(f'(x_0) + \frac{f(x_0) - f(x_0 + h')}{-h'}\right) \]
\[ = \lim_{h'\to 0} \frac12 \cdot \left(f'(x_0) + \frac{f(x_0 + h') - f(x_0)}{h'}\right) \]
\[ = \frac12 (f'(x_0) + f'(x_0)) \]
\[ = f'(x_0) \]

\subsection{Exercise 4.b}
\[
  \lim_{h\to0} \frac{f(x_0 + rh) - f(x_0 + sh)}{h}
    = \lim_{h\to0} \frac{f(x_0 + rh) - f(x_0)}{h}
    + \lim_{h\to0} \frac{f(x_0) - f(x_0 + sh)}{h}
\]
\[ h_1 = rh \qquad h_2 = sh \]
\[ = \lim_{h_1\to0} \frac{f(x_0 + h_1) - f(x_0)}{\frac1r \cdot h_1} + \lim_{h_2\to0} \frac{f(x_0) - f(x_0 + h_2)}{\frac1s \cdot h_2} \]
\[ = r \cdot f'(x_0) - s \cdot f'(x_0) \]
\[ = (r-s) \cdot f'(x_0) \]

\section{Exercise 5}
\begin{ex}
  Let $D \subseteq \mathbb R$ be an open interval. $f: D \to \mathbb R$ is differentiable
  and $f$ is twice differentiable in $x_0 \in D$.
\end{ex}

\subsection{Exercise 5.a}
\begin{ex}
  Show that
  \[ \lim_{h\to0} \frac{f(x_0 + h) - 2f(x_0) + f(x_0 - h)}{h^2} = f''(x_0) \]
\end{ex}

$f$ is differentiable, therefore continuous, and $h$ goes to $0$. So we have ,,$\frac00$''.
All conditions to apply L'H\^opital's rule are satisfied.

\[ \lim_{h\to0} \frac{f'(x_0 + h) - f'(x_0 - h)}{2h} \approx \frac00 \]

We can apply L'H\^opital's Rule again or just use the result of exercise~4a.
\[ \xRightarrow{4a} f''(x_0) \]

\subsection{Exercise 5.b}
%
\begin{ex}
  Show that the limes from exercise 5.a can also exist, even if $f''(x_0)$ does not exist.
  Use the result from Exercise~1.
\end{ex}
\[ f(x) = \begin{cases} x^2 & x > 0 \\ 0 & x = 0 \\ -x^2 & x < 0 \end{cases} \]

%For $x > 0$:
%\[
%  \lim_{h\to0} \frac{(x + h)^2 - 2x^2 + (x - h)^2}{h^2}
%    = \frac{x^2 + 2xh + h^2 - 2x^2 + x^2 - 2xh + h^2}{h^2}
%    = \lim_{n\to0} \frac{2h^2}{h^2} = \lim_{h\to0}
%\]

We know that it is not twice differentiable. But we want to show that the limes exists.

We are only concerned with $x=0$.
\[ \lim_{h\to0} f(x_0) = 0 \]
\[
  \lim_{h\to0} \frac{h^2 - h^2}{h^2} = \frac{0}{h^2} = 0
\]

So if we traverse the graph from both sides at the same time $\frac{f(x_0 + h) - f(x_0 - h)}{h}$.

\section{Exercise 6}
\begin{ex}
  Determine the following limit for arbitrary $c \in \mathbb R$:
  \[ \lim_{n\to\infty} \frac{n}{\ln{n}} \left(\sqrt[n]{n^c} - 1\right). \]
\end{ex}

\[ \lim_{n\to\infty} \frac{n}{\ln{n}} \left(\sqrt[n]{n^c} - 1\right) \]
\[
  \lim_{n\to\infty} \frac{n}{\ln{n}} \left(\sqrt[n]{n^c} - 1\right)
    = \lim_{n\to\infty} \frac{e^{\frac cn \cdot \ln{n}} - 1}{\frac{\ln{n}}{n}}
\]
and
\[
  \left(e^{\frac cn \cdot \ln{n}}\right)'
    = e^{\frac cn \cdot \ln{n}} \cdot \left(-\frac c{n^2} \cdot \ln{n} + \frac cn \cdot \frac1n\right)
    = \frac c{n^2} e^{\frac cn \cdot \ln{n}} \cdot (1 - \ln(n))
\]

All conditions are satisfied to apply L'H\^opital's rule (\enquote{$\frac00$}):

\[
  \lim_{n\to\infty} \frac{\frac c{n^2} e^{\frac cn \cdot \ln{n}} \cdot (1 - \ln{n})}{\frac{\frac{1}{n} \cdot n - \ln{n}}{n^2}}
\] \[
  = \lim_{n\to\infty} \frac{c \cdot e^{\frac cn \cdot \ln{n}} (1 - \ln(n))}{1 - \ln{n}}
  = \lim_{n\to\infty} c \cdot e^{\frac cn \cdot \ln{n}}
  = c \cdot 1
\]

\section{Exercise 7}
\begin{ex}
  \begin{itemize}
    \item Show that $e^x \geq 1 + x$ holds for all $x \in \mathbb R$. \\
      \emph{Hint:} On demand, use the Mean Value Theorem.
    \item Prove that for all $x > 0$, the following estimates hold:
      \[ \ln{x} \leq x - 1 \]
      and for all $k \in \mathbb N_+$ it holds that
      \[ k\left(1 - \frac{1}{\sqrt[k]{x}}\right) \leq \ln{x} \leq k \left(\sqrt[k]{x} - 1\right) \]
  \end{itemize}
\end{ex}

\begin{description}
  \item[$x \geq 0$]
    Choose $f(x) = e^x$ in $[0,x)$. Mean value theorem:
    \[ \exists x_0: f'(x_0) = \frac{f(b) - f(a)}{b - a} \quad \text{ for } a < x_0 < b \]
    \[ f'(x_0) = e^{x_0}   \qquad   e^{x_0} \geq 1   \qquad   x_0 \geq 0 \]
    \[ e^{x_0} = \frac{f'(x) - f(0)}{x - 0} = \frac{e^x - e^0}{x} = \frac{e^x - 1}{x} \Rightarrow \frac{e^x - 1}{x} \geq 1 \]

    Or alternatively: $f$ is convex and therefoer $f''(x) > 0$.
\end{description}

Consider $f(x) = x - 1 - \ln{x}$
\[ f'(x) = 1 - \frac1x \qquad f''(x) = \frac1{x^2} \]
\[ f'(x) \stackrel!= 0 \]
\[ 1 - \frac1x = 0 \Leftrightarrow x = -1 \]
\[ f''(1) = 1 > 0 \Rightarrow \text{ minimum and because } f(1) = 0 \Rightarrow \forall x: x - 1 - \ln{x} \geq 0 \]

Or alternatively:
\[ y \coloneqq x - 1 \]
\[ x = y + 1 \]
Show that $\ln(y + 1) \leq y \Leftrightarrow y + 1 \leq e^y$.

$e^x$ is monotonically increasing $\Rightarrow x \leq y \Leftrightarrow e^x \leq e^y$.

And this has been proven previously.

\subsection{Exercise 7.b}
\[ \ln(x) \leq k \left(\frac[k]{x} - 1\right) \]
\[ \ln(\sqrt[k]{x}) \leq \sqrt[k]{x} - 1 \Leftrightarrow \ln(y) \leq y - 1 \]
And this has been proven in Exercise~a.

The second part following analogously.

\section{Exercise 8}
\begin{ex}
  Let $f: D \to \mathbb R$ with $D \subseteq \mathbb R$. Show: If $f$ is continuous in an environment
  $U$ of $a \in D$, differentiable in $U \setminus \set{a}$ and there exists $\lim_{x\to a} f'(x)$,
  such that $f$ in $a$ differentiable and
  \[ f'(a) = \lim_{x\to a} f'(x). \]
  \emph{Hint:} On demand, use the Mean Value Theorem.
\end{ex}

Let $h_n$ be an arbitrary zero-sequence (with $h_n(x) > 0 \quad \forall x \in D$) and due to Mean Value Theorem $\exists \xi_n \in D$ with $f'(\xi_n) = \frac{f(a + hn) - f(a)}{h_n}$.
\[ \lim_{n\to\infty} f'(\xi_n) = \lim_{x\to a} f'(x) = \lim_{n\to\infty} \frac{f(a + h_n) - f(a)}{h_n} = f'(a) \]
\[ \lim_{n\to\infty} \frac{f(a + h_n) - f(a)}{h_n} = \lim_{n\to\infty} f'(\xi_n) = \lim_{x\to a} f'(x) = z \]

For the arbitrary zero-sequence, we really need to consider it arbitrary (otherwise we just show it for the one sequence). Consider this counterexample:
\[
  f(x) = \begin{cases}
    0 & x = \frac1n \text{ for } n \in \mathbb N \\
    1 & \text{else}
  \end{cases}
\]

\subsection{Alternative approach}
Application of \enquote{Schrankensatz}.

\[ \exists \lim{f'(x)} = \alpha \]
Hence for arbitrary $\varepsilon > 0: \exists \delta > 0 \forall x \in (a - \delta, a + \delta) \setminus \set{a}: \abs{f'(x) - \alpha} < \varepsilon$.
Hence $\alpha - \varepsilon < f'(x) < \alpha + \varepsilon$.

\begin{itemize}
  \item
    \[
      \forall x \in (a, a + \delta):
      \alpha - \varepsilon \leq \frac{f(x) - f(a)}{x - a} \leq \alpha + \varepsilon
    \]
  \item
    \[
      \forall x \in (a - \delta, a):
      \alpha - \varepsilon \leq \frac{f(x) - f(a)}{x - a} \leq \alpha + \varepsilon
    \]
\end{itemize}

\[ \Rightarrow \forall x \in (a - \delta, a + \delta) \setminus \set{a}: \abs{\frac{f(x) - f(a)}{x - a} - \alpha} \leq \varepsilon \]
\[ \Rightarrow \lim_{x\to a} \frac{f(x) - f(a)}{x - a} = \alpha \]

\subsection{Second alternative approach}
%
\[
  \lim_{f(a + h) - f(a)}{h}
\]
If I know $f$ is continuous, then $f(a + h) \to f(a)$. So,
\[ \enquote{\frac 00} \]
\[ \lim_{h\to 0} \frac{f'(a + h) - 0}{1} = \lim_{h\to0} f'(a + h) = \lim_{x\to a} f'(x) \]

\section{Exercise 9}
\begin{ex}
  Let $f: [a,b] \to \mathbb R$, $a < b$, differentiable with $f(a) > 0$, $f'(a) > 0$ and $f(b) = 0$.
  Prove that there exists $\xi \in (a,b): f'(\xi) = 0$.
\end{ex}

First, we want to show that $f'(a) > 0 \Rightarrow \exists \delta > 0 \forall x \in (a, a + \delta): f(x) > f(a)$.

\[ \exists \delta > 0 \forall x \in (a, a + \delta): \frac{f(x) - f(a)}{x - a} > \frac{f'(a)}{2} > 0 \]
\[ \Rightarrow f(x) - f(a) > \frac{f'(a)}{2} (x - a) > 0 \]

Indeed, $f(x)$ satisfies this property.

Secondly, we want to show that,
\[ \exists \eta \in (a + \delta, b): f(a) = f(\eta) \]
\[ \exists \xi \in [a,\eta] \forall x_1 \in [a,\eta]: f(\xi) \geq f(x_1) \]
\[ \exists \xi \in (a, \eta): \frac{f(\eta) - f(a)}{\eta - a} = f'(\eta) = 0 \]

There might be more than this one $\xi$, so the $\xi$ between the second and third line might be different.
Anyways, we found a $\xi$ with the desired property.

\section{Exercise 10}
\begin{ex}
  Determine the pointwise limit of the following function sequences $f_n: [0,\infty) \to \mathbb R$
  and determine its uniform convergence:
  \begin{itemize}
    \item $f_n(x) = \sqrt[n]{x}$
    \item $f_n(x) = \frac{1}{1 + nx}$
    \item $f_n(x) = \frac{x}{1 + nx}$
  \end{itemize}
\end{ex}

\subsection{Exercise 10.a}
%
If $x \neq 0$, $\lim_{n\to\infty} \sqrt[n]{x} = 1$. \\
If $x = 0$, $\lim_{n\to\infty} \sqrt[n]{x} = \lim_{n\to\infty} 0^{\frac 1n} = 0$.

In terms of uniform convergence:
\[ \abs{\sqrt[n]{x} - 1} < \varepsilon \]
\[ \lim_{x\to\infty} \sqrt[n]{x} = \infty \]

Example:
\begin{align*}
  \abs{\sqrt[n]{x} - 1} &< \varepsilon \\
  \sqrt[n]{x} - 1 &< \varepsilon \\
  \sqrt[n]{x} &< \varepsilon + 1 \\
  \sqrt[n]{100} &< \varepsilon + 1
\end{align*}

\subsection{Exercise 10.b}
%
\[ f_n(x) = \frac{1}{1 + nx} \]
If $x \neq 0$,
\[ \lim_{n\to\infty} \frac{1}{1 + nx} = 0 \]
If $x = 0$,
\[ \lim_{n\to\infty} \frac{1}{1 + n\cdot 0} = 1 \]

Assume it it continuously convergent. Show that:
\[ \exists \varepsilon > 0 \forall N \in \mathbb N \exists x \in [0,\infty): n \geq N \land \abs{f_n(x) - f(x)} \geq \varepsilon \]
Does not hold for $\frac 9n \geq x$.

\subsection{Exercise 10.c}
%
\[ f_n(x) = \frac{x}{1 + nx} \]
If $x \neq 0$,
\[ \lim_{n\to\infty} \frac{x}{1 + nx} = \lim_{n\to\infty} \frac{1}{\frac1x + n} = 0 \]
If $x = 0$,
\[ \lim_{n\to\infty} \frac{0}{1 + n \cdot 0} = 0 \]

\[ \abs{\frac{x}{1 + nx} - 0} < \varepsilon \]
\[ \abs{\frac{x}{1 + nx}} < \abs{\frac{x}{nx}} = \abs{\frac{1}{n}} \]

Convergence is given. Uniform convergence is not given.

\emph{Advice:} The simplest approach to show convergence is to show:
\[ \abs{f_n(x) - f(x)} \leq a_n \to 0 \]
where $a_n$ is independent from $x$.

\section{Exercise 11}
\begin{ex}
  Determine $\cos{\alpha}$, $\sin{\alpha}$ and $\tan{\alpha}$ for $\alpha \in \set{\frac{\pi}{5}, \frac{2\pi}{5}}$.

  Hint: Show that $u \coloneqq \cos{\frac{2\pi}{5}}$ and $v \coloneqq \cos{\frac{\pi}{5}}$ satisfy
  the equations $u = 2v^2 - 1$ and $-2u^2 + 1 = v$.
  Determine $u,v$ this way.
\end{ex}

\begin{align*}
  u &= \cos\left(\frac{2\pi}{5}\right) = \cos\left(\frac\pi5 + \frac\pi5\right) \\
    &= \cos^2\left(\frac\pi5\right) - \sin^2\left(\frac\pi5\right) \\
    &= 2 \cos^2\left(\frac\pi5\right) - 1 \\
    &= 2v^2 - 1
\end{align*}

To show: $v + 2u^2 - 1 = 0$, $\cos\left(\frac\pi5\right) + 2\cos^2\left(\frac\pi5\right) - 1 = 0$.

\begin{align*}
  \cos\left(\frac\pi5\right) + 2\cos\frac{2\pi}{5} - 1
    &= \cos{\frac\pi5} + \cos{\frac{4\pi}{5}} \\
    &= \cos{\frac\pi5} + \cos\left(\pi - \frac15 \pi\right) \\
    &= \cos{\frac\pi5} + \cos\pi \cdot \cos\left(\frac{\pi}{5}\right) + \sin\pi \cdot \sin{\frac\pi{5}} - \cos\frac\pi5 \cdot \cos\frac\pi5 \\
    &= 0
\end{align*}

For $u + v > 0$:
\[ 2v^2 - 1 =u \]
\[ -2u^2 + 1 = v \]
\[ 2v^2 - 2u^2 = u + v \]
\[ 2(v + u)(v - u) = u + v \]
\[ 2(v - u) = 1 \Leftrightarrow v - u = \frac12 \]

\[ v - 2v^2 + \frac12 = 0 \]
\[ v^2 - \frac12 v - \frac14 = 0 \]
\[ v_{1,2} = \frac14 \pm \sqrt{\frac{1}{16} + \frac{4}{16}} = \frac{1 \pm \sqrt{5}}{4} \]

\[ 0 < \cos(\frac\pi5) = \frac{1 + \sqrt{5}}{4} \]
\[ u = \cos\frac{2\pi}{5} = v - \frac12 = \frac{-1 + \sqrt{5}}{4} \]

\[ \cos\left(\frac{2\pi}{5}\right) = \cos^2{\frac{\pi}{5}} - \sin^2{\frac{\pi}{5}} \]
\[ \Leftrightarrow \frac{-1 + \sqrt{5}}{4} = \left(\frac{\sqrt{5} + 1}{4}\right)^2 - \sin^2\left(\frac{\pi}{5}\right) \]
\[ \Leftrightarrow \sin^2\left(\frac{\pi}{5}\right) = \frac{5 + 2 \sqrt{5} + 1}{16} - \frac{-4 + 4\sqrt{5}}{16} \]
\[ = \frac{5 + 2\sqrt{5} + 1 + 4 - 4\sqrt{5}}{16} = \frac{10 - 2\sqrt{5}}{16} = \frac{5 - \sqrt{5}}{8} \]
\[ \sin\left(\frac{\pi}{5}\right) = \sqrt{\frac{5 - \sqrt{5}}{8}} \approx 0.59 \]

\[
  \sin\frac{2\pi}{5} = \sin\left(\frac{\pi}{5} + \frac{\pi}{5}\right) = \sin{\frac\pi5} \cdot \cos{\frac\pi5}
    + \cos{\frac\pi5} \cdot \sin{\frac\pi5} = 2 \sin{\frac\pi5} \cdot \cos{\frac\pi5}
\] \[
  = 2 \frac{1 + \sqrt{5}}{4} \sqrt{\frac{5 - \sqrt{5}}{8}} = \frac{1 + \sqrt{5}}{2} \cdot \frac{5 - \sqrt{5}}{8}
  = \sqrt{\frac{5 + \sqrt{5}}{8}} \approx 0.95
\]

\[
  \tan{\frac\pi5} = \frac{\sin{\frac\pi5}}{\cos{\frac\pi5}}
     = \frac{\sqrt{\frac{5 - \sqrt{5}}{8}}}{\frac{\sqrt{5} + 1}{4}} = \frac{\sqrt{2 (5 - \sqrt{5})}}{1 + \sqrt{5}}
     = \sqrt{5 - 2\sqrt{5}} \approx 0.73
\] \[
  \tan\left(\frac{2\pi}{5}\right) = \frac{\sin{\frac{2\pi}{5}}}{\cos{\frac{2\pi}{5}}}
     = \frac{4}{-1 + \sqrt{5}} \cdot \frac{1 + \sqrt{5}}{2} \cdot \sqrt{\frac{5 - \sqrt{5}}{8}}
     = \sqrt{5 + 2\sqrt{5}} \approx 3.05
\]

\section{Exercise 12}
\begin{ex}
  To which order do you have to consider values in the series expansion of cosine,
  to approximate $\cos{1}$ with an error smaller $10^{-7}$?
  Furthermore show that $\cos{1}$ is irrational.

  \textbf{Hint:} To show irrationality of $\cos{1}$, assume, $p,q \in \mathbb N_+$ with
  $\cos{1} = \frac pq$. Replace that in the estimated error of
  \[ \cos{1} - \sum_{k=0}^q \frac{(-1)^k}{(2k)!}, \]
  multiply with $(2q)!$ and derive a contradiction.
\end{ex}

\subsection{Exercise 12.a}
%
\[ \cos{x} = \sum_{k=0}^\infty \frac{x^{2k}}{(2k)!} \cdot (-1)^k \]

Consider,
\[ S_{2m} = \sum_{k=0}^{2m} \frac{1}{(2k)!} (-1)^k \]
\[ S_{2k+1} = \sum_{k=0}^{2k+1} \frac{1}{(2k)!} \cdot (-1)^k \]

So $S_{2k+1}$ has a negative, last expression. $S_{2m}$ has a positive last expression.

\[ S_{2k+1} < \cos{1} < S_{2m} \]

\[ S_{2m} - S_{2m+1} = \sum_{k=0}^{2m} \frac{1}{(2k)!} (-1)^k - \sum_{k=0}^{2m+1} \frac{1}{(2k)!} (-1)^k \]

\[ \triangle \cos(1) = -\frac{1}{(2(2m+1))!} \cdot (-1)^{2m+1} = \frac{1}{(2 \cdot (2m + 1))!} \stackrel{!}{<} 10^{-7} \]

\[ N! > 10^7 \Rightarrow N > 11 \]
\[ 2 \cdot (2m + 1) > 11 \]
\[ 2m + 1 > \frac{11}{2} = 5.5 \]

$\Rightarrow$ 10-th order because every odd expression is cancelled out.

Consider paper: ``The irrationality of $e$ and Others''.

\subsection{Exercise 12.b}
\[
  \cos(1) \not\in \mathbb Q
\]

Assume $\exists p \in \mathbb Z$, $q \in \mathbb N$:
\[ \cos(1) = \frac{p}{q} \]

\[ \abs{\cos(1) - \sum_{k=0}^n \frac{(-1)^k}{(2k)!}} \]
\[ = \abs{\frac pq - \sum_{k=0}^{q-1} \frac{(-1)^k}{(2k)!}} < \frac{1}{(2q)!} \]
\[ = \abs{\frac{p (2q)!}{q} - \sum_{k=0}^{q-1} \frac{(-1)^k \cdot (2q)!}{(2k)!}} < 1 \]
\[ \abs{x - y} < 1 \Rightarrow 0 \qquad \text{ because } x \in \mathbb Z, y \in \mathbb Z \]

Leibniz criterion requires that the limes is not achieved in the sequence,
because the functions need to be strictly monotonical.

\section{Exercise 13}
\begin{ex}
  Let $f: [\frac\pi2, \frac{3\pi}2] \to [-1,1]$, $x \mapsto \sin{x}$.
  Show that $f$ is bijective and compute (using the formula for the derivative
  of the inverse function $(f^{-1})'(y)$ at all possible points $y \in [-1,1]$).
  Also give an explicit representation for $f^{-1}$
\end{ex}

\[ \ldots = - \frac{1}{\sqrt{1 - y^2}} \]

It is important to recognize the negative sign.

\section{Exercise 14}
\begin{ex}
  Let $w,z \in \mathbb R$ with $w, z, w + z \not\in \setdef{\frac\pi2 + k\pi}{k \in \mathbb Z}$.
  Prove the addition theorem of the tangens function:
  \[ \tan(w + z) = \frac{\tan(w) + \tan(z)}{1 - \tan(w) \tan(z)}. \]
  Let $x,y \in \mathbb R$ with $xy < 1$. Show that $\arctan(x) + \arctan(y) \in (-\frac{\pi}{2}, \frac{\pi}{2})$
  and use it to prove the addition theorem for the arcustangens function:
  \[ \arctan(x) + \arctan(y) = \arctan\frac{x + y}{1 - xy}. \]
\end{ex}

\begin{enumerate}
  \item Show that $\tan(w + z) = \frac{\tan(w) + \tan(z)}{1 - \tan(w) \tan(z)}$.

    \[ \tan(w + z) = \frac{\sin(w + z)}{\cos(w + z)} = \frac{\cos(w) \cdot \sin(z) + \sin(w) \cos(z)}{\cos(w) \cos(z) - \sin(w) \sin(z)} \]
    \[ \frac{\frac{\cos(w) \sin(w)}{\cos(w) \cos(z)} + \frac{\sin(w) \cdot \cos(z)}{\cos(w) \cdot \cos(z)}}{1 - \frac{\sin(w) \sin(z)}{\cos(w) \cos(z)}} \]
    \[ = \frac{\tan(z) + \tan(w)}{1 - \tan(w) \tan(z)} \]
  \item
    \[ \arctan(x) + \arctan(y) \in (-\frac\pi2, \frac\pi2) \]
    $x,y \in \mathbb R, xy < 1$.

    Let $x = \tan(z)$ and $y = \tan(w)$.
    \[ xy = \tan(z) \cdot \tan(w) = \frac{\sin(z) \cdot \sin(w)}{\cos(z) \cdot(w)} < 1 \]
    \[ \sin(z) \cdot \sin(w) < \cos(z) \cos(w) \]
    \[ \Leftrightarrow 0 < \cos(z) \cdot \cos(w) - \sin(z) \cdot \sin(w) \]
    \[ \Leftrightarrow 0 \stackrel!< \cos(z + w) \Leftarrow z \in \left(-\frac\pi2, \frac\pi2\right) \lor w \in \left(-\frac\pi2, \frac\pi2\right) \]
    This proof is insufficient! A case distinction for $\cos(z) \cos(w) > 0$ is required.
  \item
    Show that $\arctan(x) + \arctan(y) = \arctan\frac{x + y}{1 - xy}$.
    Let $x = \tan(z)$ and $y = \tan(w)$.
    \[
      \arctan\left(\frac{x + y}{1 - xy}\right) = \arctan\left(\frac{\tan(z) + \tan(w)}{1 - \tan(z) \tan(w)}\right)
      = \arctan(\tan(z + w)) = z + w = \arctan(x) + \arctan(y)
    \]
\end{enumerate}

\section{Exercise 15}
\begin{ex}
  Compute the following integrals by approximating the integrands using a sequence
  of step functions with the given points. Let $a,b \in \mathbb R$ with $a < b$.
  \begin{enumerate}
    \item $\int_a^b e^x \, dx$ with points $x_k \coloneqq a + k(b - a)/n$.
    \item $\int_a^b x^p \, dx$ with points $x_k \coloneqq a q^k$, $q \coloneqq \sqrt[n]{b/a}$ and $p \in \mathbb R \setminus \set{-1}$.
  \end{enumerate}
\end{ex}

\subsection{Exercise 15.a}
%
\[ = \lim_{n \to \infty} \frac{b-a}{n} \sum_{k=0}^{n-1} e^{a + \frac{k(b-a)}{n}} \]
\[ = \lim_{n \to \infty} \frac{b-a}{n} e^a \sum_{k=0}^{n-1} \left(e^{\frac{b - a}{n}}\right)^k \]
\[ = \lim_{n \to \infty} e^a \cdot \frac{b - a}{n} \frac{e^{\frac{b - a}{n}} - 1}{e^{\frac{b - a}{n} - 1}} \]
\[ = \lim_{n \to \infty} e^a \left(e^{b - a} - 1\right) \cdot \underbrace{\frac{\frac{b - a}{n}}{e^{\frac{b - a}{n}} - 1}}_{\to 1} \]
\[ = e^a \cdot \frac{e^b}{e^a} - e^a = e^b - e^a \]

\[ (\forall \varepsilon > 0) (\exists N \in \mathbb N) (\forall x \in [a,b]): \abs{\varphi_n(x)  - e^x} < \varepsilon \]
\[ e^{a + (b-a)\frac{n-1}{n}} - e^b = e^{a + (b-a)(1 - \frac1n)} - e^b = e^{a + b - \frac{b}{n} - a + \frac{a}{n}} - e^b \]
\[ = e^{b - \frac{b}{n} + \frac{a}{n} - e^b} \]

\subsection{Exercise 15.b}
%
\begin{figure}[t]
  \begin{center}
    \includegraphics{img/15b.pdf}
    \caption{Illustration of 15b}
  \end{center}
\end{figure}

\[ x_k \coloneqq aq^k \qquad q \coloneqq \left(\frac{b}{a}\right)^{\frac1n} \]
\[ p \neq -1 \]
\begin{align*}
  y_k &\coloneqq x_{k+1} - x_k \\
    &= aq^{k+1} - aq^{k} \\
    &= aq^k (q - 1)
\end{align*}

\[ \sum_{k=0}^{n-1} y_k x_k^p = \sum_{k=0}^{n-1} aq^k (q - 1) (aq^k)^p = a^{p+1} (q - 1) \sum_{k=0}^{n-1} (q^{p+1})^k \]
Is a geometric series:
\[ = a^{p+1} (q - 1) \frac{1 - (q^{p+1})^{n-1}}{1 - q^{p+1}} \]

\[
  \lim_{n\to\infty} \sum_{k=0}^{n-1} y_k x_k^p
    = a^{p+1} \lim_{n\to\infty} \left(\left(\frac ba\right)^{\frac1n} - 1\right)
    \frac{1 - (\frac ba)^{\frac{n-1}{n} (p+1)}}{1 - \left(\frac ba\right)^{\frac{p+1}{n}}}
    = a^{p+1} \left(1 - \left(\frac ba\right)^{p+1}\right) \underbrace{\lim_{n\to\infty} \frac{\left(\frac ba\right)^{\frac 1n} - 1}{1 - \left(\frac ba\right)^{\frac{p+1}{n}}}}_{\enquote{0/0}}
\] \[
  \lim_{n\to\infty} \frac{\left(\frac ba\right)^{\frac1n} - 1}{1 - \left(\frac ba\right)^{\frac{p+1}{n}}} = \enquote{0/0}
\]
L'H\'opital's Rule:
\[
  = \lim_{n\to\infty} \frac{\exp\left(\frac1n \log\left(\frac ba\right)\right) - 1}{1 - \exp\left(\frac{p+1}{n} \log\left(\frac ba\right)\right)}
  = \lim_{n\to\infty} \frac{\log\left(\frac ba\right) \cdot \frac{-1}{n^2} \exp\left(\frac1n \log\frac{b}{a}\right)}{-(p+1) \log\frac{b}{a} \cdot \frac{-1}{n^2} \exp\left(\frac{p+1}{n} \log\frac ba\right)}
\] \[
  = \lim_{n\to\infty} \frac{-1}{p+1} \frac{\left(\frac ba\right)^\frac 1n}{\left(\frac ba\right)^{\frac{p+1}{n}}}
  = \frac{-1}{p+1}
\]

\[ \Rightarrow = (a^{p+1} - b^{p+1}) \cdot \frac{-1}{p+1} = \frac{b^{p+1} - a^{p+1}}{p+1} \]

The assignment explicitly asks for a step function. This approach only verifies that
\[ \int_a^b x^p \, dx \]
\[ \left.\frac{x^{p+1}}{p+1}\right|_{x=a}^{x=b} = \frac{b^{p+1}}{p+1} - \frac{a^{p+1}}{p+1} \]

We only did the approximation from one side (also upper bound is needed which works analogously):

\[ \sum_{k=0}^{n-1} y_k x_{k+1}^p = \ldots \]

\section{Exercise 16}
\begin{ex}
  For an interval $I \subseteq \mathbb R$ let $f_n: I \to \mathbb R$ be a sequence of functions
  which are uniformly continuous converging towards $f: I \to \mathbb R$. Show that the following
  statements hold or provide a counterexample:
  \begin{itemize}
    \item If all $f_n$ are uniformly continuous, then $f$ is uniformly continuous.
    \item If all $f_n$ are Lipschitz continuous, then $f$ is Lipschitz continuous.
  \end{itemize}
\end{ex}

\subsection{Exercise 16.a}
%
It holds. So a proof is given in the following.

We want to show:
\[
  \forall \varepsilon \exists \delta:
  \abs{x - x_0} < \delta \Rightarrow
  \abs{f(x) - f(x_0)} < \varepsilon
\]

\[ \abs{f(x) - f(x_0)} = \abs{f(x) - f_n(x) + f_n(x) - f(x_0) + f_n(x_0) - f(x_0)} \]
\[
  \leq
    \underbrace{\abs{f(x) - f_n(x)}}_{<\frac\varepsilon3} +
    \underbrace{\abs{f_n(x) - f_n(x_0)}}_{< \frac\varepsilon3} +
    \underbrace{\abs{f_n(x_0) - f(x_0)}}_{< \frac\varepsilon3}
\]

We need to elaborate: For which $n$ does $\frac\varepsilon3$ hold?

\[
  \forall \varepsilon > 0 \exists \overset{\text{depends on $\varepsilon$}}{n_0}:
  \forall n \geq n_0 \forall x \in I:
  \abs{f(x) - f_n(x)} < \frac\varepsilon3
\] \[
  \forall \varepsilon > 0 \forall n \exists \delta = \delta(n, \varepsilon):
  \forall x, x_0: \abs{x - x_0} < \delta \Rightarrow \abs{f_n(x) - f_n(x_0)} < \frac\varepsilon3
\]

\subsection{Exercise 16.b}
%
This does not hold. So we provide a counterexample.

Consider $f(x) = \sqrt{x}$. It is not differentiable at $x=0$, but $f(0) = 0$ is defined.
The function cannot be Lipschitz-continuous, because the Lipschitz constant grows as we tend towards $0$.
We need functions $f_n$.

Consider $f(x) = \sqrt{x + \frac1n}$. The function $f_n$ looks like $f$, but is shifted slightly to the left.
As $n$ tends towards infinity, $f_n$ becomes $f$ and we get the problem at $x=0$.

You can also consider:
\[
  f(x) = \begin{cases}
    \sqrt{x} & \text{ for } x \geq \frac1n \\
    \sqrt{\frac1n} & \text{ for } x < \frac1n
  \end{cases}
\]

\section{Exercise 17}
%
\begin{ex}
  Let $f: [0,1] \to \mathbb R$ be a regulated function continuous in $0$.
  Show the following relation:
  \[
    \lim_{n\to\infty}
    n \int_0^{\frac1n} f(s) \, ds = f(0).
  \]
\end{ex}
\[
  \lim_{n\to\infty}
  n \int_0^{\frac1n} f(s) \, ds = f(0)
  = \lim_{n\to\infty} n \cdot \left(F\left(\frac1n\right) - F(0)\right)
  = \lim_{n\to\infty} \frac{F\left(\frac1n\right) - F(0)}{\frac1n}
  = \lim_{h\to0} \frac{F(h) - F(0)}{h}
  = f(0)
\]

\subsection{Other approach}
Continuity at $x=0$:
\[
  \forall \varepsilon > 0 \exists \delta > 0 \forall \abs{x} < \delta:
  \abs{f(x) - f(0)} < \varepsilon
\] \[
  \lim_{n\to\infty} n \int_0^{\frac1n} f(x) \, dx
  \leq \lim_{n\to\infty} n \int_0^{\frac1n} (f(0) + \varepsilon) \, dx
\]
For $\frac1n < \delta$ it holds that $f(x) < f(0) + \varepsilon$ for $x \in [0,\frac1n]$.
\[
  = \lim_{n\to\infty} n (f(0) + \varepsilon) \frac1n
  = f(0) + \varepsilon
\]
holds for all $\varepsilon > 0$.
\[
  \Rightarrow
  \lim_{n\to\infty} n \int_0^{\frac1n} f(x) \, dx \leq f(0)
\]

\section{Exercise 18}
\begin{ex}
  Prove the Riemann-Lebesgue Lemma:
  For every regulated function $f: [a,b] \to \mathbb R, a < b$ it holds that
  \[ \lim_{\lambda\to\infty} \int_a^b f(x) \sin(\lambda x) \, dx = 0. \]
  \emph{Hint:} Show the following partial results:
  \begin{itemize}
    \item For all intervals $[\alpha, \beta] \subseteq [a,b]$ it holds that
      \[ \lim_{\lambda\to\infty} \int_\alpha^\beta \sin(\lambda x)\, dx = 0. \]
    \item For all step functions $g \in \tau[a,b]$ it holds that
      \[ \lim_{\lambda\to\infty} \int_a^b g(x) \sin(\lambda x) \, dx = 0. \]
  \end{itemize}
\end{ex}

\subsection{Exercise 18.a}
\[
  \left.-\frac1\lambda \cos(\lambda x)\right|_\alpha^\beta
  = \underbrace{\frac1{\lambda}}_{\to 0} \underbrace{\left(-\cos(\beta \lambda) + \cos(\alpha\lambda)\right)}_{\text{bounded}}
\]

% The image 18.pdf illustrates the behavior of sin(\lambda x)

\subsection{Exercise 18.b}
Because $g$ is a step function of $[a,b]$, there exists a decomposition
\[ a = x_0 < x_1 < \ldots < x_n = b \]
such that $g(x)$ has a constant value $c_i$ in every subinterval $[x_{i-1}, x_i)$.
\[
  \lim_{\lambda\to\infty} \int_a^b f(x) \sin(\lambda x) \, dx
  = \lim_{\lambda\to\infty} \sum_{i=1}^n c_i \int_{x_{i-1}}^{x_i} \sin(\lambda x) \, dx
\]
This can be done, because we consider a finite sum.
\[
  \sum_{i=1}^n c_i \underbrace{\int_{x_{i-1}}^{x_i} \sin(\lambda x) \, dx}_{\to 0 \forall \text{ subintervals } H(i)}
\] \[
  = \sum_{i=1}^n c_i \cdot \underbrace{\lim \int_{x_{i-1}}^{x_i} \sin(\lambda x) \, dx = 0}_{\to 0}
\]

\subsection{Conclusion}
Because $f(x)$ is a regulated function $\forall \varepsilon > 0$, there exists a step function
$g_\varepsilon(x)$ with $\abs{f(x) - g_{\varepsilon}(x)} < \varepsilon \quad \forall x \in [a,b]$.

\[
  \abs{\int_a^b f(x) \cdot \sin(\lambda x) \, dx}
  \leq \underbrace{\int_a^b \abs{\underbrace{f(x) - g_\varepsilon(x)}_{< \varepsilon}}}_{< \varepsilon (b - a)} \cdot \abs{\underbrace{\sin(\lambda x)}_{\leq 1}} \, dx
  + \underbrace{\abs{\int_a^b g_\varepsilon(x) \sin(\lambda x) \, dx}}_{\to 0 \text{ for } \lambda \to \infty}
\] \[
  \lim_{\lambda\to\infty} \abs{\int_a^b f(x) \sin(\lambda x) \, dx} \leq \varepsilon (b - a)
\]
We can choose $\varepsilon$ arbitrary, so it must tend towards $0$.

\section{Exercise 19}
\begin{ex}
  Let $I, J \subseteq \mathbb R$ be intervals, $f: I \to \mathbb R$ continuous and $g,h: J \to I$ differentiable.
  Furthermore it holds that $g \leq h$ in $J$. Prove that
  \[ A: J \to \mathbb R, \quad x \mapsto \int_{g(x)}^{h(x)} f(\xi) \, d\xi \]
  is differentiable and determine its derivative.
\end{ex}

\subsection{Exercise 19.a}
Show differentiability.

So
\[ \lim_{x\to x_0} \frac{A(x) - A(x_0)}{x - x_0} \]
exists.
\[
  \lim_{x\to x_0} \frac{A(x) - A(x_0)}{x - x_0}
  = \lim_{x\to x_0} \frac{\int_{g(x)}^{h(x)} f(\xi) \, d\xi - \int_{g(x_0)}^{h(x_0)} f(\xi) \, d\xi}{x - x_0}
  = \lim_{x\to x_0} \frac{F(h(x)) - F(g(x)) - F(h(x_0)) + F(g(x_0))}{x - x_0}
\]
$F(h(x))$ and $F(g(x))$ exists, because $h(x)$ is continuous, so a regulated function and regulated functions always have a primitive function.
\[
  \lim_{x\to x_0} \frac{F(h(x)) - F(h(x_0))}{x - x_0} -
  \lim_{x\to x_0} \frac{F(g(x)) - F(g(x_0))}{x - x_0}
\]
If $h(x)$ is continuous, then $F(h(x))$ is differentiable (analogously for $g(x)$).
And the composition is also differentiable.

\subsection{Exercise 19.b}
Determine its derivative.

\[
  (F \circ h)' (x) - (F \circ g)'(x_0)
  = f(h(x_0)) \cdot h'(x_0) - f(g(x_0)) \cdot g'(x_0)
\]

\section{Exercise 20}
\begin{ex}
  Determine the following integrals for arbitrary $a,b \in \mathbb R, a < b$:
  \begin{itemize}
    \item $\int_a^b \frac{d}{dx} \left(x^5 \cdot e^x\right) \, dx$
    \item $\int_a^b x^4 e^{x^5} \, dx$
  \end{itemize}
\end{ex}

\subsection{Exercise 20.a}
\[
  \int_a^b \frac{d}{dx} \left(x^5 e^{x}\right) \, dx
  = \int_a^b 5x^4 e^x - x^5 e^x \, dx
  = \int_a^b \underbrace{e^x}_{g'(x)} \underbrace{(5 x^4 - x^5)}_{= f(x)} \, dx
\] \[
  = \left.e^x (5x^4 - x^5) \right|_a^b - \int_a^b e^x (20 x^3 + 5x^4)
  = e^b b^5 - e^a a^5
\]

\subsection{Exercise 20.b}
\[
  \int_a^b x^4 e^{x^5} \, dx
\] \[
  u \coloneqq x^5 \Rightarrow \frac{du}{dx} = 5x^4 \qquad dx = \frac{du}{5x^4}
\] \[
  = \int_{a^5}^{b^5} x^4 e^u \, \frac{du}{5x^4}
  = \int_{a^5}^{b^5} e^u \, \frac{du}{5}
  = \frac15 \int_{a^5}^{b^5} e^u \, du
  = \frac15 \left(e^{b^5} - e^{a^5}\right)
\]

Other approach for 20.b:
\[ F = \frac15 e^{x^5} \]
\[ F' = x^4 e^{x^5} = f \]

\section{Exercise 21}
\begin{ex}
  Consider function $f$.
  \[
    f: [-1, 1] \to \mathbb R,
    \quad
    x \mapsto \begin{cases}
      0,              & x = 0, \\
      \frac{1}{n+2},  & x \in \left[-\frac1n, -\frac1{n+1}\right) \cup \left(\frac1{n+1}, \frac1n\right], n \in \mathbb N_+.
    \end{cases}
  \]
  Is $f$ a step function? Is $f$ a regulated function? Furthermore determine
  \[ \int_{-1}^1 f(x) \, dx. \]
\end{ex}

Is not a step function, because the number of intervals is not finite.

Is it a regulated function? We can approximate $f$ using the following construction:
\[
  \varphi_k(x) = \begin{cases}
    0 & x = 0 \\
    \frac{1}{n+2}  & x \in \left[-\frac1n, -\frac1{n+1}\right) \cup \left(\frac1{n+1}, \frac1n\right], n \in \mathbb N, n \leq k \\
    0 & \text{else}
  \end{cases}
\]
We choose a $k$ such that all elements smaller $k$ are nonzero.
This approximates our function $f$.

Consider
\[
  \int_{-1}^1 f(x) \, dx \overset{\varphi_n \to f \text{ uniformly}}{=}
  \lim_{h\to\infty} \int_{-1}^1 \varphi_n(x) \, dx
\]
\[
  \int_{-1}^1 \varphi_n(x) \, dx
  = \sum_{j=1}^N c_j \triangle x_j
  = \sum_{n=1}^k \frac{1}{n+2} \cdot \abs{\frac1n - \frac1{n+1}} \cdot 2
  = 2 \cdot \sum_{n=1}^k \left(\frac{1}{n (n + 2)} - \frac{1}{(n + 2) (n + 1)}\right)
\] \[
  = 2 \cdot \sum_{n=1}^k \left(\frac12 \left(\frac1n - \frac1{n+2}\right) - \left(\frac1{n + 1} - \frac1{n+2}\right)\right)
\]
Alternatively we can also split it up, because we estimate that a series with $\frac1{n^2}$ converges.
\[
  = 2 \sum_{n=1}^k \frac12 \left(\frac1n - \frac1{n+2}\right)
  - 2 \sum_{n=1}^k \left(\frac{1}{n+1} - \frac{1}{n+2}\right)
  = 1 + \frac12 - 2 \cdot \frac12 = \frac12
\]
This expression is easier to evaluate as telescoping sum.

If we take the first approach, we need to apply partial fraction decomposition.
\[
  \frac12 \int_k = \frac12 \int_{-1}^1 \varphi_k(x) \, dx
  = \frac34 - \frac12 + \frac12 \left(\underbrace{\frac1{k+1}}_{\to 0} + \underbrace{\frac1{k+2}}_{\to 0}\right) - \frac{1}{k+1} \overset{h\to\infty}{\longrightarrow} \frac34 - \frac12 = \frac14
\]

\section{Exercise 22}
\begin{ex}
  Let $I \subseteq \mathbb R$ be an interval. Determine with the idea from below
  a primitive function of
  \[ f: I \to \mathbb R, \qquad x \mapsto x^2 \sin{x^3} \cos{x^3}. \]
  \begin{itemize}
    \item For all $x \in \mathbb R$ it holds that
      \[ \sin{x} \cos{x} = \frac12 \sin(2x). \]
    \item For all $x \in \mathbb R$ it holds that
      \[ \sin{x} \cos{x} = \frac12 \frac{d}{dx} \sin^2(x). \]
  \end{itemize}
  Explain possible differences between the results.
\end{ex}

\subsection{Exercise 22.a}
%
\[
  \int x^2 \sin(x^3) \cos(x^3) \, dx
  = \int x^2 \frac12 \sin(2x^3) \, dx
\]
Substitute with $u = 2x^3$ and $dx = \frac{du}{6x^2}$.
\[
  = \int x^2 \frac12 \sin(u) \frac{du}{6x^2}
  = - \frac1{12} \cos(u) + c
  = - \frac1{12} \cos(2x^3) + c
\]

\subsection{Exercise 22.b}
%
\[ \forall x \in \mathbb R: \sin(x) \cos(x) = \frac12 \frac{d}{dx} \sin^2(x) \]
\[
  \forall x \in \mathbb R:
    \sin(x^3) \cos(x^3)
      = \frac{1}{6x^2} \frac{d}{dx} \sin^2(x^3)
      = \frac1{6x^2} 2 \sin(x^3) \cos(x^3) 3x^2
      = \sin(x^3) \cos(x^3)
\]

\[
  \int x^2 \sin(x^3) \cos(x^3) \, dx =
  \int x^2 \frac{1}{6x^2} \frac{d}{dx} \sin^2(x^3) \, dx
  = \frac16 \sin^2(x^3) + c
\]

\subsection{Exercise 22.c}

\[
  \cos(2x^3) = \cos^2(x^3) - \sin^2(x^3)
    = 1 - \sin^2(x^3) - \sin^2(x^3)
    = 1 - 2 \sin^2(x^3)
\] \[
  \Rightarrow \frac16 \sin^2(x^3) + \tilde c
\]
with $\tilde c \approx \frac1{12} + c$.

\section{Exercise 23}
\begin{ex}
  Determine the following integrals using integration by parts.
  \begin{enumerate}
    \item $\int e^x \sin{x} \, dx$
    \item $\int \arcsin{x} \, dx$
    \item $\int_0^1 x^2 \ln^3(x) \, dx$
  \end{enumerate}
\end{ex}

\subsection{Exercise 23.a}
\[ \int \underset{u}{e^x} \underset{v'}{\sin(x)} \, dx \]
with $v = -\cos{x}$ and $u' = e^x$.
\[ = e^x (-\cos{x}) - \int \underset{u}{e^x} \cdot \underset{v'}{(-\cos{x})} \, dx \]
with $u' = e^x$ and $v = -\sin{x}$.
\[ = e^x (-\cos{x}) - \left(e^x \cdot (-\sin{x})\right) + \int e^x (-\sin{x}) \, dx \]
\[ = e^x \left(-\cos{x} + \sin{x}\right) - \int e^x \sin{x} \, dx \]

\[ \Rightarrow 2 \int e^x \sin(x) \, dx = e^x \left(-\cos(x) + \sin(x)\right) + c \]

\subsection{Exercise 23.b}
\[
  \int \arcsin(x) \, dx = \int \underset{v}{\arcsin(x)} \cdot \frac{v'}{1} \, dx
\]
with $v = x$ and $u' = \frac{1}{\sqrt{1 - x^2}}$.
\[ = \arcsin(x) \cdot x - \int x \frac{1}{\sqrt{1 - x^2}} \, dx \]
Let $t = 1 - x^2$. Hence $\frac{dt}{dx} = -2x$.
\[ = \arcsin(x) \cdot x - \int \frac{x}{\sqrt{t}} \cdot \frac{1}{-2x} \, dt \]
\[ = \arcsin(x) \cdot x + \frac12 \int \frac{1}{\sqrt{t}} \, dt \]
\[ = \arcsin(x) \cdot x + \frac12 \cdot 2 \sqrt{t} + c \]
Backsubstitution:
\[ = \arcsin(x) \cdot x + \sqrt{1 - x^2} + c \]

\subsection{Exercise 23.c}
\[
  \int_0^1 \underbrace{x^2}_{f'} \underbrace{\left(\ln{x}\right)^3}_{g} \, dx
  = \left.\frac13 x^3 \left(\ln{x}\right)^3 \right|_0^1 - \int_0^1 \frac13 x^3 \left(\ln{x}\right)^2 \cdot 3 \frac1x \, dx
\] \[
  = \frac13 \cdot 0 - \frac13 \cdot \left(
    \lim_{x\to 0} (x^3 \ln^3{x})
    - \int_0^1 \underbrace{x^2}_{f'} \underbrace{\ln^2(x)}_{g} \, dx
  \right)
\]
In the end, we can apply L'H\^opital's Rule once we have expressions like $-\frac13 \varphi^3 \frac{\ln^3{\varphi}}{\ln^3{\varphi}}$.
\[
  = \frac23 \left(- \int_0^1 \frac13 x^2 \, dx\right)
  = \left. -\frac29 \cdot \frac13 x^3 \right|_0^1 = -\frac{2}{27}
\]

\section{Exercise 24}
\begin{ex}
  Determine the following integrals using appropriate substitions:
  \begin{enumerate}
    \item $\int \frac{\cos^3(x)}{1 - \sin(x)} \, dx$.
    \item $\int \frac{dx}{\sin^2(x) \cos^4(x)} \, dx$ using $t \coloneqq \tan{x}$.
    \item $\int_0^{\frac12} \frac{x^2}{\sqrt{1 - x^2}} \, dx$ using $t \coloneqq \arcsin(x)$.
  \end{enumerate}
\end{ex}

\subsection{Exercise 24.a}
\[
  \frac{\cos^3(x)}{1 - \sin(x)} \, dx
  = \int \frac{\cos(x)\left(1 - \sin^2(x)\right)}{1 - \sin(x)} \, dx
  = \int \cos'(x) \left(1 = \sin{x}\right) \, dx
\]
with $u = 1 + \sin{x}$ and $\frac{du}{dx} = \cos{x}$ we get
\[
   = \int \cos{x} \cdot u \cdot \frac{du}{\cos{x}}
   = \frac12 u^2 + c = \frac12 \left(1 + \sin{x}\right)^2 + c
\]

Do not forget $c$ for indefinite integrals!

\subsection{Exercise 24.b}
\[
  \int \frac{1}{\sin^2(x) \cdot \cos^4(x)} \, dx
\] \[
  = \int \frac{\sin^4(x) + 2 \sin^2(x) \cos^2(x) + \cos^4(x)}{\sin^2(x) \cos^4(x)} \, dx
\] \[
  = \int \frac{\sin^2(x)}{\cos^4(x)} + \frac{2}{\cos^2(x)} + \frac{1}{\sin^2(x)} \, dx
\] \[
  = \int \frac{\tan^2(x) + 2}{\cos^2(x)} \, dx + \int \frac{1}{\sin^2(x)} \, dx
\]

Consider the left-handed expression.
Consider $t = \tan(x)$ and $\frac{dt}{dx} = \frac{1}{\cos^2(x)}$.
\[
  \int \frac{t^2 + 2}{\cos^2(x)} \cdot \cos^2(x) \, dt
    = \int t^2  + 2 \, dt = \frac13 t^3 + 2t + c
    = \frac13 \tan^3(x) + 2 \tan(x) + c_1
\]

Consider the right-handed expression.
\[
  \int \frac{1}{\sin^2(x)} \, dx
    = \int \frac{\sin^2(x) + \cos^2(x)}{\sin^2(x)} \, dx
    = \int -\frac{(\cos{x})' \sin{x} - \cos{x} \cdot (\sin(x))'}{\sin^2(x)} \, dx
\] \[
  = -\int \left(\frac{\cos{x}}{\sin{x}}\right)' \, dx = -\frac{\cos{x}}{\sin{x}}  + c_2
\]

So for the overall expression it holds that
\[
  \int \frac{\tan^2(x) + 2}{\cos^2(x)} \, dx + \int \frac{1}{\sin^2(x)} \, dx
  = \frac13 \tan^3(x) + 2\tan(x) - \frac{1}{\tan(x)} + c_3
\]

\subsection{Exercise 24.b: Alternative approach}
\[ \int \frac{1}{s^2 c^4} \, dx \]
with $s = \sin(x)$, $c = \cos(x)$, $t = \tan(x)$ and $\frac{dt}{dx} = \frac{1}{c^2}$.
It holds that
\[
  t^2 = \frac{s^2}{c^2} = \frac{1 - c^2}{c^2} = \frac{1}{c^2} - 1
\] \[
  c^2 = \frac{1}{1 + t^2}
\] \[
  t^2 = \frac{s^2}{c^2} = \frac{s^2}{1 - s^2} = -1 + \frac{1}{1 - s^2}
\] \[
  1 - s^2 = \frac{1}{1 + t^2}
\] \[
  s^2 = 1 - \frac{1}{1 + t^2} = \frac{t^2}{1 + t^2}
\]

\[
  \int \frac{1}{s^2 c^4} \, dx
     = \int \frac{1}{s^2 c^2} \, dt
     = \int \frac{1 + t^2}{t^2}(1 + t^2) \, dt
\]

\subsection{Exercise 24.c}
\[
  \int_0^{\frac12} \frac{x^2}{\sqrt{1 - x^2}^3} \, dx
  = \int_0^{\frac12} \frac{x^2 + 1 - 1}{\sqrt{1 - x^2}^3} \, dx
  = - \int_0^{\frac12} \frac{1}{\sqrt{1 - x^2}} \, dx
  + \int_0^{\frac12} \frac{1}{\sqrt{1 - x^2}^3} \, dx
\]
\[
  = \left. -\arcsin(x) \right|_0^{\frac12}
  + \int_0^{\frac12} \frac{1}{\sqrt{1 - x^2}^3} \, dx
\]
with $t = \arcsin(x)$ and $\frac{dx}{dt} = \cos(t)$ with $x = \sin(t)$.
Also $\sqrt{1 - x^2}^3 = \sqrt{\cos^2(t)}^3 = \cos(t)^3$.

Recognize that $x$ must be positive for this to hold. As it turns out,
this is fine within the interval $(0, \frac12)$.

\[
  = \left. -\arcsin(x) \right|_0^{\frac12}
  + \int_{x=0}^{x = \frac12} \frac{1}{\cos^3(t)} \cdot \cos(t) \, dt
\] \[
  = \left. -\arcsin(x) \right|_0^{\frac12}
  + \int_{x=0}^{x = \frac12} \frac{1}{\cos^2(t)} \, dt
\] \[
  = \left. -\arcsin(x) \right|_0^{\frac12}
  + \left. \tan(t) \right|_{x=0}^{x=\frac12}
\] \[
  = \left. -\arcsin(x) \right|_0^{\frac12}
  + \left. \tan(\arcsin(x))  \right|_0^{\frac12}
\] \[
  = -\arcsin\left(\frac12\right) + \arcsin(0) + \tan\left(\frac\pi6\right) - \tan(0)
\] \[
  = -\frac\pi6 + 0 + \frac{\frac12}{\sqrt{\frac34}} - 0
\] \[
  = -\frac\pi6 + 0 + \frac1{\sqrt{3}} - 0
\]

\section{Exercise 25}
\begin{ex}
  Determine
  \begin{itemize}
    \item $\int \frac{\sin{x}}{\sin{x} + \cos{x}} \, dx$.
    \item $\int_{\frac\pi4}^{\frac\pi3} \frac{\sqrt{\tan{x}}}{\cos^2(x)} \, dx$.
  \end{itemize}
\end{ex}

\subsection{Exercise 25.a}
\[
  \int \frac{\sin{x}}{\sin{x} + \cos{x}}
    = \frac{\frac12 \left(\sin{x} + \cos{x}\right) - \frac12 \left(-\sin{x} + \cos{x}\right)}{\sin{x} + \cos{x}} \, dx
\] \[
  = \int \left(\frac12 + \frac{\sin{x} - \cos{x}}{2 (\sin{x} + \cos{x})}\right) \, dx
\]
With $u = \sin{x} + \cos{x}$ and $dx = \frac{du}{\cos(x) - \sin{x}}$, we get
\[ = \frac12 x + \frac12 \cdot \int -\frac{1}{u} \, du \]
\[ = \frac x2 - \frac12 \cdot \ln(\sin{x} + \cos{x}) + c \]

\section{Exercise 25.b}
\[
  \int_{\frac\pi4}^{\frac\pi3} \frac{\sqrt{\tan{x}}}{\cos^2(x)} \, dx
\]
Consider $u = \tan{x}$ and $dx = du \, \cos^2{x}$.
\[
  = \int_{x=\frac{\pi}{4}}^{x = \frac\pi3} \sqrt{u} \, du
  = \left. \frac23 (\tan(x))^{\frac32} \right|_{\frac\pi4}^{\frac\pi3}
\] \[
  = \frac23 \left(3^{\frac34} - 1\right)
\]

\section{Exercise 27}
\begin{ex}
  Investigate the following impropert integrals for convergence.
  \begin{enumerate}
    \item $\int_1^\infty \frac{x^2}{2x^4 - x + 1} \, dx$.
    \item $\int_0^\infty x^\alpha e^{-x} \, dx$, $\alpha \in \mathbb R$.
    \item $\int_0^\infty \frac{\sqrt{x}}{(1 + x)^2} \, dx$.
    \item $\int_0^\infty \left(\frac{\pi}{2} - \arctan(x)\right) \, dx$.
  \end{enumerate}
\end{ex}

\subsection{Exercise 27.a}
\[ \int_1^\infty \frac{x^2}{2x^4 - x + 1} \, dx \]
\[ \frac{x^2}{2x^4 - x + 1} = \frac{x^2}{x (2x^3 - 1) + 1} < \frac{x}{2x^3 - 1} \leq \frac{x}{x^3} = \frac{1}{x^2} \]
\[ \int_1^b \frac{x^2}{2x^4 - x + 1} \, dx < \int_1^b \frac1{x^2} \, dx = \left[-\frac1x\right]_1^b = -\frac1b + 1 < 1 \]

In general: Approximately $\int_1^\infty \frac{x^2}{2x^4 - x + 1}$ converges becomes it looks close to $\int_1^\infty \frac{x^2}{2x^4} \, dx$.

\subsection{Exercise 27.b}
\[ \int_1^\infty \frac{x^\alpha}{e^x} \, dx \qquad \alpha \in \mathbb R \]
\[ \frac{x^\alpha}{e^x} = \frac{e^{\alpha \cdot \ln{x}}}{e^x} = e^{\alpha \ln(x) - x} \]
\[ e^x = \sum_{k=0}^\infty \frac{x^k}{k!} > \sum_{k=l}^\infty \frac{x^k}{k!} \quad l \geq \alpha \]
\[ x \geq 1: \frac{x^\alpha}{e^x} \leq \frac{x^l}{e^x} = \frac{x^l}{\sum_{k=0} \frac{x^k}{k!}} < \frac{x^l}{\frac{x^{l+2}}{(l+2)!}} = \frac{(l+2)!}{x^2} \]
\[ \int_1^b \frac{x^\alpha}{e^x} \, dx < \int_1^b \frac{(l+2)!}{x^2} \, dx < (l + 2)! \]

\subsection{Exercise 27.c}
\[ \int_0^\infty \frac{\sqrt{x}}{(1 + x)^2} \, dx = \int_0^1 \frac{\sqrt{x}}{(1 + x)^2} \, dx + \int_1^\infty \frac{\sqrt{x}}{(1 + x)^2} \, dx \]
\[ x \geq 1: \frac{\sqrt{x}}{(1 + x)^2} < \frac{\sqrt{x}}{x^2} = x^{-\frac32} \]
\[ \int_1^b \frac{\sqrt{x}}{(1 + x)^2} \, dx < \int_1^b x^{-\frac32} \, dx = 2 - \frac{2}{\sqrt{b}} < 2 \]
\[ 0 \leq x \leq 1: \frac{\sqrt{x}}{(1 + x)^2} < \frac{\sqrt{x}}{1} = \sqrt{x} < 1 \]

\subsection{Exercise 27.d}
\[ \int_0^\infty \left(\frac\pi2 - \arctan(x)\right) \, dx \]
\[ \int \arctan(x) \cdot 1 \, dx = x \cdot \arctan(x) - \int \frac{x}{1 + x^2} \, dx \]
Integration by substitution:
\[ t = 1 + x^2 \qquad \frac{dt}{dx} = 2x \Rightarrow dx = \frac{dt}{2x} \]
\[ = x \cdot \arctan(x) - \int \frac{* \, dx}{t \cdot 2*} = x \cdot \arctan(x) - \frac{1}{2} \ln(1 + x^2) + c \]
\[ \Rightarrow \int_0^b (x - \arctan(x)) \, dx = \left.\frac\pi2 - x \cdot \arctan(x) + \frac12 \ln(1 + x^2) \right|_0^b \]
\[
  = \underbrace{b}_{>0} \underbrace{\left(\frac\pi2 - \arctan(b)\right)}_{>0} + \frac12 \ln(1 + b^2)
  > \frac12 \ln(1 + b^2) > M
\]

\subsection{Remark by the tutor}
\[ \int_1^\infty \frac{1}{x^c} \, dx \text{ converges} \iff c > 1 \]
\[
  \lim_{x\to\infty} \frac{\frac\pi2 - \arctan(x)}{\frac1x}
  \overset{\text{L'H\^opital}}{=}
  \lim_{x\to\infty} \frac{-\frac{1}{1 + x^2}}{-\frac{1}{x^2}} = -1
\] \[
  \exists x_0: \frac{\pi}{2} - \arctan(x) > \frac12 \cdot \frac1x
  \quad \forall x \geq x_0
\] \[
  \int_0^\infty \frac{\pi}{2} - \arctan(x) \, dx \geq \int_{x_0}^\infty \frac12 \frac1x \, dx
\]

\section{Exercise 28}
\begin{ex}
  Find all primitive functions of $f: (-1, 1) \to \mathbb R$ with $x \mapsto \frac{1}{1 - x^4}$ using partial fraction decomposition.

  Hint: To derive the partial fraction decomposition use
  \[ \frac{1}{(1 - x)(1 + x)(1 + x^2)} = \frac{a}{1 - x} + \frac{b}{1 + x} + \frac{cx + d}{1 + x^2} \]
  with constants $a,b,c,d \in \mathbb R$. Determine the values for $a,b,c,d$.
\end{ex}

\[ \int \frac{1}{1 - x^4} \, dx = \int \frac{1}{4(1 - x)} \, dx + \int \frac{1}{4(1 + x)} \, dx + \int \frac{1}{2(1 + x^2)} \, dx \]

The first resulting integrals are:
\[ \frac14 \int \frac{1}{1 - x} \, dx = -\frac14 \ln{1 - x} + c \]
\[ \frac14 \int \frac{1}{1 + x} \, dx = \frac14 \ln(1 + x) + c \]
\[ \frac12 \int \frac{1}{1 + x^2} \, dx = \frac12 \arctan(x) + c \]

\[ \int \frac{1}{1 - x^4} \, dx = -\frac14 \ln(1 - x) + \frac14 \ln(1 + x) + \frac{1}{2} \arctan(x) + c \]
\[ \frac14 \ln\left(\frac{1 + x}{1 - x}\right) + \frac12 \arctan(x) + c \]

\section{Exercise 29}
\begin{ex}
  Given a function $f: (0, \infty) \to \mathbb R$ with $x \mapsto \frac{1}{x + \sqrt{x}}$.
  \begin{itemize}
    \item Determine the Taylor polynomial of second degree $T_f^2(x;1)$ of function $f$ in point $x_0 = 1$.
    \item Determine an upper bound for the error $\abs{f(x) - T_f^2(x;1)}$ in interval $[1;2]$.
  \end{itemize}
\end{ex}

\subsection{Exercise 29.a}
\[ f'(x) = -\frac{1 + 2\sqrt{x}}{2 \cdot x^{\frac32} (1 + \sqrt{x})^2} \]
\[ f''(x) = \frac{3 + 9\sqrt{x} + 8x}{4 \cdot x^{\frac52} \cdot (1 + \sqrt{x})^3} \]

\[ T_f^2(x;1) = f(1) + \frac{f'(1)}{1!} (x - 1)^1 + \frac{f^{(2)}(1)}{2!} \cdot (x - 1)^2 \]
\[ T_f^2(x;1) = \frac12 + \frac38 (x - 1) + \frac{20}{64} (x - 1)^2 \]

\subsection{Exercise 29.b}
\[ R_f^n(x;a) = \frac{f^{(n+1)}(\xi)}{(n+1)!} \cdot (x - x_0)^{n+1} \qquad \xi \in [x_0, x] \]
\[ f'''(x) = - \frac{3 \cdot (16 x^{\frac32} + 29x + 20\sqrt{x} + 5}{8 \cdot (\sqrt{x} + 1)^4 \cdot x^{\frac52}} \]
\[ \abs{R_f^n(x;1)} = \frac{(x - 1)^3}{(3)!} \]

Upper bound (not the best, but works):
\[ \frac{3 \cdot 65 x^{\frac32} + 15}{8 \cdot 16 \cdot x^{\frac52}} \leq \frac{3 \cdot 65}{8 \cdot 16x} + \frac{15}{8 \cdot 16x^{\frac52}} \]

$f'''(x)$ is monotonically increasing and we look at the interval $[1,2]$.
So we are closest to $0$, if $x=1$.

\section{Exercise 30}
\begin{ex}
  Let $g: \mathbb R \to \mathbb R$, $x \mapsto \sin(2x)$.
  \begin{enumerate}
    \item For arbitrary $n \in \mathbb N$, determine the Taylor polynomial of $n$-th degree
      $T_g^n(x;0)$ of $g$ in $x_0 = 0$.
    \item Give a Taylor polynomial $T_g^n(x; 0)$ such that $\abs{g(x) - T_g^n(x; 0)} < 10^{-6}$
      holds for $[-\pi,\pi]$.
  \end{enumerate}
\end{ex}

\subsection{Exercise 30.a}

\begin{align*}
  g(x) &= \sin(2x) \\
  g^{(1)}(x) &= \cos(2x) \cdot 2 \\
  g^{(2)}(x) &= -\sin(2x) \cdot 2^2 \\
  g^{(3)}(x) &= -\cos(2x) \cdot 2^3 \\
  \ldots & \ldots
\end{align*}

\begin{align*}
  T_g^n(x;0) &= g(0) + \frac{g^{(1)}(0) (x - 0)}{1!} + \ldots \\
    &= \underbrace{\sin(0)}_{=0} + \underbrace{\cos(0) \cdot 2x}_{=2x} + \frac{-\sin(0) \cdot 2^2 x^2}{2} + \frac{-\cos(0) \cdot 2^3 x^3}{3!} = -\frac{2^3 x^3}{3!} + \ldots \\
  T_g^n(x;0) &= \sum_{k=0}^m (-1)^k \cdot \left(\frac{(2x)^{2k+1}}{(2k + 1)!}\right)
    \qquad \text{ s.t. } 2m + 1 \leq n, 2m + 3 \geq n
\end{align*}

Even easier: Consider the power series for $\sin$:
\[ \sin(x) = x - \frac{x^3}{3!} + \frac{x^5}{5!} - \ldots \]
\[ \sin(2x) = (2x) - \frac{(2x)^3}{3!} + \frac{(2x)^5}{5!} - \ldots \]

\subsection{Exercise 30.b}
%
\[ R_g^n(x; 0) = \abs{\frac{f^{(n+1)}(\xi)(x - 0)^{n+1}}{(n+1)!}} < 10^{-6} \]
with $x, \xi \in [-\pi, \pi]$. We look at the following approximation ($\sin(x)$ and $\cos(x)$ is at most 1 and the factor $2^{n+1}$ of the derivative remains).
Choose $\xi$ such that $\abs{f^{(n+1)}(\xi)} \leq 2^{n+1}$.
\[
  R_g^n(x;0) \leq \abs{\frac{2^{n+1} x^{n+1}}{(n+1)!}} < 10^{-6}
  \iff
  2^{n+1} x^{n+1} 10^{6} < (n+1)!
  \iff
  10^{6} < \frac{(n+1)!}{\abs{2^{n+1} x^{n+1}}} < (n+1)!
\]
In the worst case, $x$ is very large. The largest value it reaches is $\pi$. Hence,
\[
  \abs{R_g^n} \leq \frac{1}{(n+1)!} \cdot 2^{n+1} \cdot \pi^{n+1} \leq 10^{-6}
  \qquad \forall x \in [-\pi, \pi]
\]
This holds if $n \geq 26$.

\section{Exercise 26}
\begin{ex}
  Prove the following limit criterion for improper integrals.
  Let $a \in \mathbb R$ and $f,g: [a,\infty) \to \mathbb R$ functions,
  which satisfy $f \geq 0$ and $g > 0$ in $[a, \infty)$.
  Furthermore the following limit exists:
  \[ L \coloneqq \lim_{x\to\infty} \frac{f(x)}{g(x)} \in [0,\infty]. \]
  Then it holds that,
  \begin{enumerate}
    \item $L = 0 \Rightarrow \left[\int_a^\infty g(x) \, dx < \infty \Rightarrow \int_a^\infty f(x) \, dx < \infty \right]$.
    \item $L \in (0,\infty) \Rightarrow \left[
        \int_a^\infty g(x) \, dx < \infty \Leftrightarrow \int_a^\infty f(x) \, dx < \infty
      \right]$.
    \item $L = \infty \Rightarrow \left[\int_a^\infty g(x) \, dx \text{ diverges } \Rightarrow \int_a^\infty f(x) \, dx \text{ diverges}\right]$.
  \end{enumerate}
\end{ex}

\subsection{Exercise 26.a}
We provide a counterexample:
\[
  f(x) = \begin{cases}
    \frac1x & 0 < x < 1 \\
    0 & x = 0 \\
    \frac1{x^2} & x > 1
  \end{cases}
\] \[
  g(x) = \frac{1}{x^2 + 1}
\] \[
  a = 0
\]

To make this proposition work, $f$ must be continuous or even boundedness should suffice.
\[
  \forall \varepsilon > 0 \exists x_0 \forall x \geq x_0:
  \abs{\frac{f(x)}{g(x)}} < \varepsilon
\]
Both functions yield positive values:
\[
  \forall \varepsilon > 0 \exists x_0 \forall x \geq x_0:
  \frac{f(x)}{g(x)} < \varepsilon
\]

\[ \int_a^\infty f(x) \, dx = \int_a^{x_0} f(x) \, dx + \int_{x_0}^\infty f(x) \, dx \]

\[
  \forall \varepsilon > 0 \exists x_0 \forall x \geq x_0:
  \frac{f(x)}{g(x)} < \varepsilon
  \iff
  f(x) < \varepsilon \cdot g(x)
  \implies
  \int_{x_0}^\infty f(x) \, dx < \varepsilon \int^{\infty} g(x) \, dx
\]

Because of boundedness we can provide the following estimates:
\[ \int_a^\infty f(x) \, dx = \underbrace{\int_a^{x_0} f(x) \, dx}_{<\infty} + \underbrace{\int_{x_0}^\infty f(x) \, dx}_{< \infty} \]
\[ \implies < \infty \]

\subsection{Exercise 26.b}
\[
  \forall \varepsilon > 0
  \exists x_0
  \forall x \geq x_0:
  \abs{\frac{f(x)}{g(x)} - L} < \varepsilon
\] \[
  \iff L - \varepsilon < \frac{f(x)}{g(x)} < L + \varepsilon
\] \[
  \iff (L - \varepsilon) \cdot g(x) < f(x) < (L + \varepsilon) \cdot g(x)
\] \[
  (L - \varepsilon) \int_{x_0}^\infty g(x) \, dx
  \leq \int_0^\infty f(x) \, dx
  \leq (L + \varepsilon) \int_{x_0}^\infty g(x)
\]

\subsection{Exercise 27.c}
\[
  \forall n \exists x_0 \forall x \geq x_0:
  \frac{f(x)}{g(x)} > n
  \iff f(x) > n \cdot g(x)
  \implies \int_{x_0}^\infty f(x) \, dx \geq n \cdot \int_{x_0}^\infty g(x)
\]

\section{Exercise 31}
\begin{ex}
  Let $I \subseteq \mathbb R$ be an interval, $a \in I$, $n \in \mathbb N$
  and $f: I \to \mathbb R$ is differentiable $n$-times.
  Show: If a polynomial
  \[ P(x) = \sum_{k=0}^{n} a_k(x - a)^k \]
  with $a_k \in \mathbb R$, $0 \leq k \leq n$ satisfies
  \[ \lim_{x\to a} \frac{f(x) - P(x)}{(x - a)^n} = 0 \]
  then $P(x) = T_f^n(x;a)$.
\end{ex}

\[
T_f^n(x;a) = f(a) + \frac{f'(a)(x - a)}{1!} + \frac{f''(a)(x - a)^2}{2!} + \ldots + f^{(n)}(a)(x - a)^n}{n!}
= \sum_{k=0}^n \frac{f^{(k)}(a)}{k!} (x - a)^k
\]

How do derivatives of $P$ look like?
\[ P^{(1})(x) = \sum_{k=0}^n k a_k (x - a)^{k-1} \]
\[ P^{(2)}(x) = \sum_{k=0}^n a_k k (k - 1) (x - a)^{k-2} \]
\[ P^{(i)}(x) = \sum_{k=0}^n k (k-1) \cdot \ldots \cdot (k - i + 1) a_k (x - a)^{k-i}  \]
\[ P^{(i)}(a) = a_k \cdot \underbrace{k (k-1) (k-2) \cdot \ldots \cdot (1)}_{=k!} \]
\[ a_k = \frac{P^{(k)}(a)}{k!} \]
\[ P(x) = \sum_{k=0}^n a_k (x - a)^k = \sum_{k=0}^n \frac{P^{(k)}(a)}{k!} (x - a)^k \]

For $k=0$:
\[ P^{(0)}(a) = P(a) = a_k(a - a)^0 = a_k \]
\[ \lim_{x \to a} \frac{f(x) - P(x)}{(x - a)^n} = \lim_{n\to\infty} \frac{f'(x) - P'(x)}{n (x - a)^{n-1}} = \text{apply L'H\^opital's rule $n$ times} = \lim_{x\to a} \frac{f^{(n)}(x) - P^{(n)}(x)}{n! (x - a)^0} = 0 \]

Because $f$ and $p$ are continuous, we have
\[ f(x) - P(x) \overset{x \to a}{\to} 0 \]
\[ f(a) - P(a) = 0 \Leftrightarrow f(a) = P(a) \]

We need to show: $P^{(k)}(a) = f^{(k)}(a)$ for $0 \leq k \leq n$ and $P^{(0)}(a) = f^{(0)}(a)$.

\section{Exercise 32}
\begin{ex}
  Let $f_n: [a,b] \to \mathbb R$, $a < b$, be a sequence of regulated functions
  converging uniformly to $f: [a,b] \to \mathbb R$. Prove that $f$ is a regulated
  function and limes and integration can be exchanged as follows:
  \[
    \lim_{n\to\infty} \int_a^b f_n(x) \, dx
    = \int_a^b \lim_{n\to\infty} f_n(x)\, dx \left( = \int_a^b f(x) \, dx\right)
  \]
\end{ex}

\[ \forall \varepsilon > 0 \exists N = N(\varepsilon): \forall n \geq N: \ldots \]
\[ \text{Especially for } n = N(\varepsilon): \ldots \]
\[ \exists M = M(\varepsilon): \forall m \geq M: \ldots \]
\[ \text{Especially for } m = M(\varepsilon): \ldots \]
\[ \forall x: \abs{f_{N(\varepsilon)}(x) - f(x)} < \varepsilon \]
\[ \abs{f_{N(\varepsilon)}(x) - f_{N(\varepsilon),M(\varepsilon)}} < \varepsilon \]

\[ h_k \coloneqq g_{N(\frac1{k}),M(\frac{1}{k})} \]
\[ \forall x \abs{f(x) - h_k(x)} \leq \frac{2}{k} \overset{k\to\infty}{\to} 0 \]

\section{Exercise 33}

\[ P(x) \coloneqq \sum_{k=0}^\infty a_k x^k \qquad \rho = c > 0 \]

To show: \forall b \in (-c,c), a < b$:
\[ \sum_{k=0}^\infty \int_a^b a_k x^k \, dx = \int_a^b \sum_{k=0}^\infty a_k x^k \, dx = \int_a^b P(x) \, dx \]

\[ \sum_{k=0}^\infty \int_a^b a_k x^k \, dx = \lim_{n\to\infty} \sum_{k=0}^n \int_a^b a_k x^k \, dx
= \lim_{n\to\infty} \int_a^b \sum_{k=0}^n a_k x^k \, dx \overset{\text{ex. 32}}{=} \int_a^b \lim_{n\to\infty} \sum_{k=0}^n a_k x^k \, dx \]

But we need to show that the requirements for the equation in exercise~32 are satisfied.

\[ f(x) = \sum_{k=0}^\infty a_k \cdot x^k \qquad g(f) = c \]
\[ f_n(x) = \sum_{k=0}^n a_k \cdot x^k \]
\[ \abs{f(x) - f_n(x)} = \abs{\sum_{k=0}^\infty a_k x^k - \sum_{k=0}^n a_k x^k} = \abs{\sum_{k=n+1}^\infty a_n \cdot x^k} \leq \sum_{k=n+1}^\infty \abs{a_k} \cdot \abs{x^k} \]

It holds that $\abs{\frac{x}{r}}^k \leq \abs{\frac{x}{r}}^{n+1}$. Then consider
\[ \abs{r < c} = \sum_{k=n+1}^\infty \abs{a_k} \cdot \frac{\abs{x}^k}{\abs{r}^k} \cdot \abs{r}^k \leq \abs{\frac{x}{r}}^{n+1} \underbrace{\sum_{k=n+1}^\infty \abs{a_n} \cdot \abs{r}^k}_{=S} = \underbrace{\abs{\frac{x}{r}}^{n+1}}_{<\varepsilon} \cdot S \]
We chose some $c$ close enough such that our desired properties are fulfilled.

\subsection{Elaboration}
Convergence radius $c$: $\abs{x} < c$ and $\abs{x} < r < c$.
Let $a_k r^k$ be bounded and $\sum_{k} a_k r^k$ be convergent.
\[ \sum_{k=0}^\infty \abs{a_k x^k} \leq \sum_{k=0}^\infty \abs{\frac{x}{r}}^k \underbrace{\abs{a_k r^k}}_{\leq C} \leq C \sum_{k=0}^\infty \abs{\frac{x}{r}}^k = \frac{c}{1 - \abs{\frac{x}{r}}} \]

\section{Exercise 34}
\begin{ex}
  Determine the power series representation $P(x)$ of $\arctan$ in $x_0 = 0$.
  Furthermore give the largest interval $(-c, c)$ with $c > 0$ in which power
  series $P(x)$ represents arctan.

  Hint: Represent the derivative of arctan as a power series and use exercise~33.
\end{ex}

\[ \arctan(x) = \frac{1}{1 + x^2} = \sum_{n=0}^\infty (-1)^n x^{2n} \]

Geometrical series:
\[ \sum_{n=0}^\infty x^n = \frac{1}{1 - x} \qquad \text{ converges if } \abs{x} < 1 \]

\[ \sum_{n=0}^\infty (-1)^n x^{2n} = \sum_{n=0}^\infty \left(\frac{(-1)^n}{2n + 1} x^{2n+1} + c\right)' \]

\[ T_n(x,0) = \sum_{k=0}^n \frac{f^{(k)}(0)}{k!} \cdot x^k \]
\begin{align*}
  \arctan^{(2)}(x) &= \sum_{n=0}^\infty (-1)^n 2n x^{2n-1} \\
  \arctan^{(3)}(x) &= \sum_{n=0}^\infty (-1)^n 2n (2n - 1) x^{2n-2} \\
  \arctan^{(k)}(x) &= \sum_{n=0}^\infty (-1)^n (2n) (2n - 1) \ldots (2n - (k-2)) x^{(2n - (k-1))} \\
  \arctan^{(k)}(0) &= (-1)^{\frac{k-1}{k}} (k-1)(k-2) \ldots (k-1-k+2) = (k-1)! (-1)^{\frac{k-1}{2}}
\end{align*}
\[ T_n(x;0) = \sum_{k=0}^n \frac{(-1)^l (2l)!}{(2l + 1)!} x^{2l+1} = \sum_{l=0}^n \frac{(-1)^l x^{2l+1}}{2l + 1} \]


\end{document}
