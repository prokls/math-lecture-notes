\documentclass{article}
%\usepackage[top=30pt,left=30pt,right=30pt]{geometry}
\usepackage[german,english]{babel}
\usepackage[utf8]{inputenc}
\usepackage{algpseudocode}
\usepackage{algorithm}
\usepackage{graphicx}
\usepackage{caption}
\usepackage{subcaption}
\usepackage{amsmath}
\usepackage{amssymb}
\usepackage{enumitem}
\usepackage{amsthm}
\usepackage{pxfonts}
\usepackage{wasysym}
\usepackage{framed}
\usepackage{xcolor}
\usepackage{makeidx}
\usepackage{csquotes}
\usepackage[pdfborder={0 0 0}]{hyperref}
\usepackage{stmaryrd}
\usepackage{titlesec}
\titleformat{\paragraph}{\normalfont\itshape}{}{}{}

\newtheorem{ex}{Exercise} %  TODO

\newtheorem{theorem}{Theorem}  \numberwithin{theorem}{section}
\newtheorem{problem}{Problem}  \numberwithin{problem}{section}
\newtheorem{example}{Example}  \numberwithin{example}{section}
\newtheorem*{hypothesis}{Hypothesis}%  \numberwithin{hypothesis}{section}
\newtheorem{definition}{Definition}  \numberwithin{definition}{section}
\newtheorem{lemma}{Lemma}  \numberwithin{lemma}{section}
\newtheorem*{claim}{Claim}%  \numberwithin{claim}{section}
\newtheorem{remark}{Remark}  \numberwithin{remark}{section}
\newtheorem*{corollary}{Corollary}%  \numberwithin{corollary}{section}
\newtheorem{proposition}{Proposition}  \numberwithin{proposition}{section}

\algnewcommand{\algorithmicgoto}{\textbf{go to}}%
\algnewcommand{\Goto}[1]{\algorithmicgoto~\ref{#1}}%
\algrenewcommand{\algorithmiccomment}[1]{\hskip2em$\triangleright$ {\footnotesize #1}}

% definitions
\newcommand{\drawing}[1]{%
 \begin{figure}[t]
  \begin{center}
   \includegraphics{#1}
  \end{center}
 \end{figure}
}
\newcommand{\pic}[2]{%
 \begin{figure}[t]
  \begin{center}
   \includegraphics{#1}
   \caption{#2}
  \end{center}
 \end{figure}
}
\newcommand{\set}[1]{\left\{#1\right\}}
\newcommand{\setdef}[2]{\left\{\left.#1\,\right|\,#2\right\}}
\newcommand{\angel}[1]{\left\langle#1\right\rangle}
\newcommand{\norm}[1]{\left\|#1\right\|}
\newcommand{\card}[1]{\left|#1\right|}
\newcommand{\given}[1]{\textbf{Given.} #1\par}
\newcommand{\find}[1]{\textbf{Find.} #1\par}
\newcommand{\dateref}[1]{\paragraph{\textit{This lecture took place on #1.}}}
\newcommand{\exist}{\;\exists\,}
\newcommand{\fall}{\;\forall\,}
\newcommand{\noproof}[1]{A proof for Theorem~\ref{#1} is not provided.}
\newcommand{\vectwo}[2]{\begin{pmatrix} #1 \\ #2 \end{pmatrix}}
\makeatletter
\newcommand{\xRightarrow}[2][]{\ext@arrow 0359\Rightarrowfill@{#1}{#2}}
\makeatother

\newcommand{\mtn}{(\mu\times\nu)} % mu times nu

\DeclareMathOperator{\rank}{rank}
\DeclareMathOperator{\detm}{det}
\DeclareMathOperator{\perm}{det}
\DeclareMathOperator{\sign}{sign}
\DeclareMathOperator{\degree}{deg}
\DeclareMathOperator{\prop}{probability}
\DeclareMathOperator{\argmax}{argmax}
\DeclareMathOperator{\argmin}{argmin}
\DeclareMathOperator{\vol}{vol}  % volume
\DeclareMathOperator*{\bigtimes}{\vartimes}

\makeatletter
\providecommand*{\dotcup}{%
  \mathbin{%
    \mathpalette\@dotcup{}%
  }%
}
\newcommand*{\@dotcup}[2]{%
  \ooalign{%
    $\m@th#1\cup$\cr
    \hidewidth$\m@th#1\cdot$\hidewidth
  }%
}
\makeatother


% metadata
\title{
  Analysis 2 \\
  \large{Lecture notes, University (of Technology) Graz} \\
  based on the lecture by Wolfgang Ring
}
\date{\today}
\author{Lukas Prokop}

% settings
\parindent0pt
\setlength{\parskip}{0.4\baselineskip}
%\setcounter{tocdepth}{2}

\begin{document}

Florian Kruse

Sprechstunde: Tue, 14--15

\section{Exercise 01/1}

\begin{ex}
  The Euclidean norm of $v = (v^1, v^2, \dots, v^n)^T \in \mathbb R^n$ is defined as
  \[ \norm{v}_2 \coloneqq \sqrt{(v^1)^2 + (v^2)^2 + \ldots + (v^n)^2} \]
  Show: A sequence $(x_k) \subset \mathbb R^n$ converges in regards of the Euclidean norm to $x \in \mathbb R$ iff they converge componentwise to $x$
  \[ \lim_{k\to\infty} \norm{x_k - x}_2 = 0 \iff \forall j \in \set{1,\dots,n}: \lim_{k\to\infty} x_k^j = x^j \]
\end{ex}

Direction $\Rightarrow$.

Let $\lim_{k \to \infty} \norm{x_k - x} = 0$.

Consider: $\card{x_{jk} - x_j}$ for arbitrary $j \in \set{1,\dots,n}$.

It holds that
\[ 0 \leq \card{x_{jk} - x} = \sqrt{(x_{jk} - x_j)^2} \leq \sqrt{(x_{1k} - x_1)^2 + \dots + (x_{1k} - x_n)} = \norm{x_k - x} \to 0 \]
\[ \implies \lim_{k\to\infty} \card{x_{jk} - x_j} = 0 \forall j \]

Direction $\Leftarrow$.

Let $\lim_{k\to\infty} x_{jk} = x_j \forall j \in \set{1,\dots,n}$.

The square root function is continuous.
\[ \lim_{k\to\infty} \norm{x_k - x} = \sqrt{(x_{1k} - x_1)^2 + \dots + (x_{1k} - x_n)^2} \]
\[ \sqrt{(\lim_{k\to\infty} x_{1k})^2 - 2 (\lim_{k\to\infty} x_i k) x_1 + x_{1j}^2 + \dots + (\lim_{k\to\infty} x_{n_k})^2 - 2 (\lim{x_{n_k}}) x_n + x_n^2} \]
\[ = \sqrt{\underbrace{x_1^2 - 2x_1^2 + x_1^2}_{= 0} + \dots + \underbrace{x_n^2 - 2x_n^2 + x_n^2}_{= 0}} = 0 \]

\textbf{Remark:}
In $\mathbb R^n$, all norms are equivalent.
This exercise showed this property.
So it you pick two numbers in $\mathbb R^n$ and they get \enquote{closer}, they get \enquote{closer} in every norm.

\section{Exercise 01/2}

\begin{ex}
  In the lecture, we discussed the SCNF. $d_{\text{SCNF}}: \mathbb R^2 \times \mathbb R^2 \to \mathbb R$.
  For some fixed $p \in \mathbb R^2$ it is defined as
  \[
    d_{\text{SCNF}} \coloneqq \begin{cases}
      \norm{x - y}_2 & \text{ if } \exists \lambda > 0: y = p + \lambda (x - p) \\
      \norm{x - p}_2 + \norm{y - p}_2 & \text{ else}
    \end{cases}
  \]
  For $p \coloneqq (0,0)^T$ and $x \coloneqq (1,1)^T$, sketch the set $B_R(x)$ for $R=1$ and $R=2$.
  \[ B_R(x) \coloneqq \setdef{y \in \mathbb R^2}{d_{\text{SCNF}} < R} \]
\end{ex}

\section{Exercise 01/3}

\begin{ex}
  Let $(M, d)$ be a metric space and $x \in M$.
  Furthermore let $(x_k) \subset M$ be a sequence with property that every subsequence of $(x_k)$ contains a subsequence converging to $x$.
  Prove by contradiction, that $(x_k)$ converges to $x$.
\end{ex}

$x_0 \not\to x$.

There exists $\varepsilon_0 > 0$ for infinitely many $n \in \mathbb N: d(x_n, x) \geq \varepsilon_0$.
Choose a subsequence $(x_{u_j})_{j\in\mathbb N}$ with $d(x_{n_j}, x) \geq \varepsilon_0 \forall j \in \mathbb N$.
Then there does not exist a subsequence of $(x_{n_j})$ with limit $x$.

\section{Exercise 01/4}

\begin{ex}
  Let $(M, d)$ be a metric space and complete space. The diameter of a nonempty set $A \subset M$ is given by
  \[ \operatorname{diam}(A) \coloneqq \sup\setdef{d(x,y)}{x,y \in A} \]
  Let $(A_j)_{j\in\mathbb N}$ be a sequence of nonempty, closed sets in $M$ with $A_{j+1} \subset A_j$ for all $j \in \mathbb N$.
  Furthermore it holds that $\operatorname{diam}(A_j) \to 0$ for $j \to \infty$. Prove that $x \in M$ exists with
  $\bigcap_{j=1}^\infty A_j = \set{x}$ and that $x$ is unique.
\end{ex}

$A_j \subseteq M$, because its a complete, metric space.
\[ \implies \bigcap_{j=1}^\infty A_j \neq \emptyset \iff \exists x_0 \in M: \forall j \]
Assume $\exists y_0 \in M: y_0 \neq x_0 \implies d(y_0, x_0) \geq \varepsilon > 0$
\[ \forall j \in \mathbb N: \operatorname{diam}(A_j) \geq \varepsilon \]
This is a contradiction.
However, this is not the equality, we are looking for.
Assume $\bigcap_{j=1}^\infty A_j = \set{x_0} = \set{y_0} \implies x_0 = y_0$.
This is the equality, that was meant to be proven.

\subsection{Prove $\bigcap_{j=1}^\infty A_j \neq \emptyset \iff \exists x_0 \in M: \forall j$}

\textbf{Hint:} If the assignment mentions that completeness must be proven, usually you have to construct a Cauchy sequence.

Construct $(x_j)_{j \in \mathbb N}$. Choose for $x_j$ some element of $A_j$.
Choose $x_j \in A_j$ for $j \in \mathbb N$.
This defines a Cauchy sequence $(x_j)_{j \in \mathbb N}$.
Let $j \in \mathbb N$.
$x_i \in A_j \supset A_{j+1}$ and $x_{j+1} \in A_{j+1} \forall i \in \mathbb N$.
\[ \implies d(x_j, x_{j+i}) \leq \operatorname{diam}(A_j) \forall i \in \mathbb N \]
where $\operatorname{diam}(A_j) \to 0$ with $i \to \infty$.
\[ \implies \exists x \in M: \lim_{j \to \infty}(x_j) = x \]
Because $(x_j)_{j\geq J} \subseteq A_j$ and $\lim_{j\to\infty} (x_j)_{j\geq J} = x$,
it follows that $x \in A_j$ and then it follows that $x \in \bigcap_{j=1}^\infty A_j$.

\dateref{2018/03/22}

\section{Exercise 02/1}

\subsection{Blackboard solution}

Let $B$ be bounded.
\[ \operatorname{diam}(B) < \infty \qquad \operatorname{diam}(B) = \operatorname{sup}(\setdef{d(x,y)}{x,y \in B}) \]
\[ d(B_k, B_{k+1}) = \operatorname{inf}(\setdef{d(x,y)}{x \in B_k, y \in B_{k+1}}) \]

Exercise (a).

Prove:
\[ \sum_{k=1}^\infty \operatorname{diam}(B_k) < \infty \land \sum_{k=1}^\infty d(B_k, B_{k+1}) \implies \operatorname{diam}(\bigcup_{k=1}^{\infty} B_k) < \infty \]
\[ \operatorname{diam}(B_k \cup B_{k+1}) \leq \operatorname{diam}(B_k) + d(B_k, B_{k+1}) + \operatorname{diam}(B_{k+1}) \]
We distinguish 3 cases:
\begin{enumerate}
  \item $x \in B_k, y \in B_k: d(x,y) \leq \operatorname{diam}(B_k) \leq \operatorname{diam}(B_k) + d(B_k, B_{k+1}) + \operatorname{diam}(B_{k+1})$
  \item $x \in B_{k+1}, y \in B_{k+1}, d(x, y) \leq \operatorname{diam}(B_k) + d(B_k, B_{k+1}) + \operatorname{diam}(B_{k+1})$
  \item $\forall x \in B_{k} \forall y \in B_{k+1}$
\end{enumerate}
Choose $x_0$ and $y_0$ on the border of sets $B_k$ and $B_{k+1}$ respectively.
But $x_0, y_0$ do not necessarily exist if compactness is not given.
But let $\varepsilon > 0$. Find $x_0, y_0$ with $d(x_0, y_0) \leq d(B_k, B_{k+1}) + \varepsilon$.
\[ d(x,y) \leq \underbrace{d(x,x_0)}_{\leq \operatorname{diam}(B_k)} + \underbrace{d(x_0, y_0)}_{\leq d(B_k, B_{k+1}) + \varepsilon} + \underbrace{d(x_0, y)}_{\leq \operatorname{diam(B_k)}} \leq \operatorname{diam}(B_k) + d(B_k, B_{k+1}) + \operatorname{diam}(B_{k+1}) + \varepsilon \]

Laurent Pfeiffer continued the following solution (until Exercise 2):

\[ \operatorname{diam}((B_k \cup B_{k+1}) \cup B_{k+2}) \leq \operatorname{diam}(B_k \cup B_{k+1}) + \underbrace{d((B_k \cup B_{k+1}), B_{k+2})}_{\leq d(B_{k+1}, B_{k+2})} + \operatorname{diam}(B_{k+2}) \]
\[ \leq \operatorname{diam}(B_k) + d(B_k, B_{k+1}) + \operatorname{diam}(B_{k+1}) + d((B_k \cup B_{k+1}), B_{k+2}) + \operatorname{diam}(B_{k+2}) \]
By induction it follows that
\[ \operatorname{diam}(B_k \cup B_{k+1} \cup \dots \cup B_n) \leq \operatorname{diam}(B_k) + d(B_k, B_{k+1}) + \operatorname{diam}(B_{k+1})  + d(B_{k+2}) + d(B_{n-1}, B_n) + \operatorname{diam}(B_n) \]
\[ \operatorname{diam}(B_k \cup \dots \cup B_n) \leq \underbrace{\sum_{i=1}^n \operatorname{diam}(B_i) + d(B_i, B_{i+1})}_{D} \]

Choose $x,y \in \bigcup_{i=1}^\infty B_i$. Then there exists some $k \in \mathbb N$ such that $x \in B_k$. There exists $n$ such that $y \in B_n$.
\[ d(x,y) \leq \operatorname{diam}(B_k) + \dots + \operatorname{diam}(B_n) \leq D \]

Exercise (b).

Let $x \in M$. We define: $B_{k+1} = B_{k+2} = \dots = \set{x}$.
For all $i \geq k$ it holds that
\[ \operatorname{diam}(B_i) = 0 \]
\[ d(B_i, B_{i+1}) = 0 \]
Therefore,
\[ \sum_{i=1}^\infty \operatorname{diam}(B_i) = \sum_{i=1}^k \underbrace{\operatorname{diam}(B_i)}_{<+\infty} < +\infty \]
What about the distances?
\[ \int_{i=1}^\infty d(B_i, B_{i+1}) = \sum_{i=1}^k d(B_i, B_{i+1}) < +\infty \]
By (a), it follows that
\[ \left(\bigcup_{i=1}^\infty B_i\right) \text{ is bounded } \implies \left(\bigcup_{i=1}^k B_i\right) \subseteq \left(\bigcup_{i=1}^\infty B_i\right) \text{ is also bounded}  \]

Exercise (c).

We define
\[ B_i = \left[\sum_{j=1}^i \frac1j , \sum_{j=1}^{i+1} \frac1j\right] \]
Then it holds that
\[ \operatorname{diam}(B_i) = \frac1{i+1} \xRightarrow{i\to\infty} 0 \]
\[ \sum_{i=1}^\infty \operatorname{diam}(B_i) = \infty \]
\[ B_i \cap B_{i+1} = \set{\sum_{j=1}^{i+1} \frac1j} \implies d(B_i, B_{i+1}) = 0 \]
\[ B_1 \cup \dots \cup B_i = \left[1, \underbrace{\sum_{j=1}^{i+1} \frac1j}_{\to\infty}\right] \implies \underbrace{\bigcup_{i=1}^\infty B_i}_{\text{not bounded}} = [1,\infty) \]

We define $B_i = \set{\sum_{j=1}^i \frac1{j}}$. For all $i$:
\begin{itemize}
  \item $\operatorname{diam}(B_i) = 0 \implies \sum_{i=1}^\infty \operatorname{diam}(B_i) = 0$
  \item
    \[ d(B_i, B_{i+1}) = \left(\sum_{j=1}^{i+1} \frac1j\right) - \left(\sum_{j=1}^i \frac1j\right) = \frac1{i+1} \xRightarrow{i\to\infty} 0 \]
    \[ \sum_{i=1}^\infty d(B_i, B_{i+1}) = \sum_{i=1}^\infty \frac{1}{i+1} = \infty \]
    The union is \emph{not} bounded, because $\sum_{j=1}^i \frac1j \in \bigcup_{j=1}^\infty B_j$.
\end{itemize}

\section{Exercise 02/2}

\begin{ex}
  Let $(X, d)$ be a sequentially compact, metric space. Show:
  \begin{enumerate}
    \item[a.] $X$ is bounded.
    \item[b.]
  \end{enumerate}
\end{ex}

\subsection{Blackboard solution}

Exercise (a).

Let $X$ be unbounded. Hence, there exists a tuple $(x_N, y_N) \in X \times X$ for every $N \in \mathbb N$ with $d(x_N, y_N) > N$.
Because $(X, d)$ is sequentially compact, there exists a convergent subsequence $(x_{N_k}, y_{N_{k_i}})$ we can choose such that
\[ \lim_{k\to\infty} x_{N_k} = \infty \qquad \lim_{i\to\infty} y_{N_{K_i}} = y_0 \qquad \lim_{i\to\infty} (x_{N_{k_i}}) = x_0 \]
\[ \implies \underbrace{N_{k_i}}_{\xrightarrow{i\to\infty} \infty} < d(x_{N_{k_i}}, y_{N_{k_i}}) \xrightarrow{i\to\infty} d(x_0, y_0) \]
By this contradiction, it follows that $X$ is bounded.

Exercise (b).

Let $(x_n)_{n\in\mathbb N}$ be a Cauchy sequence in $X$. Let $X$ be sequence compact $\implies$ there exists a convergent subsequence $x_{n_{k}} \xrightarrow{k\to\infty} x \in X$. Show that $x_n \xrightarrow{n\to\infty} x$.

Let $\varepsilon > 0$ be arbitrary. Choose $N \in \mathbb N$ such that $\forall n,m \geq N: d(x_n, x_m) < \frac\varepsilon2$.
Choose $k \in \mathbb N$ such that $n_k \geq N$ and $d(x_{n_k}, x) < \frac\varepsilon2$.
\[ \forall n \geq n_k: d(x, x_n) \leq d(x, x_{n_k}) + d(x_{n_k}, x_n) < \varepsilon \]

Exercise (c).

Show that $A \subset X$ is sequentially compact iff $A$ is closed.

\begin{description}
  \item[$\Rightarrow$]
    Let $(x_n)_{n \in \mathbb N}$ be a convergent sequence, $(x_n)_{n \in \mathbb N} \subset A$, $\lim_{n\to\infty} x_n = x_0 \in X$.
    Show that $x_0 \in A$.

    Set $A$ is sequentially compact. Choose subsequence $(x_{n_k})_{k \in \mathbb N} \subset A$, $\lim_{k\to\infty} x_{n_k} = x_0 \in A \implies A$ is closed.
  \item[$\Leftarrow$]
    $A$ is closed. Show that $A$ is sequentially compact.

    Let $(x_n)_{n\in\mathbb N} \subset A$ and there exists subsequence $(x_{n_k})_{k \in \mathbb N}$ with $\lim_{k\to\infty} x_{n_k} = x_0 \in X$, because $X$ is sequentially compact.
    $(x_{n_k})_{k \in \mathbb N} \subset A \implies A$ is sequentially compact.
\end{description}

\section{Exercise 02/2}

\begin{ex}
  Let $f: \mathbb R \to \mathbb R$, $f(x) = \sqrt{1 + x^2}$.
  \begin{enumerate}
    \item Show that $\card{f(x) - f(y)} < \card{x - y} \forall x, y \in \mathbb R$ with $x \neq y$
    \item Investigate which conditions of Banach's Fixed Point Theorem are [not] met.
    \item Is Banach's Fixed Point Theorem applicable? Does $f$ have a fixed point?
  \end{enumerate}
\end{ex}

Exercise (a).

\[ \card{f(x) - f(y)} < \card{x - y} \qquad x,y \in \mathbb R, x \neq y \]
\[ \card{\sqrt{1 + x^2} - \sqrt{1 + y^2}} < \card{x - y} \]
\[ 1 + x^2 + 1 + y^2 - 2 \sqrt{(1 + x^2) (1 + y^2)} < x^2 + y^2 - 2xy \]
\[ 2 - 2\sqrt{(1 + x^2)(1 + y^2)} < -2xy \]
\[ 1 + xy < \sqrt{(1 + x^2)(1 + y^2)} \]
We need to distinguish 2 cases here ($x$ and $y$ have same signum, $x$ and $y$ have different signum). This is trivial.
\[ 1 + 2xy + x^2 y^2 < 1 + x^2 + y^2 + x^2 y^2 \]
\[ 0 < x^2 + y^2 - 2xy \]
\[ 0 < (x - y)^2 \]

Exercise (b and c).

Let $x \in \mathbb R$.
\[ f(x) = x \]
\[ \sqrt{1 + x^2} = x \]
\[ 1 + x^2 = x^2 \]
\[ 1 = 0 \]

\dateref{2018/04/12}

\section{Exercise 03/4}
\begin{ex}
  Let $(X, d)$ be a metric space and $x_0 \in X$. A function $f: X \to \mathbb R$ is called half-continuous from below in $x_0$, if for every $\varepsilon > 0$ some $\delta > 0$ exists, such that $d(x, x_0) < \delta$ implies $f(x_0) - f(x) < \varepsilon$. If $f$ is half-continuous from below in every $x_0 \in X$, then $f$ is called half-continuous from below.
\end{ex}

Obviously, continuity implies half-continuity.

\subsection{Exercise 03/4a}
\begin{ex}
  Give some half-continuous from below $f: [-1, 1] \to \mathbb R$ such that $f$ is non-continuous.
\end{ex}

Let $f: [-1,1] \to \mathbb R$.
\[
  x \mapsto \begin{cases}
    -1 & x = -1 \\
    -x & x \neq -1
  \end{cases}
\]
\[ \underbrace{f(-1)}_{=-1} - \underbrace{f(x)}_{\geq -1} \leq 0 < \varepsilon \]

\subsection{Exercise 03/4b}
\begin{ex}
  Give some half-continuous from below $f: [-1, 1] \to \mathbb R$, but does not have a maximum.
\end{ex}
Same $f$ can be chosen.

\subsection{Exercise 03/4c}
\begin{ex}
  Give some half-continuous from below $f: [-1, 1] \to \mathbb R$, but does not have a minimum.
\end{ex}
$f$ as $f|_{[-1,1]}$ can be chosen.

\subsection{Exercise 03/4d}
\begin{ex}
  Prove that every half-continuous from below function in a compact set has a minimum.
\end{ex}

\textbf{Hint:} It is assumed that cover-compactness seems to be more cumbersome than sequential compactness. \\
\textbf{Remark:} This is a generalization of the theorem, that every continuous, compact function has a minimum and maximum.

Let $K \subseteq X$ be compact. $f: K \to \mathbb R$ is half-continuous from below.

Show that $f^k = \operatorname{inf}(f(K)) \in f(K)$.

\[ \exists (x_n)_{n\in\mathbb N} \subseteq K \text{ with } f(x_n) - f^k < \frac1n \]
$K$ is compact. Hence, there exists $(x_{n_k})_{k \in \mathbb N}$ with $\lim_{k\to\infty} x_{n_k} \coloneqq x^* \in K$.
Let $\varepsilon > 0$ be arbitrary.
By half-continuity from below, it follows that $\exists \delta > 0: d(x^*, x) < \delta \implies f(x^*) - f(x) < \varepsilon$.
\[ \exists K \in \mathbb N \forall k \geq K: d(x^k, x_{n_k}) < \delta \implies f(x^k) - f(x_{n_k}) < \varepsilon \iff f(x^*) < f(x_{n_k}) + \varepsilon \]
\[ \implies f(x^*) \leq \lim_{k\to\infty} f(x_{n_k}) \implies f(x^*) \leq \lim_{n\to\infty} f(x_n) = f^* \]
\[ \implies f(x^*) = f^* \implies f^* \text{ is minimum of } f(X) \]

\section{Exercise 03/3}
\begin{ex}
  Let $(X, d)$ and $(Y, e)$ be metric spaces, where $d: X \to \mathbb R$ is a discrete metric,
  hence
  \[
    d(x_1, x_2) = \begin{cases}
      0 & \text{if } x_1 = x_2 \\
      1 & \text{if } x_1 \neq x_2
    \end{cases}
  \]
\end{ex}

\subsection{Exercise 03/3a}
\begin{ex}
  Every map $f: X \to Y$ is continuous.
\end{ex}

Let $f: X \to Y$ be arbitrary.
Let $x_0 \in X$ and $\varepsilon > 0$ be arbitrary.
Show that
\[ \exists \delta > 0: d(x, x_0) < \delta \implies d(f(x), f(x_0)) < \varepsilon \]
\[ K_{\frac12}(x_0) = \set{x_0} \]

\subsection{Exercise 03/3b}
\begin{ex}
  A map $f: X \to Y$ is not necessarily bounded.
\end{ex}

$M \geq 0$ arbitrary. $\exists x,y \in f(X): e(x,y) > M$.

\[ f: \mathbb Z \to \mathbb Z \qquad x \mapsto x \]
\[ f(x) = \mathbb Z \qquad x = 0 \qquad y = M + 1 \]
$e = \card{\cdot}$.

\subsection{Exercise 03/3c}
\begin{ex}
  Every map $g: Y \to X$ is bounded.
\end{ex}
Let $g: Y \to X$ be arbitrary.
Show that $\exists M \geq 0 \forall x,y \in g(Y): d(x,y) \leq M$.
Choose $M = 2$. $\forall x,y \in X: d(x,y) \leq 1 \leq 2$.

\subsection{Exercise 03/3d}
\begin{ex}
  In case $(Y, e) = (\mathbb R, \card{\cdot})$, every non-constant map $g: Y \to X$ is non-continuous.
\end{ex}

We show: continuity implies constant.

Let $g: \mathbb R \to X$ continuous.
Let $x_0 \in \mathbb R$ be arbitrary and $\varepsilon = \frac12$.
$\exists \delta_0 > 0: \card{x_0 - x} < \delta \implies d(g(x_0), g(x)) < \frac12$
for $x_0 \in \mathbb R$ there exists $\delta_0$ such that 
$\forall x \in (x_0 - \delta, x_0 + \delta): g(x) = g(x_0)$.
\[ \operatorname{sup}\setdef{s \in [x_0, \infty)}{g(x) = g(x_0) \forall x \in [x_0, s)} \]

\section{Exercise 03/2}
\begin{ex}
  Let $V$ be the vector space of bounded, complex sequences, hence
  \[
    V \coloneqq \setdef{(a_k)_{k \in \mathbb N} \subset \mathcal C}{\exists M \in \mathbb R \text{ with } \card{a_k} \leq M \forall k \in \mathbb N}
  \]
  additionally with norm
  \[ \norm{(a_k)_{k \in \mathbb N}}_{\infty} \coloneqq \sup\setdef{\card{a_k}}{k \in \mathbb N} \]
\end{ex}

This solution was done by Mr. Kruse himself.

\subsection{Exercise 03/2b}
\begin{ex}
  The unit sphere in $(V, \norm{\cdot}_{\infty})$,
  \[ B_1(0) = \setdef{a \in V}{\norm{a}_{\infty} \leq 1} \]
  is closed and bounded, but not sequentially compact.
\end{ex}

We need to prove boundedness.

Let $C, D \in B_1(0)$.
\[ \implies \norm{\underbrace{C}_{= (c_k)} - \underbrace{D}_{=(d_k)}}_{\infty} \leq 2 \]
\[ \sup\set{\card{\underbrace{c_k - d_k}_{\leq \underbrace{\card{c_k}}_{\leq 1 \forall k} + \underbrace{\card{d_k}}_{\leq 1 \forall k} \leq 2 \forall k } }: k \in \mathbb N} \leq 2 \]

We need to prove closedness.

\[ (A^n)_{n \in \mathbb N} \subset B_1(0) \text{ with } \lim_{n\to\infty} A^n = A \]
Show that $A \in B_1(0)$.
\[ \text{For every } A^n \coloneqq (a^n_k)_{k \in \mathbb N} \text{ it holds that } \norm{\underbrace{(a^n_k)_{k \in \mathbb N}}_{= \sup\set{\card{a^n_k}: k \in \mathbb N} \leq 1}}_{\infty} \leq 1 \]

\[ (A^n)_{n \in \mathbb N} \subset B_1(0) \text{ with } \lim_{n\to\infty} A^n = A \]
\[ \iff \lim_{n \to \infty} \norm{A^n - A}_{\infty} = 0 \]

$\card{a^n_k}$ in
\[ \sup\set{\card{a^n_k}: k \in \mathbb N}  \]
converges to $\card{a_k} \leq 1$ for $n \to \infty$.

We need to prove sequentially non-compact of $B_1(0)$.
So we only need to find some sequence that does not have some converging subsequence.

We define
\[
  A^n \coloneqq (a^n_k)_{k \in \mathbb N} \coloneqq \begin{cases}
    0 & \text{ if } k \neq n \\
    1 & \text{ if } k = n
  \end{cases}
\]
for every $n \in \mathbb N$. As such we get a sequence
\[ \implies (A^n)_{n \in \mathbb N} \subset B_1(0) \]
but it holds that $\norm{A^n - A^m}_{\infty} = 1 \forall n \neq m$.
This is also not a Cauchy sequence.

\section{Exercise 03/1}
\begin{ex}
  Let $(X,d)$ be a metric space. A set $K \subset X$ is called cover-compact,
  if for every family of open sets $(U_i)_{i \in I} \subset X$ with $K \subset \bigcup_{i \in I} U_i$ it holds that:
  There exists a finite set $J \subset I$ with $K \subset \bigcup_{i \in J} U_i$. Let $K \subset X$ be cover-compact.
\end{ex}

\subsection{Exercise 03/1a}
\begin{ex}
  Show that $K$ is totally bounded, hence for every $r>0$, there exists $x_1, \dots, x_n$ in $K$
  with $K \subset \bigcup_{i=1}^n B_r(x_i)$.
\end{ex}

Construct a family of open spheres ($\left(\mathcal B_r(x)\right)_{x \in K} \subset K$ covering $K$).
By cover-compactness it follows there exists some finite $J \subset K$ with $K \subset \bigcup_{x \in J} B_r(x)$.

\subsection{Exercise 03/1b}
\begin{ex}
  Prove that $K$ is sequentially compact.
\end{ex}

Proof by contradiction: Assume $K$ is not sequentially compact.

Then there exists a sequence $(x_n)_{n \in \mathbb N} \in K$ which has a subsequence $(x_{n_k})_{k \in \mathbb N} \to c \not\in K$.
\[ \forall x \in K: \exists r_x > 0: B_{r_x}(x) \text{ contains finitely many sequence elements} \]
Because $\bigcup_{x \in K} B_{r_x}(x) \supset K$ it holds: there exists $J \subset K$ finite $\bigcup_{x \in J} B_{r_x}(x) \supset K$.
This contradicts with $(x_n)_{n \in \mathbb N} \subset K$.

\subsection{Exercise 04/1}
\begin{ex}
  Let $(M,d)$ be a complete metric space and $(A_k)_{k \in \mathbb N} \subset M$
  is a sequence of closed sets. Use Cantor's Theorem to prove:
  $\bigcup_{k \in \mathbb N} A_k$ contains an open set if at least one $A_k$
  contains an open set. Illustrate this statement for $(M,d) = (\mathbb R, \card{\cdot})$.
\end{ex}

First we illustrate it in $\mathbb R$.

\[ (A_k) = \set{a_k} \]
where $a_k \in \mathbb R$.

Consider some 

\section{Exercise 04/2}
\begin{ex}
  Let $f: [-1,1] \to \mathbb C$ be continuous and $O \subset \mathbb C$ is an open set.
  In the lecture we have seen that $f^{-1}(O)$ is open.
  Review the result and prove for $O = \mathbb C$.
  \begin{enumerate}
    \item The set $O$ is open.
    \item It holds that $f^{-1}(O) = [-1,1]$
    \item The set $[-1,1] \subset \mathbb R$ is not open.
    \item The statement of the lecture about $f^{-1}(O)$ is still correct.
  \end{enumerate}
\end{ex}

\subsection{Exercise 04/2a}
Show that $\mathbb C$ is open.

Let $z \in \mathbb C$. $\exists \varepsilon > 0$,
\[ B(z, \varepsilon) \subseteq \mathbb C \]

\subsection{Exercise 04/2b}
Follows from the definition of a function.

\subsection{Exercise 04/2c}
If it is an open set, there must be a neighborhood of arbitrary $\varepsilon$
such that this neighborhood is completely in the set.

Let $\varepsilon > 0$. Choose $x \in B(1, \varepsilon)$ with $x = 1 + \frac{\varepsilon}2$.
\[ \implies x \in B(1, \varepsilon) \land x \not\in [-1,1] \]

\subsection{Exercise 04/2d}
Let $(X,d)$ and $(Y,e)$ be metric spaces and $f: X \to Y$ continuous
then $f^{-1}(O)$ is open $\forall O \subseteq Y$ open.

Show:
\[ \forall x \in [-1,1] \exists \varepsilon > 0: \underbrace{B(x, \varepsilon)}_{= \setdef{z \in [-1,1]}{d(x, z) < \varepsilon}} \subseteq [-1,1] \]
So the difference is the domain of $z$ ($[-1,1]$ unlike exercise c, where we used $\mathbb R$).

The point was to illustrate how to read the theorem properly.

\section{Exercise 04/3}
\begin{ex}
  Let $\Omega$ be a non-empty set and $B(\Omega)$ the vector space of real-valued bounded functions on $\Omega$.
  Hence,
  \[ B(\Omega) \coloneqq \setdef{f: \Omega \to \mathbb R}{\exists M \in \mathbb R \text{ with } \card{f(x)} \leq M \forall x \in \Omega} \]
  with norm
  \[ \norm{f}_{\infty} \coloneqq \sup\setdef{\card{f(x)}}{x \in \Omega} \]
  Prove the following statements:
  \begin{enumerate}
    \item $(B(\Omega), \norm{\cdot}_{\infty})$ is a complete normed vector space.
    \item The unit circle $U$ in $B(\Omega)$ is closed and bounded.
      \[ U = \setdef{f \in B(\Omega)}{\norm{f}_{\infty} \leq 1} \]
    \item The unit circle is sequentially compact if and only if $\Omega$ is finite.
  \end{enumerate}
\end{ex}

\subsection{Exercise 04/3a}
Given $\Omega \neq 0$.
\[ B(\Omega) \coloneqq \setdef{f: \Omega \to \mathbb R}{\exists M \in \mathbb R: \card{f(x)} \leq M \quad \forall x \in \Omega} \]

First, we show that $\norm{\cdot}_{\infty}$ is indeed a norm. We just show absolute homogeneity for illustrative purposes:
\begin{align*}
  \norm{\lambda f}_{\infty}
    &= \sup\setdef{\card{\lambda \cdot f(x)}}{x \in \Omega} \\
    &= \sup\setdef{\card{\lambda} \cdot \card{f(x)}}{x \in \Omega} \\
    &= \card{\lambda} \cdot \sup\set{\card{f(x)}}{x \in \Omega} \\
    &= \card{\lambda} \cdot \norm{f}
\end{align*}

We show completeness of $(B(\Omega), \norm{\cdot}_{\infty})$.
Equivalently, all Cauchy sequences in $B(\Omega)$ are convergent.
Equivalently, for all Cauchy sequences $(f_n)_{n \in \mathbb N}: \exists f \in B(\Omega): \norm{f_n - f}_{\infty} \to 0$ for $n \to \infty$.

Let $(f_n)_{n \in \mathbb N}$ be an arbitrary Cauchy sequence. Hence,
\[ \forall \varepsilon > 0 \exists N \in \mathbb N: n,m > N \implies \norm{f_n - f_m}_{\infty} = \sup\setdef{(f_n - f_m)(x)}{x \in \Omega} < \varepsilon \]
\[ \forall \varepsilon > 0: n,m > N \]
\[ \forall x \in \Omega: \card{(f_n - f_m)(x)} < \varepsilon \]
\[ \implies \forall x \in \Omega: (f_n(x))_{n \in \mathbb N} \subseteq R \]
is a Cauchy sequence in $\mathbb R$.
\[ \iff \forall x \in \Omega: (f_n(x))_{n \in \mathbb N} \text{ converges} \]
\[ \forall x \in \Omega: (f_n(x)))_{n\in\mathbb N} \to f(x) \forall \varepsilon > 0 \exists N \in \mathbb N: n > N \implies \card{f_n(x) - f(x)} < \varepsilon \]
\[ \exists N \in \mathbb N \forall n > N: \norm{f_n - f}_{\infty} < 1 \]
\[ \norm{f}_{\infty} = \norm{f - f_N + f_N}_{\infty} \leq \underbrace{\norm{f - f_N}_{\infty}}_{<1} + \underbrace{\norm{f_N}}_{\leq M} < 1 + M \]

\subsection{Exercise 04/3b}
Let $K_1 \coloneqq \setdef{f \in B(\Omega)}{\norm{f}_{\infty} \leq 1}$.
Show $K_1$ is bounded and closed.

\subsubsection{$K_1$ is bounded}
Let $f,g \in K_1$ be arbitrary.
\[ \norm{f - g}_{\infty} \leq \norm{f}_{\infty} + \norm{g}_{\infty} \leq 1 + 1 = 2 \]
$2$ is a boundary and therefore $K_1$ is bounded.

\subsubsection{$K_1$ is closed}
Let $(f_n)_{n \in \mathbb N}$ be a convergent sequence in $K_1$
with $\lim_{n\to\infty} f_n = f \iff \lim_{n\to\infty} \norm{f_n - f} = 0$.

Show $f \in K_1$.

\[ \forall f_n \in K_1: \norm{f_n} \leq 1 \]
\[ \norm{f}_{\infty} = \norm{f - f_n}_{\infty} \leq \underbrace{\norm{f - f_n}_{\infty}}_{\xrightarrow{n \to \infty} 0} + \underbrace{\norm{f_n}_{\infty}}_{\leq 1} \leq 1 \]
\[ \implies \norm{f}_{\infty} \leq 1 \implies f \in K_1 \]

\subsection{Exercise 04/c}
$f$ is sequentially compact if and only if $\Omega$ is finite?
Equivalently, every sequence $(f_n)_{n \in \mathbb N} \subseteq K_1$ has a convergent subsequence with limit in $K_1$.

Direction $\implies$.

Let $\Omega$ be infinite. Then $\exists$ a sequence $(f_n)_{n \in \mathbb N}$ without convergent subsequence.
We build a sequence $(f_n)_{n \in \mathbb N}$ in $K_1$.

Let $(x_i)_{i \in \mathbb N}$ be an arbitrary sequence in $\Omega$ with $x_i \neq x_j \forall i \neq j$.
\[
  f_n(x) \coloneqq \begin{cases}
    1 & \text{ if } x = x_n \\
    0 & \text{ else}
  \end{cases}
\]
Then it holds that $\forall n \neq m$,
\[ \norm{f_n - f_m}_{\infty} = 1 \]

Assume there exists a convergent subsequence in $(f_{n_k})_{k \in \mathbb N}$ of $(f_n)_{n \in \mathbb N}$ with limit $f$.
\[ \implies \exists M > 0: k > M: \norm{f_{n_k} - f}_{\infty} < \frac12 \]
Let $k,l > M$ with $k \neq l$
\[ \implies \norm{f_{n_k} - f_{n_l}}_{\infty} \leq \norm{f_{n_k} - f}_{\infty} + \norm{f_{n_l} - f}_{\infty} < \frac12 + \frac12 = 1 \]
This is a contradiction to $\norm{f_n - f_m}_{\infty} = 1$.

Direction $\impliedby$.

Let $(f_n)_{n \in \mathbb N}$ be a sequence in $K_1$ without limit.
Let $n \in \mathbb N$.
\[ \Omega = \set{x_1, \ldots, x_n} \implies \card{\set{f_n(x_1), \ldots, f_n(x_n)}} < \infty \]
Let $f_n \in K_1 \implies \card{f_n(x_i)} \leq 1 \forall i \in \set{1, \ldots, m} \forall n \in \mathbb N$.

Consider $x_1 \in \Omega$.
\[
  (f_n(x_1)) = y_n^1 \in [-1,1]
\] \[
  [-1,1] \text{ compact }
  \implies (y_n^1)_{n \in \mathbb N} \text{ has convergent subsequence } (y^1_{n_k})_{k \in \mathbb N} \to \tilde y^1
\] \[
  (f_{n_k}(x_1))_{k \in \mathbb N} = (y^1_{n_k})_{k \in \mathbb N} \to \tilde y_1 \coloneqq f(x_1)
\]
and this goes on up to
\[ (f_{n_{\ddots_z}}(x_m))_{z \in \mathbb N} \to f(x_m) \]
For every $\varepsilon > 0$
\[ \exists N_1: \forall n \in N_1: \card{f_{n_{\ddots_{2}}}(x_1) - f(x_1)} < \varepsilon \]
\[ \vdots \]
\[ \exists N_m: \forall n \in N_m: \card{f_{n_{\ddots_{2}}}(x_m) - f(x_m)} < \varepsilon \]

Choose $N \coloneqq \max{N_1, \ldots, N_m}$. For all $n \geq N$,
\[ \implies \norm{f_{n_{\ddots_{2}}}}_{\infty} < \varepsilon \]

\section{Exercise 04/4}
\begin{ex}
  Let $k \in \mathbb N$. Show: $\exists \phi_k: \sqrt{k\pi} \leq \xi_k \leq \sqrt{(k+1)\pi}$ such that
  \[ \int_{\sqrt{k\pi}}^{\sqrt{(k+1)\pi}} \sin(x^2) \, dx = \frac{(-1)^k}{\xi_k} \]
\end{ex}

\[
  \int_{\sqrt{k\pi}}^{\sqrt{(k+1)\pi}} \sin(x^2) \, dx
  = \int_{\sqrt{k\pi}}^{\sqrt{(k+1) \pi}} \frac{x \cdot \sin(x^2)}{x} \, dx
  = \frac{1}{\xi_k} \cdot \int_{\sqrt{k\pi}}^{\sqrt{(k+1)\pi}} x \cdot \sin(x^2) \, dx
\]
But this IVT is unconventional.
\[
  = \left. \frac{1}{\xi_k} \cdot \left(-\frac12 \cdot \cos(x^2)\right) \right|_{\sqrt{k\pi}}^{\sqrt{(k+1)\pi}}
\]

If $k$ is even:
\[ \frac{1}{\xi_k} \left(\frac12 + \frac12\right) = \frac{1}{\xi_k} \]
If $k$ is odd:
\[ \frac{1}{\xi_k} \left(-\frac12 - \frac12\right) = -\frac{1}{\xi_k} \]

This implies a boundary of
\[ \frac{(-1)^k}{\xi_k} \]

\dateref{2018/04/26}

\section{Sheet 5, Exercise 1}
\begin{ex}
  Let $\mathcal R[a,b]$ be the vector space of real-valued regulated functions on $[a,b] \subseteq \mathbb R$, hence
  \[ \mathcal R[a,b] \coloneqq \setdef{f: [a,b] \to \mathbb R}{f \text{ is a regulated function}} \]
  annotated with a norm $\norm{\cdot}_\infty$ of Sheet 4 Exercise 3.
  Prove that $(\mathcal R[a,b], \norm{\cdot}_\infty)$
  is a complete normed vector space with a sequentially non-compact unit sphere.
\end{ex}

\section{Sheet 5, Exercise 2}
\begin{ex}
  Let $f, b \in \mathcal R[a,b]$ with
  \[ f_+(x) = g_+(x) \qquad \forall x \in [a,b) \]
  \[ f_-(x) = g_-(x) \qquad \forall x \in (a,b] \]
  \begin{enumerate}
    \item For $\alpha,\beta \in [a,b]: \int_\alpha^\beta f(x) \, dx = \int_\alpha^\beta g(x) \, dx$ holds.
    \item For every antiderivative $F: [a,b] \to \mathbb R$ of $f$ there exists an antiderivative $G: [a,b] \to \mathbb R$ of $g$ with $F(x) = G(x)$ for all $x \in [a,b]$.
  \end{enumerate}
\end{ex}

\section{Sheet 5, Exercise 2a}

Let $f,g \in \mathcal R[a,b]$.
\[ F'_+(x) \coloneqq f_+(x) = g_+(x) \]
\[ F'_-(x) \coloneqq f_-(x) = g_-(x) \]
Show: $\int_\alpha^\beta f(x) \, dx = \int_\alpha^\beta g(x) \, dx$.

In general $f_+(x) \neq f(x) \neq f_-(x)$.

\[ F \coloneqq \int f(x) \, dx \]
\[ G \coloneqq \int g(x) \, dx \]

\[ \int_\alpha^\beta f(x) \, dx = \left. F \right|_\alpha^\beta \overset{(b)}{=} \underbrace{F(\beta) + K}_{G(\beta)} - \underbrace{(F(\alpha) - K)}_{G(\alpha)} = \int_\alpha^\beta g(x) \, dx \]

\section{Sheet 5, Exercise 2b}
$F$ is an antiderivative of $f$ if and only if
\[ F = \int f(x) \, dx \]
\[ F_+'(x) = f_+(x) = g_+(x) = g_+(x) \qquad \forall x \in [a,b) \]
\[ F_-'(x) = f_-(x) = g_-(x) = g_-(x) \qquad \forall x \in (a,b] \]


\section{Sheet 5, Exercise 3}
\begin{ex}
  \begin{enumerate}
    \item Let $f: [a,b] \to \mathbb R$ continuously differentiable with $f(x) \neq 0 \forall x \in [a,b]$. Show that
      \[ \int_a^b \frac{f'(x)}{f(x)} \, dx = \ln{\card{f(b)}} - \ln{\card{f(a)}} \]
    \item Determine the value of $I$ using $\cos(x) = \frac12 (\sin{x} + \cos{x} + \cos{x} - \sin{x})$
      \[ I \coloneqq \int_0^{\frac\pi2} \frac{\cos{x}}{\sin{x} + \cos{x}} \, dx \]
    \item Determine $I$ using the substitution $y(x) = \frac\pi2 - x$.
  \end{enumerate}
\end{ex}

\section{Sheet 5, Exercise 3a}
\[ \int_a^b \frac{f'(x)}{f(x)} \, dx = \begin{vmatrix} t = f(x) \\ dt = f'(x) \, dx \end{vmatrix} = \int_{f(a)}^{f(b)} \frac{1}{t} \, dt \]
\[ = \left[\ln\card{t}\right]_{f(a)}^{f(b)} = \ln\card{f(b)} - \ln\card{f(a)} \]

\section{Sheet 5, Exercise 3b}
\[
  \int_0^{\frac\pi2} \frac{\cos(x)}{\sin(x) + \cos(x)} = \underbrace{\frac12 \int_0^{\frac\pi2} \frac{\sin(x) + \cos(x)}{\sin(x) + \cos(x)}}_{\frac\pi4} + \frac12 \int_0^{\frac\pi2} \underbrace{\frac{\cos(x) - \sin(x)}{\cos(x) + \sin(x)}}_{f(x)}
\] \[
  = \frac\pi4 + \ln{\card{\cos(\frac\pi4) + \sin(\frac\pi2)} - \ln\card{\cos(0) + \sin(0)}}
\] \[
  = \frac\pi4 + 0
\]

\section{Sheet 5, Exercise 3c}
\[ u(x) = \frac\pi2 - x \]
\begin{align*}
  \int_0^{\frac\pi2} \frac{\cos(x)}{\sin(x) + \cos(x)} \, dx
  &= \int_{\frac\pi2}^0 -\frac{\cos(\frac\pi2 - u)}{\sin(\frac\pi2 - u) + \cos(\frac\pi2 - u)} \, du \\
  &= \int_0^{\frac\pi2} \frac{\cos(\frac\pi2 - u)}{\sin(\frac\pi2 - u) + \cos(\frac\pi2 - u)} \, du \\
  &= \int_0^{\frac\pi2} \frac{\sin(u)}{\sin(u) + \cos(u)} \, du \\
  \implies 2I &= \int_0^{\frac\pi2} \frac{\sin(u)}{\sin(u) + \cos(u)} \, du + \int_0^{\frac\pi2} \frac{\cos(u)}{\sin(u) + \cos(u)} \, du \\
  2I &= \int_0^{\frac\pi2} \frac{\sin(u) + \cos(u)}{\sin(u) + \cos(u)} \, du \\
  2I = \frac\pi2 &\iff I = \frac\pi4
\end{align*}

\section{Sheet 5, Exercise 4}
\begin{ex}
  \begin{enumerate}
    \item Evaluate using integration by parts: $\int_0^\pi (\sin{x})^2 \, dx$
    \item Determine (for $n \in \mathbb N$) by integration by parts: $\int_0^{\frac\pi2} (\cos{x})^{2n} \, dx$
    \item Determine by integration by parts followed by substitution: $\int_0^1 \log(x  + 1) \, dx$
  \end{enumerate}
\end{ex}

\section{Sheet 5, Exercise 4a}

Let $u \coloneqq \sin(x)$, $u' = \cos(x)$, $v' \coloneqq \sin(x)$ and $v = -\cos(x)$.

\begin{align*}
  \int_0^\pi (\sin(x))^2 \, dx
    &= \left[-\sin(x) \cos(x)\right]_0^\pi - \int_0^\pi -\cos(x) \cos(x) \, dx \\
    &= \int_0^\infty 1 - \int_0^\pi \sin(x)^2 \, dx \\
  \iff \int_0^\pi 2 \cdot \sin(x)^2 \, dx = \int_0^\infty 1 = \pi \\
    &= \frac\pi2
\end{align*}

\section{Sheet 5, Exercise 4b}

Let $n \in \mathbb N \setminus \set{0}$.
\[ \int_0^{\frac\pi2} (\cos(x))^{2n} \, dx \]

We prove by complete induction:
Consider $n = 0$.
\[ \int_0^{\frac\pi2} (\cos(x))^{2n} \, dx = \frac\pi2 \]

Consider $n - 1 \to n$.
\[ \int_0^{\frac\pi2} \cos(x)^{2n+2} \, dx = \int_0^{\frac\pi2} \underbrace{\cos(x)^{2n+1}}_{u} \underbrace{\cos(x)}_{v'} \, dx \]
\begin{align*}
  \int_0^{\frac\pi2} (\cos(x))^2 &= \frac\pi4 \\
  \text{By induction hypothesis } \int_0^{\frac\pi2} \cos(x)^{2n} \, dx &= \frac{2n-1}{2n} \int_0^{\frac\pi2} \cos(x)^{2(n-1)} \\
    &= \begin{vmatrix}
      u' &= -(2n + 1) \sin(x) \cos(x)^{2n} \\
      v &= \sin(x)
    \end{vmatrix}
\end{align*}


\[
  [\cos(x)^{2n+1} \cdot \sin(x)]_0^{\frac\pi2} + (2n+1) \cdot \int_0^{\frac\pi2} \cos(x)^{2n} \cdot \sin(x)^2 \, dx
    = (2n + 1) \cdot \int_0^{\frac\infty2} \cos(x)^{2n} \, dx - (2n+1) \int_0^{\frac\pi2} \cos(x)^{2n+2} \, dx
\] \[
  \implies (2n + 2) \int_0^{\frac\pi2} \cos(x)^{2n+2} \, dx = (2n + 1) \int_0^{\frac\pi2} \cos(x)^{2n} \, dx
\] \[
  \implies \int_0^{\frac\pi2} \cos(x)^{2n+2} \, dx = \frac{(2n+1)}{2n+2}\int_0^{\frac\pi2} \cos(x)^{2n} \, dx
\] \[
  \frac{2n-1}{2n} \cdot \frac{2n-3}{2n-2} \cdot \ldots \cdot \frac12 \cdot \frac\pi2
\]

\section{Sheet 5, Exercise 4c}

\[
  \int_0^1 x \cdot \log(x + 1) \, dx
    = \begin{vmatrix}
      u' = x  \qquad & u = \frac{x^2}{2}   \\
      v = \log(x+1) \qquad & v' = \frac1{1+x}
    \end{vmatrix}
\] \[
  \left[\frac{x^2}{2} \log(x+1)\right]_0^1 - \int_0^1 \left(\frac{x^2}2 \cdot \frac{1}{1+x}\right) \, dx \qquad u(x) = 1 + x
\] \[
  = \left[\frac{x^2}{2} \log(x+1)\right]_0^1 - \frac12 \underbrace{\int_1^2 (u - 1)^2 \cdot \frac1u \, du}_{\int_1^2 \left(\frac{u^2 + 1 - 2u}{u}\right) \, du = \int_1^2 u + \frac1u - 2 \, du}
\] \[
  \frac{\log(2)}{2} - \frac12 \left[\frac{u^2}{2} + \log(u) - 2u\right]_1^2 = \frac14
\]

It is valid to assume that $\log = \ln$ in this exercise, because it is not specified otherwise.
But you can also consider a factor $a$, which normalizes it to $\ln$.

\end{document}