\documentclass{article}
%\usepackage[top=30pt,left=30pt,right=30pt]{geometry}
\usepackage[german,english]{babel}
\usepackage[utf8]{inputenc}
\usepackage{algpseudocode}
\usepackage{algorithm}
\usepackage{graphicx}
\usepackage{caption}
\usepackage{subcaption}
\usepackage{amsmath}
\usepackage{amssymb}
\usepackage{enumitem}
\usepackage{amsthm}
\usepackage{pxfonts}
\usepackage{wasysym}
\usepackage{framed}
\usepackage{xcolor}
\usepackage{makeidx}
\usepackage{csquotes}
\usepackage[pdfborder={0 0 0}]{hyperref}
\usepackage{stmaryrd}
\usepackage{titlesec}
\titleformat{\paragraph}{\normalfont\itshape}{}{}{}

\newtheorem{ex}{Exercise} %  TODO

\newtheorem{theorem}{Theorem}  \numberwithin{theorem}{section}
\newtheorem{problem}{Problem}  \numberwithin{problem}{section}
\newtheorem{example}{Example}  \numberwithin{example}{section}
\newtheorem*{hypothesis}{Hypothesis}%  \numberwithin{hypothesis}{section}
\newtheorem{definition}{Definition}  \numberwithin{definition}{section}
\newtheorem{lemma}{Lemma}  \numberwithin{lemma}{section}
\newtheorem*{claim}{Claim}%  \numberwithin{claim}{section}
\newtheorem{remark}{Remark}  \numberwithin{remark}{section}
\newtheorem*{corollary}{Corollary}%  \numberwithin{corollary}{section}
\newtheorem{proposition}{Proposition}  \numberwithin{proposition}{section}

\algnewcommand{\algorithmicgoto}{\textbf{go to}}%
\algnewcommand{\Goto}[1]{\algorithmicgoto~\ref{#1}}%
\algrenewcommand{\algorithmiccomment}[1]{\hskip2em$\triangleright$ {\footnotesize #1}}

% definitions
\newcommand{\drawing}[1]{%
 \begin{figure}[t]
  \begin{center}
   \includegraphics{#1}
  \end{center}
 \end{figure}
}
\newcommand{\pic}[2]{%
 \begin{figure}[t]
  \begin{center}
   \includegraphics{#1}
   \caption{#2}
  \end{center}
 \end{figure}
}
\newcommand{\set}[1]{\left\{#1\right\}}
\newcommand{\setdef}[2]{\left\{\left.#1\,\right|\,#2\right\}}
\newcommand{\angel}[1]{\left\langle#1\right\rangle}
\newcommand{\norm}[1]{\left\|#1\right\|}
\newcommand{\card}[1]{\left|#1\right|}
\newcommand{\given}[1]{\textbf{Given.} #1\par}
\newcommand{\find}[1]{\textbf{Find.} #1\par}
\newcommand{\dateref}[1]{\paragraph{\textit{This lecture took place on #1.}}}
\newcommand{\exist}{\;\exists\,}
\newcommand{\fall}{\;\forall\,}
\newcommand{\noproof}[1]{A proof for Theorem~\ref{#1} is not provided.}
\newcommand{\vectwo}[2]{\begin{pmatrix} #1 \\ #2 \end{pmatrix}}
\makeatletter
\newcommand{\xRightarrow}[2][]{\ext@arrow 0359\Rightarrowfill@{#1}{#2}}
\makeatother

\newcommand{\mtn}{(\mu\times\nu)} % mu times nu

\DeclareMathOperator{\rank}{rank}
\DeclareMathOperator{\detm}{det}
\DeclareMathOperator{\perm}{det}
\DeclareMathOperator{\sign}{sign}
\DeclareMathOperator{\degree}{deg}
\DeclareMathOperator{\prop}{probability}
\DeclareMathOperator{\argmax}{argmax}
\DeclareMathOperator{\argmin}{argmin}
\DeclareMathOperator{\vol}{vol}  % volume
\DeclareMathOperator*{\bigtimes}{\vartimes}

\makeatletter
\providecommand*{\dotcup}{%
  \mathbin{%
    \mathpalette\@dotcup{}%
  }%
}
\newcommand*{\@dotcup}[2]{%
  \ooalign{%
    $\m@th#1\cup$\cr
    \hidewidth$\m@th#1\cdot$\hidewidth
  }%
}
\makeatother


% metadata
\title{
  Analysis 2 Practicals \\
  \large{Notes, University (of Technology) Graz} \\
  based on the lecture by Wolfgang Ring
}
\date{\today}
\author{Lukas Prokop}

% settings
\parindent0pt
\setlength{\parskip}{0.4\baselineskip}
%\setcounter{tocdepth}{2}

\begin{document}
\maketitle
\tableofcontents

\section{Practicals}
\begin{itemize}
  \item Florian Kruse
  \item Analysis 2 practicals, every Thu, 15:00--16:30
  \item Sprechstunde: Tue, 14--15
\end{itemize}

\section{Sheet 1, Exercise 1}

\begin{ex}
  The Euclidean norm of $v = (v^1, v^2, \dots, v^n)^T \in \mathbb R^n$ is defined as
  \[ \norm{v}_2 \coloneqq \sqrt{(v^1)^2 + (v^2)^2 + \ldots + (v^n)^2} \]
  Show: A sequence $(x_k) \subset \mathbb R^n$ converges in regards of the Euclidean norm to $x \in \mathbb R$ iff they converge componentwise to $x$
  \[ \lim_{k\to\infty} \norm{x_k - x}_2 = 0 \iff \forall j \in \set{1,\dots,n}: \lim_{k\to\infty} x_k^j = x^j \]
\end{ex}

Direction $\Rightarrow$.

Let $\lim_{k \to \infty} \norm{x_k - x} = 0$.

Consider: $\card{x_{jk} - x_j}$ for arbitrary $j \in \set{1,\dots,n}$.

It holds that
\[ 0 \leq \card{x_{jk} - x} = \sqrt{(x_{jk} - x_j)^2} \leq \sqrt{(x_{1k} - x_1)^2 + \dots + (x_{1k} - x_n)} = \norm{x_k - x} \to 0 \]
\[ \implies \lim_{k\to\infty} \card{x_{jk} - x_j} = 0 \forall j \]

Direction $\Leftarrow$.

Let $\lim_{k\to\infty} x_{jk} = x_j \forall j \in \set{1,\dots,n}$.

The square root function is continuous.
\[ \lim_{k\to\infty} \norm{x_k - x} = \sqrt{(x_{1k} - x_1)^2 + \dots + (x_{1k} - x_n)^2} \]
\[ \sqrt{(\lim_{k\to\infty} x_{1k})^2 - 2 (\lim_{k\to\infty} x_i k) x_1 + x_{1j}^2 + \dots + (\lim_{k\to\infty} x_{n_k})^2 - 2 (\lim{x_{n_k}}) x_n + x_n^2} \]
\[ = \sqrt{\underbrace{x_1^2 - 2x_1^2 + x_1^2}_{= 0} + \dots + \underbrace{x_n^2 - 2x_n^2 + x_n^2}_{= 0}} = 0 \]

\textbf{Remark:}
In $\mathbb R^n$, all norms are equivalent.
This exercise showed this property.
So it you pick two numbers in $\mathbb R^n$ and they get \enquote{closer}, they get \enquote{closer} in every norm.

\section{Sheet 1, Exercise 2}

\begin{ex}
  In the lecture, we discussed the SCNF. $d_{\text{SCNF}}: \mathbb R^2 \times \mathbb R^2 \to \mathbb R$.
  For some fixed $p \in \mathbb R^2$ it is defined as
  \[
    d_{\text{SCNF}} \coloneqq \begin{cases}
      \norm{x - y}_2 & \text{ if } \exists \lambda > 0: y = p + \lambda (x - p) \\
      \norm{x - p}_2 + \norm{y - p}_2 & \text{ else}
    \end{cases}
  \]
  For $p \coloneqq (0,0)^T$ and $x \coloneqq (1,1)^T$, sketch the set $B_R(x)$ for $R=1$ and $R=2$.
  \[ B_R(x) \coloneqq \setdef{y \in \mathbb R^2}{d_{\text{SCNF}} < R} \]
\end{ex}

\section{Sheet 1, Exercise 3}

\begin{ex}
  Let $(M, d)$ be a metric space and $x \in M$.
  Furthermore let $(x_k) \subset M$ be a sequence with property that every subsequence of $(x_k)$ contains a subsequence converging to $x$.
  Prove by contradiction, that $(x_k)$ converges to $x$.
\end{ex}

$x_0 \not\to x$.

There exists $\varepsilon_0 > 0$ for infinitely many $n \in \mathbb N: d(x_n, x) \geq \varepsilon_0$.
Choose a subsequence $(x_{u_j})_{j\in\mathbb N}$ with $d(x_{n_j}, x) \geq \varepsilon_0 \forall j \in \mathbb N$.
Then there does not exist a subsequence of $(x_{n_j})$ with limit $x$.

\section{Sheet 1, Exercise 4}

\begin{ex}
  Let $(M, d)$ be a metric space and complete space. The diameter of a nonempty set $A \subset M$ is given by
  \[ \operatorname{diam}(A) \coloneqq \sup\setdef{d(x,y)}{x,y \in A} \]
  Let $(A_j)_{j\in\mathbb N}$ be a sequence of nonempty, closed sets in $M$ with $A_{j+1} \subset A_j$ for all $j \in \mathbb N$.
  Furthermore it holds that $\operatorname{diam}(A_j) \to 0$ for $j \to \infty$. Prove that $x \in M$ exists with
  $\bigcap_{j=1}^\infty A_j = \set{x}$ and that $x$ is unique.
\end{ex}

$A_j \subseteq M$, because its a complete, metric space.
\[ \implies \bigcap_{j=1}^\infty A_j \neq \emptyset \iff \exists x_0 \in M: \forall j \]
Assume $\exists y_0 \in M: y_0 \neq x_0 \implies d(y_0, x_0) \geq \varepsilon > 0$
\[ \forall j \in \mathbb N: \operatorname{diam}(A_j) \geq \varepsilon \]
This is a contradiction.
However, this is not the equality, we are looking for.
Assume $\bigcap_{j=1}^\infty A_j = \set{x_0} = \set{y_0} \implies x_0 = y_0$.
This is the equality, that was meant to be proven.

\subsection{Prove $\bigcap_{j=1}^\infty A_j \neq \emptyset \iff \exists x_0 \in M: \forall j$}

\textbf{Hint:} If the assignment mentions that completeness must be proven, usually you have to construct a Cauchy sequence.

Construct $(x_j)_{j \in \mathbb N}$. Choose for $x_j$ some element of $A_j$.
Choose $x_j \in A_j$ for $j \in \mathbb N$.
This defines a Cauchy sequence $(x_j)_{j \in \mathbb N}$.
Let $j \in \mathbb N$.
$x_i \in A_j \supset A_{j+1}$ and $x_{j+1} \in A_{j+1} \forall i \in \mathbb N$.
\[ \implies d(x_j, x_{j+i}) \leq \operatorname{diam}(A_j) \forall i \in \mathbb N \]
where $\operatorname{diam}(A_j) \to 0$ with $i \to \infty$.
\[ \implies \exists x \in M: \lim_{j \to \infty}(x_j) = x \]
Because $(x_j)_{j\geq J} \subseteq A_j$ and $\lim_{j\to\infty} (x_j)_{j\geq J} = x$,
it follows that $x \in A_j$ and then it follows that $x \in \bigcap_{j=1}^\infty A_j$.

\dateref{2018/03/22}

\section{Sheet 2, Exercise 1}

\subsection{Blackboard solution}

Let $B$ be bounded.
\[ \operatorname{diam}(B) < \infty \qquad \operatorname{diam}(B) = \operatorname{sup}(\setdef{d(x,y)}{x,y \in B}) \]
\[ d(B_k, B_{k+1}) = \operatorname{inf}(\setdef{d(x,y)}{x \in B_k, y \in B_{k+1}}) \]

Exercise (a).

Prove:
\[ \sum_{k=1}^\infty \operatorname{diam}(B_k) < \infty \land \sum_{k=1}^\infty d(B_k, B_{k+1}) \implies \operatorname{diam}(\bigcup_{k=1}^{\infty} B_k) < \infty \]
\[ \operatorname{diam}(B_k \cup B_{k+1}) \leq \operatorname{diam}(B_k) + d(B_k, B_{k+1}) + \operatorname{diam}(B_{k+1}) \]
We distinguish 3 cases:
\begin{enumerate}
  \item $x \in B_k, y \in B_k: d(x,y) \leq \operatorname{diam}(B_k) \leq \operatorname{diam}(B_k) + d(B_k, B_{k+1}) + \operatorname{diam}(B_{k+1})$
  \item $x \in B_{k+1}, y \in B_{k+1}, d(x, y) \leq \operatorname{diam}(B_k) + d(B_k, B_{k+1}) + \operatorname{diam}(B_{k+1})$
  \item $\forall x \in B_{k} \forall y \in B_{k+1}$
\end{enumerate}
Choose $x_0$ and $y_0$ on the border of sets $B_k$ and $B_{k+1}$ respectively.
But $x_0, y_0$ do not necessarily exist if compactness is not given.
But let $\varepsilon > 0$. Find $x_0, y_0$ with $d(x_0, y_0) \leq d(B_k, B_{k+1}) + \varepsilon$.
\[ d(x,y) \leq \underbrace{d(x,x_0)}_{\leq \operatorname{diam}(B_k)} + \underbrace{d(x_0, y_0)}_{\leq d(B_k, B_{k+1}) + \varepsilon} + \underbrace{d(x_0, y)}_{\leq \operatorname{diam(B_k)}} \leq \operatorname{diam}(B_k) + d(B_k, B_{k+1}) + \operatorname{diam}(B_{k+1}) + \varepsilon \]

Laurent Pfeiffer continued the following solution (until Exercise 2):

\[ \operatorname{diam}((B_k \cup B_{k+1}) \cup B_{k+2}) \leq \operatorname{diam}(B_k \cup B_{k+1}) + \underbrace{d((B_k \cup B_{k+1}), B_{k+2})}_{\leq d(B_{k+1}, B_{k+2})} + \operatorname{diam}(B_{k+2}) \]
\[ \leq \operatorname{diam}(B_k) + d(B_k, B_{k+1}) + \operatorname{diam}(B_{k+1}) + d((B_k \cup B_{k+1}), B_{k+2}) + \operatorname{diam}(B_{k+2}) \]
By induction it follows that
\[ \operatorname{diam}(B_k \cup B_{k+1} \cup \dots \cup B_n) \leq \operatorname{diam}(B_k) + d(B_k, B_{k+1}) + \operatorname{diam}(B_{k+1})  + d(B_{k+2}) + d(B_{n-1}, B_n) + \operatorname{diam}(B_n) \]
\[ \operatorname{diam}(B_k \cup \dots \cup B_n) \leq \underbrace{\sum_{i=1}^n \operatorname{diam}(B_i) + d(B_i, B_{i+1})}_{D} \]

Choose $x,y \in \bigcup_{i=1}^\infty B_i$. Then there exists some $k \in \mathbb N$ such that $x \in B_k$. There exists $n$ such that $y \in B_n$.
\[ d(x,y) \leq \operatorname{diam}(B_k) + \dots + \operatorname{diam}(B_n) \leq D \]

Exercise (b).

Let $x \in M$. We define: $B_{k+1} = B_{k+2} = \dots = \set{x}$.
For all $i \geq k$ it holds that
\[ \operatorname{diam}(B_i) = 0 \]
\[ d(B_i, B_{i+1}) = 0 \]
Therefore,
\[ \sum_{i=1}^\infty \operatorname{diam}(B_i) = \sum_{i=1}^k \underbrace{\operatorname{diam}(B_i)}_{<+\infty} < +\infty \]
What about the distances?
\[ \int_{i=1}^\infty d(B_i, B_{i+1}) = \sum_{i=1}^k d(B_i, B_{i+1}) < +\infty \]
By (a), it follows that
\[ \left(\bigcup_{i=1}^\infty B_i\right) \text{ is bounded } \implies \left(\bigcup_{i=1}^k B_i\right) \subseteq \left(\bigcup_{i=1}^\infty B_i\right) \text{ is also bounded}  \]

Exercise (c).

We define
\[ B_i = \left[\sum_{j=1}^i \frac1j , \sum_{j=1}^{i+1} \frac1j\right] \]
Then it holds that
\[ \operatorname{diam}(B_i) = \frac1{i+1} \xRightarrow{i\to\infty} 0 \]
\[ \sum_{i=1}^\infty \operatorname{diam}(B_i) = \infty \]
\[ B_i \cap B_{i+1} = \set{\sum_{j=1}^{i+1} \frac1j} \implies d(B_i, B_{i+1}) = 0 \]
\[ B_1 \cup \dots \cup B_i = \left[1, \underbrace{\sum_{j=1}^{i+1} \frac1j}_{\to\infty}\right] \implies \underbrace{\bigcup_{i=1}^\infty B_i}_{\text{not bounded}} = [1,\infty) \]

We define $B_i = \set{\sum_{j=1}^i \frac1{j}}$. For all $i$:
\begin{itemize}
  \item $\operatorname{diam}(B_i) = 0 \implies \sum_{i=1}^\infty \operatorname{diam}(B_i) = 0$
  \item
    \[ d(B_i, B_{i+1}) = \left(\sum_{j=1}^{i+1} \frac1j\right) - \left(\sum_{j=1}^i \frac1j\right) = \frac1{i+1} \xRightarrow{i\to\infty} 0 \]
    \[ \sum_{i=1}^\infty d(B_i, B_{i+1}) = \sum_{i=1}^\infty \frac{1}{i+1} = \infty \]
    The union is \emph{not} bounded, because $\sum_{j=1}^i \frac1j \in \bigcup_{j=1}^\infty B_j$.
\end{itemize}

\section{Sheet 2, Exercise 2}

\begin{ex}
  Let $(X, d)$ be a sequentially compact, metric space. Show:
  \begin{enumerate}
    \item[a.] $X$ is bounded.
    \item[b.]
  \end{enumerate}
\end{ex}

\subsection{Blackboard solution}

Exercise (a).

Let $X$ be unbounded. Hence, there exists a tuple $(x_N, y_N) \in X \times X$ for every $N \in \mathbb N$ with $d(x_N, y_N) > N$.
Because $(X, d)$ is sequentially compact, there exists a convergent subsequence $(x_{N_k}, y_{N_{k_i}})$ we can choose such that
\[ \lim_{k\to\infty} x_{N_k} = \infty \qquad \lim_{i\to\infty} y_{N_{K_i}} = y_0 \qquad \lim_{i\to\infty} (x_{N_{k_i}}) = x_0 \]
\[ \implies \underbrace{N_{k_i}}_{\xrightarrow{i\to\infty} \infty} < d(x_{N_{k_i}}, y_{N_{k_i}}) \xrightarrow{i\to\infty} d(x_0, y_0) \]
By this contradiction, it follows that $X$ is bounded.

Exercise (b).

Let $(x_n)_{n\in\mathbb N}$ be a Cauchy sequence in $X$. Let $X$ be sequence compact $\implies$ there exists a convergent subsequence $x_{n_{k}} \xrightarrow{k\to\infty} x \in X$. Show that $x_n \xrightarrow{n\to\infty} x$.

Let $\varepsilon > 0$ be arbitrary. Choose $N \in \mathbb N$ such that $\forall n,m \geq N: d(x_n, x_m) < \frac\varepsilon2$.
Choose $k \in \mathbb N$ such that $n_k \geq N$ and $d(x_{n_k}, x) < \frac\varepsilon2$.
\[ \forall n \geq n_k: d(x, x_n) \leq d(x, x_{n_k}) + d(x_{n_k}, x_n) < \varepsilon \]

Exercise (c).

Show that $A \subset X$ is sequentially compact iff $A$ is closed.

\begin{description}
  \item[$\Rightarrow$]
    Let $(x_n)_{n \in \mathbb N}$ be a convergent sequence, $(x_n)_{n \in \mathbb N} \subset A$, $\lim_{n\to\infty} x_n = x_0 \in X$.
    Show that $x_0 \in A$.

    Set $A$ is sequentially compact. Choose subsequence $(x_{n_k})_{k \in \mathbb N} \subset A$, $\lim_{k\to\infty} x_{n_k} = x_0 \in A \implies A$ is closed.
  \item[$\Leftarrow$]
    $A$ is closed. Show that $A$ is sequentially compact.

    Let $(x_n)_{n\in\mathbb N} \subset A$ and there exists subsequence $(x_{n_k})_{k \in \mathbb N}$ with $\lim_{k\to\infty} x_{n_k} = x_0 \in X$, because $X$ is sequentially compact.
    $(x_{n_k})_{k \in \mathbb N} \subset A \implies A$ is sequentially compact.
\end{description}

\section{Sheet 2, Exercise 2}

\begin{ex}
  Let $f: \mathbb R \to \mathbb R$, $f(x) = \sqrt{1 + x^2}$.
  \begin{enumerate}
    \item Show that $\card{f(x) - f(y)} < \card{x - y} \forall x, y \in \mathbb R$ with $x \neq y$
    \item Investigate which conditions of Banach's Fixed Point Theorem are [not] met.
    \item Is Banach's Fixed Point Theorem applicable? Does $f$ have a fixed point?
  \end{enumerate}
\end{ex}

Exercise (a).

\[ \card{f(x) - f(y)} < \card{x - y} \qquad x,y \in \mathbb R, x \neq y \]
\[ \card{\sqrt{1 + x^2} - \sqrt{1 + y^2}} < \card{x - y} \]
\[ 1 + x^2 + 1 + y^2 - 2 \sqrt{(1 + x^2) (1 + y^2)} < x^2 + y^2 - 2xy \]
\[ 2 - 2\sqrt{(1 + x^2)(1 + y^2)} < -2xy \]
\[ 1 + xy < \sqrt{(1 + x^2)(1 + y^2)} \]
We need to distinguish 2 cases here ($x$ and $y$ have same signum, $x$ and $y$ have different signum). This is trivial.
\[ 1 + 2xy + x^2 y^2 < 1 + x^2 + y^2 + x^2 y^2 \]
\[ 0 < x^2 + y^2 - 2xy \]
\[ 0 < (x - y)^2 \]

Exercise (b and c).

Let $x \in \mathbb R$.
\[ f(x) = x \]
\[ \sqrt{1 + x^2} = x \]
\[ 1 + x^2 = x^2 \]
\[ 1 = 0 \]

\dateref{2018/04/12}

\section{Sheet 3, Exercise 4}
\begin{ex}
  Let $(X, d)$ be a metric space and $x_0 \in X$. A function $f: X \to \mathbb R$ is called half-continuous from below in $x_0$, if for every $\varepsilon > 0$ some $\delta > 0$ exists, such that $d(x, x_0) < \delta$ implies $f(x_0) - f(x) < \varepsilon$. If $f$ is half-continuous from below in every $x_0 \in X$, then $f$ is called half-continuous from below.
\end{ex}

Obviously, continuity implies half-continuity.

\subsection{Sheet 3, Exercise 4a}
\begin{ex}
  Give some half-continuous from below $f: [-1, 1] \to \mathbb R$ such that $f$ is non-continuous.
\end{ex}

Let $f: [-1,1] \to \mathbb R$.
\[
  x \mapsto \begin{cases}
    -1 & x = -1 \\
    -x & x \neq -1
  \end{cases}
\]
\[ \underbrace{f(-1)}_{=-1} - \underbrace{f(x)}_{\geq -1} \leq 0 < \varepsilon \]

\subsection{Sheet 3, Exercise 4b}
\begin{ex}
  Give some half-continuous from below $f: [-1, 1] \to \mathbb R$, but does not have a maximum.
\end{ex}
Same $f$ can be chosen.

\subsection{Sheet 3, Exercise 4c}
\begin{ex}
  Give some half-continuous from below $f: [-1, 1] \to \mathbb R$, but does not have a minimum.
\end{ex}
$f$ as $f|_{[-1,1]}$ can be chosen.

\subsection{Sheet 3, Exercise 4d}
\begin{ex}
  Prove that every half-continuous from below function in a compact set has a minimum.
\end{ex}

\textbf{Hint:} It is assumed that cover-compactness seems to be more cumbersome than sequential compactness. \\
\textbf{Remark:} This is a generalization of the theorem, that every continuous, compact function has a minimum and maximum.

Let $K \subseteq X$ be compact. $f: K \to \mathbb R$ is half-continuous from below.

Show that $f^k = \operatorname{inf}(f(K)) \in f(K)$.

\[ \exists (x_n)_{n\in\mathbb N} \subseteq K \text{ with } f(x_n) - f^k < \frac1n \]
$K$ is compact. Hence, there exists $(x_{n_k})_{k \in \mathbb N}$ with $\lim_{k\to\infty} x_{n_k} \coloneqq x^* \in K$.
Let $\varepsilon > 0$ be arbitrary.
By half-continuity from below, it follows that $\exists \delta > 0: d(x^*, x) < \delta \implies f(x^*) - f(x) < \varepsilon$.
\[ \exists K \in \mathbb N \forall k \geq K: d(x^k, x_{n_k}) < \delta \implies f(x^k) - f(x_{n_k}) < \varepsilon \iff f(x^*) < f(x_{n_k}) + \varepsilon \]
\[ \implies f(x^*) \leq \lim_{k\to\infty} f(x_{n_k}) \implies f(x^*) \leq \lim_{n\to\infty} f(x_n) = f^* \]
\[ \implies f(x^*) = f^* \implies f^* \text{ is minimum of } f(X) \]

\section{Sheet 3, Exercise 3}
\begin{ex}
  Let $(X, d)$ and $(Y, e)$ be metric spaces, where $d: X \to \mathbb R$ is a discrete metric,
  hence
  \[
    d(x_1, x_2) = \begin{cases}
      0 & \text{if } x_1 = x_2 \\
      1 & \text{if } x_1 \neq x_2
    \end{cases}
  \]
\end{ex}

\subsection{Sheet 3, Exercise 3a}
\begin{ex}
  Every map $f: X \to Y$ is continuous.
\end{ex}

Let $f: X \to Y$ be arbitrary.
Let $x_0 \in X$ and $\varepsilon > 0$ be arbitrary.
Show that
\[ \exists \delta > 0: d(x, x_0) < \delta \implies d(f(x), f(x_0)) < \varepsilon \]
\[ K_{\frac12}(x_0) = \set{x_0} \]

\subsection{Sheet 3, Exercise 3b}
\begin{ex}
  A map $f: X \to Y$ is not necessarily bounded.
\end{ex}

$M \geq 0$ arbitrary. $\exists x,y \in f(X): e(x,y) > M$.

\[ f: \mathbb Z \to \mathbb Z \qquad x \mapsto x \]
\[ f(x) = \mathbb Z \qquad x = 0 \qquad y = M + 1 \]
$e = \card{\cdot}$.

\subsection{Sheet 3, Exercise 3c}
\begin{ex}
  Every map $g: Y \to X$ is bounded.
\end{ex}
Let $g: Y \to X$ be arbitrary.
Show that $\exists M \geq 0 \forall x,y \in g(Y): d(x,y) \leq M$.
Choose $M = 2$. $\forall x,y \in X: d(x,y) \leq 1 \leq 2$.

\subsection{Sheet 3, Exercise 3d}
\begin{ex}
  In case $(Y, e) = (\mathbb R, \card{\cdot})$, every non-constant map $g: Y \to X$ is non-continuous.
\end{ex}

We show: continuity implies constant.

Let $g: \mathbb R \to X$ continuous.
Let $x_0 \in \mathbb R$ be arbitrary and $\varepsilon = \frac12$.
$\exists \delta_0 > 0: \card{x_0 - x} < \delta \implies d(g(x_0), g(x)) < \frac12$
for $x_0 \in \mathbb R$ there exists $\delta_0$ such that 
$\forall x \in (x_0 - \delta, x_0 + \delta): g(x) = g(x_0)$.
\[ \operatorname{sup}\setdef{s \in [x_0, \infty)}{g(x) = g(x_0) \forall x \in [x_0, s)} \]

\section{Sheet 3, Exercise 2}
\begin{ex}
  Let $V$ be the vector space of bounded, complex sequences, hence
  \[
    V \coloneqq \setdef{(a_k)_{k \in \mathbb N} \subset \mathcal C}{\exists M \in \mathbb R \text{ with } \card{a_k} \leq M \forall k \in \mathbb N}
  \]
  additionally with norm
  \[ \norm{(a_k)_{k \in \mathbb N}}_{\infty} \coloneqq \sup\setdef{\card{a_k}}{k \in \mathbb N} \]
\end{ex}

This solution was done by Mr. Kruse himself.

\subsection{Sheet 3, Exercise 2b}
\begin{ex}
  The unit sphere in $(V, \norm{\cdot}_{\infty})$,
  \[ B_1(0) = \setdef{a \in V}{\norm{a}_{\infty} \leq 1} \]
  is closed and bounded, but not sequentially compact.
\end{ex}

We need to prove boundedness.

Let $C, D \in B_1(0)$.
\[ \implies \norm{\underbrace{C}_{= (c_k)} - \underbrace{D}_{=(d_k)}}_{\infty} \leq 2 \]
\[ \sup\set{\card{\underbrace{c_k - d_k}_{\leq \underbrace{\card{c_k}}_{\leq 1 \forall k} + \underbrace{\card{d_k}}_{\leq 1 \forall k} \leq 2 \forall k } }: k \in \mathbb N} \leq 2 \]

We need to prove closedness.

\[ (A^n)_{n \in \mathbb N} \subset B_1(0) \text{ with } \lim_{n\to\infty} A^n = A \]
Show that $A \in B_1(0)$.
\[ \text{For every } A^n \coloneqq (a^n_k)_{k \in \mathbb N} \text{ it holds that } \norm{\underbrace{(a^n_k)_{k \in \mathbb N}}_{= \sup\set{\card{a^n_k}: k \in \mathbb N} \leq 1}}_{\infty} \leq 1 \]

\[ (A^n)_{n \in \mathbb N} \subset B_1(0) \text{ with } \lim_{n\to\infty} A^n = A \]
\[ \iff \lim_{n \to \infty} \norm{A^n - A}_{\infty} = 0 \]

$\card{a^n_k}$ in
\[ \sup\set{\card{a^n_k}: k \in \mathbb N}  \]
converges to $\card{a_k} \leq 1$ for $n \to \infty$.

We need to prove sequentially non-compact of $B_1(0)$.
So we only need to find some sequence that does not have some converging subsequence.

We define
\[
  A^n \coloneqq (a^n_k)_{k \in \mathbb N} \coloneqq \begin{cases}
    0 & \text{ if } k \neq n \\
    1 & \text{ if } k = n
  \end{cases}
\]
for every $n \in \mathbb N$. As such we get a sequence
\[ \implies (A^n)_{n \in \mathbb N} \subset B_1(0) \]
but it holds that $\norm{A^n - A^m}_{\infty} = 1 \forall n \neq m$.
This is also not a Cauchy sequence.

\section{Sheet 3, Exercise 1}
\begin{ex}
  Let $(X,d)$ be a metric space. A set $K \subset X$ is called cover-compact,
  if for every family of open sets $(U_i)_{i \in I} \subset X$ with $K \subset \bigcup_{i \in I} U_i$ it holds that:
  There exists a finite set $J \subset I$ with $K \subset \bigcup_{i \in J} U_i$. Let $K \subset X$ be cover-compact.
\end{ex}

\subsection{Sheet 3, Exercise 1a}
\begin{ex}
  Show that $K$ is totally bounded, hence for every $r>0$, there exists $x_1, \dots, x_n$ in $K$
  with $K \subset \bigcup_{i=1}^n B_r(x_i)$.
\end{ex}

Construct a family of open spheres ($\left(\mathcal B_r(x)\right)_{x \in K} \subset K$ covering $K$).
By cover-compactness it follows there exists some finite $J \subset K$ with $K \subset \bigcup_{x \in J} B_r(x)$.

\subsection{Sheet 3, Exercise 1b}
\begin{ex}
  Prove that $K$ is sequentially compact.
\end{ex}

Proof by contradiction: Assume $K$ is not sequentially compact.

Then there exists a sequence $(x_n)_{n \in \mathbb N} \in K$ which has a subsequence $(x_{n_k})_{k \in \mathbb N} \to c \not\in K$.
\[ \forall x \in K: \exists r_x > 0: B_{r_x}(x) \text{ contains finitely many sequence elements} \]
Because $\bigcup_{x \in K} B_{r_x}(x) \supset K$ it holds: there exists $J \subset K$ finite $\bigcup_{x \in J} B_{r_x}(x) \supset K$.
This contradicts with $(x_n)_{n \in \mathbb N} \subset K$.

\subsection{Sheet 4, Exercise 1}
\begin{ex}
  Let $(M,d)$ be a complete metric space and $(A_k)_{k \in \mathbb N} \subset M$
  is a sequence of closed sets. Use Cantor's Theorem to prove:
  $\bigcup_{k \in \mathbb N} A_k$ contains an open set if at least one $A_k$
  contains an open set. Illustrate this statement for $(M,d) = (\mathbb R, \card{\cdot})$.
\end{ex}

First we illustrate it in $\mathbb R$.

\[ (A_k) = \set{a_k} \]
where $a_k \in \mathbb R$.

Consider some 

\section{Sheet 4, Exercise 2}
\begin{ex}
  Let $f: [-1,1] \to \mathbb C$ be continuous and $O \subset \mathbb C$ is an open set.
  In the lecture we have seen that $f^{-1}(O)$ is open.
  Review the result and prove for $O = \mathbb C$.
  \begin{enumerate}
    \item The set $O$ is open.
    \item It holds that $f^{-1}(O) = [-1,1]$
    \item The set $[-1,1] \subset \mathbb R$ is not open.
    \item The statement of the lecture about $f^{-1}(O)$ is still correct.
  \end{enumerate}
\end{ex}

\subsection{Sheet 4, Exercise 2a}
Show that $\mathbb C$ is open.

Let $z \in \mathbb C$. $\exists \varepsilon > 0$,
\[ B(z, \varepsilon) \subseteq \mathbb C \]

\subsection{Sheet 4, Exercise 2b}
Follows from the definition of a function.

\subsection{Sheet 4, Exercise 2c}
If it is an open set, there must be a neighborhood of arbitrary $\varepsilon$
such that this neighborhood is completely in the set.

Let $\varepsilon > 0$. Choose $x \in B(1, \varepsilon)$ with $x = 1 + \frac{\varepsilon}2$.
\[ \implies x \in B(1, \varepsilon) \land x \not\in [-1,1] \]

\subsection{Sheet 4, Exercise 2d}
Let $(X,d)$ and $(Y,e)$ be metric spaces and $f: X \to Y$ continuous
then $f^{-1}(O)$ is open $\forall O \subseteq Y$ open.

Show:
\[ \forall x \in [-1,1] \exists \varepsilon > 0: \underbrace{B(x, \varepsilon)}_{= \setdef{z \in [-1,1]}{d(x, z) < \varepsilon}} \subseteq [-1,1] \]
So the difference is the domain of $z$ ($[-1,1]$ unlike exercise c, where we used $\mathbb R$).

The point was to illustrate how to read the theorem properly.

\section{Sheet 4, Exercise 3}
\begin{ex}
  Let $\Omega$ be a non-empty set and $B(\Omega)$ the vector space of real-valued bounded functions on $\Omega$.
  Hence,
  \[ B(\Omega) \coloneqq \setdef{f: \Omega \to \mathbb R}{\exists M \in \mathbb R \text{ with } \card{f(x)} \leq M \forall x \in \Omega} \]
  with norm
  \[ \norm{f}_{\infty} \coloneqq \sup\setdef{\card{f(x)}}{x \in \Omega} \]
  Prove the following statements:
  \begin{enumerate}
    \item $(B(\Omega), \norm{\cdot}_{\infty})$ is a complete normed vector space.
    \item The unit circle $U$ in $B(\Omega)$ is closed and bounded.
      \[ U = \setdef{f \in B(\Omega)}{\norm{f}_{\infty} \leq 1} \]
    \item The unit circle is sequentially compact if and only if $\Omega$ is finite.
  \end{enumerate}
\end{ex}

\subsection{Sheet 4, Exercise 3a}
Given $\Omega \neq 0$.
\[ B(\Omega) \coloneqq \setdef{f: \Omega \to \mathbb R}{\exists M \in \mathbb R: \card{f(x)} \leq M \quad \forall x \in \Omega} \]

First, we show that $\norm{\cdot}_{\infty}$ is indeed a norm. We just show absolute homogeneity for illustrative purposes:
\begin{align*}
  \norm{\lambda f}_{\infty}
    &= \sup\setdef{\card{\lambda \cdot f(x)}}{x \in \Omega} \\
    &= \sup\setdef{\card{\lambda} \cdot \card{f(x)}}{x \in \Omega} \\
    &= \card{\lambda} \cdot \sup\set{\card{f(x)}}{x \in \Omega} \\
    &= \card{\lambda} \cdot \norm{f}
\end{align*}

We show completeness of $(B(\Omega), \norm{\cdot}_{\infty})$.
Equivalently, all Cauchy sequences in $B(\Omega)$ are convergent.
Equivalently, for all Cauchy sequences $(f_n)_{n \in \mathbb N}: \exists f \in B(\Omega): \norm{f_n - f}_{\infty} \to 0$ for $n \to \infty$.

Let $(f_n)_{n \in \mathbb N}$ be an arbitrary Cauchy sequence. Hence,
\[ \forall \varepsilon > 0 \exists N \in \mathbb N: n,m > N \implies \norm{f_n - f_m}_{\infty} = \sup\setdef{(f_n - f_m)(x)}{x \in \Omega} < \varepsilon \]
\[ \forall \varepsilon > 0: n,m > N \]
\[ \forall x \in \Omega: \card{(f_n - f_m)(x)} < \varepsilon \]
\[ \implies \forall x \in \Omega: (f_n(x))_{n \in \mathbb N} \subseteq R \]
is a Cauchy sequence in $\mathbb R$.
\[ \iff \forall x \in \Omega: (f_n(x))_{n \in \mathbb N} \text{ converges} \]
\[ \forall x \in \Omega: (f_n(x)))_{n\in\mathbb N} \to f(x) \forall \varepsilon > 0 \exists N \in \mathbb N: n > N \implies \card{f_n(x) - f(x)} < \varepsilon \]
\[ \exists N \in \mathbb N \forall n > N: \norm{f_n - f}_{\infty} < 1 \]
\[ \norm{f}_{\infty} = \norm{f - f_N + f_N}_{\infty} \leq \underbrace{\norm{f - f_N}_{\infty}}_{<1} + \underbrace{\norm{f_N}}_{\leq M} < 1 + M \]

\subsection{Sheet 4, Exercise 3b}
Let $K_1 \coloneqq \setdef{f \in B(\Omega)}{\norm{f}_{\infty} \leq 1}$.
Show $K_1$ is bounded and closed.

\subsubsection{$K_1$ is bounded}
Let $f,g \in K_1$ be arbitrary.
\[ \norm{f - g}_{\infty} \leq \norm{f}_{\infty} + \norm{g}_{\infty} \leq 1 + 1 = 2 \]
$2$ is a boundary and therefore $K_1$ is bounded.

\subsubsection{$K_1$ is closed}
Let $(f_n)_{n \in \mathbb N}$ be a convergent sequence in $K_1$
with $\lim_{n\to\infty} f_n = f \iff \lim_{n\to\infty} \norm{f_n - f} = 0$.

Show $f \in K_1$.

\[ \forall f_n \in K_1: \norm{f_n} \leq 1 \]
\[ \norm{f}_{\infty} = \norm{f - f_n}_{\infty} \leq \underbrace{\norm{f - f_n}_{\infty}}_{\xrightarrow{n \to \infty} 0} + \underbrace{\norm{f_n}_{\infty}}_{\leq 1} \leq 1 \]
\[ \implies \norm{f}_{\infty} \leq 1 \implies f \in K_1 \]

\subsection{Sheet 4, Exercise c}
$f$ is sequentially compact if and only if $\Omega$ is finite?
Equivalently, every sequence $(f_n)_{n \in \mathbb N} \subseteq K_1$ has a convergent subsequence with limit in $K_1$.

Direction $\implies$.

Let $\Omega$ be infinite. Then $\exists$ a sequence $(f_n)_{n \in \mathbb N}$ without convergent subsequence.
We build a sequence $(f_n)_{n \in \mathbb N}$ in $K_1$.

Let $(x_i)_{i \in \mathbb N}$ be an arbitrary sequence in $\Omega$ with $x_i \neq x_j \forall i \neq j$.
\[
  f_n(x) \coloneqq \begin{cases}
    1 & \text{ if } x = x_n \\
    0 & \text{ else}
  \end{cases}
\]
Then it holds that $\forall n \neq m$,
\[ \norm{f_n - f_m}_{\infty} = 1 \]

Assume there exists a convergent subsequence in $(f_{n_k})_{k \in \mathbb N}$ of $(f_n)_{n \in \mathbb N}$ with limit $f$.
\[ \implies \exists M > 0: k > M: \norm{f_{n_k} - f}_{\infty} < \frac12 \]
Let $k,l > M$ with $k \neq l$
\[ \implies \norm{f_{n_k} - f_{n_l}}_{\infty} \leq \norm{f_{n_k} - f}_{\infty} + \norm{f_{n_l} - f}_{\infty} < \frac12 + \frac12 = 1 \]
This is a contradiction to $\norm{f_n - f_m}_{\infty} = 1$.

Direction $\impliedby$.

Let $(f_n)_{n \in \mathbb N}$ be a sequence in $K_1$ without limit.
Let $n \in \mathbb N$.
\[ \Omega = \set{x_1, \ldots, x_n} \implies \card{\set{f_n(x_1), \ldots, f_n(x_n)}} < \infty \]
Let $f_n \in K_1 \implies \card{f_n(x_i)} \leq 1 \forall i \in \set{1, \ldots, m} \forall n \in \mathbb N$.

Consider $x_1 \in \Omega$.
\[
  (f_n(x_1)) = y_n^1 \in [-1,1]
\] \[
  [-1,1] \text{ compact }
  \implies (y_n^1)_{n \in \mathbb N} \text{ has convergent subsequence } (y^1_{n_k})_{k \in \mathbb N} \to \tilde y^1
\] \[
  (f_{n_k}(x_1))_{k \in \mathbb N} = (y^1_{n_k})_{k \in \mathbb N} \to \tilde y_1 \coloneqq f(x_1)
\]
and this goes on up to
\[ (f_{n_{\ddots_z}}(x_m))_{z \in \mathbb N} \to f(x_m) \]
For every $\varepsilon > 0$
\[ \exists N_1: \forall n \in N_1: \card{f_{n_{\ddots_{2}}}(x_1) - f(x_1)} < \varepsilon \]
\[ \vdots \]
\[ \exists N_m: \forall n \in N_m: \card{f_{n_{\ddots_{2}}}(x_m) - f(x_m)} < \varepsilon \]

Choose $N \coloneqq \max{N_1, \ldots, N_m}$. For all $n \geq N$,
\[ \implies \norm{f_{n_{\ddots_{2}}}}_{\infty} < \varepsilon \]

\section{Sheet 4, Exercise 4}
\begin{ex}
  Let $k \in \mathbb N$. Show: $\exists \phi_k: \sqrt{k\pi} \leq \xi_k \leq \sqrt{(k+1)\pi}$ such that
  \[ \int_{\sqrt{k\pi}}^{\sqrt{(k+1)\pi}} \sin(x^2) \, dx = \frac{(-1)^k}{\xi_k} \]
\end{ex}

\[
  \int_{\sqrt{k\pi}}^{\sqrt{(k+1)\pi}} \sin(x^2) \, dx
  = \int_{\sqrt{k\pi}}^{\sqrt{(k+1) \pi}} \frac{x \cdot \sin(x^2)}{x} \, dx
  = \frac{1}{\xi_k} \cdot \int_{\sqrt{k\pi}}^{\sqrt{(k+1)\pi}} x \cdot \sin(x^2) \, dx
\]
But this IVT is unconventional.
\[
  = \left. \frac{1}{\xi_k} \cdot \left(-\frac12 \cdot \cos(x^2)\right) \right|_{\sqrt{k\pi}}^{\sqrt{(k+1)\pi}}
\]

If $k$ is even:
\[ \frac{1}{\xi_k} \left(\frac12 + \frac12\right) = \frac{1}{\xi_k} \]
If $k$ is odd:
\[ \frac{1}{\xi_k} \left(-\frac12 - \frac12\right) = -\frac{1}{\xi_k} \]

This implies a boundary of
\[ \frac{(-1)^k}{\xi_k} \]

\dateref{2018/04/26}

\section{Sheet 5, Exercise 1}
\begin{ex}
  Let $\mathcal R[a,b]$ be the vector space of real-valued regulated functions on $[a,b] \subseteq \mathbb R$, hence
  \[ \mathcal R[a,b] \coloneqq \setdef{f: [a,b] \to \mathbb R}{f \text{ is a regulated function}} \]
  annotated with a norm $\norm{\cdot}_\infty$ of Sheet 4 Exercise 3.
  Prove that $(\mathcal R[a,b], \norm{\cdot}_\infty)$
  is a complete normed vector space with a sequentially non-compact unit sphere.
\end{ex}

\section{Sheet 5, Exercise 2}
\begin{ex}
  Let $f, b \in \mathcal R[a,b]$ with
  \[ f_+(x) = g_+(x) \qquad \forall x \in [a,b) \]
  \[ f_-(x) = g_-(x) \qquad \forall x \in (a,b] \]
  \begin{enumerate}
    \item For $\alpha,\beta \in [a,b]: \int_\alpha^\beta f(x) \, dx = \int_\alpha^\beta g(x) \, dx$ holds.
    \item For every antiderivative $F: [a,b] \to \mathbb R$ of $f$ there exists an antiderivative $G: [a,b] \to \mathbb R$ of $g$ with $F(x) = G(x)$ for all $x \in [a,b]$.
  \end{enumerate}
\end{ex}

\subsection{Sheet 5, Exercise 2a}

Let $f,g \in \mathcal R[a,b]$.
\[ F'_+(x) \coloneqq f_+(x) = g_+(x) \]
\[ F'_-(x) \coloneqq f_-(x) = g_-(x) \]
Show: $\int_\alpha^\beta f(x) \, dx = \int_\alpha^\beta g(x) \, dx$.

In general $f_+(x) \neq f(x) \neq f_-(x)$.

\[ F \coloneqq \int f(x) \, dx \]
\[ G \coloneqq \int g(x) \, dx \]

\[ \int_\alpha^\beta f(x) \, dx = \left. F \right|_\alpha^\beta \overset{(b)}{=} \underbrace{F(\beta) + K}_{G(\beta)} - \underbrace{(F(\alpha) - K)}_{G(\alpha)} = \int_\alpha^\beta g(x) \, dx \]

\subsection{Sheet 5, Exercise 2b}
$F$ is an antiderivative of $f$ if and only if
\[ F = \int f(x) \, dx \]
\[ F_+'(x) = f_+(x) = g_+(x) = g_+(x) \qquad \forall x \in [a,b) \]
\[ F_-'(x) = f_-(x) = g_-(x) = g_-(x) \qquad \forall x \in (a,b] \]

\section{Sheet 5, Exercise 3}
\begin{ex}
  \begin{enumerate}
    \item Let $f: [a,b] \to \mathbb R$ continuously differentiable with $f(x) \neq 0 \forall x \in [a,b]$. Show that
      \[ \int_a^b \frac{f'(x)}{f(x)} \, dx = \ln{\card{f(b)}} - \ln{\card{f(a)}} \]
    \item Determine the value of $I$ using $\cos(x) = \frac12 (\sin{x} + \cos{x} + \cos{x} - \sin{x})$
      \[ I \coloneqq \int_0^{\frac\pi2} \frac{\cos{x}}{\sin{x} + \cos{x}} \, dx \]
    \item Determine $I$ using the substitution $y(x) = \frac\pi2 - x$.
  \end{enumerate}
\end{ex}

\subsection{Sheet 5, Exercise 3a}
\[ \int_a^b \frac{f'(x)}{f(x)} \, dx = \begin{vmatrix} t = f(x) \\ dt = f'(x) \, dx \end{vmatrix} = \int_{f(a)}^{f(b)} \frac{1}{t} \, dt \]
\[ = \left[\ln\card{t}\right]_{f(a)}^{f(b)} = \ln\card{f(b)} - \ln\card{f(a)} \]

\subsection{Sheet 5, Exercise 3b}
\[
  \int_0^{\frac\pi2} \frac{\cos(x)}{\sin(x) + \cos(x)} = \underbrace{\frac12 \int_0^{\frac\pi2} \frac{\sin(x) + \cos(x)}{\sin(x) + \cos(x)}}_{\frac\pi4} + \frac12 \int_0^{\frac\pi2} \underbrace{\frac{\cos(x) - \sin(x)}{\cos(x) + \sin(x)}}_{f(x)}
\] \[
  = \frac\pi4 + \ln{\card{\cos(\frac\pi4) + \sin(\frac\pi2)} - \ln\card{\cos(0) + \sin(0)}}
\] \[
  = \frac\pi4 + 0
\]

\subsection{Sheet 5, Exercise 3c}
\[ u(x) = \frac\pi2 - x \]
\begin{align*}
  \int_0^{\frac\pi2} \frac{\cos(x)}{\sin(x) + \cos(x)} \, dx
  &= \int_{\frac\pi2}^0 -\frac{\cos(\frac\pi2 - u)}{\sin(\frac\pi2 - u) + \cos(\frac\pi2 - u)} \, du \\
  &= \int_0^{\frac\pi2} \frac{\cos(\frac\pi2 - u)}{\sin(\frac\pi2 - u) + \cos(\frac\pi2 - u)} \, du \\
  &= \int_0^{\frac\pi2} \frac{\sin(u)}{\sin(u) + \cos(u)} \, du \\
  \implies 2I &= \int_0^{\frac\pi2} \frac{\sin(u)}{\sin(u) + \cos(u)} \, du + \int_0^{\frac\pi2} \frac{\cos(u)}{\sin(u) + \cos(u)} \, du \\
  2I &= \int_0^{\frac\pi2} \frac{\sin(u) + \cos(u)}{\sin(u) + \cos(u)} \, du \\
  2I = \frac\pi2 &\iff I = \frac\pi4
\end{align*}

\section{Sheet 5, Exercise 4}
\begin{ex}
  \begin{enumerate}
    \item Evaluate using integration by parts: $\int_0^\pi (\sin{x})^2 \, dx$
    \item Determine (for $n \in \mathbb N$) by integration by parts: $\int_0^{\frac\pi2} (\cos{x})^{2n} \, dx$
    \item Determine by integration by parts followed by substitution: $\int_0^1 \log(x  + 1) \, dx$
  \end{enumerate}
\end{ex}

\subsection{Sheet 5, Exercise 4a}

Let $u \coloneqq \sin(x)$, $u' = \cos(x)$, $v' \coloneqq \sin(x)$ and $v = -\cos(x)$.

\begin{align*}
  \int_0^\pi (\sin(x))^2 \, dx
    &= \left[-\sin(x) \cos(x)\right]_0^\pi - \int_0^\pi -\cos(x) \cos(x) \, dx \\
    &= \int_0^\infty 1 - \int_0^\pi \sin(x)^2 \, dx \\
  \iff \int_0^\pi 2 \cdot \sin(x)^2 \, dx = \int_0^\infty 1 = \pi \\
    &= \frac\pi2
\end{align*}

\subsection{Sheet 5, Exercise 4b}

Let $n \in \mathbb N \setminus \set{0}$.
\[ \int_0^{\frac\pi2} (\cos(x))^{2n} \, dx \]

We prove by complete induction:
Consider $n = 0$.
\[ \int_0^{\frac\pi2} (\cos(x))^{2n} \, dx = \frac\pi2 \]

Consider $n - 1 \to n$.
\[ \int_0^{\frac\pi2} \cos(x)^{2n+2} \, dx = \int_0^{\frac\pi2} \underbrace{\cos(x)^{2n+1}}_{u} \underbrace{\cos(x)}_{v'} \, dx \]
\begin{align*}
  \int_0^{\frac\pi2} (\cos(x))^2 &= \frac\pi4 \\
  \text{By induction hypothesis } \int_0^{\frac\pi2} \cos(x)^{2n} \, dx &= \frac{2n-1}{2n} \int_0^{\frac\pi2} \cos(x)^{2(n-1)} \\
    &= \begin{vmatrix}
      u' &= -(2n + 1) \sin(x) \cos(x)^{2n} \\
      v &= \sin(x)
    \end{vmatrix}
\end{align*}


\[
  [\cos(x)^{2n+1} \cdot \sin(x)]_0^{\frac\pi2} + (2n+1) \cdot \int_0^{\frac\pi2} \cos(x)^{2n} \cdot \sin(x)^2 \, dx
    = (2n + 1) \cdot \int_0^{\frac\infty2} \cos(x)^{2n} \, dx - (2n+1) \int_0^{\frac\pi2} \cos(x)^{2n+2} \, dx
\] \[
  \implies (2n + 2) \int_0^{\frac\pi2} \cos(x)^{2n+2} \, dx = (2n + 1) \int_0^{\frac\pi2} \cos(x)^{2n} \, dx
\] \[
  \implies \int_0^{\frac\pi2} \cos(x)^{2n+2} \, dx = \frac{(2n+1)}{2n+2}\int_0^{\frac\pi2} \cos(x)^{2n} \, dx
\] \[
  \frac{2n-1}{2n} \cdot \frac{2n-3}{2n-2} \cdot \ldots \cdot \frac12 \cdot \frac\pi2
\]

\subsection{Sheet 5, Exercise 4c}

\[
  \int_0^1 x \cdot \log(x + 1) \, dx
    = \begin{vmatrix}
      u' = x  \qquad & u = \frac{x^2}{2}   \\
      v = \log(x+1) \qquad & v' = \frac1{1+x}
    \end{vmatrix}
\] \[
  \left[\frac{x^2}{2} \log(x+1)\right]_0^1 - \int_0^1 \left(\frac{x^2}2 \cdot \frac{1}{1+x}\right) \, dx \qquad u(x) = 1 + x
\] \[
  = \left[\frac{x^2}{2} \log(x+1)\right]_0^1 - \frac12 \underbrace{\int_1^2 (u - 1)^2 \cdot \frac1u \, du}_{\int_1^2 \left(\frac{u^2 + 1 - 2u}{u}\right) \, du = \int_1^2 u + \frac1u - 2 \, du}
\] \[
  \frac{\log(2)}{2} - \frac12 \left[\frac{u^2}{2} + \log(u) - 2u\right]_1^2 = \frac14
\]

It is valid to assume that $\log = \ln$ in this exercise, because it is not specified otherwise.
But you can also consider a factor $a$, which normalizes it to $\ln$.

\section{Sheet 6, Exercise 1}
\begin{ex}
  Let $\mathcal R[a,b]$ be the set of regulated functions, $\mathcal C[a,b]$ be the set of continuous functions
  and $\mathcal M[a,b]$ be the set of montonic functions on $[a,b] \subset \mathbb R$. Show:
  \begin{enumerate}
    \item $f \in \mathcal C[a,b] \implies f \in \mathcal R[a,b]$
    \item $f \in \mathcal M[a,b] \implies f \in \mathcal R[a,b]$
    \item $f \in \mathcal C[a,b], g \in \mathcal R[a,b] \land g([a,b]) \subset [a,b] \implies f \circ g \in \mathcal R[a,b]$
  \end{enumerate}
\end{ex}

\subsection{Sheet 6, Exercise 1a}
Assume $f \in \mathcal C[a,b]$.
For all $x \in [a,b]$, $f$ has one-sided limits.

\subsection{Sheet 6, Exercise 1b}
Let $x \in [a,b]$. Consider $x_{n \in \mathbb N} \nearrow x$.
Show that $\lim_{n\to\infty} f(x_n)$ exists.
We consider a monotonic subsequence
\[ f(x_{n_k}) \geq f(x_{n_{k+1}}) \forall k \in \mathbb N \]
\[ f(x) \leq f(x_{n_k}) \forall k \in \mathbb N \]

\subsection{Sheet 6, Exercise 1c}
$(x_n)_{n\in\mathbb N} \nearrow x$.
\[ \lim_{n\to\infty} f(g(x_n)) \text{ exists} \]
\[ \lim_{n\to\infty} \underbrace{g(x_n)}_{\eqqcolon y_n} = y \in \mathbb R \]
\[ \lim_{n\to\infty} f(y_n) = f(\lim_{n\to\infty}) = f(\lim_{n\to\infty} y_n) TODO \]

$g: [a,b] \to [a,b]$.
$f \in \mathcal R[a,b]$, $g \in \mathcal C([a,b]), g([a,b]) \subset [a,b]$.


\section{Sheet 6, Exercise 2}
\begin{ex}
  Determine all antiderivatives:
  \begin{align}
    &\int \frac{1}{x (\ln{x})^3} \, dx \qquad (x > 0) \\
    &\int \sin^3(x) \cos^4(x) \, dx \\
    &\int \operatorname{arsinh}(x) \, dx
  \end{align}
\end{ex}

\subsection{Sheet 6, Exercise 2a}

We apply integration by substitution:
\[ \int_{g(a)}^{g(b)} f(x) \, dx = \int_a^b f(g(u)) \cdot g'(u) \, du \]

We consider:
\[ f(x) = \left(\frac1{x^3}\right) = \frac1{x^3} \]
\[ g(x) = \ln(x) \qquad g'(x) = \frac1x \]

\[ \int \frac{1}{x (\ln{x})^3} \, dx = \int \left(\frac1{u^3}\right) \, du = \int u^{-3} \, du = \frac{u^{-2}}{-2} + c = \frac{1}{-2 \cdot u^2} + c = \frac{1}{-2 \cdot \ln(x)^2} + c \]

\textbf{Hint.} Because we apply Backsubstitution, we do what we usually do by computing the integral over some specified limits. Therefore the improper integral is exact as well.

\subsection{Sheet 6, Exercise 2b}

\begin{align*}
  \int \sin(x)^2 \cdot \sin(x) \cdot \cos(x)^4 \, dx
    &= \int (1 - \cos(x)^2) \cdot \cos(x)^4 \cdot \sin(x) \, dx \\
    &= \int (\cos(x)^4 - \cos(x)^6) \cdot \sin(x) \, dx \\
    &\begin{vmatrix} u = \cos(x) \\ u' = -\sin(x) \\ du = dx \cdot u' \end{vmatrix} \\
    &= \int (u^4 - u^6) \cdot (-1) \, du = \int (-u^4 + u^6) \, du \\
    &= \frac{u^7}{7} - \frac{u^5}{5} + c = \frac{\cos(x)^7}{7} - \frac{\cos(x)^5}{5} + c
\end{align*}

\subsection{Sheet 6, Exercise 2c}

\begin{align*}
  \int \operatorname{arsinh}(x) \, dx
    &= \int \ln(x + \sqrt{x^2 + 1}) \, dx \\
    &\begin{vmatrix} u = \ln(x + \sqrt{x^2+1}) \\ v' = 1 \\ v = x \\ u' = \frac{1}{\sqrt{x^2 + 1}} \end{vmatrix} \\
    &= \ln(x + \sqrt{x^2 + 1}) x - \int \frac{1}{\sqrt{x^2 + 1}} x \, dx \\
    &\begin{vmatrix} u = x^2 + 1 \\ u' = 2x \\ du = 2x \, dx \end{vmatrix} \\
    &= \operatorname{arsinh}(x) \cdot x - \int \frac1{\sqrt{u}} \frac12 \, du \\
    &= \operatorname{arsinh}(x) \cdot x - \sqrt{u} + c \\
    &= \operatorname{arsinh}(x) \cdot x - \sqrt{x^2 + 1} + c
\end{align*}

\section{Sheet 6, Exercise 3}
\begin{ex}
  For $a = 0$ and $a > 0$, determine all antiderivatives:
  \[ \int \frac{\ln(x)}{\sqrt{a + x}} \, dx \qquad (x > 0) \]
\end{ex}

Case $a = 0$:

\begin{align*}
  \int \frac{ln(x)}{\sqrt{x}}
    &\begin{vmatrix} u' = \frac1{\sqrt{x}} & u = 2\sqrt{x} \\ v = \ln(x) & v' = \frac{1}{x} \end{vmatrix} \\
    &= \ln(x) \cdot 2 \sqrt{x} ... \\
    &= \ln(x) \cdot \sqrt{x} - 4 \sqrt{x} + c
\end{align*}

Case $a > 0$:

\begin{align*}
  \int \frac{\ln(x)}{\sqrt{x + a}}
    &= \int \frac{\ln(x)}{\sqrt{x + a}} \cdot 2 \sqrt{x + a} \, du \\
    &\begin{vmatrix} u = \sqrt{x + a} \\ \frac{du}{dx} = \frac{1}{2\sqrt{x + a}} \implies dx = 2\sqrt{x + a} \, du \\ u = \sqrt{x + a} \implies x = u^2 - a \end{vmatrix} \\
    &= 2\int \ln(x) \, du \\
    &= 2 \ln(u^2 - a) \, du \\
    &= 2 \int \ln(u + \sqrt{a}) + \ln(u - \sqrt{a}) \, du \\
    &= 2 \left(\int (u + \sqrt{a}) \, du + \int \ln(u - \sqrt{a}) \, du\right)
\end{align*}

We compute separately:
\begin{align*}
  \int \ln(x + c) \, dx
    &= \int 1 \cdot \ln(x + c) \, dx \\
    &\begin{vmatrix} u' = 1 \implies u = x \\ v = \ln(x + c) \implies v' = \frac{1}{x + c} \end{vmatrix} \\
    &= x \ln(x + c) - \int \frac{x + c - c}{x + c}  \\
    &= x \ln(x + c) - x + c \ln(x + c) \\
    &= (x + c) \ln(x + c) - x + c
\end{align*}
with
\[ \int \frac{x + c}{x + c} - \frac{c}{x + c} = \int 1 - \frac{c}{x + c} = x - c \ln(x + c) + c \]

We continue:
\begin{align*}
  &= 2 ((u + \sqrt{a}) \ln(u + \sqrt{a}) - (u + \sqrt{a}) + (u - \sqrt{a}) \ln (u - \sqrt{a}) - (u - \sqrt{a})) + c \\
  &= 2 (u \ln(u^2 - a) + \sqrt{a} \ln\left(\frac{u + \sqrt{a}}{u - \sqrt{a}}\right) - 2u) + c \\
  &= 2\sqrt{x + a} \ln(x) + \sqrt{a} \ln\left(\frac{\sqrt{x + a} + \sqrt{a}}{\sqrt{x + a} - \sqrt{a}}\right) - 4\sqrt{x + a} + c
\end{align*}

\section{Sheet 6, Exercise 4}
\begin{ex}
  Let $k \in \mathbb Z$, $I_k \coloneqq ((2k - 1) \pi, (2k + 1)\pi)$ and
  \[ f: \mathbb R \to \mathbb R, \qquad f(x) \coloneqq \frac{1}{3\cos(x) + 5} \]
  \begin{enumerate}
    \item Prove for all $x \in I_k$ the identity
      \[ \cos(x) = \frac{1 - \tan(x/2)^2}{1 + \tan(x/2)^2} \]
    \item Determine all antiderivatives:
      \[ \int f(x) \, dx, x \in I_k \]
      Begin by integration by substitution with $u(x) = \tan(\frac x2)$.
    \item Construct a continuous function $F: \mathbb R \to \mathbb R$,
      that is an antiderivative of $f$ on every compact interval.
  \end{enumerate}
\end{ex}

\subsection{Sheet 6, Exercise 4a}

\[ \tan\left(\frac x2\right) = \frac{\sin x}{1 + \cos(x)} \]
Proof:
Let $u = \frac x2$ and $x = 2u$.
\[ \tan(u) = \frac{\sin{2u}}{1 + \cos(2u)} = \frac{2 \sin(u) \cos(u)}{1 + \cos^2(u) - \sin^2(u)} = \frac{2 \sin(u) \cos(u)}{2 \cos^2(u)} = \frac{\sin(u)}{\cos(u)} = \tan(u) \]

Then,
\begin{align*}
  \frac{1 - \tan(x/2)^2}{1 + \tan(x/2)^2}
    &= \frac{1 - \frac{\sin^2(x)}{1 + \cos(x)}}{1 + \frac{\sin^2(x)}{(1 + \cos(x))^2}} \\
    &= \frac{(1 + \cos(x))^2 - \sin^2(x)}{(1 + \cos(x))^2 + \sin^2(x)} \\
    &= \frac{1 + 2\cos(x) + \cos(x)^2 - \sin(x)}{1 + 2\cos(x) + \underbrace{\cos(x)^2 + \sin^2(x)}_{=1}} \\
    &= \frac{2 \cos(x) (1 + \cos(x))}{2(1 + \cos(x))} \\
    &= \cos(x)
\end{align*}

\subsection{Sheet 6, Exercise 4b}

Let $x \in I_k$.
\begin{align*}
  \int f(x) \, dx
    &= \int \frac{1}{3 \cos(x) + 5} \, dx \\
    &= \int \frac{1}{3 \left(\frac{1 - \tan^2(x/2)}{1 + \tan^2(x/2)}\right) + 5} \, dx \\
    &\begin{vmatrix} u = \tan(x/2) \\ du = \frac{1}{2 \cos^2(x/2)} \, dx \end{vmatrix} \\
    &= \int \frac{1}{3 \left(\frac{1 - u^2}{1 + u^2} + 5\right)} 2 \cos^2(x/2) \, du \\
    &= 2 \int \frac{1}{\left(3 \left(\frac{1 - u^2}{1 + u^2} + 5\right) + 5\right)\left(1 + u^2\right)} \, du \\
  \cos(x) = \frac{1}{1 + \tan^2(x)}
\end{align*}

We compute separately:
\[
  \left(\frac{3 (1 - u^2) + 5}{1 + u^2} + 5\right)(1 + u^2) = \frac{3 (1 - u^2)}{1 + u^2} (1 + u^2) + 5(1 + u^2) = 2 (4 + u^2)
\] \[
  = 2 \int \frac12 \frac{1}{4 + u^2} \, du = \int \frac{1}{4 + u^2} \, du = \begin{vmatrix} t = \frac u2 \\ dt = \frac12 du \end{vmatrix}
  = 2 \int \frac{1}{4 + 4 t^2} \, dt = \frac24 \int \frac{1}{1 + t^2} \, dt
\]

\[ \frac12 \arctan(t) + c = \frac12 \arctan\left(\frac u2\right) + c = \frac12 \arctan\left(\frac{\tan(x/2)}{2}\right) + c \]

Is expected to be continuously differentiable.

\section{Sheet 7, Exercise 1}
\begin{ex}
  Use the direct comparison criterion to determine the convergence of these integrals:
  \[ (a) \int_1^{\infty} \frac{1}{x^2 + 5x + 1} \, dx \qquad (b) \int_0^\infty \frac{1}{x^s + x^{\frac1s}} \, dx \quad s \in \mathbb R \setminus \set{0} \]
\end{ex}

\subsection{Sheet 7, Exercise 1a}
\[ \int_1^\infty \frac{1}{x^2 + 5x + 1} \, dx \leq \int_1^\infty \frac{1}{x^2} \, dx \]
\[ \int_1^\infty \frac1{x^p} < \infty \iff p > 1 \]

\subsection{Sheet 7, Exercise 1b}
\begin{description}
  \item[Case $s = 1$]
    \begin{align*}
      \int_0^\infty \frac{1}{x + x} \, dx &= \frac12 \int_0^\infty \frac{1}{x} = \frac12 \left(\int_0^1 \frac1x + \int_1^\infty \frac1x\right) \\
        &= \frac12 \lim_{t\to\infty} \int_0^t \frac1x \, dx \\
        &= \frac12 \left(\lim_{t\to\infty} \int_1^t \frac1x \, dx + \lim_{t\to\infty} \int_t^1 \frac1x \, dx \right)
    \end{align*}
  \item[Case $s < 0$]
    \[ \int_0^\infty \frac{1}{x^s + x^{\frac1s}} \, dx \]
    Because $s < 0$, $x^s + x^{\frac1s}$ is monotonically decreasing and positive.
    \[ \frac{1}{x^s + x^{\frac1s}} \]
    is monotonically increasing.
    More specifically:
    \[ \int_0^1 \underbrace{\frac{1}{x^s + x^{\frac1s}}}_{\geq 0} + \int_1^\infty \underbrace{\frac{1}{x^s + x^{\frac1s}}}_{\geq 1} \]
    \[ \int_1^\infty 1 \, dx = \infty \]
\end{description}

\section{Sheet 7, Exercise 2}
\begin{ex}
  Prove the following statements:
  \begin{enumerate}
    \item $\forall k \in \mathbb N \cup \set{0}: \int_{k\pi}^{(k+1)\pi} \card{\operatorname{sinc}(x)} \, dx \geq \frac{2}{(k+1)\pi}$.
    \item The improper integral $\int_0^\infty \card{\operatorname{sinc}(x)} \, dx$ does not exist.
  \end{enumerate}
\end{ex}

\subsection{Sheet 7, Exercise 2a}

We apply the Mean Value Theorem:
\[ \exists \xi \in [k\pi, (k+1)\pi]: I = \frac1\xi \int_{u\pi}^{(u+1)\pi} \card{\sin(x)} \, dx \]
\[ \int_{k\pi}^{(k+1)\pi} \card{\sin(x)} \, dx = \card{\int_{k\pi}^{(k+1)\pi} \sin(x) \, dx} = \card{\left. -\cos(x) \right|_{k\pi}^{(k+1)\pi}} = 2 \]
\[ \implies I = \frac1\xi 2 \geq \frac{2}{(k+1)\pi} \forall n \in \mathbb N \]

\[ \overbrace{\operatorname{sinc}(0)}^{=1} \geq \frac{\overbrace{\sin(0)}^{=0}}{\pi} \]

Let $k=0$:
\[ \int_0^\infty \operatorname{sinc}(x) \, dx \overbrace{\implies}^{\text{ for } x \neq 0} \operatorname{sinc}(x) = \frac{\sin(x)}{x} \geq \frac{\sin(x)}{\pi} \forall x \in (0, \pi] \]
We can exclude the case $x = 0$, because individual finitely many values don't matter.
\[ \geq \int_0^\pi \frac{\sin(x)}{\pi} \, dx = \frac2\pi = \frac{2}{(k+1)\pi} \]
\[ \implies \operatorname{sinc}(x) \geq \frac{\sin(x)}{\pi} \forall x \in [0,\pi] \]

\subsection{Sheet 7, Exercise 2b}
Sketch of the proof (but it lacks details acc. to the tutor)
\[ \int_0^\infty \card{\sin(x)} \, dx = \sum_{k=0}^\infty \int_{k\pi}^{(k+1)\pi} \card{\operatorname{sinc}(x)} \, dx \]
\[ \geq \lim_{n\to\infty} \sum_{k=0}^N \underbrace{\frac{2}{(k+1)\pi}}_{\to \infty} = \sum_{k=0}^\infty \frac{2}{k\pi + \pi} \]
\[ \lim_{N\to\infty} \frac2\pi \sum_{k=1}^N \frac{1}{k} \]
\[ \int_{k\pi}^{(k+1)\pi} \card{\sin(x)} \geq \frac{2}{(k+1)\pi} \]

We add some details:
\[ \lim_{N\to\infty} \sum_{k=0}^N T_n \cdot \triangle x \eqqcolon \int f \]
\[ \int_a^b f \, dx + \int_b^c f \, dx = \int_a^b f \, dx \]
\[ \lim_{R\to\infty} \int_0^{R\pi} \card{\operatorname{sinc}(x)} \, dx \geq \lim_{N\to\infty} \sum_{k=0}^{N-1} \int_{k\pi}^{(k+1)\pi} \card{\operatorname{sinc}(x)} \, dx \]

\section{Sheet 7, Exercise 3}
\begin{ex}
\end{ex}

\section{Sheet 7, Exercise 4}
\begin{ex}
  Let $n \in \mathbb N$. For $k \in \set{0,1,\dots,n}$, we define $x_k \coloneqq \frac{k}{n}$ and the step function
  \[ T_n: [0,1] \to \mathbb R \qquad T_n(x) \coloneqq \begin{cases} x_k^2 & \text{ if } x \in [x_{k-1}, x_k) \\ 1 & \text{ if } x = 1 \end{cases} \]
  \begin{enumerate}
    \item Show that: For every $\varepsilon > 0$ there exists $N \in \mathbb N$ such that $\norm{T_n(x) - x^2} < \varepsilon$ for all $n \geq N$.
    \item Determine $\int_0^1 x^2 \, dx$ using sequence $(T_n)_{n\in\mathbb N}$.
  \end{enumerate}
\end{ex}

\subsection{Sheet 7, Exercise 4a}

\[ \forall \varepsilon > 0 \exists N \in \mathbb N: \forall n \geq N: \norm{T_n(x) - x^2}_{\infty} < \varepsilon \]
\[ \norm{T_n(x) - x^2}_{\infty} = 1 - x_{n-1}^2 \]
\[ \norm{T_n(x) - x^2}_{\infty} = 1 - x_{n-1}^2 = 1 - \left(\frac{n-1}{n}\right)^2 = \frac{2n-1}{n^2} \]
\begin{enumerate}
  \item $\forall x \in [x_{k-1}, x_k): \card{T_n(x) - x^2} \leq x_k^2 - x_{k-1}^2 = \left(\frac kn\right)^2 - \left(\frac{k-1}{n}\right)^2 = \frac{2k-1}{n^2}$. \\
    Remark: Also $\frac{2k-1}{n^2} \leq \frac{2n - 1}{n^2} \to 0$. \\
    Remark: $x_k^2 - x_{k-1}^2 = (x_k - x_{k-1})(x_k + x_{l-1}) = \frac1n \delta$ with $0 \leq \delta \leq 2$.
  \item $\forall k \in \set{0,1,\dots,n-2}: x_{k+1}^2 - x_k^2 < x_{k+2}^2 - x_{k+1}^2$
    \[ \left(\frac{k+1}{n}\right)^2 - \left(\frac{k}{n}\right)^2 = \frac{k^2 + 2k + 1 - k^2}{n^2} < \frac{2k+3}{n^2} \]
    \[ = \frac{k^2 + 4k + 4 - (k^2 + 2k + 1)}{n^2} = \left(\frac{k+2}{n}\right)^2 - \left(\frac{k+1}{n}\right)^2 \]
\end{enumerate}

\subsection{Sheet 7, Exercise 4b}
\[ \int_0^1 x^2 \, dx \]
By exercise (4a), it follows that $\lim_{n\to\infty} \norm{T_n - x^2}_{\infty} = 0$.
\[ \int_0^1 x^2 \, dx = \lim_{n\to\infty} \int_0^1 T_n(x) \, dx = \lim_{n\to\infty} \frac1n \sum_{k=1}^n x_k^2 = \lim_{n\to\infty} \frac{1}{n^3} \sum_{k=1}^n k^2 = \frac13 \]
\[ \lim_{n\to\infty} \frac{1}{n^3} \sum_{k=1}^n k^2 = \frac{n (n+1) (2n + 1)}{6} = \frac{n^3}{6} \cdot \left[1 \cdot \left(1 + \frac1n\right) \cdot \left(2 + \frac1n\right)\right] \]
The integral is independent of the particular chosen approximating sequence (see lecture notes).

\subsection{Remark on integrals}
You are allowed to change a regulated function in countable infinite many points. Its limit won't change.
\[ \int_a^b f \, dx \]
\[
  \tilde f \coloneqq \begin{cases}
    f(x) & x \in (a,b] \\
    0 & x = a
  \end{cases}
\]
Then $\int_a^b f \, dx = \int \tilde f \, dx$.

\dateref{2018/05/24}

\section{Sheet 8, Exercise 1}
\begin{ex}
  Let $f: \mathbb R \to \mathbb R$ be given by $f(x) \coloneqq \cosh(2x)$.
  \begin{enumerate}
    \item Determine $f^{(n)}(x)$ and $T_f^n(x; 0)$ for $n \geq 0$ and furthermore $T_f(x; 0)$
    \item Show that for all $x \in \mathbb R$ it holds that $R_f^{n+1}(x; 0) \to 0$ for $n \to \infty$.
      You can use the Lagrange representation of the Taylor remainder $R_f^{n+1}$.
  \end{enumerate}
\end{ex}

\subsection{Sheet 8, Exercise 1a}
\[
  T_f^n(x; 0) = \sum_{k=0}^n \frac{1}{k!} f^{(k)}(0) x^k = \sum_{\substack{k=0 \\ k \text{ even}}}^n \frac{1}{k!} 2^k \underbrace{\operatorname{cosh}(0)}_{=1} x^k + \sum_{\substack{k=0 \\ k \text{ odd}}}^n \frac1{k!} 2^k \underbrace{\operatorname{sinh}(0)}_{= 0} x^k
\] \[
  = \sum_{\substack{k=0 \\ k \text{ even}}}^n \frac{2^k}{k!} x^{-k}
\] \[
  T_f(x; 0) = \sum_{k=0}^\infty \frac{2^{2k}}{(2k)!} x^{2k}
\]

\subsection{Sheet 8, Exercise 1b}
\[
  R_p^{n+1}(x; 0) = \frac{1}{(n+1)!} f^{n+1}(\xi) x^{n+1}
\] \[
  \xi \in (x, 0) \cup (0, x)
\] \[
  \card{\frac{1}{(n+1)!} f^{n+1}(\xi) x^{n+1}} \leq \frac{x^{n+1}}{(n+1)!} 2^{n+1} \card{\operatorname{cosh}(2\xi)}
\] \[
  \leq \frac{\card{x^{n+1}}}{(n+1)!} 2^{n+1} \underbrace{\operatorname{cosh}(2x)}_{\text{constant}} \xrightarrow{n \to \infty} 0
\]

\subsection{Sheet 8, Exercise 1c}
\[ T_f(x; 0) = \sum_{k=0}^\infty \frac{2^{(2k)}}{(2k)!} x^{2k} \]
\[ \underbrace{\lim_{n\to\infty} R_f^{n+1}(x; 0)}_{0} = \lim_{n\to\infty} (f(x) - T_f^n(x; 0)) = f(x) - T_f(x; 0) \]
\[ 0 = f(x) - \lim_{n\to\infty} T_f^n(x; 0) \]
with $\lim_{n\to\infty} (f(x) - T_f^n(x; 0)) = \lim_{n\to\infty} T_f^n(x; 0)$.
As $\lim_{n\to\infty} (f(x) - T_f^n(x; 0))$ converges, it holds that $\lim_{n\to\infty} (f(x) - T_f^n(x; 0)) = \lim_{n\to\infty} f(x) - \lim_{n\to\infty} T_f^n(x; 0)$. So we do not need to show convergence of $\lim_{n\to\infty} T_f^n(x; 0)$.

\subsection{Sheet 8, Exercise 1d}
Show that
\[ \card{f(x) - T_f^8(x; 0)} < \frac{\card{x}^9}{700} \card{\operatorname{sinh}(2x)} < \frac{\card{x}^9}{1400} e^{2 \card{x}} \]

\[ \card{R_f^9(x; 0)} = \frac{1}{9!} \card{f^{(9)}(\xi) x^9} \qquad \xi \in (0, x) \lor (x, 0) \]
\[ = \frac{\card{x}^9}{9!} 2^9 \card{\operatorname{sinh}(2\xi)} < \frac{\card{x}^9}{700} \card{\operatorname{sinh}(2\xi)} \]

Show:
\begin{enumerate}
  \item \[ \frac{2^9}{9!} \overset!< \frac{1}{700} \]
    \[ \frac{2 \cdot 2^2 \cdot 2^3 \cdot 2^3 \cdot 4}{2 \cdot 3 \cdot 4 \cdot 5 \cdot 6 \cdot 7 \cdot 8 \cdot 9} = \frac{4}{81 \cdot 35} < \frac{4}{80 \cdot 35} = \frac1{700} \]
  \item \[ \card{\operatorname{sinh}(2\xi)} < \card{\operatorname{sinh}(2x)}  \]
    \[ x > 0 \implies \xi > 0 \]
    because of monotonicity.
    \[ x < 0 \implies x < \xi < 0 \]
\end{enumerate}

\subsection{Sheet 8, Exercise 1e}
\[ \operatorname{sinh}(2x) = \frac12 (\underbrace{e^{2x}}_{>0} - \underbrace{e^{-2x}}_{> 0}) < \frac12 e^{2x} \]
\[ \frac12 \card{e^{2x}} = \frac12 \]

\[ x < 0: \card{\operatorname{sinh}(2x)} < + \frac{e^{2\card{x}}}{2} \qquad \forall x \in [-1,1] \]
\[ \card{f(x) - T_f^8(x; 0)} < \frac{6}{1000} \qquad x = 0 \]
\[ \card{f(x) - T_f^8(x; 0)} < \frac{\card{x}^9}{1400} e^{2 \card{x}} \]
\[ \frac{\card{x}^9}{1400} e^{2 \card{x}} \leq \frac{\card{x}^9}{1400} e^2 < \frac{2.8^2}{1400} = \frac{28 \cdot 26}{140000} = \frac{7}{1250} < \frac{6}{1000} \]

\section{Sheet 8, Exercise 2}
\begin{ex}
  Let $n \in \mathbb N \cup \set{0}$, $a > 0$, $I \coloneqq [-a, a]$ and $f: I \to \mathbb R$ n-times differentiable.
  \begin{enumerate}
    \item Show: If $f$ is even, i.e. $f(x) = f(-x) \forall x \in I$, $T_f^n(x; 0)$ is even
    \item Show: If $f$ is odd, i.e. $f(x) = -f(-x) \forall x \in I$, $T_f^n(x; 0)$ is odd
    \item Prove that the inverse statements of (a) and (b) are wrong.
      Use $g: I \to \mathbb R, g(x) \coloneqq x^{n+1}$ for $x > 0$, $g(x) = 0$.
    \item Prove that $a$ and $b$ also hold for $T_f(x; 0)$ instead of $T_f^n(x; 0)$ if $f$ is arbitrary often differentiable.
    \item Show that the inverse of statements (a) and (b) are also wrong for $T_f(x; 0)$ instead of $T_f^n(x; 0)$, if $f$ is arbitrarily often differentiable.
  \end{enumerate}
\end{ex}

\subsection{Sheet 8, Exercise 2a}

\[ T_f^n = \ln(x_0) + \sum_{k=1}^n \underbrace{\frac{(-1)^{k+1}}{x_0^k \cdot k}}_{a_k} (x - x_0)^k \]

\subsection{Sheet 8, Exercise 2b}

\[ \text{Cauchy-Hadamard } \implies \rho = \left(\limsup_{k \to \infty} \sqrt[k]{\card{a_k}}\right)^{-1} \]
Area of convergence: $(0, 2x)$

Outside the area of convergence, the series diverges.

\[ \left(\limsup_{k\to\infty} \frac{1}{\card{x_0} \cdot \sqrt[k]{k}}\right)^{-1} = \left(\frac{1}{\card{x_0}}\right)^{-1} = x_0 \]

Consider $x = 2x_0$:
\[ \sum_{k=1}^\infty \frac{(-1)^{k+1}}{x_0^k \cdot k} \cdot x_k^k \implies \text{ converges} \]
Consider $x = 0$:
\[ \sum_{k=1}^\infty \frac{(-1)^{2k} (-1)}{x_0^k \cdot k} x_0^k \implies \text{ diverges} \]

Thus, the actual area of converge is $(0, 2x_0]$.

\subsection{Sheet 8, Exercise 2c}
Show that:
\[ \lim_{n\to\infty} R_f^{n+1}(x; x_0) = 0 \]

\begin{align*}
  \card{R_f^{n+1}(x; x_0)}
    &= \card{\frac{1}{n!} \int_{x_0}^x (x - t)^n \cdot f^{(n+1)}(t) \, dt} \\
    &= \card{\frac{1}{n!} \int_{x_0}^x (x - t)^n \frac{(-1)^n n!}{t^{n+1}} \, dt} \\
    &= \card{\int_{x_0}^x \frac{(x - t)^n}{t^{n+1}} \, dt} \\
    &= \card{\int_{x_0}^x \frac{1}{t} \cdot \left(\underbrace{\frac{x}{t} - 1}_{\eqqcolon q}\right)^n \, dt}
\end{align*}

\[ \sup\set{\frac{x}{t} - 1 \middle| *} \]
\[ t \in [x_0, x] \]
\[ x \in [x_0, 2x_0) \]
\[ = \underbrace{\frac{x}{x_0}}_{<2} - 1 < 1 \]

Whence, consider $x = x_0$,
\[ \card{\int_{x_0}^x \frac{1}{t} \cdot \left(q\right)^n \, dt} \leq \card{\tilde q^n} \cdot {\card{\ln(x) - \ln(x_0)}} \xrightarrow{n \to \infty} 0 \]

The identity in the assignment implies that $T_f(x; x_0)$ converges.

$T_f(x; x_0)$ does not converge at $x = 0$.

\section{Sheet 8, Exercise 3}
\begin{ex}
  Let $n \in \mathbb N \cup \set{0}, a > 0, I \coloneqq [-a, a]$ and $f: I \to \mathbb R$ n-times differentiable.
  \begin{enumerate}
    \item Show: If $f$ is even, i.e. $f(x) = f(-x) \forall x \in I$, $T_f^n(x; 0)$ is even.
    \item Show: If $f$ is odd, i.e. $f(x) = -f(-x) \forall x \in I$, $T_f^n(x; 0)$ is odd.
  \end{enumerate}
\end{ex}

\subsection{Sheet 8, Exercise 3a}
$f(x)$ is even, then $f'(x)$ is odd $\iff f'(x) = -f'(-x)$. How?
\[ f(x) = f(-x) \iff f(x) = f((-1) \cdot (x)) \implies f'(x) = -f'(-x) \]

\begin{align*}
  f(x) &= \lim_{h\to0} \frac{f(x + h) - f(x)}{h} \\
  T_f^n(x; 0) &= \sum_{k=0}^n \frac{1}{k!} f^{(k)}(0) \cdot x^k \\
  &= \sum_{\substack{k=0 \\ k \bmod{2} \equiv 0}}^n \frac{1}{k!} f^{(k)}(0) \cdot x^k + \sum_{\substack{k=1 \\ k \bmod{2} = 1}}^n \overbrace{\frac{1}{k!}f^{(k)}(0) \cdot x^k}^{=0} \\
  &= \sum_{\substack{k = 0 \\ k \bmod{2} \equiv 0}}^n \frac{1}{k!} f^{(k)}(0) \cdot (x)^k = T_f^n(-x, 0)
\end{align*}

\subsection{Sheet 8, Exercise 3b}
Analogous to Exercise 3a.

\subsection{Sheet 8, Exercise 3c}
\[ g: I \to \mathbb R \]
\[ x \mapsto \begin{cases} x^{n+1} & x > 0 \\ 0 & x \leq 0 \end{cases} \]
\[ \sum_{k=0}^n \frac{1}{k!} g^{(k)}(0) \cdot x^k \]
\[ g^{(0)} = 0 \qquad g^{(k)}(0) = 0 \forall k \leq n \]

Do not skip to show that $x=0$ in all derivatives is zero.

\subsection{Sheet 8, Exercise 3d}

\[ f(x) = f(-x) \overset!\implies T_f(x_0) = T_f(-x, 0) \]
\[ T_f(x, 0) = \lim_{n\to\infty} T_f^n(x, 0) = \lim_{n\to\infty} \left(T_f^n(-x, 0\right) \]

This implies that the Taylor series converges.

\subsection{Sheet 8, Exercise 3e}

Find a function that is differentiable infinitely often, is even and odd and $f(0) = 0$.

\[ k(x) = \begin{cases} e^{-\frac1x} & x > 0 \\ 0 & x \leq 0 \end{cases} \]

\section{Sheet 9, Exercise 3}
\subsection{Sheet 9, Exercise 3a}
\[ n \in \mathbb N, t \in \mathbb R: \frac{1}{1 + t^2} = \frac{(-t^2)^{n+1}}{1 + t^2} + \sum_{k=0}^n (-t^2)^k \]
Let $z \coloneqq -t^2$ and we are done (the domains of $-t^2$ and $z$ also match).

\subsection{Sheet 9, Exercise 3b}
We already know:
\[ \frac{d}{dx} \arctan(x) = \frac{1}{1 + x^2} \]
By the fundamental theorem of differential and integration calculus, variant 1:
\[ \implies \int_0^x \left(\frac{d}{dt} \arctan(x)\right) \, dt = \arctan(x) \]
\begin{align*}
  \arctan(x) &= \int_0^x \left[\frac{(-t^2)^{n+1}}{1 + t^2} + \sum_{k=0}^n (-t^2)^k\right] \\
    &= \int_0^x \frac{(-t^2)^{n+1}}{1 + t^2} \, dt + \sum_{k=0}^n \int_0^x (-t^2)^k \, dt \\
    &= \int_0^x \frac{(-t^2)^{n+1}}{1 + t^2} \, dt + \sum_{k=0}^n (-1)^k \frac{x^{2k+1}}{2k + 1}
\end{align*}

\subsection{Sheet 9, Exercise 3c}
\[ \forall x \in [-1, 1]: \arctan(x) = \sum_{k=0}^\infty \frac{(-1)^k x^{2k+1}}{2k + 1} \]
Show that $\lim_{n\to\infty} \int_0^x \frac{(-t^2)^{n+1}}{1 + t^2} \, dt = 0$.
\[ \card{\int_0^x \frac{(-t^2)^{n+1}}{1 + t^2} \, dt} \leq \card{\int_0^x (-t^2)^{n+1} \, dt} = \card{\int_0^x \card{-t^2}^{n+1} \, dt} \]
\[ = \card{\int_0^x t^{2n+2} \, dt} = \card{\left.\frac{t^{2n+3}}{2n+3} \right|_0^x} = \frac{\card{x}^{2n + 3}}{2n + 3} \leq \frac{1}{2n+3} \xrightarrow{n \to \infty} 0 \]

\subsection{Sheet 9, Exercise 3d}
\[ 1 - \frac13 + \frac15 - \dots = \sum_{k=0}^\infty \frac{(-1)^k \cdot 1^{2k+1}}{2k + 1} = \arctan(x) \]

\section{Sheet 9, Exercise 4}
\subsection{Sheet 9, Exercise 4a}
\[ P(z) = \sum_{k=0}^\infty a_k (z - z_0)^k \qquad Q(z) = \sum_{k=0}^\infty a_k (f(z) - z_0)^k \]
\[ f: \substack{\mathbb C \to \mathbb C \\ \hat z \mapsto \overline z} \]

Show that $P(z)$ converges $\iff Q(\hat z)$ converges.

\begin{align*}
  P(z) &= \sum_{k=0}^\infty a_k(z - z_0)^k = \lim_{n\to\infty} \sum_{k=0}^n a_k(\overline z - z_0)^k = \lim_{n\to\infty} \sum_{k=0}^\infty a_k (f(\hat z) - z_0)^k \\
    &= \sum_{k=0}^\infty a_k (f(\hat z) - z_0)^k = Q(\hat z)
\end{align*}

\subsection{Sheet 9, Exercise 4b}
\[ P(z) = \sum_{k=0}^\infty z^k  \qquad Q(z) = \sum_{k=0}^\infty (-1)^k (z^2)^k = \sum_{k=0}^\infty (-z^2)^k \]
with $P(z) = \frac{1}{1 - z}$.
\[ f(z) = -z^2 \]
\[ \card{z} < 1 \iff \card{z^2} < 1 \iff \card{-z^2} < 1 \]

\subsection{Sheet 9, Exercise 4c}
Determine the root function of $Q(z)$ with $z \in \mathbb R$.
\[ \int Q(z) \, dz = \sum_{k=0}^\infty \frac{(-1)^k z^{2k+1}}{2k - 1} + C \]

These are all root functions. But are these all root functions? Yes.
There is some $C$ such that this integral becomes $\arctan$, specifically $C = 0$.
\[ \sum_{k=0}^\infty \frac{(-1)^k z^{2k+1}}{2k+1} = \arctan(z) \qquad \forall z \in (-1, 1) \]

\section{Sheet 9, Exercise 2}
\subsection{Sheet 9, Exercise 2a}
\[ f: \mathbb R \setminus \set{-1, 2} \to \mathbb R \]
\[ f(x) = \frac{x + 3}{x^2 - x - 2} \]
\[ \frac{-\frac23}{x + 1} + \frac{\frac53}{x - 2} \]
\[ f^{(n)}(x) = \frac{\frac23 n! \cdot (-1)^n}{(x + 1)^{n+1}} + \frac{\frac53 n! (-1)}{(x - 2)^{n + 1}} \]
\[ T_f^n(x; 0) = \sum_{k=0}^n \frac{-\frac23 k! (-1)^k + \frac53 k! (-1)^k}{(-2)^{k+1}}{k!} \cdot x^k \]

\subsection{Sheet 9, Exercise 2b}
\[ T_f^2(x; 0) = -\frac32 + \frac x4 - \frac78 x^2 \]
$\xi \in (0, x)$.
\[ R_3 = \card{\frac{f^3(\xi)}{3!} x^3} = \card{\frac{\frac{4}{(\xi + 1)^4} - \frac{10}{(\xi - 2)^4}}{6} x^3} = \frac{\frac{5}{(\xi - 2)^4} - \frac{2}{(\xi + 1)^4}}{3} x^3 \]
\[ \frac{5}{(1 - 2)^4} - \frac{2}{(1 + 1)^4} = \frac{\frac{39}8}{3} \]
\[ \card{\frac{16}{375} - \frac{32}{3}} = \frac{10.624}{1000} \]

\subsection{Sheet 9, Exercise 2c}
\[ T_f(x, 0) = f(x) \forall x \in [0, 1) \]
\[ T_f^n(x; 0) - f(x) = R^{n+1}_f(x; 0) \]
\[ R^{n+1} = \frac{f^{(n+1)}(\xi)}{(n+1)!} x^{n+1} = \frac{-\frac23}{(\xi + 1)^{n+2}} + \frac{\frac53}{(\xi - 2)^{n+2}} \card{x}^{n+1} = 0 \]

\subsection{Sheet 9, Exercise 2d}
Not so easy.

\subsection{Sheet 9, Exercise 2e}
If convergence radius $>1$, then function is continuous and series is continuous in all points (smaller the radius). Contradicts with $f(-1)$ is excluded from the set. But another approach works better: Continuous functions on compact sets are bounded.

Cauchy-Hadamard:
\[ \leadsto \sqrt[n]{\card{-\frac53 \cdot \frac{(-1)(-1)^n}{2^{n+1}} + \frac23}}^{-1} \leq \sqrt[n]{\frac23} \underbrace{\sqrt[n]{1 + \frac{5}{2^{n+2}}}}_{\leq \sqrt[n]{3}} \]

\section{Sheet 10, Exercise 2}
\begin{ex}
  Show that the following functions are differentiable and determine the corresponding Jacobi matrix.
  \begin{enumerate}
    \item $f: \mathbb R \to \mathbb R^4, f(x) \coloneqq \begin{pmatrix} 3x^2 \\ \sin(3x) \\ 42 \\ \cos(x^2) \end{pmatrix}$
    \item $g: \mathbb R^3 \to \mathbb R^2, g(x) \coloneqq \begin{pmatrix} 4x^2 y^3 \\ xye^z + e^x y \end{pmatrix}$
    \item $h: (0, \infty) \times \mathbb R^2 \to \mathbb R, h(x, y, z) \coloneqq \sin(zx) \ln(x + y^2)$
  \end{enumerate}
\end{ex}

\subsection{Sheet 10, Exercise 2a}
\[ f'(x) = \begin{pmatrix} 6x \\ \cos(3x) \cdot 3 \\ -\sin(x^2) \cdot 2x \end{pmatrix} \]

\subsection{Sheet 10, Exercise 2b}
\[ g'(x, y, z) = \begin{pmatrix} 8xy^3 & 12 x^2 y^2 & 0 \\ ye^z + e^xy & xe^z + e^x & xye^z \end{pmatrix} \]

\subsection{Sheet 10, Exercise 2c}
\[ h'(x, y, z) = \begin{pmatrix} \cos(zx) \cdot z \cdot \ln(x + y^2) + \frac{\sin(zx)}{x + y^2} \\ \frac{\sin(zx) \cdot 2y}{x + y^2} \\ \cos(zx) \cdot x \cdot \ln(x + y^2) \end{pmatrix} \]

\subsection{Remark on differentiability}
Continuous partial differentiability implies total differentiability implies continuity.
Continuous partial differentiability implies total differentiability implies partial differentiability.

\section{Sheet 10, Exercise 3}
\begin{ex}
  Consider the function $f: \mathbb R^2 \to \mathbb R$ defined by
  \[ f(x, y) \coloneqq \begin{cases} \frac{xy^3}{x^2 + y^2} & \text{ if } (x, y) \neq (0, 0) \\ 0 & \text{ if } (x, y) = (0, 0) \end{cases} \]
  Show that,
  \begin{enumerate}
    \item $f$ is continuous and continuously partially differentiable.
    \item $f$ is two times partially differentiable and $\partial_x \partial_y f(0, 0) \neq \partial_y \partial_x f(0, 0)$.
    \item $f \in \mathcal C^2(\mathbb R^2 \setminus \set{0})$.
    \item Computationally: One of the second partial derivatives of $f$ is non-continuous in $(0, 0)$.
    \item The statement of (d) without calculations.
  \end{enumerate}
\end{ex}

\subsection{Sheet 10, Exercise 3a}
\[ \frac{\partial f}{\partial x} (0, 0) = \lim_{x \to 0} \frac{\overbrace{f(x, 0)}^{0} - \overbrace{f(0, 0)}^{0}}{x} = \lim_{x \to 0} = 0 \]
\[ \frac{\partial f}{\partial x} (0, 0) = \lim_{y \to 0} \frac{\overbrace{f(0, y)}^{0} - \overbrace{f(0, 0)}^{0}}{y} = \lim_{y \to 0} 0 = 0 \]
Let $(x, y) \neq (0, 0)$.
\[ \frac{\partial f}{\partial x} (x, y) = \frac{y^3 (x^2 + y^2) - 2x (xy^3)}{(x^2 + y^2)^2} \]
\[ \frac{\partial f}{\partial y} \frac{3xy^2 (x^2 + y^2) - 2y (xy^3)}{(x^2 + y^2)^2} \]

The partial derivatives exist everywhere except $(0, 0)$.
\[ \lim_{(x,y) \to (0,0)} \frac{y^3 (x^2 + y^2) - 2x (xy^3)}{(x^2 + y^2)^2} = \lim_{(x,y)\to(0, 0)} \frac{x^2 y^3 + y^5 - 2x^2 y^3}{x^4 + 2x^2 y^2 + y^4} \]

Continuity follows from total differentiability.

\subsection{Sheet 10, Exercise 3b}
It is not obvious that this fraction is two times differentiable. $\frac1{x}$ is not differentiable on 
\[ (x, y) \neq (0, 0) \qquad \frac{\partial f^2}{\partial x \partial y} = \frac{-3x^4 y^2 + 6x^2 y^4 + y^6}{(x^2 + y^2)^3} = \frac{\partial f^2}{\partial y \partial x} \]
The Schwarz' Theorem requires that $f \in \mathcal C^2$.

Either we only determine one partial derivative and copy it for the second derivative.
Or we compute both of them, but then they need to equate and we can use this for verification.

\[ \lim_{y \to 0} \frac{-3x^4 y^2 + 6x^2 y^4 + y^6}{(x^2 + y^2)^3} \]
Continuous function, because $x \neq 0$ and $y \neq 0$.

\[ \partial x \partial y f(0, 0) = \lim_{h \to 0} \frac{1}{h} (\partial x f(h, 0) - \partial x f(0, 0)) = \lim_{h\to 0} \frac{1}{h} \left(\frac{-3 h^4 \cdot 0^2 + 6 h^2 \cdot 0^4 + 0^4}{(h^2 + 0^2)^3}\right) = \lim_{h\to0} \frac1h 0 = 0 \]
\[ \partial y \partial x f(0, 0) = \lim_{h \to 0} \frac1h (\partial x f(0, h) - \partial x f(0, 0)) = \lim_{h\to 0} \frac1h \cdot \frac{h^5}{h^4} = 1 \]

\subsection{Sheet 10, Exercise 3c}
We prove it using polar coordinates.
\[ 1 = \sin(\varphi_n^2) + \cos(\varphi_n^2) \]
\[ x_n = r_n \cos(\varphi_n) \qquad y_n = r_n \sin(\varphi_n) \]
\[ (x_n, y_n) \to \vec 0 \iff (r_n \cos(\varphi_n), r_n \sin(\varphi_n)) \to \vec 0 \iff r_n \to 0 \]

Let $\set{(x_n, y_n)}$ be a sequence with limit $\vec 0$.
Show: $\lim_{n\to\infty} \frac{\partial f}{\partial x} (x_n, y_n) = \frac{\partial f}{\partial x} f(0, 0) = 0$.

\[ \frac{\partial f}{\partial x} (x_n, y_n) = \frac{r_n^5 \sin(\varphi_n^5) - r_n^2 \cos(\varphi_n^2) r_n^3 \sin(\varphi_n^3)}{(rn^2 \cos(\varphi_n^2) + r_n^2 \sin(\varphi_n^2))^2} \]
\[ = \underbrace{r_n}_{\to 0} \underbrace{(\sin(\varphi_n^5) - \cos(\varphi_n^2) \sin(\varphi_n^3))}_{\text{bounded}} \to 0 \]

\[ \card{\frac{y^3 (x^2 + y^2) - 2x (xy^3)}{(x^2 + y^2)^2}} \]

\subsection{Sheet 10, Exercise 3d}
\[ \partial x \partial y f(x, y) = \frac{-3x^4 y^2 + 6x^2 y^4 + y^6}{(x^2 + y^2)^3} \]

Let $\set{(X_h, X_h)}$ be a sequence of $(X_n, X_n) \to \vec 0$ for $n \to \infty$.

Show that: $\lim_{x_n \to 0} \partial_{xy} f(x_n, x_n) = \partial_{xy} f(0, 0) = 0$.

\subsection{Sheet 10, Exercise 3e}
We know $f \in \mathcal C^2(\mathbb C \setminus \set{0})$ and $\partial_{xy} f(0, 0) \neq \partial yx f(0, 0)$.


We use the inverse ($p \implies q$ also gives $\not q \implies \not p$) of the Schwarz' Theorem: Because symmetry of derivatives ($f \not\in \mathcal C^2(\mathbb R^2)$) is not given, the two-times partial derivative is non continuous in $(0, 0)$.

Why is the point of non-continuity $(0, 0)$? Well, we showed continuity for $\mathbb R \setminus \set{\vec 0}$, so $(0, 0)$ remains.

\[ \exists (x, y) \in \mathbb R: \partial x \partial y f(x, y) \neq \partial y \partial x f(x, y) \]

\section{Sheet 10, Exercise 4}
\begin{ex}
  Let $f, g: \mathbb R^2 \to \mathbb R$ be given by
  \[ f(x, y) \coloneqq \sqrt{\card{xy}} \qquad g(x, y) \coloneqq \card{x} + \card{y} \]
  \begin{enumerate}
    \item Determine the level sets $\setdef{(x, y)^T \in \mathbb R^2}{f(x, y) = c}$ and accordingly $\setdef{(x, y)^T \in \mathbb R^2}{g(x, y) = c}$ for $c \in \mathbb R$.
    \item Determine the gradients $\nabla f$ and $\nabla g$, assuming they exist.
  \end{enumerate}
\end{ex}

\section{Sheet 10, Exercise 4a}
\[ y = \pm (c - \card{x}) \]
Consider $c = 0$.
\[ \card{x} + \card{y} = 0 \iff x = 0 \land y = 0 \]
Consider $c < 0$.
\[ \Gamma_0 = \emptyset \]
Consider $c > 0$.
\[ \card{x} + \card{y} = c \]

\[ \Gamma = \angel{(x, y) \in \mathbb R^2 \middle| \card{x} + \card{y} = c} \]

\section{Sheet 11, Exercise 4b}

The gradients do not exist on $\mathbb R^2$. But the gradients exist if $x \neq 0 \lor y \neq 0$.
\[ \nabla g(x, y) = \begin{bmatrix} \frac{\partial g}{\partial x} \\ \frac{\partial g}{\partial y} \end{bmatrix} \]
\[ \frac{\partial g}{\partial x} = \begin{cases} 1 & \text{ if } x > 0 \\ -1 & \text{ if } x < 0 \end{cases} \]
The gradients exist if $x \neq 0 \land y \neq 0$:
\[ \frac{\partial g}{\partial x}(0, y) = \frac{d}{dx} \card{x} \]
\[ f(x) = \card{x} \qquad \frac{df}{dx}(0) \]

\section{Sheet 11, Exercise 1}
\begin{ex}
  Let $D \subset \mathbb R^n$ be open and $v: D \to \mathbb R^n$.
  A partial differentiable function $f: D \to \mathbb R$ is called antiderivative of $v$ on $D$,
  if $\forall x \in D: \nabla f(x) = v(x)$.
  \begin{enumerate}
    \item Show that $v: D \to \mathbb R^3$ with $v(x, y, z) \coloneqq (xy, yz, xz)^T$
      has no antiderivative on an open set $D \subset \mathbb R^3$.
    \item Give an antiderivative of $v: \mathbb R^3 \to \mathbb R^3$ with $v(x, y, z) \coloneqq (yz, xz, xy)^T$.
    \item Investigate whether an antiderivative exists for
      \[ v: \mathbb R^2 \to \mathbb R^2 \quad v(x, y) \coloneqq e^x \begin{pmatrix} y \sin{y} - x \cos{y} \\ x \sin{y} + y \cos{y} \end{pmatrix} \]
  \end{enumerate}
\end{ex}

\subsection{Sheet 11, Exercise 1a}

\[ v(x, y, z) = \begin{pmatrix} xy \\ yy \\ xz \end{pmatrix} \]
\[ \partial_{xx} = y \qquad \partial_{yx} = x \qquad \partial_{yy} = z \qquad \partial_{xy} = 0 \]
The Jacobi matrix is not symmetric.

\subsection{Sheet 11, Exercise 1b}

\[ v(x, y, z) = \begin{pmatrix} yz \\ xz \\ xy \end{pmatrix} \]
\[ f(x) = xyz + c \]

\subsection{Sheet 11, Exercise 1c}

\[ v(x, y) = \begin{pmatrix} y \sin(y) - x \cos(y) \\ x \sin(y) + y \cos(y) \end{pmatrix} \]
\[ f = \int e^x (y \sin(y) - x \cos(y)) \, dx = e^x (y \sin(x) - x \cos(y) + \cos(y)) + c(y) \]
\[ e^x (y \cos(y) + \sin(y) + x \sin(y) - \sin(y)) + c'(y) \]

\section{Sheet 11, Exercise 2}
\begin{ex}
  Let $f: \mathbb R^3 \to \mathbb R^2$ be given over
  \[
    f(x_1, x_2, x_3) \coloneqq \begin{pmatrix}
      4x_1^2 + x_2^2 + x_3^2 - 5 \\
      x_1 x_3 - 1
    \end{pmatrix}
  \]
  \begin{enumerate}
    \item Determine the Jacobian matrix of $f$.
    \item Determine at which points $(x_1, x_2, x_3) \in \mathbb R^3$ the requirements of the Implicit Function Theorem for $f$ are given.
    \item Find an open set $D \subset \mathbb R$ and a function $g: D \to \mathbb R^2$ such that $f(g_1(x_3), g_2(x_3), x_3) = (0, 0)^T$ for $x_3 \in D$ is true (solve for $(x_1, x_2)$).
    \item Determine the derivative
  \end{enumerate}
\end{ex}

\[ f: \mathbb R^3 \to \mathbb R^2 \]
\[ f\begin{pmatrix} x_1 \\ x_2 \\ x_3 \end{pmatrix} = \begin{pmatrix} 4x^2_1 + x_2^2 + x_3^2 - 5 \\ x_1 x_3 - 1 \end{pmatrix} \]

\subsection{Sheet 11, Exercise 2a}
\[ Df(x) = \begin{bmatrix} 8x_1 & 2x_2 & 2x_3 \\ x_3 & 0 & x_1 \end{bmatrix} \]
to apply the Implicit Function Theorem, we can actually choose any square submatrix. Not necessarily the right-most one.

\subsection{Sheet 11, Exercise 2b}
By $(x_1, x_2)$ it must hold, that:
\[
  f\begin{pmatrix} x_1 \\ x_2 \\ x_3 \end{pmatrix} = \begin{pmatrix} 0 \\ 0 \end{pmatrix}
  \quad\land\quad
  \begin{vmatrix} 8x_1 & 2x_2 \\ x_3 & 0 \end{vmatrix} \neq 0 \qquad (*)
\]
\[ (*) \iff -2x_2 x_3 \neq 0 \iff x_2 \neq 0 \land x_3 \neq 0 \]
\[ x_1 x_3 = 1 \implies x_1 \neq 0 \iff x_1 = \frac1{x_3}  \]
\[ 4 \begin{pmatrix} \frac1{x_3} \end{pmatrix}^2 + x_2^2 + x_3^2 - 5 = 0 \]
\[ x_2^2 = -\frac{4}{x_3^2} - x_3^2 + 5 \]
\[ x_2 = \pm \sqrt{-\frac{4}{x_3^2} - x_3^2 + 5} \]
\[ \frac{4}{x_3^2} + x_3^2 \leq 5 \]
\[ 1 \leq x_3 \leq 2 \lor -2 \leq x_3 \leq -1 \]

In which points are the requirements of the Implicit Function Theorem satisfied?

For $x_3 \in (1,2)$ or $x_3 \in (-1, -2)$:
\[ \begin{pmatrix} \frac{1}{x_3} \\ \pm \sqrt{-\frac{4}{x_3^2} - x_3^2 + 5} \\ x_3 \end{pmatrix} \]

\subsection{Sheet 11, Exercise 2c}
\[ g: (1,2) \to \mathbb R^2 \qquad x_3 \mapsto \begin{pmatrix} \frac{1}{x_3} \\ \sqrt{-\frac{4}{x_3^2} - x_3^2 + 5} \end{pmatrix} \]

\subsection{Sheet 11, Exercise 2d}
\[
  Dg(x_3) = \begin{pmatrix}
    -\frac{1}{x_3^2} \\
    -\frac{x_3^4 + 4}{x_3^3 \sqrt{5 - \frac{4}{x_3^2} - x_3^2}}
  \end{pmatrix}
\]

Remark:
The Implicit Function Theorem actually includes a theorem giving the derivative without explicit evaluation of $g$:
\[ F(x, g(x)) = 0 \forall x \in D \implies \frac{d}{dx} F(x, g(x)) = 0 \forall x \in D \]
Also: $g$ is continuously differentiable.

\section{Sheet 11, Exercise 3}
\begin{ex}
  Determine the Taylor polynomial of order $n$ for the following functions:
  \begin{enumerate}
    \item $f: \mathbb R^3 \to \mathbb R, \: f(x, y, z) \coloneqq 2 + y - xz - 2x^2y - 4xyz^2, n=3$.
    \item $g: \mathbb R^2 \to \mathbb R, \: g(x, y) \coloneqq \frac{\exp(x + y)}{\exp(x) + \exp(y)}, n = 2$.
  \end{enumerate}
\end{ex}

It was meant to be evaluated for $(0, 0)$.

\[ T_{(0,0)}^2(x, y) = g(0, 0) + \frac{dg}{dx} (0, 0) (x - 0) + \frac{dg}{dy} (0,0) \cdot y + \frac12 \left(\frac{dg}{dxx} (x - 0)^2 + \frac{2dg}{dxy} (y - 0)(x - 0) + \frac{dg}{dyy} (y - 0)^2\right) \]

\[ g(x, y) = \frac{e^{x + y}}{e^x + e^y} \qquad n = 2 \]
\[ = \frac12 + \frac14 x + \frac14 y + \frac12 \cdot \left(\frac14 xy\right) \]

\[ T_{(0,0,0)}^3(x, y, z) = f(0, 0, 0) + \nabla f(0, 0, 0)^T \begin{pmatrix} x \\ y \\ z \end{pmatrix} + \frac12 (x, y, z) \nabla^2 f(0, 0, 0) \begin{pmatrix} x \\ y \\ z \end{pmatrix} \]
\[ + \frac16 (f_{xxx} + f_{yyy} + f_{zzz} + 3_{xxy} + 3_{xxz} + 3_{yyx} + 3_{yyz} + 3_{zzx} + 3_{zzy} + 3_{xyz}) \]

Actually, we don't need to partial derivatives.
Well, the function is given as a polynomial. We want to determine the Taylor expansion of a polynomial, which is a polynomial itself. Thus, we actually retrieve the same polynomial (the representation might be different).
\[ T^4 (x, y, z; 0) = f(x, y, z) = 2 + y - xz - 2x^2 y - 4xyz^2 \]
$\implies$ $-4xyz^2$ is the only expression of degree $4$. Thus $2 + y - xz - 2x^2 y$ is $T^3(x, y, z; 0)$.
Does this also work for $\neq (0, 0, 0)$? Yes.

\section{Sheet 11, Exercise 4}
\begin{ex}
  We consider the function $f: \mathbb R^{n+1} \to \mathbb R$ given by
  \[ f(x, b_0, \dots, b_{n-1}) = \sum_{j=0}^{n-1} b_j x^j + x^n \]
  Show the following claim:
  If the polynomial $a_0 + \dots + a_{n-1} x^{n-1} + x^n$ has $n$ different roots,
  then there exists some $\delta > 0$ such that for $b_j \in B_{\delta}(a_j)$ with $j = 0, \dots, n-1$
  also polynomial $b_0 + \dots + b_{n-1} x^{n-1} + x^n$ has $n$ different roots.
  \begin{enumerate}
    \item Determine the Jacobian matrix $J_f(\xi_i, a_0, \dots, a_{n-1})$ for one of the $n$ roots $\xi_i$, $i = 1,\dots,n$.
    \item Verify that the requirements of the Implicit Function Theorem are satisfied and apply the theorem on each of the $n$ roots.
    \item Choose the neighborhoods $V_i$ resulting from (b) in root $\xi_0$ such that those are disjoint. Construct the desired neighborhoods of all $a_j$
  \end{enumerate}
\end{ex}

By cruising Kruse himself.

\subsection{Sheet 11, Exercise 4a}
\[ J_f(x, b_0, \dots, b_{n-1}) = \left[p'(x), 1, x, x^2, \dots, x^{n-1}\right] \in \mathbb R^{1 \times (n+1)} \]
\[ p: \mathbb R \to \mathbb R \qquad p(x) \coloneqq \sum_{j=0}^{n-1} b_j x^j + x^n \]

Gradient is a row vector. Jacobian matrix is a column vector.

\subsection{Sheet 11, Exercise 4b}
Requirement 1.
\[ \left[J_f(\xi_i, a_0, \dots, a_{n-1}\right]_{1,1} = p'(\xi_i) \neq 0 \text{ satisfied} \]

Requirement 2.
If all roots are different (all of them are simple roots), the roots of the derivative cannot be the same (for each of them). This follows by Rolle's Theorem. If you have two zeros, there is one point \emph{in between} with derivative zero.
\[ f(\xi_1, a_j, \dots, a_{n-1}) = 0 \text{ satisfied} \]
\[ \implies \forall i \exists g_i: \underbrace{U_i}_{R^n} \to \underbrace{V_i}_{\mathbb R} \text{ with } \underbrace{f(g(b_0, \dots, b_{n-1}), b_0, \dots, b_{n-1}) = 0 \forall (b_0, \dots, b_{n-1}) \in U_i}_{(*)} \]
where $V_i$ is a neighborhood of $\xi_i$.
We apply it $n$ times. We get $n$ roots.

\subsection{Sheet 11, Exercise 4c}

Without loss of generality: $\bigcap_{i=1}^n V_i = \emptyset$ by miniaturization.
Miniaturize $U_i$ by $U_i = g_i^{-1}(V_i)$ with $U_i$ being still open (because $V_i$ is open) $(**)$.
Because $g_i \in \mathbb C^{1}$ by the Implicit Function Theorem

\[ \exists \delta > 0: B_{\delta}(a_0, \dots, a_{n-1}) \subset U_i \forall i \]
Let $(b_0, \dots, b_{n-1}) \in B_{\delta}(a_0, \dots, a_{n-1})$. $(*) \implies f(g(b_0, \dots, b_{n-1}), b_0, \dots, b_{n-1}) = 0 \forall i$.
\[ (b_0, \dots, b_{n-1}) \in B_{\delta}(a_0, \dots, a_{n-1}) \subset U_i \forall i \]
\[ g_i(U_i) = V_i \]

\dateref{2018/06/28}

\section{Sheet 12, Exercise 1}
\begin{ex}
  Let $a > \frac12 + e^{-2}$, $D \coloneqq (0, \infty) \times \mathbb R$, $G \coloneqq \set{(x, y)^T \in D: x + a = 1}$ and
  \[ f: (0, \infty) \times \mathbb R \to \mathbb R, \qquad f(x,y) \coloneqq \ln(x + a) y^2 e^y + (x - a)^2. \]
  \begin{enumerate}
    \item Argue that $f$ is two times partially differentiable and compute $\nabla f$ and $\nabla^2 f$.
    \item Show that $f$ on $D \setminus G$ has exactly two stationary points and that those are a strictly local minimum and a saddle point respectively. \\
      \emph{Hint:} Show that $x + a > 1$ for the stationary point with $y = -2$.
    \item Show that $(a, 0)^T$ is not a global minimum of $f$ on $D$ if $a < 1$.
  \end{enumerate}
\end{ex}

\subsection{Sheet 12, Exercise 1a}
\[
  \nabla f = \begin{bmatrix}
    \frac{1}{x + a} \cdot 1 \cdot y^2 \cdot e^y + 2 (x - a) \\
    \ln(x + a) (2y + y^2) \cdot e^y
  \end{bmatrix}
\] \[
  \nabla^2 f = \begin{bmatrix}
    -\frac{1}{(x + a)^2} \cdot y^2 \cdot e^y + 2  & \frac{1}{(x + a)} \cdot (2y + y^2) \cdot e^y \\
    \frac{1}{(x + a)} \cdot (2y + y^2) \cdot e^y  & \ln(x + a) (2 + 4y + y^2) \cdot e^y
  \end{bmatrix}
\]

\subsection{Sheet 12, Exercise 1b}
\[ \frac{1}{x + a} \cdot y^2 e^y + 2 (x - a) = 0 \]
This equation is more helpful to retrieve stationary points:
\[ \ln(x + a) \cdot (2y + y^2) \cdot e^y = 0 \]

Consider $x = a \land y = 0$.
\[
  \nabla^2 f(x, y) =
  \begin{bmatrix}
    2 & 0 \\
    0 & 2 \cdot \ln(2a)
  \end{bmatrix} \qquad \text{ positive definite}
\] \[
  \implies \text{strict local minimum}
\]
It is symmetric. So it is positive definite iff all eigenvalue are positive.

Consider $y = -2$.
\[ \frac{1}{x + a} \cdot 4 \cdot \frac{1}{e^2} + 2 (x - a) \overset!= 0 \]
\[ \frac{4}{(x + a) e^2} + 2 (x - a) \overset!= 0 \]
\[ \frac{4}{e^2} + 2 \cdot (x^2 - a^2) \overset!= 0 \]
\[ x = \sqrt{a^2 - \frac{2}{e^2}} \]

Show that: $x + a \overset!> 1$.

\[ \sqrt{a^2 - 2e^{-2}} + a \]
\[ a^2 > \frac14 + e^{-4} + e^{-2} \]
\[ \sqrt{\frac14 - e^{-2} + e^{-4}} + a > 1 \]
\[ \frac12 - e^{-2} + a > 1 \]

By $x + a > 1$, $x = \sqrt{a^2  - \frac{2}{e^2}}$ and $y = -2$ is in $D \setminus G$.

\[
  \nabla^2 f \begin{pmatrix} \sqrt{a^2 - 2e^{-2}} \\ -2 \end{pmatrix}
  = \begin{bmatrix}
    \underbrace{2 - \frac{4e^{-2}}{(x + a)^2}}_{>0} & 0 \\
    0 & \underbrace{\underbrace{\ln(\underbrace{x + a}_{>1})}_{>0} \cdot e^{-2} \cdot (-2)}_{<0}
  \end{bmatrix}
\]
So indefinite Hesse matrix, so it is a saddle point.

\subsection{Sheet 12, Exercise 1c}
\[ f(a, 0) = 0 \]

\section{Sheet 12, Exercise 2}
\begin{ex}
  Let $A \in \mathbb R^{m \times n}$ and $b \in \mathbb R^{m}$. We consider the least square problem
  \[ \min_{x \in \mathbb R^n} \norm{Ax - b}_2^2 \]
  \begin{enumerate}
    \item Show that $g: \mathbb R^n \to \mathbb R$, $g(x_1, \dots, x_n) \coloneqq \sum_{j=1}^n x_j^2$ is arbitrari often continuously partially differentiable and determine $\nabla g$ and $\nabla g$.
    \item Use the chain rule to prove that
      \[ f: \mathbb R^n \to \mathbb R, \qquad f(x_1, \dots, x_n) \coloneqq \norm{Ax - b}_2^2 \]
      is arbitrarily often differentable and determine $\nabla f$ and $\nabla^2 f$. Verify that your result has the proper dimensions. For example, $\nabla^2 f(x) \in \mathbb R^{n \times n}$.
    \item Prove that every local minimum $x^* \in \mathbb R^n$ of $f$ necessarily satisfies the normal system $A^T Ax^* = A^T b$.
    \item Show:
      \begin{itemize}
        \item Every $x^* \in \mathbb R^n$ with $A^T A x^* = A^T b$ is a global minimum of $f$
        \item If $A^T A$ is positive definite, so $f$ has at most one global minimum $x^*$.
      \end{itemize}
      \emph{Hint:} Taylor expansion of $f$ in $x^*$.
    \item Show that $f$ can have several local or global minima.
  \end{enumerate}
\end{ex}

\subsection{Sheet 12, Exercise 2a}

\[ \nabla g(x) = 2x = (2 I) x \]
\[ \nabla^2 g(x) = 2I \]

\subsection{Sheet 12, Exercise 2b}
\[ \norm{Ax - b}_2^2 = g(\underbrace{Ax - b}_{\eqqcolon h(x)}) \]

\begin{align*}
  Dh(x) &= A \\
  Df(x) &= Dg(h(x)) \cdot Dh(x) \\
    &= 2h(x)^T \cdot A \\
    &= 2(Ax - b)^T \cdot A \\
    &= 2(x^T A^T - b^T) \cdot A \\
  \nabla f(x) &= 2 \cdot \left(A^T A x - A^T b\right) \\
  \nabla f^2(x) &= 2A^T A
\end{align*}

\subsection{Sheet 12, Exercise 2c}

\[ D f(x^*) = [0, 0, \dots, 0] \]
\begin{align*}
  0 &= 2 (A^T Ax^* - A^T b) \\
  0 &= A^T Ax^* - A^T b \\
  A^T Ax^* &= A^T b
\end{align*}

\subsection{Sheet 12, Exercise 2d}

\[ f(x) = f(x^*) + \underbrace{\nabla f(x^*)^T}_{=0 \text{ because } A^T Ax^* = A^T b} (x - x^*) + \frac12 (x - x^*)^T \underbrace{\nabla^2 f(\xi)}_{2A^TA} (x - x^*) \]
with $\xi = x^* + \Theta(x - x^*)$ with $0 < \Theta < 1$.
\[ = f(x^*) + \underbrace{(x - x^*)^T A^T A (x - x^*)}_{\norm{A(x - x^*)}_2^2 \geq 0} \]

1: $\geq f(x^*) \forall x \in \mathbb R^n$ \\
2: $> f(x^*) \forall x \in \mathbb R^n \setminus \set{x^*}$

\subsection{Sheet 12, Exercise 2e}
Choose $n = m = 1$, $A = 0 = b$.
Then $f$ is the zero function in $\mathbb R$.

$A^T A$ is also positive semidefinite.

\section{Sheet 12, Exercise 4}
\subsection{Sheet 12, Exercise 4a}
A rolled up snake around $(0, 0)$ with opposite directions for $c=1$ and $c=-1$ each.

\subsection{Sheet 12, Exercise 4b}
\[ S_\gamma = \int_a^b \norm{\gamma'(t)}_2 \, dt = \sqrt{1 + c^2} \cdot \int_a^b e^{ct} \, dt = \sqrt{1 + c^2} \cdot \left.\frac{e^{ct}}{c}\right|_a^b = \frac{\sqrt{1 + c^2}}{c} (e^{bc} - e^{ac})\]


\end{document}
