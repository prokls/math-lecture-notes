\documentclass[a4paper]{article}
\usepackage[utf8]{inputenc}
\usepackage[LGR,T1]{fontenc}
\usepackage{amsmath}
\usepackage{amssymb}
\usepackage{amsfonts}
\usepackage{amsthm}
\usepackage{baskervald}
\usepackage{bbold}
\usepackage{csquotes}
\usepackage{enumerate}
\usepackage{faktor}
\usepackage{fancyhdr}
\usepackage[margin=1in]{geometry}
\usepackage[pdfborder={0 0 0},colorlinks=true,citecolor=red]{hyperref}
\usepackage{imakeidx} % before hyperref
\usepackage{mathalfa}
\usepackage{mathtools}
\usepackage{mdframed}
\usepackage[bigdelims,vvarbb]{newtxmath}
\usepackage{rotating}
\usepackage{stmaryrd}
\usepackage{pifont}
\usepackage{wasysym}
\usepackage{xcolor}

\renewcommand*\oldstylenums[1]{\textosf{#1}}

\theoremstyle{definition}
\newmdtheoremenv[%
  backgroundcolor=white,
  linecolor=white!60!black,
  linewidth=3pt]{ex}{Exercise}

\DeclareMathOperator\kernel{kernel}

\title{Linear Algebra 2 -- Practicals}
\author{Lukas Prokop}
\date{summer term 2016}

\newcommand\meta[3]{This #1 took place on #2 (#3).\par}
\newcommand\abs[1]{|\,#1\,|}
\newcommand\set[1]{\left\{#1\right\}}
\newcommand\setdef[2]{\left\{#1\,\middle|\,#2\right\}}
\newcommand\card[1]{\left|\,#1\,\right|}
\newcommand\divides[2]{#1\,\mid\,#2}
\newcommand\mathspace{\hspace{20pt}}
\newcommand\functional[1]{\left\langle{#1}\right\rangle}
\newcommand\Q{\mathbb{Q}}
\newcommand\nope{\lightning}
\newcommand\vecfour[4]{\begin{pmatrix} #1 \\ #2 \\ #3 \\ #4 \end{pmatrix}}
\newcommand{\textgreek}[1]{\begingroup\fontencoding{LGR}\selectfont#1\endgroup}

\parindent0pt
\parskip7pt
\setcounter{tocdepth}{1}

\begin{document}
\maketitle
\tableofcontents

\section{Solution of the last lecture exam of Analysis~1}
\subsection{Exam: Exercise 1}
\begin{ex}
  Determine the limes of
  \[ \sum_{n=2}^\infty \frac{1}{n^2 - 1} \]
\end{ex}

\[ \frac{1}{3} + \frac{1}{8} + \frac{1}{15} + \frac{1}{24} + \ldots \]
does not help us. What about this representation?
\[ \frac{1}{n^2-1} = \frac{1}{(n+1)(n-1)} = \frac{a}{n+1} + \frac{b}{n-1} = \frac{a(n-1) + b(n+1)}{(n+1)(n-1)} \]
\[ a(n-1) + b(n+1) = 1 \]
\[ (a + b)n + (b-a) = 1 \]
\[ \Rightarrow a + b = 0 \land b - a = 1 \]
\[ \Rightarrow a = -\frac12 \quad b = \frac12 \]

Followingly,
\[ \sum_{n=2}^\infty \frac{1}{n^2 - 1} = \sum_{n=2}^\infty \frac{1}{(n+1)(n-1)} = \sum_{n=2}^\infty \left(\frac{\frac12}{n-1} - \frac{\frac12}{n+1}\right) \]

Okay, how to proceed? Let's build a pre-factor:
\[ \frac12 \sum_{n=2}^\infty \left(\frac{1}{n-1} - \frac{1}{n+1}\right) \]
\[ = \left(\frac11 - \frac13\right) + \left(\frac12 - \frac14\right) + \left(\frac13 - \frac15\right) + \left(\frac14 - \frac16\right) + \ldots \]
\[ = \frac11 + \frac12 = \frac32 \]

Let's describe this process of cancelling out formally as telescoping sum:
\[
  S_m \coloneqq \frac12 \sum_{n=2}^m \left(\frac1{n-1} - \frac1{n+1}\right)
  = \frac12 \sum_{n=2}^m \frac{1}{n-1} - \frac12 \sum_{n=2}^m \frac{1}{n+1}
\]

Please be aware that we explicitly define $S_m$ because we want to work with finite sums.
Only in finite sums, we are always allowed to split up sums.

\[ = \frac12 \sum_{n=2}^m \frac{1}{n-1} - \frac12 \sum_{n=4}^{m+2} \frac{1}{n-1} \]
\[ = \frac12 \left(\frac11 + \frac12\right) - \frac12 \left(\frac1m + \frac1{m+1}\right) \]

We already know $\frac1m \xrightarrow{m\to\infty} 0$. Also $\frac1{m+1} \xrightarrow{m\to\infty} 0$.
Followingly also $\frac12 \left(\frac1m + \frac1{m+1}\right) \xrightarrow{m\to\infty} 0$.



\subsection{Exam: Exercise 2}
\begin{ex}
  A recursive definition of a sequence is given:
  \[ a_0 \in \mathbb R, a_0 > 1, (a_n)_{n\in\mathbb N} \]
  \[ a_{n+1} = \frac12 \left(a_n +  1\right) \]
\end{ex}

As an example, we look at the sequence with $a_0 = 2$:
\[ a_0 = 2 \qquad a_1 = \frac32 \qquad a_2 = \frac54 \qquad a_3 \frac98 \]
Another example is $a_0 = 7$:
\[ a_0 = 7 \qquad a_1 = 4 \qquad a_2 = \frac52 \qquad a_3 \frac74 \]

\begin{ex}
  a) Show that $1 \stackrel!< a_n \stackrel!{\leq} a_0 \quad \forall n \in \mathbb N$
\end{ex}

Our examples suggest that this claim might hold.

We use induction over $n$ to prove this statement:
\begin{description}
  \item[induction base] $1 < a_0 \leq a_0$ holds trivially.
  \item[induction step]
    We are given $1 < a_n \leq a_0$ by the induction hypothesis.

    \begin{align*}
      a_{n+1} &= \frac12 (a_n + 1) \\
              &\leq \frac12 (a_0 + a_0) &[\text{induction hypothesis and } 1 < a_0]
    \end{align*}

    \begin{align*}
      a_{n+1} &= \frac12 (a_n + 1) \\
              &> \frac12 (1 + 1) & [\text{induction hypothesis}] \\
              &= 1
    \end{align*}
\end{description}

\begin{ex}
  b) Prove that $a_{n+1} \stackrel{!}{<} a_n \quad \forall n \in \mathbb N$
\end{ex}

\begin{align*}
  a_{n+1} &= \frac12 (a_n + 1) \\
          &< \frac12 (a_n + a_n)   &[\text{we have proven: } a_n > 1]
\end{align*}

\begin{ex}
  c) Does this series converge? If so, give its limit.
\end{ex}

Yes, because it is monotonically decreasing (according to exercise b)
and bounded below (according to exercise a).

\[ b_n \coloneqq a_n - 1 \qquad \forall n \in \mathbb N \]
\[ b_0 \coloneqq a_0 - 1 \]
\[ b_{n+1} = a_{n+1} - 1 = \frac12 (a_n + 1) - 1 = \frac12 (b_n + 1 + 1) - 1 = \frac12 b_n \]

\[ b_n = \frac1{2^n} b_0 \to 0 \cdot b_0 = 0 \]
\[ \Rightarrow b_n \to 0 \]
\[ \Rightarrow a_n = b_n + 1 \to 1 \]


Does it work to just show: $1 = \frac12 (1 + 1)$?
Nope, because in points of continuity this might be true even though $1$ is not its limes.

Let $a_n \to a$ and $a_{n+1} = \frac12 (a_n + 1)$.
\[ a_{n+1} \to a \qquad \frac12 (a_n + 1) \to \frac12 (a + 1) \qquad a = \frac12 (a + 1) \]

\subsection{Exam: Exercise 3}
\begin{ex}
  $f: \mathbb R \to \mathbb R$ with $x \mapsto 2x^2 + 5x- 3$.
  Show continuity with an $\varepsilon$-$\delta$-proof.
\end{ex}

If we don't need an $\varepsilon$-$\delta$-proof, we would argue with the Algebraic Continuity Theorem:
The function $f$ is a composition of continuous functions, hence a continuous function itself.

$\varepsilon$-$\delta$-definition:
\[
  \forall x_0 \in \mathbb R \forall \varepsilon > 0 \exists \delta > 0:
  \abs{x - x_0} < \delta \Rightarrow \abs{f(x) - f(x_0)} < \varepsilon
\]

If $\abs{x - x_0} < \delta$,
\begin{align*}
  \abs{f(x) - f(x_0)} &= \abs{2x^2 + 5x - 3 - (2x_0^2 + 5x_0 - 3)} \\
    &= \abs{2x^2 + 5x - 2x_0^2 - 5x_0} \\
    &\leq 2 \abs{x^2 - x_0^2} + 5 \abs{x - x_0} \\
    &= 2 \abs{(x + x_0) (x - x_0)} + 5 \abs{x - x_0} \\
    &= 2 \abs{x + x_0} \abs{x - x_0} + 5 \abs{x - x_0} \\
    &\leq 2 (\abs{x} + \abs{x_0}) \abs{x - x_0} + 5 \abs{x - x_0} \\
    &\leq 2 \left(\abs{x_0} + \delta + \abs{x_0}\right) + 5 \delta \\
  \intertext{Our goal: we are able to claim $\stackrel!{<} \varepsilon$}
    &= 4 \abs{x_0} \delta + 2 \delta^2 + 5 \delta \\
    &= 2\delta^2 + (4 \abs{x_0} + 5) \delta
\end{align*}

In general (here it does not apply), that $x_0$ might be zero. So division is not allowed and requires case distinctions (cumbersome!).

The following steps work only because we know $\varepsilon > 0$ and $\delta > 0$:
\[ 2 \delta^2 < \frac{\varepsilon}{2} \]
\[ \delta < \frac{\sqrt{\varepsilon}}{2} \]
\[ (4 \abs{x_0} + 5) \delta < \varepsilon \]
\[ \delta < \frac{\varepsilon}{4 \abs{x_0} + 5} \]

Then we can submit those results as solution:

Let $\varepsilon > 0$ and $\delta \coloneqq \min\left(\frac{\sqrt{\varepsilon}}{5}, \frac{\varepsilon}{4 \abs{x_0} + 6}\right)$.
Then the $\varepsilon$-$\delta$ definition shows that $f$ is continuous.

\section{Exam: Exercise 4}
%
\begin{ex}
  Let $f: [0,1] \to \mathbb R$ be continuous and $f(0) = f(1)$.
  Show that $\exists \xi \in [0,\frac12]$ with $f(\xi) = f(\xi + \frac12)$.

  Hint: Consider $h: [0, \frac12] \to \mathbb R$ with $h(x) = f(x) - f(x + \frac12)$.
\end{ex}

Intuition:
Let $\xi = 0$ with $f(\xi) = 0$ and $\xi = \frac12$ with $f(\xi) = \frac1{16}$.
Then the difference $f(0) - f(\frac12)$ is negative. At the same time $f(\frac12) - f(1)$ is positive.
So at some point between $x=0$ and $x=1$ the difference must be zero.

\[ \exists \xi \in [0,\frac12]: h(\xi) = 0 \]
\begin{align*}
  h(0) &= f(0) - f\left(\frac12\right) \\
  h(1) &= f\left(\frac12\right) - f(1) = f\left(\frac12\right) - f(0) = -h(0)
\end{align*}

$f(x)$ is continuous in $[0,\frac12]$. $f(x + \frac12)$ is continuous in $[0,\frac12]$.
Therefore $h$ is continuous, because it is a composition of continuous functions.

\begin{description}
  \item[Case 1: $h(0) < 0$]
    Then $h(\frac12) > 0$ and $h(0) < 0 < h(\frac12)$.
    Due to Intermediate Value Theorem it holds that
    \[ \exists \xi \in [0, \frac12]: h(\xi) = 0 \]
    \[ \Rightarrow f(\xi) = f(\xi + \frac12) \]
  \item[Case 2: $h(0) > 0$]
    Then $h(\frac12) < 0$. Remaining part analogous.
  \item[Case 3: $h(0) = 0$]
    Then by definition $f(0) = f(\frac12)$, so choose $\xi = 0$.
\end{description}

\clearpage
\section{Exercise 1}
\begin{ex}
  Investigate the function $f: \mathbb R \to \mathbb R, x \mapsto \frac12 (x \abs{x} + x^2)$
  in terms of multiple differentiability in all points $x_0 \in \mathbb R$.
\end{ex}



\section{Exercise 2}
\begin{ex}
  Determine, possibly using l'H\^{o}pital's rule, the following limits:
  \begin{enumerate}
    \item $\lim_{x\to1} \frac{\ln{x}}{x - 1}$
    \item $\lim_{x\to0^+} \frac{1}{x} - \frac{1}{\sin{x}}$
    \item $\lim_{x\to\frac\pi2^-} \frac{\ln(\cos{x})}{\ln(1 - \sin{x})}$
    \item $\lim_{x\to1^-} x^\frac{1}{1-x}$
    \item $\lim_{\substack{n\to\infty \\ n \in \mathbb N}} n^{\frac{1}{\sqrt{n}}}$
    \item $\lim_{x\to\infty} \frac{e^x - e^{-x}}{e^x + e^{-x}}$
  \end{enumerate}
\end{ex}

\end{document}
