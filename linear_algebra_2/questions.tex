\documentclass[a4paper]{article}
\usepackage[top=30pt,left=30pt,right=30pt]{geometry}
\usepackage[german,english]{babel}
%\usepackage[utf8]{inputenc}
\usepackage{amsmath}
\usepackage{amssymb}
\usepackage{amsthm}
\usepackage{graphicx}
\usepackage{caption}
\usepackage{fontspec}
\usepackage{mdframed}
\usepackage{pxfonts}
\usepackage{wasysym}
\usepackage{framed}
\usepackage{xcolor}
\usepackage{makeidx}
\usepackage{csquotes}
\usepackage[pdfborder={0 0 0}]{hyperref}
\usepackage{stmaryrd}
\usepackage{titlesec}
\titleformat{\paragraph}{\normalfont\itshape}{}{}{}

\setmainfont{Alegreya}

% definitions
\newcommand{\set}[1]{\left\{#1\right\}}
\newcommand{\setdef}[2]{\left\{\left.#1\,\right|\,#2\right\}}
\newcommand{\ip}[2]{\left\langle#1,#2\right\rangle} % inner product
\newcommand{\angel}[1]{\left\langle#1\right\rangle}
\newcommand{\norm}[1]{\left\|#1\right\|}
\newcommand{\card}[1]{\left|#1\right|}
\newcommand{\given}[1]{\textbf{Given.} #1\par}
\newcommand{\find}[1]{\textbf{Find.} #1\par}
\newcommand{\dateref}[1]{%
  \begin{mdframed}[backgroundcolor=gray!10,innerbottommargin=0pt,innertopmargin=0pt]
    \paragraph{\textit{$\downarrow$ This lecture took place on #1.}}%
  \end{mdframed}%
}
\newcommand{\exist}{\;\exists\,}
\newcommand{\fall}{\;\forall\,}
\newcommand{\noproof}[1]{A proof for Theorem~\ref{#1} is not provided.}
\newcommand{\vectwo}[2]{\begin{pmatrix} #1 \\ #2 \end{pmatrix}}
\newcommand{\vecthree}[3]{\begin{pmatrix} #1 \\ #2 \\ #3 \end{pmatrix}}
\makeatletter
\newcommand{\xRightarrow}[2][]{\ext@arrow 0359\Rightarrowfill@{#1}{#2}}
\makeatother
\newcommand{\rh}[1]{\vec{#1}}
\newcommand{\sout}[1]{#1} % TODO define a strike-through for math mode
\newcommand{\mtn}{(\mu\times\nu)} % mu times nu
\newcommand{\divides}{\,\big|\,} % mu times nu
\def\braket#1{\mathinner{\langle{#1}\rangle}}

\DeclareMathOperator{\rank}{rank}
\DeclareMathOperator{\diag}{diag}
\DeclareMathOperator{\detm}{det}
\DeclareMathOperator{\perm}{perm}
\DeclareMathOperator{\sign}{sign}
\DeclareMathOperator{\degree}{deg}
\DeclareMathOperator{\im}{im}
\DeclareMathOperator{\ke}{kern}
\DeclareMathOperator{\prop}{probability}
\DeclareMathOperator{\Hom}{Hom}
\DeclareMathOperator{\argmax}{argmax}
\DeclareMathOperator{\argmin}{argmin}
\DeclareMathOperator{\vol}{vol}  % volume
\DeclareMathOperator*{\bigtimes}{\vartimes}

\makeatletter
\providecommand*{\dotcup}{%
  \mathbin{%
    \mathpalette\@dotcup{}%
  }%
}
\newcommand*{\@dotcup}[2]{%
  \ooalign{%
    $\m@th#1\cup$\cr
    \hidewidth$\m@th#1\cdot$\hidewidth
  }%
}
\makeatother


% metadata
\title{
  Linear Algebra 2 \\
  List of potential exam questions
}
\date{\today}
\author{Lukas Prokop}

% settings
\parindent0pt
\setlength{\parskip}{0.4\baselineskip}

\setcounter{section}{6}
\makeindex

\begin{document}
\maketitle

\section{Determinants}

\begin{itemize}
  \item Give the three determinant properties
  \item Give the definition of determinant forms and multilinearity
  \item Prove $\triangle(a_1, \dots, a_k + \lambda a_i, \dots, a_n) = \triangle(a_1, \dots, a_k, \dots, a_n) \forall \lambda \in \mathbb K, \forall i \neq k$
  \item Prove $\triangle(a_1, \dots, a_i, \dots, a_j, \dots, a_n) = -\triangle(a_1, \dots, a_j, \dots, a_i, \dots, a_n)$
  \item Define permutations
  \item Define transpositions
  \item Prove: Every permutation can be represented as product of transpositions
  \item Is the decomposition of a permutation as transpositions unique?
  \item Define and determine the signature of a permutation
  \item Prove $\sign(\pi) = \prod_{\substack{i,j \\ i < j}} \frac{\pi(j) - \pi(i)}{j-i}$
  \item Prove: every transposition has sign $-1$
  \item Prove: $\sign(\pi \circ \sigma) = \sign(\pi) \cdot \sign(\sigma)$
  \item Prove: $\forall \sigma \in \sigma_n: \triangle(a_{\sigma(1)}, \dots, a_{\sigma(n)}) = \sign(a) \cdot \triangle(a_1, \dots, a_n)$
  \item Define Leibniz' formula for determinants
  \item Prove Leibniz' formula
  \item Prove: Let $B$ and $C$ be two bases of a vector space. The determinant of $B$ is non-trivial iff the determinant of $C$ is non-trivial.
  \item Prove: $\triangle \text{ non-trivial } \iff \triangle(b_1, \dots, b_n) \neq 0 \text{ for every basis}$
  \item Prove: Let $\triangle$ be a non-trivial determinant form $\triangle(v_1, \dots, v_n) \neq 0 \iff v_1, \dots, v_n$ is linearly independent.
  \item Prove: Two determinant forms are different only by some factor
  \item Prove: $f: V \to V$ is invertible $\iff \det(f) \neq 0$.
  \item Prove: For a matrix $A \in \mathbb K^{n\times n}$ it holds that $\det{A} \neq 0 \iff$ A has full rank.
  \item Prove: $f, g: V \to V$ linear. $\implies \det(f \circ g) = \det(f) \cdot \det(g)$
  \item Prove $\det(A \cdot B) = \det(A) \cdot \det(B)$ directly
  \item Prove: $\det(A^{-1}) = \frac{1}{\det(A)}$ if invertible
  \item Prove: $\det(A) = 0 \iff \rank(A) < n$
  \item Prove: $\det(A^t) = \det(A)$
  \item Define $\operatorname{perm}(A)$
  \item Prove: Let $A$ be an upper triangular matrix, hence $a_{ij} = 0$ if $i > j$. $\implies \det(A) = a_{11} a_{22} \dots a_{nn}$.
  \item Prove: $\begin{vmatrix} &   &   &   & 0 \\ &   & B &   & 0 \\ &   &   &   & 0 \\ & * & * & * & a_{nn} \end{vmatrix} = \det(B) \cdot a_{nn}$
  \item Define Laplace Expansion
  \item Define the cofactor of a matrix
  \item Define the complementary matrix
  \item Prove: $A^{-1} = \frac{1}{\det{A}} \hat{A}$
  \item Give and prove Cramer's rule
\end{itemize}

\section{Inner products}

\begin{itemize}
  \item Give a geometrical proof of the Pythagorean theorem
  \item Define the scalar product in $\mathbb R^2$ and $\mathbb R^3$
  \item Prove $\langle a, b\rangle = \langle b, a\rangle$, $\langle \lambda a, b\rangle = \lambda \langle a, b\rangle = \langle a, \lambda b\rangle$ and $\langle a + b, c \rangle = \langle a, c \rangle + \langle b, c \rangle$ in $\mathbb R^2$
  \item Prove that the scalar product in $\mathbb R^3$ is the dot product.
  \item Give the law of Cosines
  \item Define an outer product (in $\mathbb R^3$)
  \item Prove $b \times a = -a \times b$, $(\lambda a) \times b = \lambda (a \times b) = a \times (\lambda b)$ and $(a + b) \times c = a \times c + b \times c$
  \item Define bilinearity
  \item Define antisymmetry
  \item Define an inner product
  \item The function value of an inner product is an element of which domain?
  \item Define positive/negative-(semi)definite and indefinite inner products (in terms of inner products and in terms of matrices).
  \item Define a scalar product in terms of inner products.
  \item What is a positive definite inner product in Hermitian form?
  \item What is a unitary product?
  \item When does the scalar product satisfy $\ip ab = \ip ba$?
  \item Why is the scalar product defined with sesquilinearity and not bilinearity?
  \item Define norms
  \item Every $\square{}$ induces a $\square{}$. How?
  \item Define $l$-norms
  \item Prove: $\norm{x} \coloneqq \sqrt{\angel{x, x}}$ is a norm on $V$
  \item Give and prove: CBS inequality
  \item Prove: Let $V$ be a vector space over $\mathbb K = \mathbb R$ or $\mathbb C$. Let $B = \set{b_1, \dots, b_n}$ be a basis. $\angel{,}$ is an inner product.
    There exists a unique matrix $A$ in Hermitian form such that $\forall x,y \in V: \angel{x,y} = \Phi_B(x)^T \cdot A \cdot \overline{\Phi_B(y)}$.
    Additionally show: If $\angel{,}$ is positive definite, $A$ is invertible.
  \item Define the conjugate transpose of a matrix.
  \item Define self-adjoint matrices.
  \item Define symmetrical matrices.
  \item Define Hermitian matrices.
  \item Give one simple example each for a (positive/negative) (semi)definite matrix and indefinite matrix.
  \item Define congruent matrices.
  \item Prove: Every Hermitian matrix is congruent to a diagonal matrix $D$ of form $\operatorname{diag}(D) = (1, \dots, 1, -1, \dots, -1, 0, \dots, 0)$.
  \item Give and prove Sylvester's law of inertia.
  \item Define the index and signature of a matrix.
  \item Prove: $A > 0 \iff A \hat={} I \iff \operatorname{ind}(A) = n$
  \item Prove: $A \geq 0 \iff \operatorname{ind}(A) = \operatorname{sign}(A) = \operatorname{rank}(A)$
  \item Prove: $A \hat={} B \iff \operatorname{ind}(A) = \operatorname{ind}(B) \land \operatorname{sign}(A) = \operatorname{sign}(B)$
  \item Prove: $\det(C^*) = \overline{\det(C)}$
  \item Prove: $A = A^* \implies \det(A) \in \mathbb R$
  \item Prove: $A = A^*, B = B^*, A \hat B \implies \sign{\det(A)} = \sign{\det(B)}$
  \item Prove: $A > 0 \implies \det(A) > 0$
  \item Define the minor of a matrix.
  \item Let $A = A^*$. Prove: $A > 0 \iff$ all first minors $A_r$ satisfy $\det(A_r) > 0$
  \item All minors of a positive definite matrix are $\square$
  \item Let $A = A^*$. Prove: $A < 0 \iff (-1)^r \det(A_r) > 0 \forall r \in \set{1, \dots, n}$
  \item What is an Euclidean space? What is a unitary space?
  \item What is a Hilbert space?
  \item Give the parallelogram law.
  \item Define orthogonal and orthonormal families of vectors.
  \item Define orthonormal bases of vectors.
  \item Let $(v_i)_{i \in I} \subseteq V$, $v_i \neq 0 \forall i$. Prove: $(v_i)_{i \in I}$ is orthogonal, then $(v_i)_{i \in I}$ is linear independent.
  \item Let $B = (b_1, \ldots, b_n)$ is an orthonormal basis of an finite-dimensional vector space over $\mathbb K$.
    For $v \in V$, let $\Phi_B(v) = \begin{pmatrix} \lambda_1 & \dots & \lambda_n \end{pmatrix}^T$.
    For $w \in V$, let $\Phi_B(w) = \begin{pmatrix} \mu_1 & \dots & \mu_n \end{pmatrix}^T$. Prove:
    \begin{enumerate}
      \item $\lambda_i = \ip{v}{b_i}$
      \item $\ip vw = \sum_{i=1}^n \lambda_i \overline{\mu_i}$
    \end{enumerate}
  \item Define orthogonal complements
  \item Let $V$ be a vector space with scalar product. $M, N \subseteq V$ are partitions.
    \begin{enumerate}
      \item $M^\bot$ is a subspace.
      \item $M \subseteq N \implies N^\bot \subseteq M^\bot$ \\
        $(M_1 \cup M_2)^\bot = M_1^\bot \cap M_2^\bot$
      \item $\set{0}^\bot = V$
      \item $V^\bot = \set{0}$
      \item $M \cap M^\bot \subseteq \set{0}$
      \item $M^\bot = \mathcal L(M)^\bot$
      \item $M \subseteq (M^\bot)^\bot$
    \end{enumerate}
  \item Prove: Let $U \subseteq V$ be a subspace. $U + U^\bot$ is a direct sum in $\mathbb R^n$ such that $U + U^\bot = \mathbb R^n$.
  \item Define convexity of functions and sets.
  \item Let $V = U \dot+ U^{\bot}$. Prove: $\forall x, y \in V: \ip{x}{\pi_{U}(y)} = \ip{\pi_U(x)}{y} = \ip{\pi_U(x)}{\pi_U(y)}$
  \item Let $V = U \dot+ U^{\bot}$. Prove: $\norm{\pi_U(x)} \leq \norm{x} \land \norm{\pi_U(x)} = \norm{x} \iff x \in U$
  \item Define the Gram matrix.
  \item Let $v_1, \ldots, v_m \in V$. $G = \operatorname{Gram}(v_1, \ldots, v_m)$. Prove: $G = G^*$ is Hermitian, positive \emph{semi}definite.
  \item Let $v_1, \ldots, v_m \in V$. $G = \operatorname{Gram}(v_1, \ldots, v_m)$. Prove: $\xi \in \ker{G} \iff \sum_{i=1}^m \overline{\xi_i} v_i = 0$
  \item Let $v_1, \ldots, v_m \in V$. $G = \operatorname{Gram}(v_1, \ldots, v_m)$. Prove: $G$ is positive definite iff $(v_1, \ldots, v_m)$ are linear independent.
  % \item Theorem 8.54
  \item Give Bessel's inequality. Give an intuition what the inequality says/when it can be useful.
  \item Give Parseval's identity. Give an intuition what the inequality says/when it can be useful.
  \item Give and prove the Gram--Schmidt process for orthogonalization
  \item Define Laguerre polynomials
  \item Define Hermite polynomials
  \item How are Laguerre and Hermite polynomials related to the Gram--Schmidt process?
  \item Give Riesz representation theorem
  \item Prove Riesz representation theorem
  \item Does Riesz representation theorem hold for infinite-dimensional spaces?
  \item Prove: $v = 0 \iff \forall w \in V: \ip vw = 0$
  \item Prove: $\ip xy = \ip xz \forall x \implies y = z$
  \item Prove: $\norm{v} = \sup\setdef{\card{\ip vw}}{\norm{w} \leq 1}$
  \item Define adjoint maps
  \item Prove: Let $(V, \ip{\cdot}{\cdot}_V)$ and $(W, \ip{\cdot}{\cdot}_W)$ be spaces with a scalar product. $\dim{V}, \dim{W} < \infty$.
    $T: W \to V$ linear. Prove: For every $v \in V$ the map $w \mapsto \ip{T(w)}{v}_V$ is linear.
  \item Prove: Let $(V, \ip{\cdot}{\cdot}_V)$ and $(W, \ip{\cdot}{\cdot}_W)$ be spaces with a scalar product. $\dim{V}, \dim{W} < \infty$.
    $T: W \to V$ linear. Prove: $\forall v \in V \exists! u \in W \forall w \in W: \ip{T(w)}{v}_V = \ip{w}{u}_W$ and $T^*(v) = u$.
  \item Prove: Let $(V, \ip{\cdot}{\cdot}_V)$ and $(W, \ip{\cdot}{\cdot}_W)$ be spaces with a scalar product. $\dim{V}, \dim{W} < \infty$.
    $T: W \to V$ linear. Show: $T^* \in \operatorname{Hom}(V, W)$
  \item Prove: Let $(V, \ip{\cdot}{\cdot}_V)$ and $(W, \ip{\cdot}{\cdot}_W)$ be spaces with a scalar product. $\dim{V}, \dim{W} < \infty$.
    $T: W \to V$ linear. Prove: the map $\operatorname{Hom}(W, V) \mapsto \operatorname{Hom}(V, W)$ with $T \mapsto T^*$ is antilinear and $T^{**} = T$.
  \item Does every linear map have an adjoint map?
  \item Define involutions
  \item Prove: Let $B \subseteq V, C \subseteq W$ be orthonormal bases. $f \in \operatorname{Hom}(V, W)$.
    \[ \Phi_B^C(f^*) = \Phi_C^B(f)^* = \overline{\Phi_C^B(f)^T} \]
  \item Let $U, V, W$ be finite-dimensional. $U \xrightarrow f V \xrightarrow g W$. Prove: $(g \circ f)^* = f^* \circ g^*$
  \item Let $U, V, W$ be finite-dimensional. $U \xrightarrow f V \xrightarrow g W$. Prove: $f^{**} = f$
  \item Let $U, V, W$ be finite-dimensional. $U \xrightarrow f V \xrightarrow g W$. Prove: $\ker{f} = (\im{f^*})^\bot$
  \item Let $U, V, W$ be finite-dimensional. $U \xrightarrow f V \xrightarrow g W$. Prove: $\im{f} = (\ke{f^*})^\bot$
  \item Let $U, V, W$ be finite-dimensional. $U \xrightarrow f V \xrightarrow g W$. Prove: $f$ injective $\iff f^*$ surjective
  \item Let $U, V, W$ be finite-dimensional. $U \xrightarrow f V \xrightarrow g W$. Prove: $f$ surjective $\iff f^*$ injective
  \item Define self-adjoint matrices
  \item Define linear isometries (= unitary transformations)
  \item Prove: unitary transformations are injective
  \item If $\dim{V} = \dim{W} < \infty$ and $f: V \to W$ is linear and unitary, then $f$ is invertible and $f^{-1} = f^*$.
  \item If $\dim{V} = \infty$, $f: V \to V$ is isometry, it does not imply that $f$ is invertible.
  \item Define unitary matrices
  \item Let $U$ be a unitary matrix. Prove: $UU^* = I \iff U^* U = I$.
  \item Define orthogonal matrices
  \item Eigenspaces are always $\square{}$
  \item Prove: $T \in \mathbb C^{n \times n}$ is unitary $\implies T$ is unitary
  \item Prove: $T \in \mathbb C^{n \times n}$. $\forall x,y \in \mathbb C^n: \ip{Tx}{Ty} = \ip xy \iff$ the columns of $T$ define an orthonormal basis of $\mathbb C^n$
  \item Define isometries in metric spaces.
  \item Is translation an isometry?
  \item Is translation unitary?
  \item What does the counterclockwise transformation matrix in $\mathbb R^2$ look like?
  \item Define the orthogonal group
  \item Define the unitary group
  \item Define the special orthogonal group
  \item Define the special unitary group
  \item Define the general linear group
  \item Define the special linear group
  \item Prove: $U \in \mathcal U(n) \implies \card{\det(U)} = 1$
  \item Give the general layout of a quaternion
  \item What is the multiplicative identity element of quaternions?
\end{itemize}

\section{Polynomials}

\begin{itemize}
  \item Define algebras
  \item Does an algebra satisfy associativity?
  \item Does an algebra satisfy commutativity?
  \item Define Jordan algebras
  \item Define the polynomial algebra
  \item Define the algebra of formal power series
  \item Define the degree of polynomials. Give the degree of polynomials $0, 1, x$ and $x^2$.
  \item Prove: Let $p(x)$ and $q(x)$ be polynomials. $\deg(p(x)) + \deg(q(x)) = \deg(g(x) \cdot q(x))$
  \item Define zero division freedom for polynomials
  \item Prove: Every polynomial induces a polynomial function and the map $p(x) \mapsto p_f(x)$ is linear and multiplicative.
  \item Define algebra homomorphisms
  \item Give and prove the insertion theorem for polynomials
  \item Define the root of a polynomial
  \item There is a formula to find roots of any quadratic equation. What is the highest polynomial degree such that a formula exists?
  \item What is the idea of Cardanos cubic formula?
  \item Prove polynomial division with remainder: $p(x), q(x) \in \mathbb K[x]$, $q(x) \neq 0$. Then there exists exactly one polynomial $s(x), r(x) \in \mathbb K[x]$, $p(x) = s(x) \cdot q(x) + r(x)$.
  with $\deg{r(x)} < \deg{q(x)}$.
  \item Give Ruffini-Horner's method
  \item Define reducibility of polynomials
  \item Give the Fundamental Theorem of Algebra
  \item Define the greatest common divisor for polynomials
  \item Prove Bezout's identity for polynomials
\end{itemize}

\section{Eigenvalues and eigenvectors}

\begin{itemize}
  \item Define eigenvalues and eigenvectors
  \item Give an intuitive notion of eigenvalues and eigenvectors
  \item Can 0 be an eigenvalue? Can $\vec 0$ be an eigenvector?
  \item Does every matrix have eigenvalues and eigenvectors?
  \item Give simple examples in $\mathbb R^{2 \times 2}$ and $\mathbb Q^{2 \times 2}$ without eigenvectors and eigenvalues.
  \item Define spectrums of linear maps
  \item Let $A, B \in \mathbb C^{n \times n}$. Prove: $\operatorname{spec}(AB) = \operatorname{spec}(BA)$
  \item What kind of linear map do you need to consider eigenspaces (necessary, but not sufficient condition)?
  \item Define eigenspaces of linear maps
  \item What is the eigenvalue of the matrix $\lambda \cdot \operatorname{id}$?
  \item Let $b_1, \dots, b_n$ be a basis of $V$. Let $\lambda_1, \dots, \lambda_n \in \mathbb K$. Then there exists a linear map $f$ such that $f(b_i) = \lambda_i \cdot b_i$ and this map is $\square{}$.
  \item Prove: Left-sided eigenvalue $\iff$ right-sided eigenvalue
  \item Write it down formally: The spectrum does not depend on the choice of the basis
  \item Prove: The spectrum does not depend on the choice of the basis
  \item Define characteristic polynomials of matrices
  \item Let $A \in \mathbb K^{n \times n}$. What is the degree of the characteristic polynomial of $A$?
  \item Complete and prove: $\lambda$ is eigenvalue $\iff \chi_A(\lambda) = \square{}$
  \item Prove: A square matrix $A$ is invertible if and only if $0$ is not an eigenvalue of $A$
  \item Define symmetrical minors
  \item Prove: $\chi_{T^{-1} AT}(x) = \chi_A(x)$
  \item Define diagonalizable matrices
  \item Define equivalent matrices
  \item Define similar matrices
  \item Prove: $A$ is diagonalizable $\iff \exists$ basis of eigenvectors.
  \item Define exponentiation of matrices
  \item Define Fibonacci sequences
  \item Give an explanation of the relationship of rabbit populations and the Fibonacci sequence
  \item Give the iteration matrix of the Fibonacci sequence
  \item The golden ratio is the root of which polynomial?
  \item Give the golden ratio
  \item Prove: eigenvectors corresponding to different eigenvalues are linear independent.
  \item Prove: an $n\times n$ matrix with $n$ different eigenvalues is diagonalizable.
  \item Define the geometric and algebraic multiplicity of an eigenvalue
  \item Prove: A matrix is diagonalizable iff for different eigenvalues $\lambda_1, \dots, \lambda_r$ it holds that $\sum_{i=1}^r d(\lambda_i) = n$
  \item Give an argument why there are at most $n$ eigenvalues for a matrix in $\mathbb K^{n \times n}$
  \item Prove: For every eigenvalue $\lambda$, it holds that $d(\lambda) \leq k(\lambda)$
  \item Define nilpotent matrices
\end{itemize}

\section{Jordan normal form}

\begin{itemize}
  \item Define invariant subspaces
  \item Show: Eigenspaces are invariant subspaces
  \item Prove: Let $A \in \mathbb K^{n\times n}, V = \mathbb K^n$. If $U \subseteq V$ is invariant and $p(x) \in \mathbb K[x]$, then $U$ is invariant under $p(A)$.
  \item Prove: Let $A \in \mathbb K^{n\times n}, V = \mathbb K^n$. $U_1, \dots, U_k$ are invariant subspaces. Then $\bigcap_{i=1}^k U_i$ and ${\text{\huge +}}_{i=1}^k U_i$ are invariant with respect to $A$.
  \item Prove: Let $f: V \to V$ and let $U \subseteq V$ be an invariant subspace. Then $f|_U: U \to U$ is a homomorphism
  \item Prove: Let $f: V \to V$. If $V$ can be decomposed into a direct sum of invariant subspaces, then $A$ can be transformed into block diagonal form.
  \item Let $\dim{V} = n, f \in \operatorname{End}(V)$. Prove: $\set{0} \subseteq \ker{f} \subseteq \ker{f^2} \subseteq \ker{f^3} \subseteq \dots$ and $\im{f} \supseteq \im{f^2} \supseteq \im{f^3} \supseteq \dots$
  \item Let $\dim{V} = n, f \in \operatorname{End}(V)$. Prove: $\exists m \leq n: \ker{f^m} = \ker{f^{m+1}}$
  \item Let $\dim{V} = n, f \in \operatorname{End}(V)$. Prove:
    \[ \ker{f^m} = \ker{f^{m+1}} \iff \im{f^m} = \im{f^{m+1}} \iff \ker{f^m} = \ker{f^{m+k}} \forall k \geq 1 \iff \]
    \[ \im{f^m} = \im{f^{m+k}} \forall k \geq 1 \iff \ker{f^m} \cap \im{f^m} = \set{0} \iff V = \ker{f^m} \dot{+} \im{f^m} \]
  \item Explain Fitting's Lemma in a few words.
  \item Define generalized spaces. Define generalized eigenvectors. How do generalized eigenvectors differ from eigenvectors?
  \item Prove: Let $\lambda_1, \dots, \lambda_k$ be different eigenvalues of $A$ and $\ker(\lambda I - A)^{r_i}$ the corresponding generalized spaces where
    \[ \ker(\lambda_i I - A)^{r_{i-1}} \subsetneq \ker(\lambda_i I - A)^{r_i} = \ker(\lambda_i \cdot I - A)^{r_i + 1} \]
    \[ \implies \bigcap_{i=1}^k \im(\lambda_i - A)^{r_i} \cap \ker((\lambda_1 I - A)^{r_1} (\lambda_2 I - A)^{r_2} \dots (\lambda_k I - A)^{r_k}) = \set{0} \]
  \item Prove: $\forall \lambda \neq \mu \in \operatorname{spec}(A) \forall k,l \geq 1: \ker(\lambda I - A)^k \cap \ker(\mu I - A)^l = \set{0}$
  \item Prove: The sum $\sum_{i=1}^k \ker(\lambda_i I - A)^{r_i}$ is direct for arbitrary pairwise different $\lambda_1, \dots, \lambda_k$.
  \item Prove: Let $\lambda_1, \dots, \lambda_k$ be pairwise different eigenvalues of $A \in \mathbb K^{n\times n}$. Let $W \coloneqq \bigcap_{i=1}^k \im(\lambda_i I - A)^n$. $V = \ker(\lambda_1 I - A)^{n} \oplus \dots \oplus \ker(\lambda_k I - A)^n \oplus \bigcap_{i=1}^k \im(\lambda_i I - A)^n$
  \item Prove: $W$ is invariant under $A$ and $\lambda_i \not\in \operatorname{spec}(A|_W) \forall i \in \set{1, \dots, k}$
  \item Prove: Let $\mathbb K$ be algebraically closed and let $\lambda_1, \dots, \lambda_k$ be all eigenvalues of a matrix $A \in \mathbb K^{n\times n}$. Then $\mathbb K^{n} = \ker(\lambda_1 I - A)^n \oplus \dots \oplus \ker(\lambda_k I - A)^n$.
  \item What is the index of a linear map?
  \item Define nilpotent matrices.
  \item The sum of nilpotent matrices is $\square{}$
  \item The product of nilpotent matrices is $\square{}$
  \item Assume all generalized spaces are eigenspaces. What about diagonizability?
  \item Prove: Let $\ker(f^m) \subseteq \ker(f^{m+1}) \subseteq \ker(f^{m+2})$
    \begin{align*}
      u_1 \dots u_p & \dots \text{ basis of } \ker{f^n} \\
      u_1 \dots u_p v_1 \dots v_k & \dots \text{ basis of } \ker{f^{m+1}} \\
      u_1 \dots u_p v_1 \dots v_k w_1 \dots w_r & \dots \text{ basis of } \ker{f^{m+2}}
    \end{align*}
    Then $(u_1 \dots u_p, f(w_1), \dots, f(w_r))$ is linear independent.
  \item What kind of matrix is the Jordan normal form of a matrix?
  \item Prove: Let $\dim{V} = n$. $f: V \to V$ is nilpotent of index $p$ ($f^p = 0$). $d = \dim\ker{f}$. Then there exists a basis $B$ of $V$ such that
    \[ \operatorname{diag}\left(\Phi_B^B(f)\right) = ([N_1], [N_2], \dots, [N_d]) \]
    where
    \[
      N_i = \begin{bmatrix} 0 & 1 & \ddots & 0 \\  & 0 & 1 & \\ & & \ddots & 1 \\ 0 & & \ddots & 0 \end{bmatrix}_{n_i \times n_i}
      \qquad p = n_1 \geq n_2 \geq \dots \geq n_d \geq 1
      \qquad n_1 + \dots + n_d = n
    \]
  \item Define Jordan blocks of length $k$ of a given eigenvalue $\lambda$.
  \item Prove: Let $\mathbb K$ be an algebraically closed field. Then every matrix $A \in \mathbb K^{n+m}$ is similar to a matrix of Jordan normal form.
  \item Let $B^{-1} AB = \begin{bmatrix} J_1 & & \\ & \ddots & \\ & & J_q \end{bmatrix} \in \mathbb K^{n + m}$ be a Jordan normal form with $J_i = J_{k_i}(\lambda_i)$. Prove: $\sum_{i=1}^q k_i = n$
  \item Let $B^{-1} AB = \begin{bmatrix} J_1 & & \\ & \ddots & \\ & & J_q \end{bmatrix} \in \mathbb K^{n + m}$ be a Jordan normal form with $J_i = J_{k_i}(\lambda_i)$. Prove: Geometric multiplicity of $\lambda$ equals the number of corresponding Jordan blocks. Algebraic multiplicity of $\lambda$ equals the num of sizes of corresponding Jordan blocks.
  \item Let $B^{-1} AB = \begin{bmatrix} J_1 & & \\ & \ddots & \\ & & J_q \end{bmatrix} \in \mathbb K^{n + m}$ be a Jordan normal form with $J_i = J_{k_i}(\lambda_i)$. Show: The smallest exponent $r$ such that $\ker((\lambda I - A)^r) = \ker((\lambda I - A)^{r+1})$ is the largest length of a corresponding Jordan block.
  \item Let $B^{-1} AB = \begin{bmatrix} J_1 & & \\ & \ddots & \\ & & J_q \end{bmatrix} \in \mathbb K^{n + m}$ be a Jordan normal form with $J_i = J_{k_i}(\lambda_i)$. Prove: Let $k \in \mathbb N$. $\#\set{i: \lambda_i = \lambda \land k_i \geq k + 1} = \rank(\lambda I - A)^k - \rank(\lambda I - A)^{k+1}$
  \item Let $B^{-1} AB = \begin{bmatrix} J_1 & & \\ & \ddots & \\ & & J_q \end{bmatrix} \in \mathbb K^{n + m}$ be a Jordan normal form with $J_i = J_{k_i}(\lambda_i)$. Prove: The Jordan blocks are uniquely determined (except for the order)
  \item Let $A \in \mathbb K^{n + n}$ matrix. $\Psi_A: \substack{\mathbb K[x] \to \mathbb K^{n + n} \\ p(x) \mapsto p(A)}$ and $a_0 + a_1 x + \dots + a_k x^k \mapsto a_0 \cdot I + a_1 A + \dots + a_k A^k$. Prove: $p\left(\begin{bmatrix} \lambda_1 & & 0 \\ & \ddots & \\ 0 & & \lambda_n \end{bmatrix}\right)$.
  \item Let $A \in \mathbb K^{n + n}$ matrix. $\Psi_A: \substack{\mathbb K[x] \to \mathbb K^{n + n} \\ p(x) \mapsto p(A)}$ and $a_0 + a_1 x + \dots + a_k x^k \mapsto a_0 \cdot I + a_1 A + \dots + a_k A^k$. Prove: $p\left(\begin{bmatrix} \lambda_1 & & 0 \\ & \ddots & \\ 0 & & \lambda_n \end{bmatrix}\right) = \begin{bmatrix} p(\lambda_1) & & \\ & \ddots & \\ & & p(\lambda_n) \end{bmatrix}$.
  \item Let $A \in \mathbb K^{n + n}$ matrix. $\Psi_A: \substack{\mathbb K[x] \to \mathbb K^{n + n} \\ p(x) \mapsto p(A)}$ and $a_0 + a_1 x + \dots + a_k x^k \mapsto a_0 \cdot I + a_1 A + \dots + a_k A^k$. Prove: $p\left(\begin{bmatrix} A_1 & & \\ & \ddots & \\ & & A_n \end{bmatrix}\right) = \begin{bmatrix} p(A_1) & & \\ & \ddots & \\ & & p(A_n) \end{bmatrix}$
  \item Let $A \in \mathbb K^{n + n}$ matrix. $\Psi_A: \substack{\mathbb K[x] \to \mathbb K^{n + n} \\ p(x) \mapsto p(A)}$ and $a_0 + a_1 x + \dots + a_k x^k \mapsto a_0 \cdot I + a_1 A + \dots + a_k A^k$. Prove: $A = T^{-1} BT \implies p(A) = T^{-1} p(B) T$
  \item For some Jordan block $J_k(\lambda)$ it holds that,
    \[
      p(J_k(\lambda))_{i,j} = \begin{cases}
        \frac{p^{(j - i)}(\lambda)}{(j - i)} & j > i \\
        p(\lambda) & j = i \\
        0 & j < i \text{ (below the diagonal)}
      \end{cases}
    \]
  \item Let $A \in \mathbb K^{N \times N}$. Prove: $\exists p(x) \in \mathbb K[x]: p(A) = 0$
  \item Define the annihilator of a matrix.
  \item Let $A \in \mathbb K^{N \times N}$. Prove: $\exists$ a unique polynomial $m_A(x) \in \mathbb K[x]$ with minimal degree and leading coefficients $1$ and $p(x) \in \operatorname{Ann}(A) \iff m_A(x) \divides{} p(x)$
  \item Let $A \in \mathbb K^{N \times N}$. Prove: $m_A(\lambda) = 0 \forall \lambda \in \operatorname{spec}(A)$
  \item Give and prove the Cayley-Hamilton Theorem
  \item Prove: $m_A(x) \divides{} \chi_A(x)$
  \item Conclude: the roots of $m_A(x)$ are the eigenvalues of $A$
  \item Recognize: The minimal polynomial has the structure $m_A(x) = \prod (\lambda - \lambda_i)^{m_i}$ where $m_i$ is the smallest exponent for $\ker(\lambda_i - A)^m = \ker(\lambda_i - A)^{m+1}$, hence this equals the largest length of a Jordan block for eigenvalue $\lambda_i$.
  \item Recognize: $A$ is diagonalizable $\iff$ all $m_i = 1 \iff m_A(x) = \prod_{i=1}^k (\lambda - \lambda_i) \iff m_A(x)$ has only simple roots.
  \item $p(A) \cdot x = p(\lambda) \cdot x \qquad \text{ if } \lambda \in \operatorname{spec}(A) \implies p(\lambda) \in \operatorname{spec}(p(A))$
  \item What is an important precondition for the spectral mapping theorem?
  \item Give and prove the spectral mapping theorem
\end{itemize}

\section{Normal matrices}

\begin{itemize}
  \item Define normal matrices
  \item If a matrix is self-adjoint, then is it necessarily $\square{}$
  \item A real-valued self-adjoint matrix is called $\square{}$
  \item Are unitary matrices normal?
  \item Prove: $A$ is normal $\iff$ $A$ is unitarily diagonalizable.
  \item The sum of normal matrices is $\square{}$
  \item The product of normal matrices is $\square{}$
  \item $A \in \mathbb C^{n \times n}$ is normal. Prove: $\ker{A} = \ker{A^*}$
  \item $A \in \mathbb C^{n \times n}$ is normal. Prove: $\ker{A} = \ker{A^2}$
  \item Let $A \in \mathbb C^{n \times n}$ is normal. Prove: $\ker(\lambda I - A) = \ker(\overline{\lambda} I - A^*)$
  \item Let $A \in \mathbb C^{n \times n}$ is normal. Prove: $\ker(\lambda I - A)^2 = \ker(\lambda I - A)$
  \item Let $A$ be normal. Prove: $\lambda \neq \mu \in \operatorname{spec}{A} \implies \ker(\lambda I - A) \bot \ker(\mu I - A)$
  \item Matrix $A$ is normal. What can you say about generalized eigenspaces?
  \item Let $A \in \mathbb C^{n \times n}$ be normal. Prove: Eigenvectors of different eigenvalues are orthogonal to each other.
  \item Let $A \in \mathbb C^{n \times n}$. Prove: $A$ is normal $\implies$ $\exists$ orthonormal basis of eigenvectors
  \item Let $A \in \mathbb C^{n \times n}$. Prove: $\exists$ orthonormal basis of eigenvectors $\implies$ $\exists$ unitary matrix $U$ such that $U^* AU = \operatorname{diag}(\lambda_1, \dots, \lambda_n)$.
  \item Let $A \in \mathbb C^{n \times n}$. Prove: $\exists$ unitary matrix $U$ such that $U^* AU = \operatorname{diag}(\lambda_1, \dots, \lambda_n) \implies A$ is normal.
  \item Define Schur's decomposition
  \item If $A \in \mathbb R^{n \times n}$ and $\chi_A(\lambda)$ decomposes into linear factors. $\exists U \in \mathcal U(n): U^* AU = R$ is an upper triangular matrix. Then $U$ $\square{}$
  \item Prove: A matrix is normal $\iff$ Schur normal form = diagonal matrix.
  \item Prove: Let $A \in \mathbb C^{n \times n}, A = A^* \implies \operatorname{spec}(A) \subseteq \mathbb R$.
  \item Let $A \in \mathbb C^{n \times n}$ self-adjoint. Prove: $A \geq 0 \iff \operatorname{spec}(A) \subseteq [0, \infty)$.
  \item For which matrices is the square root of a matrix defined?
  \item Prove: The square root of a positive definite matrix exists and is unique.
  \item Define Cholesky decompositions
  \item Define Schur complements of $D$ in $M$. Give the dimensions of $D$ and $M$.
  \item Let $A \in \mathbb C^{n \times n}$. $A > 0$, $b \in \mathbb C^n$, $\gamma > 0$.
    \[ \det\left[\begin{array}{c|c}A & B \\ \hline b^* & \gamma \end{array}\right] = \det A \cdot (\gamma - b^* A^{-1} b)  \]
  \item Cholesky decomposition is a $\square{}$ decomposition
  \item Prove: The lower triangular matrix in the Cholesky decomposition is unique
  \item Prove: If $A > 0$, then $\det{A} \leq a_{11} a_{22} \dots a_{nn}$.
  \item Give and prove Hadamard's inequality
  \item Which matrices have polar decompositions?
  \item Define the polar decomposition of a matrix $A$.
  \item Prove: Let $A \in \mathbb C^{n \times n}$. $\card{A} \coloneqq (A^* A)^{\frac12}$. Then $\exists U \in \mathcal U(n)$ such that $A = U \cdot \card{A}$.
  \item Which matrices have singular values?
  \item Define singular value decompositions of matrices.
\end{itemize}

\section{Eigenvalue estimates}

\begin{itemize}
  \item Define the numerical range of a matrix $A$
  \item Define the numerical radius of matrix $A$
  \item Prove: $\operatorname{spec}(A) \subseteq W(A)$
  \item Give and prove the Theorem by Toeplitz--Hausdorff
  \item Define Rayleigh quotients.
  \item Which requirements are given for matrix $A$ in the Rayleigh-Ritz Theorem and the Courant--Fischer--Weyl min--max principle?
  \item Give and prove the Rayleigh-Ritz Theorem.
  \item Give and prove the Courant--Fischer--Weyl min--max principle.
  \item Which one is more generic: Rayleigh-Ritz Theorem or Courant--Fischer--Weyl min--max principle?
  \item Give and prove the Cauchy interlacing theorem
  \item Prove: $A, B$ are self-adjoint $\in \mathbb C^{n \times n}$. $\lambda_{k}(A) + \lambda_1(B) \leq \lambda_k(A + B) \leq \lambda_k(A) + \lambda_n(B)$
  \item Define Ger\v{s}gorin discs
  \item Give and prove the Ger\v{s}gorin theorem
  \item What is the spectral gap of a matrix?
\end{itemize}

\section{Matrix norms}

\begin{itemize}
  \item Give one scalar product for matrices in $\mathbb C^{m \times n}$.
  \item Define Frobenius norms (= Schur norms) (= Hilbert-Schmidt norms)
  \item Let $V, W$ be normed vectorspaces. $f \in \operatorname{Hom}(V, W)$. Defined the so-called induced norm.
  \item Define optimal norms
  \item Define the spectral radius of a matrix
  \item What does every compatible matrix norm satisfy with respect to the spectral radius.
  \item Prove: The spectral radius is not a norm.
  \item Prove: $\forall A \in \mathbb C^{n \times n} \: \forall \varepsilon > 0 \:\exists \text{ norm on } \mathbb C^n:$ the induced matrix norm satisfies $\norm{A} \leq \rho(A) + \varepsilon$
  \item Define condition numbers of matrices
\end{itemize}

\section{Non-negative matrices}

\begin{itemize}
  \item Define non-negative matrices
  \item Give the Perron--Frobenius theorem
\end{itemize}

\section{Generic/recall question}

\begin{itemize}
  \item Which matrices satisfy $A B = B A$?
  \item Which matrices satisfy $A^{-1} = A$?
  \item Which matrices satisfy $A^{T} = A$?
  \item Which matrices satisfy $AA^* = I$?
  \item Which matrices satisfy $AA^* = A^*A$?
  \item Which matrices satisfy $A = A^*$?
\end{itemize}

\end{document}
